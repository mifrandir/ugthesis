% UG project example file, February 2022
% Do not change the first two lines of code, except you may delete "logo," if causing problems.
% Understand any problems and seek approval before assuming it's ok to remove ugcheck.
\documentclass[bsc,singlespacing,parskip,logo]{infthesis}
\usepackage{ugcheck}

% Include any packages you need below, but don't include any that change the page
% layout or style of the dissertation. By including the ugcheck package above,
% you should catch most accidental changes of page layout though.
\usepackage{microtype} % recommended, but you can remove if it causes problems
\usepackage{quiver}
\usepackage{hyperref}

\begin{document}
\begin{preliminary}

  \title{Towards Strong Relative Pseudomonads}

  \author{Franz Miltz}

  % CHOOSE YOUR DEGREE a):
  % please leave just one of the following un-commented
  %\course{Artificial Intelligence}
  %\course{Artificial Intelligence and Computer Science}
  %\course{Artificial Intelligence and Mathematics}
  %\course{Artificial Intelligence and Software Engineering}
  %\course{Cognitive Science}
  %\course{Computer Science}
  %\course{Computer Science and Management Science}
  \course{Computer Science and Mathematics}
  %\course{Computer Science and Physics}
  %\course{Software Engineering}
  %\course{Master of Informatics} % MInf students

  % CHOOSE YOUR DEGREE b):
  % please leave just one of the following un-commented
  %\project{MInf Project (Part 1) Report}  % 4th year MInf students
  %\project{MInf Project (Part 2) Report}  % 5th year MInf students
  \project{4th Year Project Report}        % all other UG4 students

  \date{\today}

  \abstract{
    Synthetic measure theory uses commutative monads to develop an entirely categorical
    language of measures and integration. This language has proven practically useful
    in the development of higher-order statistical programming languages.

    There is another purely categorical notion of integration: coends. Certain coends
    arise from the presheaf construction, a monad-like structure that fails to be a
    model of synthetic measure theory for several reasons. There are multiple ways
    in which one could attempt to fix these problems.

    In this report, we make a start at defining a strong relative pseudomonad that
    would be suitable as the backbone of an extended synthetic measure theory. At each
    step, we show how the presheaf construction gives rise to the required structure and
    how this structure satisfies the necessary axioms.

    Unfortunately, the strength of a monad does not lend itself to being generalised
    to relative pseudomonads. We are thus neither able to give a complete definition
    of a strong relative pseudomonad nor do we manage to extend synthetic measure theory
    to admit the presheaf construction as a model.
  }

  \maketitle

  \newenvironment{ethics}
  {\begin{frontenv}{Research Ethics Approval}{\LARGE}}
  {\end{frontenv}\newpage}

  \begin{ethics}
    This project was planned in accordance with the Informatics Research
    Ethics policy. It did not involve any aspects that required approval
    from the Informatics Research Ethics committee.

    \standarddeclaration
  \end{ethics}


  \begin{acknowledgements}
    I am incredibly grateful to Dr Ohad Kammar, my supervisor, who started to
    talk to me about category theory well before I began my work on this
    project. Thank you for your patience, your encouragement, and the vast
    amount of knowledge that you have been able to share with me.

    Thanks should also go to Mr Matthias K\"onig. Thank you for setting me on
    the magnificent path towards abstract mathematics. I would not be where I am
    today without you.

    Finally, I would like to express my sincerest appreciation to Mary
    and Fred. You have never failed to put a smile on my face.
  \end{acknowledgements}


  \tableofcontents
\end{preliminary}

\chapter{Introduction}\label{sec:introduction}

\section{Synthetic measure theory}

Kock observed that it is possible to characterise measure theory abstractly by a commutative monad
over a locally small cartesian closed category satisfying two further axioms~\cite{kock2011}.

For practical purposes, it makes sense to consider models that account for traditional Lebesgue integration.
However, function spaces are never measurable~\cite{aumann1961} so the category of measurable
spaces is not cartesian closed. This means that conventional measure theory is not a model of synthetic measure theory.
It is therefore interesting to find models that are close to conventional measure theory.

Measurable function spaces are desirable for other reasons, e.g.\ to formalise higher-order probability
theory, which is a problem that arises naturally when considering probabilisitic programming languages.
It is possible to extend measurable spaces to quasi-Borel spaces in order to achieve cartesian closure~\cite{heunen2017}. While some care is required when comparing quasi-Borel spaces to measure theory
and rephrasing probability theory in these new terms, they bring with them several convenient properties.
In particular, quasi-Borel spaces form a model of synthetic measure theory and the associated integral
is precisely the measure space integral~\cite{scibor2018}.

Since synthetic measure theory turns out to be a practically useful tool, it is now also
valuable to consider models that are far removed from conventional measure theory. One
may think of this as testing the limits of the theory.

A classical example of a monad from functional programming is the list monad.
This fails to be commutative. If we forget about the ordering of elements, we
obtain the powerset monad taking sets to their powersets. This monad may be
endowed with a commutative strong structure that, moreover, forms a model of synthetic
measure theory. In this model, the measures over a set $X$ are subsets
$\mu\subseteq X$ and the integral of a function $f:X\to\cc{0,1}$ with respect
to $\mu$ is
\begin{align*}
  \int_X \mu\rr{dx}f\rr{x} = \begin{cases}
    0 & \text{if }\forall  x\in \mu. f\rr{x}=0 \\
    1 & \text{if }\exists  x\in \mu. f\rr{x}=1
  \end{cases}
\end{align*}
In other words, if we consider $f$ to be the characteristic function of a subset $B\subseteq X$
then the integral is zero if and only if $\mu\cap B=\emptyset$.

It is worth noting that Kock's synthetic measure theory is not the only attempt at
formalising probability theory categorically. An alternative approach was taken by
Fritz in~\cite{fritz2020} who developed Markov categories by generalising probability
theory directly. This allows for more abstract reasoning,
thereby improving clarity. In particular, Fritz' theory unifies different types of probability
theory and does not rely on measure theoretic probability theory to the same degree.

\section{Presheaves}

Category theory has its own notion of integration: coends. A coend of a
functor $\catop{C}\times\cat{C}\to\cat{D}$ is a particular colimit in
$\cat{D}$. We tend to think of coends as integrals because they behave
like such.

However, as of right now there have been little to no formal developments
relating coends to integration. The motivation for this project is to
establish such a connection by considering a suitable model of synthetic
measure theory. To construct this model, we require the following:

\begin{enumerate}
  \item a cartesian closed category $\cat{C}$ and
  \item a commutative monad $T$ on $\cat{C}$ such that
  \item the extension operation of $T$ involves coends.
\end{enumerate}

These conditions are contradictory. Coends are not unique, so defining an
extension operation involves choosing particular coends. It is then not
reasonable to demand that this choice is consistent in the sense that all the
monad laws are satisfied up to equality. Thus we will not obtain a monad in the
strict sense. Fortunately, monads have been generalised to pseudomonads to
account for this problem~\cite{marmolejo2013}.

Now the obvious candidate model is the presheaf construction, taking each small
category $\scat{C}$ to its presheaf category $\widehat{\scat C} =
\bb{\scatop{C}, \Set}$. This is a great choice because a functor
$\scat{C}\to\widehat{\scat C}$ may be thought of as a functor
$\scatop{C}\times\scat{C}\to\Set$. Thus coends are closely related to
presheaves. Secondly, it is known that the presheaf construction exhibits
monad-like behaviour~\cite{fiore2017}. Unfortunately, the presheaf category of
a small category is itself not necessarily small. Thus the presheaf
construction fails to be a pseudomonad.

One way to turn the presheaf construction into a pseudomonad is by considering
small presheaves~\cite{day2007}. This is still insufficient for our pruposes,
as the functor category $\bb{\scat{C},\scat{D}}$ of small categories $\scat{C}$
and $\scat{D}$ is itself not small. This violates our first requirement of
cartesian closure. We have tried and failed to find a cartesian closed
2-category on which the presheaf construction remains a pseudomonad.

We thus opt for relative pseudomonads~\cite{fiore2017}. This generalisation
allows for the possibility that the pseudofunctor underlying a pseudomonad is
itself not an endomorphism. We know that the presheaf construction is a
relative pseudomonad. Before we can extend synthetic measure theory to include
relative pseudomonads, we need to define the corresponding notion of strength.
This turns out to be a difficult problem that we are not able to solve
entirely.

\section{Related work}

There are four pieces of work that this project builds on:

\begin{itemize}
  \item Kocks synthetic measure theory from~\cite{kock2011} is our main motivation. It
    is therefore especially helpful in making sure what requirements our theory needs
    to satisfy and what simplifications we are able to make.
  \item Fiore et al.~developed the theory theory of relative pseudomonads in~\cite{fiore2017}.
    This is our main reference point for the first part of our project.
  \item  Paquet and Saville defined strong pseudomonads in~\cite{saville2023}. This
    serves as a guide for the second part of our theory and will most likely be a
    valuable reference for future work.
  \item Uustalu outlined in~\cite{tarmo} how to construct a strong relative monad. While
    this extended abstract does justify the correctness of the definition, it has allowed
    us to come up with a sensible structure of a strong relative pseudomonad nonetheless.
\end{itemize}

\section{Contributions}

The following developments in this report are worth noting:

\begin{itemize}
  \item We partially define strong relative pseudomonads with a particular 
    focus on generalising synthetic measure theory.
  \item We show that our definition coincides with relative pseudomonads
    in the sense of~\cite{fiore2017}.
  \item We show how our definition yields the structure of a strong
    pseudomonad in the sense of~\cite{saville2023}.
  \item We demonstrate how to endow the presheaf construction with the
    necessary structure and verify that it satisfies the corresponding
    axioms.
\end{itemize}

\section{Overview}

We structure this report as follows:
\begin{itemize}
  \item Chapter 2 revisits introductory category theory. We define commutative
    monads and relative monads in order to develop an intuition for later
    developments. We then focus our attention towards coends, getting used to
    the notation and proving results that will be useful later on.
  \item Chapter 3 rigorously introduces 2-categories, pseudofunctors, and
    pseudonatural transformations. We pay particular attention to the strictness
    requirements, making sure that our definitions are not more general than required.
  \item Chapter 4 consists entirely of novel developments: step by step we work towards
    the definition of a strong relative pseudomonad. Along the way we show how the
    structures and axioms relate to previous work and to the presheaf construction.
  \item Chapter 5 contains a critical evaluation.
  \item Chapter 6 summarises the remaining work and lists possible extensions.
\end{itemize}




\chapter{1-categories}\label{sec:synthetic_measure_theory_and_presheaves}

There are two purely categorical notions of integration that we are interested
in: the integral arising from Kock's synthetic measure theory~\cite{kock2011}
and coends as an integral of certain functors.

We begin this chapter by revisiting some introductory category theory. We then
proceed to define commutative monads, the backbone of synthetic measure theory.
After that, we shift our attention to coends, outlining in what sense the coend
of a functor may be thought of as an integral.

\section{Categories, functors, and natural transformations}\label{sec:categories}

This section presents standard constructions from the literature such as~\cite{maclane1997}.

Recall that a category consists of objects, morphisms between objects, and a
composition operation that is associative and has identities.

Let $\cat{C}$ be a category. For objects $X,Y\in\cat{C}$ we denote by
$\Hom_{\cat{C}}(X,Y)$ the class of all morphisms $X\to Y$ and by $\id_X$ the
identity $X\to X$. Moreover, we will drop the composition $\circ$ and
subscripts whenever it is convenient to do so. Such notational conventions will
prove particularly useful when considering larger structures later on.

One of the most important categories is \Set{} whose objects are sets and whose
morphisms are functions. Composition is just the usual composition of functions
and the identities are the usual identity functions.

Each category $\cat{C}$ has a dual $\catop{C}$ called the opposite category.
The objects of $\catop{C}$ are exactly the objects of $\cat{C}$ but the
morphisms $Y\to X$ in $\catop{C}$ are exactly the morphisms $X\to Y$ in $\cat{C}$.
Composition in $\catop{C}$ is the same as in $\cat{C}$.

Given two categories $\cat{C}$ and $\cat{D}$, we have the product category
$\cat{C}\times\cat{D}$ which has as objects all pairs $(X,Y)$ with $X\in\cat{C}$
and $Y\in\cat{D}$ and hom-sets $\Hom((X,Y),(X',Y'))=\Hom(X,Y)\times\Hom(X',Y')$.
Composition is defined pointwise so the identites are $(\id,\id)$.

Functors are morphisms between categories. They act on objects and morphisms in
a way that preserves identities and distributes over composition. One
particularly important functor is $\Hom : \scatop{C}\times\scat{C}\to\Set$,
where $\scatop C$ is the opposite category. It takes a pairs of objects in a small
category to its hom-set and the action on morphisms is given by composition:
$\Hom(f,g)(h) = ghf$. We may also fix an object $X\in\scat{C}$ to obtain the
functors
\begin{align*}
  \Hom(X,-):\scat C\to\Set, \hs \Hom(-,X):\scatop C\to\Set.
\end{align*}

Let $F:\cat{C}\to\cat{D}$ and $G:\cat{D}\to\cat{E}$ be functors. The composite
$GF:\cat{C}\to\cat{E}$ is defined as $GF = G(F-)$. We thus have the category
$\CAT$ of all categories and all functors between them and its full subcategory
$\Cat$ of small categories.

Let $F,G:\cat{C}\to\cat{D}$ be parallel functors. A natural transformation $\phi
: F \Rightarrow G$ consists of morphisms $\phi_X : FX \to GX$ in $\cat{D}$, for
all $X\in\cat{C}$, that satisfy the naturality condition: for all $f:X\to Y$ in
$\cat{C}$, $\phi_Y\circ Ff = Gf\circ \phi_X$.

We can compose natural transformations $F\Rightarrow G$ and $G\Rightarrow H$ by
composing the components. Thus, for all categories $\cat{C}$ and $\cat{D}$, we
have the functor category $\bb{\cat{C},\cat{D}}$ with functors
$\cat{C}\to\cat{D}$ as objects and natural transformations as morphisms.

Consider functors $F,G:\cat{C}\to\bb{\cat{D},\cat{E}}$. A natural
transformation $\phi : F\Rightarrow G$ has components $\phi_C : FC\to GC$ in
$\bb{\cat{D},\cat{E}}$. This means that each $\phi_C$ is itself a natural
transformation with components $(\phi_C)_D : (FC)D \to (GC)D$ in
$\cat{E}$. Such functors and natural transformations will play a major role in
later chapters. We will make sure to keep our notation precise to assist the
reader in peeling back the various layers of indirection.

\section{Cartesian structure}\label{sec:cartesian_categories}

Now recall that the product of objects $X,Y\in\cat{C}$ is an object $X\times Y$
together with morphisms $\pi_1 : X\times Y \to X$ and $\pi_2 : X\times Y \to
Y$. It is universal in the sense that, for all other pairs of morphisms $f:W\to
X$ and $g:W\to Y$, there is a unique morphism $\aa{f,g}:W\to X\times Y$ such
that the following commutes:
\begin{equation}\label{eq:product_diagram}
  % https://q.uiver.app/?q=WzAsNCxbMiwwLCJXIl0sWzIsMSwiWFxcdGltZXMgWSJdLFswLDEsIlgiXSxbNCwxLCJZIl0sWzEsMywiXFxwaV8yIiwyXSxbMSwyLCJcXHBpXzEiXSxbMCwxLCJcXGxhbmdsZSBmLGdcXHJhbmdsZSIsMV0sWzAsMiwiZiIsMl0sWzAsMywiZyJdXQ==
  \begin{tikzcd}
  && W \\
    X && {X\times Y} && Y
    \arrow["{\pi_2}"', from=2-3, to=2-5]
    \arrow["{\pi_1}", from=2-3, to=2-1]
    \arrow["{\langle f,g\rangle}"{description}, from=1-3, to=2-3]
    \arrow["f"', from=1-3, to=2-1]
    \arrow["g", from=1-3, to=2-5]
  \end{tikzcd}
\end{equation}

We say that a diagram commutes if every two paths from the same source to the
same sink are equal. We will make heavy use of commutative diagrams, both as
axioms that we postulate and as statements that we prove. In the latter case,
the commutativity may be left implicit. Unfortunately, diagrams tend to get
much more complex in later chapters. Therefore it will not be possible to write
down every proof in full detail. We will, however, occasionally include links
to quiver~\cite{quiver}, a graphical editor for commutative diagrams. For
example, the diagram \ref{eq:product_diagram} corresponds to \href{
  https://q.uiver.app/?q=WzAsNCxbMiwwLCJXIl0sWzIsMSwiWFxcdGltZXMgWSJdLFswLDEsIlgiXSxbNCwxLCJZIl0sWzEsMywiXFxwaV8yIiwyXSxbMSwyLCJcXHBpXzEiXSxbMCwxLCJcXGxhbmdsZSBmLGdcXHJhbmdsZSIsMV0sWzAsMiwiZiIsMl0sWzAsMywiZyJdXQ==
}{this quiver link}. The purpose of these links is twofold: On the one hand
they should serve as a useful guide through the less obvious proofs and on the
other we hope that they may serve as resource for future reference. However, we
believe that this report is entirely self-contained and therefore the reader is
not required to engage with the quiver links whatsoever.

We now take the product of morphisms $f:X\to X'$ and $g:Y\to Y'$ to be the
morphism $f\times g:X\times X'\to Y\times Y'$ given by $f\times
g=\aa{f\pi_1,g\pi_2}$. Thus we have a functor $\cat{C}\times\cat{C}\to\cat{C}$.
It is important to realises that, in general, products are not unique.
Fortunately, they are unique \emph{up to canonical isomorphism}. Therefore
specifying such a product functor amounts to choosing a product for each pair
of objects.

As the product functor depends on the choice of products, we do not expect it to
be associative or commutative in the strict sense. However, there are natural
isomorphisms with components
\begin{align}\label{eq:product_alpha_gamma}
  \alpha_{X,Y,Z} : (X\times Y)\times Z \cong X\times (Y\times Z), \hs
  \gamma_{X,Y} : X\times Y \cong Y\times X.
\end{align}
It is worth pointing out that $\gamma$ is its own inverse. That is,
$\inv\gamma_{X,Y}=\gamma_{Y,X}$.

An object $1\in\cat{C}$ is terminal if, for all $X\in\cat{C}$, there is a unique
morphism $X\to 1$. It turns out that terminal objects are units for
multiplication. That is, there is a natural isomorphism with components
\begin{align*}
  \lambda_X : 1\times X \cong X.
\end{align*}

Combining this with $\gamma$ from (\ref{eq:product_alpha_gamma}) yields the
right unitor $\rho=\lambda\gamma:X\times 1\cong X$.

To tie all of this together, a cartesian category is a category with a choice of
all (binary) products and a distinguished terminal object.

\begin{example}
  Two particular cartesian structures are going to be of interest to us: on $\Set$
  and on $\CAT$ (and thereby $\Cat$). In $\Set$, we take the product of two sets
  to be the cartesian product with the usual projections $(x,y)\mapsto x$ and
  $(x,y)\mapsto y$. The terminal object $1\in\Set$ is a distinguished singleton
  $1=\cc{*}$. In $\CAT$, the product of categories $\cat{C}$ and $\cat{D}$ is the
  product category $\cat{C}\times\cat{D}$ with the obvious projection functors and
  the terminal category $\mathbb 1\in\CAT$ is a distinguished category with a
  single object and a single (identity) morphism. Note that, for small categories
  $\scat C$ and $\scat D$, $\scat C\times\scat D$ is small so $\Cat$ inherits the
  cartesian structure.
\end{example}

\section{Monads}\label{sec:monads}

Monads are central to the study of category theory. They arise naturally in practical settings,
e.g.\ to model computation with side effects, as well as for theoretical
purposes, e.g.\ monad algebras as a generalisation of algebraic theories.

We adapt the no-iteration definition of a monad~\cite{marmolejo2010}
because it more easily generalises to relative monads~\cite{altenkirch2015}.

\begin{definition}\label{def:monad}
  Let $\cat{C}$ be a category. A \emph{monad $T$ on $\cat{C}$} consists of
  \begin{enumerate}
    \item for all $X\in\cat{C}$, an object $TX\in\cat{C}$;
    \item for all $X\in\cat{C}$, a morphism $\eta_X : X \to TX$ in $\cat{C}$;
    \item for all $X,Y\in\cat{C}$, a map
      \begin{align*}
        \extend{\rr{-}}{*}_{X,Y}:\Hom\rr{X,TY}\to \Hom\rr{TX,TY}
      \end{align*}
  \end{enumerate}
  such that, for all $X,Y,Z\in{\cat{C}}$, $f:X\to TY$, and $g:Y\to TZ$,
  the following commute:
  \begin{equation}
    % https://q.uiver.app/?q=WzAsMixbMCwwLCJUWCJdLFsyLDAsIlRYIl0sWzAsMSwiXFxldGFeKiIsMix7ImN1cnZlIjozfV0sWzAsMSwiIiwwLHsiY3VydmUiOi0zLCJsZXZlbCI6Miwic3R5bGUiOnsiaGVhZCI6eyJuYW1lIjoibm9uZSJ9fX1dXQ==
    \begin{tikzcd}
      TX && TX
      \arrow["{\eta^*}"', curve={height=18pt}, from=1-1, to=1-3]
      \arrow[curve={height=-18pt}, Rightarrow, no head, from=1-1, to=1-3]
    \end{tikzcd}
    \hs
    % https://q.uiver.app/?q=WzAsMyxbMCwwLCJYIl0sWzIsMCwiVFkiXSxbMiwyLCJUWSJdLFswLDEsIlxcZXRhIl0sWzEsMiwiZl4qIl0sWzAsMiwiZiIsMl1d
    \begin{tikzcd}
      X && TY \\
      \\
        && TY
        \arrow["\eta", from=1-1, to=1-3]
        \arrow["{f^*}", from=1-3, to=3-3]
        \arrow["f"', from=1-1, to=3-3]
    \end{tikzcd}
    \hs
    % https://q.uiver.app/?q=WzAsMyxbMCwwLCJUWCJdLFsyLDAsIlRZIl0sWzIsMiwiVFoiXSxbMCwxLCJmXioiXSxbMSwyLCJnXioiXSxbMCwyLCIoZ14qZileKiIsMl1d
    \begin{tikzcd}
      TX && TY \\
      \\
         && TZ
         \arrow["{f^*}", from=1-1, to=1-3]
         \arrow["{g^*}", from=1-3, to=3-3]
         \arrow["{(g^*f)^*}"', from=1-1, to=3-3]
    \end{tikzcd}
  \end{equation}
\end{definition}

The double line in the first diagram refers to the identity $TX\to TX$.

Let us now investigate the powerset monad which we touched on in chapter \ref{sec:introduction}.

\begin{example}
  The powerset construction $\P$ takes each set $X$ to its powerset $\P X$.
  We add a monad structure like so: the unit map takes elements
  to singletons, i.e. $\eta\rr{x}=\cc{x}$. The extension of a
  function $f:X\to\P Y$ takes $A\in\P X$ to the union
  $\extend{f}{T}\rr{A}=\bigcup \cc{f\rr{x} : x \in A}$.
  This is known as the powerset monad.
\end{example}

Every monad $T$ on $\cat{C}$ gives rise to an endofunctor $\cat{C}\to\cat{C}$
with object map $X\mapsto TX$ and morphism map $f\mapsto\extend{\rr{\eta
f}}{T}$. We abuse notation and denote this endofunctor simply by $T$ and write
$Tf:TX\to TY$, as usual. In the case of the powerset monad, the functorial
action takes a function $f:X\to Y$ to the direct image map $A \mapsto
\cc{f\rr{x} : x \in A}$.

\section{Commutative monads}\label{sec:commutative_monads}

Given that a monad is just a monoid in the category of
endofunctors~\cite{maclane1997}, one might expect that the commutativity of a
monad is related to the commutativity of the corresponding monoid. This is not
the case. The commutativity required for synthetic measure theory refers to the
strength of a monad. For our purposes, a monad is strong if it interacts well
with the cartesian structure of the corresponding category.

Fix a cartesian category $\cat{C}$.
A strength $\sigma$ for a monad $T$ on $\cat{C}$ is a natural
transformation with components
\begin{align*}
  \sigma_{X,Y}:X\times TY \to T\rr{X\times Y}
\end{align*}
following certain conditions. A strong monad is a monad $T$ equipped with a
strength $\sigma$.

Such a strong monad is then called commutative~\cite{kock1970}, if the costrength
\begin{align*}
  \tau_{X,Y}:TX\times Y\to T(X\times Y)
\end{align*}
given by the composite
\begin{equation}
  % https://q.uiver.app/?q=WzAsNCxbMCwwLCJUWFxcdGltZXMgWSJdLFsyLDAsIlQoWFxcdGltZXMgWSkiXSxbMCwxLCJZXFx0aW1lcyBUWCJdLFsyLDEsIlQoWVxcdGltZXMgWCkiXSxbMiwzLCJcXHNpZ21hIiwyXSxbMywxLCJUXFxnYW1tYSIsMl0sWzAsMiwiXFxnYW1tYSIsMl0sWzAsMSwiXFx0YXUiXV0=
  \begin{tikzcd}
    {TX\times Y} && {T(X\times Y)} \\
    {Y\times TX} && {T(Y\times X)}
    \arrow["\sigma"', from=2-1, to=2-3]
    \arrow["T\gamma"', from=2-3, to=1-3]
    \arrow["\gamma"', from=1-1, to=2-1]
    \arrow["\tau", from=1-1, to=1-3]
  \end{tikzcd}
\end{equation}
makes the diagram below commute:

\begin{equation}\label{eq:commutative_monad}
  % https://q.uiver.app/?q=WzAsNixbMCwwLCJUWFxcdGltZXMgVFkiXSxbMiwwLCJUKFRYXFx0aW1lcyBZKSJdLFs0LDAsIlReMihYXFx0aW1lcyBZKSJdLFs0LDEsIlQoWFxcdGltZXMgWSkiXSxbMiwxLCJUXjIoWFxcdGltZXMgWSkiXSxbMCwxLCJUKFhcXHRpbWVzIFRZKSJdLFswLDUsIlxcdGF1IiwyXSxbMCwxLCJcXHNpZ21hIl0sWzEsMiwiVFxcdGF1Il0sWzUsNCwiVFxcc2lnbWEiLDJdLFs0LDMsIlxcdGV4dHtpZH1eKiIsMl0sWzIsMywiXFx0ZXh0e2lkfV4qIl1d
  \begin{tikzcd}
    {TX\times TY} && {T(TX\times Y)} && {T^2(X\times Y)} \\
    {T(X\times TY)} && {T^2(X\times Y)} && {T(X\times Y)}
    \arrow["\tau"', from=1-1, to=2-1]
    \arrow["\sigma", from=1-1, to=1-3]
    \arrow["T\tau", from=1-3, to=1-5]
    \arrow["T\sigma"', from=2-1, to=2-3]
    \arrow["{\text{id}^*}"', from=2-3, to=2-5]
    \arrow["{\text{id}^*}", from=1-5, to=2-5]
  \end{tikzcd}
\end{equation}

\begin{example}
  In the case of $\Set$, the product is the usual cartesian product of sets. The powerset
  monad is a commutative monad with strength and costrength given by
  $\sigma\rr{x,B}=\cc{x}\times B$ and $\tau\rr{A,y}=A\times\cc{y}$, respectively.
\end{example}

\section{Relative monads}

This section closely follows~\cite{altenkirch2015}.

In section~\ref{sec:monads}, we chose to present the no-iteration definition of
a monad because it avoids repeated applications of the object map. This now
leads to an obvious generalisation of monads that are functors
$\cat{J}\to\cat{C}$ rather than endofunctors. We require another functor
$\cat{J}\to\cat{C}$ to relate objects in $\cat{J}$ to those in the image of
the relative monad. The resulting definition is remarkably similar to~\ref{def:monad}.

\begin{definition}\label{def:relative_monad}
  Let $J:\cat{J}\to\cat{C}$ be a functor. A \emph{relative monad $T$ over $J$}
  consists of
  \begin{enumerate}
    \item for all $X\in\cat{J}$, an object $TX\in\cat{C}$;
    \item for all $X\in\cat{J}$, a morphism $\eta_X : JX \to TX$ in $\cat{C}$;
    \item for all $X,Y\in\cat{J}$, a map
      \begin{align*}
        \extend{\rr{-}}{*}_{X,Y}:\Hom\rr{JX,TY}\to \Hom\rr{TX,TY}
      \end{align*}
  \end{enumerate}
  such that, for all $X,Y,Z\in{\cat{J}}$, $f:JX\to TY$, and $g:JY\to TZ$,
  the following commute:
  \begin{equation}
    % https://q.uiver.app/?q=WzAsMixbMCwwLCJUWCJdLFsyLDAsIlRYIl0sWzAsMSwiXFxldGFeKiIsMix7ImN1cnZlIjozfV0sWzAsMSwiIiwwLHsiY3VydmUiOi0zLCJsZXZlbCI6Miwic3R5bGUiOnsiaGVhZCI6eyJuYW1lIjoibm9uZSJ9fX1dXQ==
    \begin{tikzcd}
      TX && TX
      \arrow["{\eta^*}"', curve={height=18pt}, from=1-1, to=1-3]
      \arrow[curve={height=-18pt}, Rightarrow, no head, from=1-1, to=1-3]
    \end{tikzcd}
    \hs
    % https://q.uiver.app/?q=WzAsMyxbMCwwLCJKWCJdLFsyLDAsIlRZIl0sWzIsMiwiVFkiXSxbMCwxLCJcXGV0YSJdLFsxLDIsImZeKiJdLFswLDIsImYiLDJdXQ==
    \begin{tikzcd}
      JX && TY \\
      \\
         && TY
         \arrow["\eta", from=1-1, to=1-3]
         \arrow["{f^*}", from=1-3, to=3-3]
         \arrow["f"', from=1-1, to=3-3]
    \end{tikzcd}
    \hs
    % https://q.uiver.app/?q=WzAsMyxbMCwwLCJUWCJdLFsyLDAsIlRZIl0sWzIsMiwiVFoiXSxbMCwxLCJmXioiXSxbMSwyLCJnXioiXSxbMCwyLCIoZ14qZileKiIsMl1d
    \begin{tikzcd}
      TX && TY \\
      \\
         && TZ
         \arrow["{f^*}", from=1-1, to=1-3]
         \arrow["{g^*}", from=1-3, to=3-3]
         \arrow["{(g^*f)^*}"', from=1-1, to=3-3]
    \end{tikzcd}
  \end{equation}
\end{definition}

\begin{example}
  Let $T$ be a monad on $\cat{C}$. Then $T$ is a relative monad over the identity
  $\cat{C}\to\cat{C}$. Moreover, for any functor $J:\cat{J}\to\cat{C}$, we have a
  relative monad $T'$ over $J$ with object map $T'X = TJX$.
\end{example}

Our motivation for presenting this definition is the presheaf construction. It
behaves like a relative monad $\Cat\to\CAT$ but fails to satisfy the axioms in
the strict sense because coends, which we shall introduce next, are unique only
up to isomorphism.

While our definition of a monad is easy to generalise to the relative case, the
same cannot be said for commutative monads. The axioms of strong and
commutative monads involve repeated applications of the object map (see
\ref{eq:commutative_monad}). We therefore have to find alternative conditions
to impose. Fortunately,~\cite{tarmo} already contains the definition of a
strong relative moand that we will make use of. However, we will encounter
similar problems when identifying suitable coherence conditions in chapter~
\ref{sec:strong_relative_pseudomonads}.

\section{Coends}\label{sec:coends}

Let us now turn our attention to the categorical integral that we are hoping to
capture in a model of synthetic measure theory. There are two dual notions of an
integral of a functor $\catop{C}\times\cat{C}\to\cat{D}$: ends and coends. We will
focus on the latter because coends arise naturally when considering the Yoneda
embedding, one of the most fundamental constructions in category theory.

Rather than defining coends directly, we begin by defining a more general
structure:

\begin{definition}\label{def:cowedge}
  Let $F:\catop{C}\times\cat{C}\to\cat{D}$ be a functor. A \emph{cowedge of $F$} consists of
  \begin{enumerate}
    \item an object $W\in\cat{D}$;
    \item for all $X\in\cat{C}$, a morphism $w_X:F\rr{X,X}\to W$
  \end{enumerate}
  such that, for all $f:X\to Y$ in $\cat{C}$,
  \begin{equation}\label{eq:cowedge_property}
    % https://q.uiver.app/?q=WzAsNCxbMiwxLCJXIl0sWzAsMSwiRihYLFgpIl0sWzIsMCwiRihZLFkpIl0sWzAsMCwiRihZLFgpIl0sWzEsMCwiZV9YIiwyXSxbMiwwLCJlX1kiXSxbMywyLCJGKFksZikiXSxbMywxLCJGKGYsWCkiLDJdXQ==
    \begin{tikzcd}
      {F(Y,X)} && {F(Y,Y)} \\
      {F(X,X)} && W
      \arrow["{w_X}"', from=2-1, to=2-3]
      \arrow["{w_Y}", from=1-3, to=2-3]
      \arrow["{F(Y,f)}", from=1-1, to=1-3]
      \arrow["{F(f,X)}"', from=1-1, to=2-1]
    \end{tikzcd}
  \end{equation}
  Let $(W,w_X)$ and $(W',w'_X)$ be cowedges of $F$. A morphism $h:W\to W'$ is a morphism of
  cowedges $(W,w_X)\to(W',w'_X)$ if, for all $X\in\cat{C}$, $hw_X=w'_X$.
\end{definition}

Examples of cowedges that are both interesting and specific are hard to come by.
What the following lacks in specificity it makes up for in importance.

\begin{example}\label{ex:cowedge}
  Let $P:\catop{C}\to\Set$ be a functor and $W\in\cat{C}$ an object. Consider the functor
  $\catop{C}\times\cat{C}\to\Set$ given by
  \begin{align*}
    Y,X \mapsto PY \times \Hom(W,X).
  \end{align*}
  and, for each $X\in\cat{C}$, the function
  \begin{align*}
    w_X : PX \times \Hom(W,X) \to PW
  \end{align*}
  given by $x,f\mapsto (Pf)(x)$. Now fix $f:W\to X$, $g:X\to Y$, and $y\in PY$.
  Then
  \begin{align*}
    w(y,gf) = P(gf)(y) = PfPg(y) = w(Pg(y), f).
  \end{align*}
  I.e. the diagram
  \begin{equation*}
    % https://q.uiver.app/?q=WzAsNCxbMCwwLCJQWVxcdGltZXNcXHRleHR7SG9tfShXLFgpIl0sWzAsMSwiUFlcXHRpbWVzXFx0ZXh0e0hvbX0oVyxZKSJdLFszLDAsIlBYXFx0aW1lc1xcdGV4dHtIb219KFcsWCkiXSxbMywxLCJQVyJdLFswLDIsIlBnXFx0aW1lc1xcdGV4dHtIb219KFcsWCkiXSxbMCwxLCJQWVxcdGltZXNcXHRleHR7SG9tfShXLGcpIiwyXSxbMSwzLCJ3X1kiLDJdLFsyLDMsIndfWiJdXQ==
    \begin{tikzcd}
      {PY\times\text{Hom}(W,X)} &&& {PX\times\text{Hom}(W,X)} \\
      {PY\times\text{Hom}(W,Y)} &&& PW
      \arrow["{Pg\times\text{Hom}(W,X)}", from=1-1, to=1-4]
      \arrow["{PY\times\text{Hom}(W,g)}"', from=1-1, to=2-1]
      \arrow["{w_Y}"', from=2-1, to=2-4]
      \arrow["{w_Z}", from=1-4, to=2-4]
    \end{tikzcd}
  \end{equation*}
  commutes. Thus $(PW, w_X)$ is a cowedge.
\end{example}

We notice that, for any cowedge $(W,w_X)$, postcomposition with a morphism
$f:W\to W'$ yields another cowedge $(W',fw_X)$, making $f$ into a cowedge
homomorphism. Now a coend is an initial cowedge:

\begin{definition}\label{def:coend}
  Let $F:\catop{C}\times\cat{C}\to\cat{D}$ be a functor. A \emph{coend of $F$}
  is a cowedge $(E,e_X)$ of $F$ such that, for all cowedges $(W,w_X)$ of $F$,
  there exists a unique cowedge homomorphism $(E,e_X)\to(W,w_X)$.
\end{definition}

Coends are special colimits (see~\cite[Remark 1.2.3]{loregian2015}). So, while
there is no guarantee that a particular functor has a coend, any two coends of
the same functor are canonically isomorphic. This allows us to fix a choice of
coends without loss of generality, just as we discussed for products.

Observe a useful property of coends: Consider functors
$F,G:\catop{C}\times\cat{C}\to\cat{D}$, a natural transformation
$\phi:F\Rightarrow G$ between them, and assume that we have chosen coends of $F$
and $G$. By precomposing the coend diagram of $G$ with $\phi$, we obtain a
cowedge of $F$. Thus, for each choice of coends, there exists a unique cowedge
homomorphism induced by $\phi$.

Thus a choice of coends assigns objects to functors $\catop{C}\times\cat{C}\to\cat{D}$
and morphisms to natural transformations between such functors. This
observations suggests that one might want to think of coends as a functor
$\bb{\catop{C}\times\cat{C},\cat{D}}\to\cat{D}$. There is one problem, however:
just like measure theoretic integrals, statements about coends rely on their
existence in the first place. Fortunately, all coends in this thesis will be of
functors of the form $\scatop{C}\times\scat{C}\to\Set$ where $\scat C$ is
small. As $\Set$ is cocomplete, all such coends exist and are indeed
functorial. It is possible to construct coends in \Set{} explicitly by taking
disjoint unions and quotienting them by a particular equivalence relation.
Unfortunately, the resulting construction is far from being intuitive so we
will not investigate it further.

\begin{notation}
  The notation that arises from the formal definition of coends does not scale
  well to more complicated calculations. We improve it as follows:
  \begin{itemize}
    \item For a functor $F:\catop{C}\times\cat{C}\to\cat{D}$, we will denote the
      object and morphism of a chosen coend by
      \begin{align}\label{eq:coend_integral_notation}
        \int^{C} F(C,C) \hs\text{and}\hs q_X:F(X,X)\to\int^C F(C,C).
      \end{align}
      Here the variable $C$ is bound by the integral. That is, the choice of
      symbol is arbitrary.
    \item For functors $F,G:\catop{C}\times\cat{C}\to\cat{D}$, a natural
      transformation $\phi: F\Rightarrow G$, and choices of coends, we denote
      the morphism of cowedges induced by $\phi$ by
      \begin{equation}\label{eq:coend_natural_transformation}
        \int^C \phi_{C,C} : \int^C F(C,C) \to \int^C G(C,C).
      \end{equation}
  \end{itemize}
\end{notation}

We already saw an example of a coend in \ref{ex:cowedge}. We establish the
universality by constructing a cowedge homomorphism to the chosen coend.

\begin{example}\label{ex:coend}
  The cowedge in \ref{ex:cowedge} is a coend. To see this fix a choice of coend and
  consider the morphism $h$ given by the composite
  \begin{equation}
    % https://q.uiver.app/?q=WzAsMyxbMCwwLCJQVyJdLFsyLDEsIlxcaW50XntDfSBQQ1xcdGltZXNcXHRleHR7SG9tfShXLEMpIl0sWzIsMCwiUFdcXHRpbWVzXFx0ZXh0e0hvbX0oVyxXKSJdLFswLDIsIlxcbGFuZ2xlIFBXLFxcRGVsdGFcXHRleHR7aWR9XFxyYW5nbGUiXSxbMiwxLCJxIl0sWzAsMSwiaCIsMl1d
    \begin{tikzcd}
      PW && {PW\times\text{Hom}(W,W)} \\
         && {\int^{C} PC\times\text{Hom}(W,C)}
         \arrow["{\langle PW,\Delta\text{id}\rangle}", from=1-1, to=1-3]
         \arrow["q", from=1-3, to=2-3]
         \arrow["h"', from=1-1, to=2-3]
    \end{tikzcd}
  \end{equation}
  Here $\Delta\id$ is the constant function $x\mapsto \id$.
  Now let $x\in PX$ and $f:W\to X$. Then
  \begin{align*}
    (h w)(x,f) = h(Pf(x))a = q(Pf(x),\id) = q(x,f).
  \end{align*}
  where the last equality holds due to~\ref{eq:cowedge_property}.
  Thus the diagram
  \begin{equation*}
    % https://q.uiver.app/?q=WzAsMyxbMSwwLCJQVyJdLFsxLDEsIlxcaW50XkMgUENcXHRpbWVzXFx0ZXh0e0hvbX0oVyxDKSJdLFswLDAsIlBYXFx0aW1lc1xcdGV4dHtIb219KFcsWCkiXSxbMCwxLCJoIl0sWzIsMCwidyJdLFsyLDEsInEiLDJdXQ==
    \begin{tikzcd}
      {PX\times\text{Hom}(W,X)} & PW \\
                                & {\int^C PC\times\text{Hom}(W,C)}
                                \arrow["h", from=1-2, to=2-2]
                                \arrow["w", from=1-1, to=1-2]
                                \arrow["q"', from=1-1, to=2-2]
    \end{tikzcd}
  \end{equation*}
  commutes, showing that $h$ is a cowedge homomorphism. By universality of the coend we conclude
  that we have an isomorphism
  \begin{align*}
    h:PW\cong \int^C PC\times\Hom(W,C).
  \end{align*}
  This is sometimes referred to as the co-Yoneda lemma.
\end{example}

\section{Properties of coends}

We are now in a position to make precise the functoriality of the coend:

\begin{lemma}\label{lemma:functoriality_of_coends}
  Let $\scat C$ be a small category.
  Every choice of coends for all functors $\scatop{C}\times\scat{C}\to\Set$ induces
  a functor $\bb{\scatop{C}\times\scat{C},\Set}\to\Set$ given by
  \begin{align*}
    F \mapsto \int^C F(C,C), \hs \phi \mapsto \int^C \phi_{C,C}.
  \end{align*}
  \begin{proof}
    Fix $F,G,H:\scatop{C}\times\scat{C}\to\Set$. Consider natural
    transformations $\phi:F\Rightarrow G$ and $\psi:G\Rightarrow H$. Now
    \begin{align*}
      \rr{\int^C \psi_{C,C}}\rr{\int^C \phi_{C,C}} = \int^C (\psi\phi)_{C,C}
    \end{align*}
    because both sides are cowedge homomorphisms
    \begin{align*}
      \int^C F(C,C) \to \int^C H(C,C).
    \end{align*}
    By universality of the coend on the left, this is unique.

    Preservation of identities follows by a similar argument.
  \end{proof}
\end{lemma}

While it is not at all obvious in what sense coends describe integration, we observe
that they tend to behave like integrals in the analytic sense. For example, we have the
following result about scalar multiplication:

\begin{lemma}\label{lemma:scalar_multiplication_of_coends}
  Let $F:\catop{C}\times\cat{C}\to\cat{D}$ be a functor where $\cat{D}$ is cartesian,
  and let $D\in\cat{D}$. For any choice of coend of $F$, there is a canonical isomorphism
  \begin{align*}
    D\times \int^{C} F(C,C) \cong \int^{C} D\times F(C,C).
  \end{align*}
  \begin{proof}
    We have a functor $D\times -:\cat{D}\to\cat{D}$ with inverse $\pi_2 \circ -$.
    In particular, $D\times -$ is cocontinuous. By~\cite[Theorem 1.2.7]{loregian2015},
    we have
    \begin{align*}
      (D\times -)\rr{\int^C F(C,C)} \cong \int^C (D\times -)F(C,C),
    \end{align*}
    as required.
  \end{proof}
\end{lemma}

Due to the canonical isomorphism of products $A\times B\cong B\times A$,
we immediately have a similar statement for scalar multiplication on the right.

The final property that relates coends to conventional integrals is
the Fubini rule:

\begin{theorem}[{\cite[Theorem 1.3.1]{loregian2015}}]\label{thm:fubini}
  Let $F:\catop{B}\times\cat{B}\times\catop{C}\times\cat{C}\to\cat{D}$ be a functor.
  Then there are canonical isomorphisms
  \begin{align*}
    \int^{B}\int^{C} F(B,B,C,C)
    \cong \int^{B,C} F(B,B,C,C)
    \cong \int^{C}\int^{B} F(B,B,C,C).
  \end{align*}
  That is, if one of the coends above exists, so do the other and there are
  unique isomorphisms of cowedges between them.
\end{theorem}



\chapter{2-categories}\label{sec:2categories}

To account for the possibility of axioms holding only up to a specified
isomorphism, we need to consider 2-categories. We avoid confusion by referring
to  the usual categories as 1-categories. Since 2-categorical notions tend to
be straightforward generalisations of their 1-categorical counterparts, many of
the developments in this chapter are going to seem unremarkable. The difficulty
lies in choosing which axioms we allow to hold only up to a coherent
isomorphism. This type of problem rarely arises in 1-category theory but is a
central to the study of 2-categories.

\begin{notation}
  It will be convenient to fix some notational conventions for the remainder of this report.
  Unless otherwise indicated,
  \begin{itemize}
    \item curly upper-case letters denote 1-categories, e.g. $\cat{C}$;
    \item blackboard bold upper-case letters denote small 1-categories, e.g. $\scat C$;
    \item plain upper-case letters denote objects, e.g. $X$.
  \end{itemize}
\end{notation}


\section{Definition}

The idea of a 2-category is straightforward. In mathematics, it has proven useful
to replace equality by different notions of isomorphisms. Commutative diagrams in
1-categories are nothing but equalities between morphisms, so we would like to study
isomorphisms between morphisms instead. Thus 2-categories have morphisms between morphisms.
To disambiguate we call morphisms between objects 1-cells and morphisms between morphisms
2-cells.

The concept of morphisms between morphisms should be familiar already. The 1-category
\CAT{} has functors as morphisms and we know that natural transformations are just
morphisms between functors. We will make use of this and extend \CAT{} to a 2-category
\biCAT{}.

While the additional structure brings with it a lot of freedom, it comes with a price.
Firstly, 2-categorical concepts contain more data that needs to be specified and
more axioms that need to be verified. This makes 2-category theory more complex.
Secondly, due to this additional complexity, it is not always desirable to state axioms up to
isomorphism. Thus, one has to make an informed choice about the degree of strictness
required at each step along the way.

Even when defining a 2-category itself, strictness plays a role: it is possible to
demand that the axioms hold only up to isomorphism. This consideration leads to weak
2-categories or bicategories. (see~\cite{leinster1998}) We note that the weakness of the former
has nothing to do with the strength of a monad as in~\ref{sec:commutative_monads}.
We do not require the full generality of bicategories and are thus going
to restrict our attention to (strict) 2-categories:

\begin{definition}\label{def:2category}
  A \emph{2-category} $\bicat{C}$ consists of \begin{enumerate}
    \item a class of objects $\Obj_{\bicat{C}}$;
    \item for all objects $X,Y$, a 1-category $\Hom_{\bicat C}\bb{X,Y}$ with
      composition $\bullet$ whose objects $f:X\to Y$ are called 1-cells and
      whose morphisms $\bicell u:f\Rightarrow g$ are called 2-cells;
    \item for all $X\in\bicat{C}$, an identiy 1-cell $\text{id}_X:X\to X$;
    \item for all $X,Y,Z\in\bicat{C}$, a composition functor
      \begin{equation}
        \label{eq:bicategory_composition}
        \circ_{X,Y,Z}:\Hom_{\bicat{C}}\bb{Y,Z}\times\Hom_{\bicat{C}}\bb{X,Y}
        \to\Hom_{\bicat{C}}\bb{X,Z};
      \end{equation}
  \end{enumerate}
  such that the following hold:
  \begin{enumerate}
    \item for all composable 1-cells $f,g,h$,
      $h\circ\rr{g\circ f}=\rr{h\circ g}\circ f$;
    \item for all $f:X\to Y$, $\id_Y\circ f = f = f\circ \id_X$.
  \end{enumerate}
\end{definition}

We note that there are two ways to compose 2-cells:
\begin{enumerate}
  \item For 1-cells $f,g,h:X\to Y$ and 2-cells $\bicell u: f\Rightarrow g$,
    $\bicell v: g\Rightarrow h$ we have the vertical composite
    $\bicell v\bullet\bicell u: f\Rightarrow h$ given by the composition in $\Hom\bb{X,Y}$. In a diagram:
    \begin{equation*}
      % https://q.uiver.app/?q=WzAsMixbMCwwLCJYIl0sWzIsMCwiWSJdLFswLDEsImYiLDAseyJjdXJ2ZSI6LTV9XSxbMCwxLCJoIiwyLHsiY3VydmUiOjV9XSxbMCwxLCJnIiwxXSxbMiw0LCJcXG1hdGhiZiB1IiwxLHsic2hvcnRlbiI6eyJzb3VyY2UiOjIwLCJ0YXJnZXQiOjIwfX1dLFs0LDMsIlxcbWF0aGJmIHYiLDEseyJzaG9ydGVuIjp7InNvdXJjZSI6MjAsInRhcmdldCI6MjB9fV1d
      \begin{tikzcd}
        X && Y
        \arrow[""{name=0, anchor=center, inner sep=0}, "f", curve={height=-40pt}, from=1-1, to=1-3]
        \arrow[""{name=1, anchor=center, inner sep=0}, "h"', curve={height=40pt}, from=1-1, to=1-3]
        \arrow[""{name=2, anchor=center, inner sep=0}, "g"{description}, from=1-1, to=1-3]
        \arrow["{\mathbf u}"{description}, shorten <=4pt, shorten >=4pt, Rightarrow, from=0, to=2]
        \arrow["{\mathbf v}"{description}, shorten <=4pt, shorten >=4pt, Rightarrow, from=2, to=1]
      \end{tikzcd}
    \end{equation*}
  \item For 1-cells $f,g:X\to Y$, $h,k:Y\to Z$ and 2-cells $\bicell
    u:f\Rightarrow g$, $\bicell v:h\Rightarrow k$ we have the horizontal
    composite $\bicell v\circ\bicell u:h\circ f\Rightarrow k\circ g$ given by
    the composition functor $\circ_{X,Y,Z}$. In a diagram:
    \begin{equation*}
      % https://q.uiver.app/?q=WzAsMyxbMCwwLCJYIl0sWzIsMCwiWSJdLFs0LDAsIloiXSxbMCwxLCJmIiwwLHsiY3VydmUiOi00fV0sWzAsMSwiZyIsMix7ImN1cnZlIjo0fV0sWzEsMiwiaCIsMCx7ImN1cnZlIjotNH1dLFsxLDIsImsiLDIseyJjdXJ2ZSI6NH1dLFszLDQsIlxcbWF0aGJmIHUiLDEseyJzaG9ydGVuIjp7InNvdXJjZSI6MjAsInRhcmdldCI6MjB9fV0sWzUsNiwiXFxtYXRoYmYgdiIsMSx7InNob3J0ZW4iOnsic291cmNlIjoyMCwidGFyZ2V0IjoyMH19XV0=
      \begin{tikzcd}
        X && Y && Z
        \arrow[""{name=0, anchor=center, inner sep=0}, "f", curve={height=-24pt}, from=1-1, to=1-3]
        \arrow[""{name=1, anchor=center, inner sep=0}, "g"', curve={height=24pt}, from=1-1, to=1-3]
        \arrow[""{name=2, anchor=center, inner sep=0}, "h", curve={height=-24pt}, from=1-3, to=1-5]
        \arrow[""{name=3, anchor=center, inner sep=0}, "k"', curve={height=24pt}, from=1-3, to=1-5]
        \arrow["{\mathbf u}"{description}, shorten <=6pt, shorten >=6pt, Rightarrow, from=0, to=1]
        \arrow["{\mathbf v}"{description}, shorten <=6pt, shorten >=6pt, Rightarrow, from=2, to=3]
      \end{tikzcd}
    \end{equation*}
\end{enumerate}

\begin{notation}
  We adopt similar conventions as for 1-categories:
  \begin{itemize}
    \item $X\in\bicat{C}$ means $X\in\Obj_{\bicat{C}}$;
    \item We drop the 1-cell composition and therefore the horizontal
      composition of 2-cells. Thus $\bicell v\bicell u:hf\Rightarrow kg$ means
      $\bicell v\circ\bicell u:h\circ f\Rightarrow k\circ g$.
    \item To avoid unnecessary subscripts we identify objects and 1-cells with their
      respective identities. Thus we might write $h\bicell uX:hfX\Rightarrow hgX$
      to mean
      \begin{align*}
        \textbf{id}_h\circ\bicell u\circ\textbf{id}_{\id_X}:h\circ f\circ\id_X\Rightarrow h\circ g\circ\id_X.
      \end{align*}
      Note how we identify $\textbf{id}_{\id_X}$ with $\id_X$ and thus with $X$.
  \end{itemize}
\end{notation}


\begin{example}
  The most important 2-category that we will consider is $\biCAT$ which has as
  objects all categories, as 1-cells all functors, and as 2-cells all natural transformations.
  That is, for all $\cat{C},\cat{D}\in\biCAT$, $\Hom\bb{\cat{C},\cat{D}}$ is the functor
  category $\bb{\cat{C},\cat{D}}$. Composition of 1-cells is the usual composition of functors.
\end{example}

Given two 2-categories, we obtain the product 2-category which is entirely analogous to the
product 1-category:

\begin{definition}\label{def:product_2category}
  Let $\bicat{C},\bicat{D}$ be 2-categories. The \emph{product 2-category} $\bicat{C}\times\bicat{D}$ has
  \begin{enumerate}
    \item as objects all pairs $(X,X')$ with $X\in\bicat{C}$ and $X'\in\bicat{D}$;
    \item for all objects $(X,X')$ and $(Y,Y')$, the hom-category
      \begin{align*}
        \Hom\bb{(X,Y),(X',Y')} = \Hom\bb{X,Y}\times\Hom\bb{X',Y'};
      \end{align*}
    \item for all objects $(X,X')$, the identity 1-cell $(\text{id}_X,\text{id}_{X'})$;
    \item the composition functor given by pointwise composition in the respective
      2-categories, i.e. $(g,g')(f,f')=(gf,g'f')$.
  \end{enumerate}
\end{definition}

\section{Cartesian structure}\label{sec:bicartesian_2categories}

Kock's commutative monads are strong with respect to the cartesian structure
of a 1-category. In order to formalise commutative relative pseudomonads, we require
a similar structure on 2-categories. We are fortunate in that our main 2-category
of interest, \biCAT, admits strict products. That is, the required 1-cell equations
hold up to equality and not merely up to isomorphism.

While it is straightforward to define larger product structures
(e.g.~\cite{saville2020}), the notion of strength of a monad only
requires binary products and a terminal object which serves as the
identity.

The standard definition of a 1-categorical product that we outlined in
\ref{sec:cartesian_categories} requires an alternative formulation
before we may generalise it elegantly. We invite the reader to verify
that adapting the following definition to 1-categories in the obvious
way does indeed yields a product in the usual sense.

\begin{definition}\label{def:bicategories_products}
  A \emph{cartesian structure for a 2-category $\bicat{C}$} consists of
  \begin{enumerate}
    \item an object $1\in\bicat{C}$;
    \item for all $W\in\bicat{C}$, an isomorphism of categories
      $\Hom\bb{W,1}\cong\mathbb 1$;
    \item for all $X,Y\in\bicat{C}$:
      \begin{enumerate}
        \item an object $X\times Y\in\bicat{C}$;
        \item 1-cells $\pi_1:X\times Y\to X$ and $\pi_2:X\times Y\to Y$ called projections;
        \item for all $W\in\bicat{C}$, an isomorphism of categories
          \begin{equation}\label{eq:cartesian_isomorphism}
            % https://q.uiver.app/?q=WzAsMixbMCwwLCJcXHRleHR7SG9tfVtXLFhcXHRpbWVzIFldIl0sWzIsMCwiXFx0ZXh0e0hvbX1bVyxYXVxcdGltZXNcXHRleHR7SG9tfVtXLFldIl0sWzAsMSwiKFxccGlfMVxcY2lyYy0sXFxwaV8yXFxjaXJjIC0pIiwwLHsiY3VydmUiOi0zfV0sWzEsMCwiXFxsYW5nbGUtXFxyYW5nbGUiLDAseyJjdXJ2ZSI6LTN9XSxbMiwzLCJcXGNvbmciLDEseyJzaG9ydGVuIjp7InNvdXJjZSI6MjAsInRhcmdldCI6MjB9LCJzdHlsZSI6eyJib2R5Ijp7Im5hbWUiOiJub25lIn0sImhlYWQiOnsibmFtZSI6Im5vbmUifX19XV0=
            \begin{tikzcd}
              {\text{Hom}[W,X\times Y]} && {\text{Hom}[W,X]\times\text{Hom}[W,Y]}
              \arrow[""{name=0, anchor=center, inner sep=0}, "{(\pi_1\circ-,\pi_2\circ -)}", curve={height=-18pt}, from=1-1, to=1-3]
              \arrow[""{name=1, anchor=center, inner sep=0}, "{\langle-\rangle}", curve={height=-18pt}, from=1-3, to=1-1]
              \arrow["\cong"{description}, draw=none, from=0, to=1]
            \end{tikzcd}
          \end{equation}
          The functor $\aa{-}$ is called tupling.
      \end{enumerate}
  \end{enumerate}
\end{definition}

\begin{example}
  Let $\cat{C},\cat{D}$ be 1-categories. It is known that the product of categories
  $\cat{C}\times\cat{D}$ is a product in $\CAT$. Let $\pi_1,\pi_2$ denote the corresponding
  projections in $\CAT$. These are 1-cells in $\biCAT$. Define the tupling functor by
  \begin{align*}
    \aa{F,G} &=  \rr{F-,G-} \\
    \aa{\phi,\psi}_X &= \rr{\phi_X,\psi_X}
  \end{align*}
  By using the universal properties of the 1-categorical product it is straightforward to
  verify that this is indeed an inverse to $\rr{\pi_1\circ-,\pi_2,\circ -}$. Thus the
  1-categorical cartesian structure of $\CAT$ extends to a 2-categorical cartesian
  structure of $\biCAT$.
\end{example}

\section{Pseudofunctors}\label{sec:inclusions}

Category theory is in many ways about studying morphisms rather than objects.
We are therefore interested in defining the notion of a morphism between 2-categories which we
call pseudofunctors.
Just as a functor specifies where objects and morphisms are mapped to,
a pseudofunctor is a map on objects, 1-cells, and 2-cells.

There is an important difference, however: pseudofunctors are not the
most strict morphism between 2-categories. In particular, we allow for
the possibility that the functor axioms hold only up to a canonical
isomorphism. This is achieved by specifying these isomorphisms as part
of the structure: preservation of identities is witnessed by $\bicell i$
and distributivity is witnessed by $\bicell d$.

Now that the functoriality axioms are part of the structure, the axioms of a
pseudofunctor serve a different purpose entirely. While the structural 2-cells
ensure that certain 1-cells are isomorphic, we want to avoid the possibility of
deriving multiple different isomorphisms between the same 1-cells from this
structure. This is referred to as coherence. While proving coherence directly
is hard in general, many structures come with a sufficient set of conditions
that lead to coherence.

\begin{definition}\label{def:pseudofunctor}
  Let $\bicat{C},\bicat{D}$ be 2-categories. Then a \emph{pseudofunctor $F:\bicat{C}\to\bicat{D}$} consists of
  \begin{enumerate}
    \item for all $X\in\bicat{C}$, an object $FX\in\bicat{D}$;
    \item for all $X,Y\in{\bicat{C}}$, a functor
      $F_{X,Y}:\Hom\bb{X,Y}\to\Hom\bb{FX,FY}$;
    \item for all $X\in\bicat{C}$, an invertible 2-cell
      \begin{equation}\label{eq:pseudofunctor_identity}
        % https://q.uiver.app/?q=WzAsMixbMCwwLCJGWCJdLFswLDEsIkZYIl0sWzAsMSwiIiwwLHsiY3VydmUiOi01LCJsZXZlbCI6Miwic3R5bGUiOnsiaGVhZCI6eyJuYW1lIjoibm9uZSJ9fX1dLFswLDEsIkZfe1gsWH0oXFx0ZXh0e2lkfV9YKSIsMix7ImN1cnZlIjo1fV0sWzMsMiwiXFxiaWNlbGwgaV9YIiwwLHsic2hvcnRlbiI6eyJzb3VyY2UiOjIwLCJ0YXJnZXQiOjIwfX1dXQ==
        \begin{tikzcd}
          FX \\
          FX
          \arrow[""{name=0, anchor=center, inner sep=0}, curve={height=-30pt}, Rightarrow, no head, from=1-1, to=2-1]
          \arrow[""{name=1, anchor=center, inner sep=0}, "{F_{X,X}(\text{id}_X)}"', curve={height=30pt}, from=1-1, to=2-1]
          \arrow["{\bicell i_X}", shorten <=12pt, shorten >=12pt, Rightarrow, from=1, to=0]
        \end{tikzcd}
      \end{equation}
    \item for all $f:X\to Y$ and $g:Y\to Z$ in $\bicat{C}$, an invertible 2-cell
      \begin{equation}\label{eq:pseudofunctor_distributivity}
        % https://q.uiver.app/?q=WzAsMyxbMCwwLCJGWCJdLFsyLDAsIkZZIl0sWzEsMSwiRloiXSxbMCwxLCJGZiJdLFsxLDIsIkZnIl0sWzAsMiwiRihnZikiLDJdLFs1LDQsIlxcbWF0aGJmIGRfe2YsZ30iLDAseyJzaG9ydGVuIjp7InNvdXJjZSI6MjAsInRhcmdldCI6MjB9fV1d
        \begin{tikzcd}
          FX && FY \\
             & FZ
             \arrow["Ff", from=1-1, to=1-3]
             \arrow[""{name=0, anchor=center, inner sep=0}, "Fg", from=1-3, to=2-2]
             \arrow[""{name=1, anchor=center, inner sep=0}, "{F(gf)}"', from=1-1, to=2-2]
             \arrow["{\mathbf d_{f,g}}", shorten <=6pt, shorten >=6pt, Rightarrow, from=1, to=0]
        \end{tikzcd}
      \end{equation}
  \end{enumerate}
  such that
  \begin{enumerate}
    \item for all composable 1-cells $f,g,h$,
      \begin{equation}\label{eq:pseudofunctor_coherence_associativity}
        % https://q.uiver.app/?q=WzAsNCxbMCwwLCJGKGhnZikiXSxbMCwxLCJGKGhnKSBGZiJdLFsxLDAsIkZoRihnZikiXSxbMSwxLCJGaEZnRmYiXSxbMCwxLCJcXG1hdGhiZiBkIiwyXSxbMCwyLCJcXG1hdGhiZiBkIl0sWzIsMywiRmhcXG1hdGhiZiBkIl0sWzEsMywiXFxtYXRoYmYgZEZmIiwyXV0=
        \begin{tikzcd}
          {F(hgf)} & {FhF(gf)} \\
          {F(hg) Ff} & FhFgFf
          \arrow["{\mathbf d}"', from=1-1, to=2-1]
          \arrow["{\mathbf d}", from=1-1, to=1-2]
          \arrow["{Fh\mathbf d}", from=1-2, to=2-2]
          \arrow["{\mathbf dFf}"', from=2-1, to=2-2]
        \end{tikzcd}
      \end{equation}
    \item for all 1-cells $f$,
      \begin{equation}\label{eq:pseudofunctor_coherence_identity}
        % https://q.uiver.app/?q=WzAsNCxbMCwwLCJGZkYoXFx0ZXh0e2lkfSkiXSxbMCwxLCJGKGZcXHRleHR7aWR9KSJdLFsyLDAsIkZmXFx0ZXh0e2lkfSJdLFsyLDEsIkZmIl0sWzAsMSwiXFxtYXRoYmYgZF57LTF9IiwyXSxbMCwyLCJGZlxcbWF0aGJmIGkiXSxbMiwzLCIiLDIseyJsZXZlbCI6Miwic3R5bGUiOnsiaGVhZCI6eyJuYW1lIjoibm9uZSJ9fX1dLFsxLDMsIiIsMCx7ImxldmVsIjoyLCJzdHlsZSI6eyJoZWFkIjp7Im5hbWUiOiJub25lIn19fV1d
        \begin{tikzcd}
          {FfF(\text{id})} && {Ff\text{id}} \\
          {F(f\text{id})} && Ff
          \arrow["{\mathbf d^{-1}}"', from=1-1, to=2-1]
          \arrow["{Ff\mathbf i}", from=1-1, to=1-3]
          \arrow[Rightarrow, no head, from=1-3, to=2-3]
          \arrow[Rightarrow, no head, from=2-1, to=2-3]
        \end{tikzcd}\hs
        % https://q.uiver.app/?q=WzAsNCxbMCwwLCJGKFxcdGV4dHtpZH0pRmYiXSxbMCwxLCJGKFxcdGV4dHtpZH1mKSJdLFsyLDAsIlxcdGV4dHtpZH1GZiJdLFsyLDEsIkZmIl0sWzAsMSwiXFxtYXRoYmYgZF57LTF9IiwyXSxbMCwyLCJcXG1hdGhiZiBpIEZmIl0sWzIsMywiIiwyLHsibGV2ZWwiOjIsInN0eWxlIjp7ImhlYWQiOnsibmFtZSI6Im5vbmUifX19XSxbMSwzLCIiLDAseyJsZXZlbCI6Miwic3R5bGUiOnsiaGVhZCI6eyJuYW1lIjoibm9uZSJ9fX1dXQ==
        \begin{tikzcd}
          {F(\text{id})Ff} && {\text{id}Ff} \\
          {F(\text{id}f)} && Ff
          \arrow["{\mathbf d^{-1}}"', from=1-1, to=2-1]
          \arrow["{\mathbf i Ff}", from=1-1, to=1-3]
          \arrow[Rightarrow, no head, from=1-3, to=2-3]
          \arrow[Rightarrow, no head, from=2-1, to=2-3]
        \end{tikzcd}
      \end{equation}
  \end{enumerate}
  A 2-functor is a pseudofunctor whose 2-cells $\bicell i$ and $\bicell d$ are identities.
\end{definition}

As we will be dealing with at most one pseudofunctor with non-identity 2-cells, our
notation is unambiguous. One 2-functor that will be of interest is the product 2-functor
that allows us to take products not just of objects but also of 1-cells and 2-cells:

\begin{example}\label{ex:product_2functor}
  Analogously to the 1-categorical case (see \ref{sec:cartesian_categories}),
  we observe that taking binary products in a cartesian 2-category yields a
  2-functor $\bicat{C}\times\bicat{C}\to\bicat{C}$, given by the following
  structure:
  \begin{enumerate}
    \item for all $(X,Y)\in\bicat{C}\times\bicat{C}$, we have the object $X\times Y\in\bicat{C}$;
    \item for all $X,X',Y,Y'\in\bicat{C}$, the functor on hom-categories is given
      by the composite
      \begin{equation}
        % https://q.uiver.app/?q=WzAsMyxbMCwwLCJcXHRleHR7SG9tfVtYLFldXFx0aW1lc1xcdGV4dHtIb219W1gnLFknXSJdLFsyLDEsIlxcdGV4dHtIb219W1hcXHRpbWVzIFgnLFlcXHRpbWVzIFknXSJdLFsyLDAsIlxcdGV4dHtIb219W1hcXHRpbWVzIFgnLFldXFx0aW1lc1xcdGV4dHtIb219W1hcXHRpbWVzIFgnLFknXSJdLFsyLDEsIlxcbGFuZ2xlLV9rXFxyYW5nbGUiXSxbMCwyLCIoLVxcY2lyY1xccGlfMSwtXFxjaXJjXFxwaV8yKSJdLFswLDEsIihcXHRpbWVzKV97KFgsWCcpLChZLFknKX0iLDJdXQ==
        \begin{tikzcd}
          {\text{Hom}[X,Y]\times\text{Hom}[X',Y']} && {\text{Hom}[X\times X',Y]\times\text{Hom}[X\times X',Y']} \\
                                                   && {\text{Hom}[X\times X',Y\times Y']}
                                                   \arrow["{\langle-_k\rangle}", from=1-3, to=2-3]
                                                   \arrow["{(-\circ\pi_1,-\circ\pi_2)}", from=1-1, to=1-3]
                                                   \arrow["{(\times)_{(X,X'),(Y,Y')}}"', from=1-1, to=2-3]
        \end{tikzcd}
      \end{equation}
  \end{enumerate}
  We need to verify that \ref{eq:pseudofunctor_identity} and \ref{eq:pseudofunctor_distributivity}
  are identities. We verify, for all $X,X'\in\bicat{C}$,
  \begin{align*}
    \text{id}_X\times \text{id}_{X'} = \aa{\pi_1,\pi_2} = \text{id}_{X\times X'}
  \end{align*}
  where the last equality follows from the isomorphism
  (\ref{eq:cartesian_isomorphism}). Similarly, consider
  $(f,f'):(X,X')\to(Y,Y')$ and $(g,g'):(Y,Y')\to(Z,Z')$ in
  $\bicat{C}\times\bicat{C}$. In the diagram
  \begin{equation}
    % https://q.uiver.app/?q=WzAsOSxbMiwwLCJYXFx0aW1lcyBYJyJdLFsyLDEsIllcXHRpbWVzIFknIl0sWzIsMiwiWlxcdGltZXMgWiciXSxbNCwyLCJaJyJdLFs0LDEsIlknIl0sWzQsMCwiWCciXSxbMCwwLCJYIl0sWzAsMSwiWSJdLFswLDIsIloiXSxbNyw4LCJnIiwyXSxbNiw3LCJmIiwyXSxbNSw0LCJmJyJdLFs0LDMsImcnIl0sWzEsMiwiXFxsYW5nbGUgZ1xccGlfMSxnJ1xccGlfMlxccmFuZ2xlIiwyXSxbMCwxLCJcXGxhbmdsZSBmXFxwaV8xLGYnXFxwaV8yXFxyYW5nbGUiLDJdLFswLDYsIlxccGlfMSIsMl0sWzAsNSwiXFxwaV8yIl0sWzEsNywiXFxwaV8xIiwyXSxbMSw0LCJcXHBpXzIiXSxbMiw4LCJcXHBpXzEiLDJdLFsyLDMsIlxccGlfMiJdXQ==
    \begin{tikzcd}
      X && {X\times X'} && {X'} \\
      Y && {Y\times Y'} && {Y'} \\
      Z && {Z\times Z'} && {Z'}
      \arrow["g"', from=2-1, to=3-1]
      \arrow["f"', from=1-1, to=2-1]
      \arrow["{f'}", from=1-5, to=2-5]
      \arrow["{g'}", from=2-5, to=3-5]
      \arrow["{\langle g\pi_1,g'\pi_2\rangle}"', from=2-3, to=3-3]
      \arrow["{\langle f\pi_1,f'\pi_2\rangle}"', from=1-3, to=2-3]
      \arrow["{\pi_1}"', from=1-3, to=1-1]
      \arrow["{\pi_2}", from=1-3, to=1-5]
      \arrow["{\pi_1}"', from=2-3, to=2-1]
      \arrow["{\pi_2}", from=2-3, to=2-5]
      \arrow["{\pi_1}"', from=3-3, to=3-1]
      \arrow["{\pi_2}", from=3-3, to=3-5]
    \end{tikzcd}
  \end{equation}
  each tile commutes by (\ref{eq:cartesian_isomorphism}). Now notice $gf\times g'f' =
  \aa{gf\pi_1,g'f'\pi_2}$. The claim then follows by the universal property of
  the product.
\end{example}

The presheaf construction gives rise to a pseudofunctor. We will show this in
the next chapter (see \ref{prop:induced_pseudofunctor}). For now, let us
briefly investigate what this structure looks like without spelling out the
details:

\begin{example}\label{ex:presheaf_pseudofunctor}
  Consider the pseudofunctor structure $\widehat{-} : \biCat\to\biCAT$:
  \begin{enumerate}
    \item for all $\scat X\in\biCat$, $\widehat{\scat X}=\bb{\scatop X,\Set}$;
    \item for all $F:\scat X\to\scat Y$ and $P\in\widehat{\scat X}$,
      \begin{align*}
        \widehat{F}P = \int^X PX\times\Hom\rr{-,FX};
      \end{align*}
    \item the natural isomorphism $\bicell i$ has as components natural isomorphisms
      \begin{align*}
        \int^X PX\times\Hom(-,X) \cong P;
      \end{align*}
    \item for all composable functors $F$ and $G$, the natural isomorphism $\bicell d_{F,G}$
      has as components isomorphisms
      \begin{align*}
        \int^X PX\times\Hom(-,(GF)X) \cong \int^Y\rr{\int^X PX\times\Hom(Y,FX)}\times\Hom(-,GY).
      \end{align*}
  \end{enumerate}
\end{example}

There are two functors corresponding to a particular relative monad: on the one
hand there is the functor that the monad is relative to and on the other we
have the functor induced by the monad~\cite{altenkirch2015}. Similarly, a
relative pseudomonad has two corresponding pseudofunctors. It is for this
reason that we require two degrees of strictness: while the inclusion
$\biCat\to\biCAT$ is a 2-functor, the pseudofunctor induced by the presheaf
construction is not. To limit complexity, we aim to make use of the strictness
of the former while allowing for the non-strictness of the latter.

\section{Inclusions of cartesian 2-categories}

Before we can endow the presheaf construction with a strength, we need to
capture some of the additional properties of the inclusion. It only makes sense
to define a strong pseudomonad relative to a 2-functor $\bicat J\to\bicat C$ if
both its domain and codomain are cartesian and the cartesian structures agree.
We could treat this in full generality by considering monoidal pseudofunctors,
similar to~\cite{tarmo}. However, the inclusion $\biCat\to\biCAT$ preserves
products and terminal objects we are able to impose and satisfy much stricter
conditions.

An inclusion functor is injective on objects and morphisms. Similarly, we demand an
inclusion pseudofunctor be injective on objects, 1-cells, and 2-cells. Moreover,
we require inclusions to preserve the cartesian structure.

\begin{definition}
  An \emph{inclusion of cartesian 2-categories} consists of
  \begin{enumerate}
    \item cartesian 2-categories $\bicat{J}$ and $\bicat{C}$;
    \item a 2-functor $J:\bicat{J}\to\bicat{C}$;
  \end{enumerate}
  such that
  \begin{enumerate}
    \item $J$ is injective on objects, 1-cells, and 2-cells;
    \item for all $X,Y\in\bicat{J}$,
      \begin{align*}
        J1=1,\hs
        J\rr{X\times Y}=JX\times JY,\hs
        J(\pi_1) = \pi_1,\hs J(\pi_2)=\pi_2.
      \end{align*}
  \end{enumerate}
\end{definition}

\begin{example}
  Define $\biCat$ as the 2-category with objects all small categories and, for all
  $\scat{C},\scat D\in\biCat$, the hom-category $\Hom\bb{\scat C,\scat D}=\bb{\scat C,\scat D}$.
  We now have the obvious inclusion $J:\biCat\to\biCAT$ that is the identity
  on objects and hom-categories. Define the cartesian structure
  on $\biCat$ in the same way that we defined it for $\biCAT$. This is justified as the
  product of small categories is itself small. Thus $J$ forms an inclusion of cartesian
  2-categories.
\end{example}

\section{Pseudonatural transformations}

Naturally, we have morphisms between pseudofunctors. The naturality of a
pseudonatural transformation is witnessed by a structural 2-cell which obeys a
coherence axiom. Once again, we have two levels of strictness. It is worth
noting, however, that these are not related to the strictness of the underlying
pseudofunctors.

\begin{definition}
  Let $F,G:\bicat{C}\to\bicat{D}$ be pseudofunctors. A \emph{pseudonatural transformation}
  $\phi:F\Rightarrow G$ consists of
  \begin{enumerate}
    \item for all $X\in\bicat C$, a 1-cell $\phi_X : FX\to GX$;
    \item for all $f:X\to Y$ in $\bicat C$, a naturality 2-cell
      \begin{equation}\label{eq:naturality_2cell}
        % https://q.uiver.app/?q=WzAsNCxbMCwwLCJGWCJdLFsyLDAsIkZZIl0sWzAsMSwiR1giXSxbMiwxLCJHWSJdLFswLDEsIkZmIiwxXSxbMCwyLCJcXHBoaSIsMl0sWzIsMywiR2YiLDFdLFsxLDMsIlxccGhpIl0sWzUsNywiXFxtYXRoYmYgbl9mIiwxLHsic2hvcnRlbiI6eyJzb3VyY2UiOjIwLCJ0YXJnZXQiOjIwfX1dXQ==
        \begin{tikzcd}
          FX && FY \\
          GX && GY
          \arrow["Ff"{description}, from=1-1, to=1-3]
          \arrow[""{name=0, anchor=center, inner sep=0}, "\phi"', from=1-1, to=2-1]
          \arrow["Gf"{description}, from=2-1, to=2-3]
          \arrow[""{name=1, anchor=center, inner sep=0}, "\phi", from=1-3, to=2-3]
          \arrow["{\mathbf n_f}"{description}, shorten <=13pt, shorten >=13pt, Rightarrow, from=0, to=1]
        \end{tikzcd}
      \end{equation}
  \end{enumerate}
  such that, for all composable 1-cells $f,g\in\bicat C$,
  \begin{equation}
    % https://q.uiver.app/?q=WzAsNSxbMCwwLCJHZ0dmXFxwaGkiXSxbMiwwLCJHZ1xccGhpIEZmIl0sWzQsMCwiXFxwaGkgRmcgRmYiXSxbNCwxLCJcXHBoaSBGKGdmKSJdLFswLDEsIkcoZ2YpXFxwaGkiXSxbMCw0LCJcXG1hdGhiZiBkXnstMX1cXHBoaSIsMl0sWzAsMSwiR2dcXG1hdGhiZiBuIl0sWzEsMiwiXFxtYXRoYmYgbiBGZiJdLFsyLDMsIlxccGhpXFxtYXRoYmYgZF57LTF9Il0sWzQsMywiXFxtYXRoYmYgbiIsMl1d
    \begin{tikzcd}
      GgGf\phi && {Gg\phi Ff} && {\phi Fg Ff} \\
      {G(gf)\phi} &&&& {\phi F(gf)}
      \arrow["{\mathbf d^{-1}\phi}"', from=1-1, to=2-1]
      \arrow["{Gg\mathbf n}", from=1-1, to=1-3]
      \arrow["{\mathbf n Ff}", from=1-3, to=1-5]
      \arrow["{\phi\mathbf d^{-1}}", from=1-5, to=2-5]
      \arrow["{\mathbf n}"', from=2-1, to=2-5]
    \end{tikzcd}
  \end{equation}
  A 2-natural transformation is a pseudonatural transformation whose naturality 2-cell
  $\bicell n$ is the identity.
\end{definition}

\begin{example}\label{ex:cartesian_monoidal_structure}
  Let $\bicat{C}$ be a cartesian 2-category. We saw that the product 2-functor is merely
  an extension of the usual product functor. Because $\bicell n$ is the identity,
  each natural transformation between product functors yields a 2-natural transformation
  between product 2-functors. Thus the usual isomorphisms $\lambda,\alpha,\gamma$ are
  in fact 2-natural transformations.
\end{example}


\chapter{Strong inclusion pseudomonads}\label{sec:strong_relative_pseudomonads}

We have now developed all the theory required to present our approach to defining
a generalisation of strong monads that admits the presheaf construction as a model.
We do so in three major steps, each of which we will justify by applying it to the
presheaf construction and comparing it to related work

Firstly, we present a structure that is similar to that of a relative
pseudomonad. We justify the differences and show how the presheaf construction
gives rise to such a structure. Secondly, we state the axioms that this
structure is required to satisfy in order to induce a relative pseudomonad.
Finally, we extend the structure in a way that induces a strong pseudomonad
structure in the case where the inclusion is the identity. Unfortunately, we
are not able to postulate axioms that are sufficient to show that this induced
structure satsifies the axioms of a strong pseudomonad.

Fix an inclusion of cartesian 2-categories $J:\bicat{J}\to\bicat{C}$.

\section{Prestrong structure}

A strong monad is a monad equipped with a suitable natural transformation.
Hence the obvious way to define a strong relative pseudomonad is by equipping a relative
pseudomonad with a suitable pseudonatural transformation.
This approach would have two notable benefits. Firstly, it would allow us to leverage already
existing results about relative pseudomonads without any overhead. Secondly, it would
provide us with an intuitive connection to strong pseudomonads that would, presumably,
make some of the axioms easier to postulate.

We investigate another approach. Rather than adding the strength as additional structure,
we choose to build it in to the extension operator. That is, we only allow
1-cells $f:W\times JX\to TY$ to be extended to $f^\dagger:W\times TX\to TY$.
This relates the extension to the cartesian structure directly. Another reason why we
believe that this idea is worth entertaining is because it becomes very pleasant to
describe commutativity of such a structure.
One can demand the existence of the following natural isomorphism:

\begin{equation*}
  % https://q.uiver.app/?q=WzAsOCxbMCwwLCJcXHRleHR7SG9tfVtKWFxcdGltZXMgSlksVFpdIl0sWzEsMCwiXFx0ZXh0e0hvbX1bSlhcXHRpbWVzIFRZLFRaXSJdLFsyLDAsIlxcdGV4dHtIb219W1RZXFx0aW1lcyBKWCxUWl0iXSxbMiwxLCJcXHRleHR7SG9tfVtUWVxcdGltZXMgVFgsVFpdIl0sWzIsMiwiXFx0ZXh0e0hvbX1bVFhcXHRpbWVzIFRZLFRaXSJdLFsxLDIsIlxcdGV4dHtIb219W1RYXFx0aW1lcyBKWSxUWl0iXSxbMCwyLCJcXHRleHR7SG9tfVtKWVxcdGltZXMgVFgsVFpdIl0sWzAsMSwiXFx0ZXh0e0hvbX1bSllcXHRpbWVzIEpYLFRaXSJdLFswLDcsIi1cXGNpcmNcXGdhbW1hIiwyXSxbNyw2LCIoLSleXFxkYWdnZXIiLDJdLFs2LDUsIi1cXGNpcmNcXGdhbW1hIiwyLHsiY3VydmUiOjJ9XSxbNSw0LCIoLSleXFxkYWdnZXIiLDIseyJjdXJ2ZSI6Mn1dLFswLDEsIigtKV5cXGRhZ2dlciIsMCx7ImN1cnZlIjotMn1dLFsxLDIsIi1cXGNpcmNcXGdhbW1hIiwwLHsiY3VydmUiOi0yfV0sWzIsMywiKC0pXlxcZGFnZ2VyIl0sWzMsNCwiLVxcY2lyY1xcZ2FtbWEiXSxbNywzLCJcXGNvbmciLDEseyJzaG9ydGVuIjp7InNvdXJjZSI6MjAsInRhcmdldCI6MjB9LCJsZXZlbCI6Miwic3R5bGUiOnsiYm9keSI6eyJuYW1lIjoibm9uZSJ9LCJoZWFkIjp7Im5hbWUiOiJub25lIn19fV1d
  \begin{tikzcd}
    {\text{Hom}[JX\times JY,TZ]} & {\text{Hom}[JX\times TY,TZ]} & {\text{Hom}[TY\times JX,TZ]} \\
    {\text{Hom}[JY\times JX,TZ]} && {\text{Hom}[TY\times TX,TZ]} \\
    {\text{Hom}[JY\times TX,TZ]} & {\text{Hom}[TX\times JY,TZ]} & {\text{Hom}[TX\times TY,TZ]}
    \arrow["{-\circ\gamma}"', from=1-1, to=2-1]
    \arrow["{(-)^\dagger}"', from=2-1, to=3-1]
    \arrow["{-\circ\gamma}"', curve={height=12pt}, from=3-1, to=3-2]
    \arrow["{(-)^\dagger}"', curve={height=12pt}, from=3-2, to=3-3]
    \arrow["{(-)^\dagger}", curve={height=-12pt}, from=1-1, to=1-2]
    \arrow["{-\circ\gamma}", curve={height=-12pt}, from=1-2, to=1-3]
    \arrow["{(-)^\dagger}", from=1-3, to=2-3]
    \arrow["{-\circ\gamma}", from=2-3, to=3-3]
    \arrow["\cong"{description}, draw=none, from=2-1, to=2-3]
  \end{tikzcd}
\end{equation*}

Given that synthetic measure theory requires a commutative monad, it is possible that
this alternative structure is more convenient in the context of a generalised synthetic
measure theory.

\begin{definition}\label{def:prestrong_inclusion_pseudomonad_structure}
  A \emph{strong $J$-pseudomonad structure} consists of
  \begin{enumerate}
    \item for all $X\in\bicat{J}$, an object $TX\in\bicat{C}$;
    \item for all $X\in\bicat{J}$, a 1-cell $\eta_X:JX\to TX$ in $\bicat{C}$;
    \item for all $X,Y\in\bicat{J}$ and $W\in\bicat{C}$, a functor
      \begin{align}\label{eq:strong_pseudomonad_extension}
        \extend{\rr{-}}{\dagger}_{X,Y,W}:\Hom\bb{W\times JX,TY}\to\Hom\bb{W\times TX,TY};
      \end{align}
    \item for all $f:W\times JX\to TY$ in $\bicat{C}$, an invertible 2-cell
      \begin{equation}\label{eq:strong_pseudomonad_r}
        % https://q.uiver.app/?q=WzAsMyxbMCwwLCJXXFx0aW1lcyBKWCJdLFsxLDEsIlRZIl0sWzIsMCwiV1xcdGltZXMgVFgiXSxbMiwxLCJmXlxcZGFnZ2VyIl0sWzAsMSwiZiIsMl0sWzAsMiwiV1xcdGltZXNcXGV0YSJdLFs0LDMsIlxcYmljZWxsIHJfZiIsMCx7InNob3J0ZW4iOnsic291cmNlIjoyMCwidGFyZ2V0IjoyMH19XV0=
        \begin{tikzcd}
          {W\times JX} && {W\times TX} \\
                       & TY
                       \arrow[""{name=0, anchor=center, inner sep=0}, "{f^\dagger}", from=1-3, to=2-2]
                       \arrow[""{name=1, anchor=center, inner sep=0}, "f"', from=1-1, to=2-2]
                       \arrow["W\times\eta", from=1-1, to=1-3]
                       \arrow["{\bicell r_f}", shorten <=7pt, shorten >=7pt, Rightarrow, from=1, to=0]
        \end{tikzcd}
      \end{equation}
    \item for all $X\in\bicat{J}$, an invertible 2-cell
      \begin{equation}\label{eq:strong_pseudomonad_l}
        % https://q.uiver.app/?q=WzAsMixbMCwwLCIxXFx0aW1lcyBUWCJdLFswLDIsIlRYIl0sWzAsMSwiXFxsYW1iZGEiLDIseyJjdXJ2ZSI6NX1dLFswLDEsIihcXGV0YVxcbGFtYmRhKV5cXGRhZ2dlciIsMCx7ImN1cnZlIjotNX1dLFsyLDMsIlxcbWF0aGJmIGxfWCIsMCx7InNob3J0ZW4iOnsic291cmNlIjoyMCwidGFyZ2V0IjoyMH19XV0=
        \begin{tikzcd}
          {1\times TX} \\
          \\
          TX
          \arrow[""{name=0, anchor=center, inner sep=0}, "\lambda"', curve={height=30pt}, from=1-1, to=3-1]
          \arrow[""{name=1, anchor=center, inner sep=0}, "{(\eta\lambda)^\dagger}", curve={height=-30pt}, from=1-1, to=3-1]
          \arrow["{\mathbf l_X}", shorten <=12pt, shorten >=12pt, Rightarrow, from=0, to=1]
        \end{tikzcd}
      \end{equation}
    \item for all $i:V\to W$, $f:JX\to TY$, and $g:W\times JY\to TZ$ in $\bicat{C}$,
      an invertible 2-cell
      \begin{equation}\label{eq:strong_pseudomonad_c}
        % https://q.uiver.app/?q=WzAsMyxbMCwwLCJWXFx0aW1lcyBUWCJdLFsyLDEsIlRaIl0sWzQsMCwiV1xcdGltZXMgVFkiXSxbMiwxLCJnXlQiXSxbMCwxLCIoZ15UKGlcXHRpbWVzIGYpKV5UIiwyXSxbMCwyLCJpXFx0aW1lcyAoZlxcbGFtYmRhKV5cXGRhZ2dlclxcbGFtYmRhXnstMX0iXSxbNCwzLCJcXG1hdGhiZiBjX3tmLGcsaX0iLDAseyJzaG9ydGVuIjp7InNvdXJjZSI6MjAsInRhcmdldCI6MjB9fV1d
        \begin{tikzcd}
          {V\times TX} &&&& {W\times TY} \\
                       && TZ
                       \arrow[""{name=0, anchor=center, inner sep=0}, "{g^T}", from=1-5, to=2-3]
                       \arrow[""{name=1, anchor=center, inner sep=0}, "{(g^T(i\times f))^T}"', from=1-1, to=2-3]
                       \arrow["{i\times (f\lambda)^\dagger\lambda^{-1}}", from=1-1, to=1-5]
                       \arrow["{\mathbf c_{f,g,i}}", shorten <=13pt, shorten >=13pt, Rightarrow, from=1, to=0]
        \end{tikzcd}
      \end{equation}
  \end{enumerate}
\end{definition}

We note that a priori this extension operator is defined on a narrower selection of 1-cells
than the extension operator of a relative pseudomonad. This is only a temporary limitation.
Using the left unitor $\lambda$, we obtain the usual extension functor
\begin{align}\label{eq:relative_pseudomonad_extension}
  \extend{\rr{-}}{*}_{X,Y} = \extend{\rr{-\lambda_{JX}}}{\dagger}\inv\lambda_{JX};
\end{align}
as in the diagram
\begin{equation}
  % https://q.uiver.app/?q=WzAsNCxbMCwwLCJcXHRleHR7SG9tfVtKWCxUWV0iXSxbMiwwLCJcXHRleHR7SG9tfVtUWCxUWV0iXSxbMCwxLCJcXHRleHR7SG9tfVsxXFx0aW1lcyBKWCxUWV0iXSxbMiwxLCJcXHRleHR7SG9tfVsxXFx0aW1lcyBUWCxUWV0iXSxbMywxLCItXFxjaXJjXFxsYW1iZGFeey0xfSIsMl0sWzIsMywiKC0pXlxcZGFnZ2VyIiwyXSxbMCwyLCItXFxjaXJjXFxsYW1iZGEiLDJdLFswLDEsIigtKV4qIl1d
  \begin{tikzcd}
    {\text{Hom}[JX,TY]} && {\text{Hom}[TX,TY]} \\
    {\text{Hom}[1\times JX,TY]} && {\text{Hom}[1\times TX,TY]}
    \arrow["{-\circ\lambda^{-1}}"', from=2-3, to=1-3]
    \arrow["{(-)^\dagger}"', from=2-1, to=2-3]
    \arrow["{-\circ\lambda}"', from=1-1, to=2-1]
    \arrow["{(-)^*}", from=1-1, to=1-3]
  \end{tikzcd}
\end{equation}
This will help us state the axioms in the following section more cleanly.

The purpose of \ref{def:prestrong_inclusion_pseudomonad_structure} is to define a theory that admits
the presheaf construction as a model. While one might expect this result to be a straightforward
statement, some work is required to define the structure itself. One must pay special
attention when working with several layers of indirection.
For example, we have a natural transformation whose components are functors
$\lambda_{\scat X}:1\times\scat X\to\scat X$. This problem is amplified by the fact that
our main point of study are objects $[\scatop X,\Set]$ which are themselves categories
with functors as objects and natural transformations as morphisms.

Nonetheless, we define the entire structure in detail:

\begin{example}\label{ex:prestrong_presheaves}
  The \emph{prestrong presheaf construction} consists of:
  \begin{enumerate}
    \item for all $\scat X\in\biCat$, the object $\widehat{\scat X}=\bb{\scatop X,\Set}$;
    \item for all $\scat X\in\biCat$, the unit functor $\eta_{\scat X}$ given
      by the Yoneda embedding $\eta_{\scat X}(X) = \Hom\rr{-,X}$;
    \item for all $\scat X,\scat Y\in\biCat$ and $\cat W\in\biCAT$, the extension functor
      $\extend{\rr{-}}{\dagger}_{\scat X,\scat Y,\cat W}$ given by
      \begin{enumerate}
        \item for all $F:\cat W\times\scat X\to \widehat{\scat Y}$, $W\in\cat W$,
          and $P\in\widehat{\scat X}$
          \begin{align}
            \extend{F}{\dagger}\rr{W,P} = \int^{X} PX\times F\rr{W,X}\rr{-}
          \end{align}
          and, for all morphisms $f$ in $\cat W$ and $\phi$ in $\widehat{\scat X}$,
          \begin{equation}
            \extend{F}{\dagger}\rr{f,\phi} = \int^{X} \phi_X\times F\rr{f,X}\rr{-};
          \end{equation}
        \item for all $F,G:\cat W\times\scat X\to\widehat{\scat Y}$,
          $\phi:F\Rightarrow G$, $W\in\cat W$, and $P\in\widehat{\scat X}$,
          \begin{align}
            \extend{\phi}{\dagger}_{W,P} = \int^{X} PX \times \rr{\phi_{W,P}}_{(-)};
          \end{align}
      \end{enumerate}
    \item for all $F:\cat W\times\scat X\to\widehat{\scat Y}$, $W\in\cat W$, and $X\in\scat X$,
      the natural transformation $(\bicell r_F)_{W,X}$ has the components
      \begin{equation}\label{eq:strong_presheaf_r}
        % https://q.uiver.app/?q=WzAsMyxbMCwwLCJGKFcsWClZIl0sWzIsMCwiXFx0ZXh0e0hvbX0oWCxYKVxcdGltZXMgRihXLFgpWSJdLFsyLDEsIlxcaW50XntYJ31cXHRleHR7SG9tfShYJyxYKVxcdGltZXMgRihXLFgnKVkiXSxbMCwxLCJcXGxhbmdsZSBcXERlbHRhXFx0ZXh0e2lkfSxGKFcsWClZXFxyYW5nbGUiXSxbMSwyLCJxIl0sWzAsMiwiKChcXG1hdGhiZiByX0YpX3tXLFh9KV9ZIiwyXV0=
        \begin{tikzcd}
          {F(W,X)Y} && {\text{Hom}(X,X)\times F(W,X)Y} \\
                    && {\int^{X'}\text{Hom}(X',X)\times F(W,X')Y}
                    \arrow["{\langle \Delta\text{id},F(W,X)Y\rangle}", from=1-1, to=1-3]
                    \arrow["q", from=1-3, to=2-3]
                    \arrow["{((\mathbf r_F)_{W,X})_Y}"', from=1-1, to=2-3]
        \end{tikzcd}
      \end{equation}
    \item for all $P\in\widehat{\scat X}$, the natural transformation $\rr{\bicell l_{\scat X}}_P$ has the components
      \begin{equation}\label{eq:strong_presheaf_l}
        % https://q.uiver.app/?q=WzAsMyxbMCwwLCJQWCJdLFsyLDEsIlxcaW50XntYJ30gUFgnXFx0aW1lc1xcdGV4dHtIb219KFgsWCcpIl0sWzIsMCwiUFhcXHRpbWVzXFx0ZXh0e0hvbX0oWCxYKSJdLFswLDIsIlxcbGFuZ2xlIFBYLFxcRGVsdGFcXHRleHR7aWR9XFxyYW5nbGUiXSxbMiwxLCJxIl0sWzAsMSwiKChcXG1hdGhiZiBsX3tcXG1hdGhiYiBYfSlfUClfWCIsMl1d
        \begin{tikzcd}
          PX && {PX\times\text{Hom}(X,X)} \\
             && {\int^{X'} PX'\times\text{Hom}(X,X')}
             \arrow["{\langle PX,\Delta\text{id}\rangle}", from=1-1, to=1-3]
             \arrow["q", from=1-3, to=2-3]
             \arrow["{((\mathbf l_{\mathbb X})_P)_X}"', from=1-1, to=2-3]
        \end{tikzcd}
      \end{equation}
    \item for all $F:\scat X\to\widehat{\scat Y}$, $G:\cat W\times\scat Y\to\widehat{\scat Z}$,
      $I:\cat V\to\scat W$, $P\in\widehat{\scat X}$, and $V\in\cat V$, the natural isomorphism
      $\rr{\bicell c_{F,G,I}}_{V,P}$ has as components the canonical cowedge
      isomorphisms given by
      \begin{equation*}
        % https://q.uiver.app/#q=WzAsNixbMCwyLCJcXGludF5YIFBYXFx0aW1lc1xcaW50XlkgKEZYKVlcXHRpbWVzIEcoSVYsWSlaIl0sWzEsMiwiXFxpbnReWVxcbGVmdChcXGludF5YIFBYXFx0aW1lcyAoRlgpWVxccmlnaHQpXFx0aW1lcyBHKElWLFkpWiJdLFswLDEsIlBYXFx0aW1lcyBcXGludF5ZIChGWClZXFx0aW1lcyBHKElWLFkpWiJdLFswLDAsIlBYXFx0aW1lc1xcbGVmdCgoRlgpWVxcdGltZXMgRyhJVixZKVpcXHJpZ2h0KSJdLFsxLDAsIihQWFxcdGltZXMgKEZYKVkpXFx0aW1lcyBHKElWLFkpWiJdLFsxLDEsIlxcbGVmdChcXGludF5YIFBYXFx0aW1lcyAoRlgpWVxccmlnaHQpXFx0aW1lcyBHKElWLFkpWiJdLFswLDEsIigoXFxtYXRoYmYgY197RixHLEl9KV97VixQfSlfWiIsMSx7ImN1cnZlIjozfV0sWzIsMCwicSIsMl0sWzUsMSwicSJdLFszLDIsIlBYXFx0aW1lcyBxIiwyXSxbNCw1LCJxXFx0aW1lcyBHKElWLFkpWiJdLFszLDQsIlxcY29uZyIsMSx7ImN1cnZlIjotM31dXQ==
        \begin{tikzcd}
          {PX\times\left((FX)Y\times G(IV,Y)Z\right)} & {(PX\times (FX)Y)\times G(IV,Y)Z} \\
          {PX\times \int^Y (FX)Y\times G(IV,Y)Z} & {\left(\int^X PX\times (FX)Y\right)\times G(IV,Y)Z} \\
          {\int^X PX\times\int^Y (FX)Y\times G(IV,Y)Z} & {\int^Y\left(\int^X PX\times (FX)Y\right)\times G(IV,Y)Z}
          \arrow["{((\mathbf c_{F,G,I})_{V,P})_Z}"{description}, curve={height=18pt}, from=3-1, to=3-2]
          \arrow["q"', from=2-1, to=3-1]
          \arrow["q", from=2-2, to=3-2]
          \arrow["{PX\times q}"', from=1-1, to=2-1]
          \arrow["{q\times G(IV,Y)Z}", from=1-2, to=2-2]
          \arrow["\cong"{description}, curve={height=-18pt}, from=1-1, to=1-2]
        \end{tikzcd}
      \end{equation*}
  \end{enumerate}
\end{example}

Even from this detailed description it is not clear that we have provided the
necessary structure: we have to make sure that the structural 2-cells are in fact
invertible. For $\bicell c$, this is immediate and we already showed the result for
$\bicell l$ in \ref{ex:coend}. For $\bicell r$, the proof is similar:

\begin{lemma}\label{lemma:strong_presheaf_r_is_iso}
  The function (\ref{eq:strong_presheaf_r}) is an isomorphism.
  \begin{proof}
    For each $X'\in\scat X$, consider the function
    \begin{align*}
      w\rr{X'}:\Hom(X',X)\times F(W,X')Y&\to F(W,X)Y
    \end{align*}
    given by $\rr{f,u}\mapsto (F(W,f)Y)u$. We note that, for all 1-cells $f:X_1\to X_2$
    in $\scat X$,
    the following commutes:
    \begin{equation*}
      % https://q.uiver.app/?q=WzAsNCxbMSwxLCJGKFcsWClZIl0sWzAsMCwiXFx0ZXh0e0hvbX0oWF8yLFgpXFx0aW1lcyBGKFcsWF8xKVkiXSxbMSwwLCJcXHRleHR7SG9tfShYXzEsWClcXHRpbWVzIEYoVyxYXzEpWSJdLFswLDEsIlxcdGV4dHtIb219KFhfMixYKVxcdGltZXMgRihXLFhfMilZIl0sWzEsMywiXFx0ZXh0e0hvbX0oWF8yLFgpXFx0aW1lcyBGKFcsZilZIiwyXSxbMSwyLCJcXHRleHR7SG9tfShmLFgpXFx0aW1lcyBGKFcsWF8xKVkiLDAseyJjdXJ2ZSI6LTJ9XSxbMywwLCJ3KFhfMikiLDIseyJjdXJ2ZSI6Mn1dLFsyLDAsIncoWF8xKSJdXQ==
      \begin{tikzcd}
        {\text{Hom}(X_2,X)\times F(W,X_1)Y} & {\text{Hom}(X_1,X)\times F(W,X_1)Y} \\
        {\text{Hom}(X_2,X)\times F(W,X_2)Y} & {F(W,X)Y}
        \arrow["{\text{Hom}(X_2,X)\times F(W,f)Y}"', from=1-1, to=2-1]
        \arrow["{\text{Hom}(f,X)\times F(W,X_1)Y}", curve={height=-12pt}, from=1-1, to=1-2]
        \arrow["{w(X_2)}"', curve={height=12pt}, from=2-1, to=2-2]
        \arrow["{w(X_1)}", from=1-2, to=2-2]
      \end{tikzcd}
    \end{equation*}
    Thus $F(W,X)Y$ and $w$ define a cowedge. Consider the diagram
    \begin{equation}\label{eq:r_iso_proof}
      % https://q.uiver.app/?q=WzAsNCxbMSwwLCJGKFcsWClZIl0sWzAsMCwiXFx0ZXh0e0hvbX0oWCcsWClcXHRpbWVzIEYoVyxYJylZIl0sWzEsMiwiXFxpbnRee1gnfVxcdGV4dHtIb219KFgnLFgpXFx0aW1lcyBGKFcsWCcpWSJdLFsxLDEsIlxcdGV4dHtIb219KFgsWClcXHRpbWVzIEYoVyxYKVkiXSxbMSwwLCJ3KFgnKSJdLFsxLDIsInEiLDIseyJjdXJ2ZSI6NX1dLFszLDIsInEiLDJdLFswLDMsIlxcbGFuZ2xlIFxcRGVsdGFcXHRleHR7aWR9LEYoVyxYKVlcXHJhbmdsZSIsMl0sWzAsMiwiKChcXG1hdGhiZiByX0YpX3tXLFh9KV9ZIiwwLHsiY3VydmUiOi01fV1d
      \begin{tikzcd}
        {\text{Hom}(X',X)\times F(W,X')Y} & {F(W,X)Y} \\
                                          & {\text{Hom}(X,X)\times F(W,X)Y} \\
                                          & {\int^{X'}\text{Hom}(X',X)\times F(W,X')Y}
                                          \arrow["{w(X')}", from=1-1, to=1-2]
                                          \arrow["q"', curve={height=60pt}, from=1-1, to=3-2]
                                          \arrow["q"', from=2-2, to=3-2]
                                          \arrow["{\langle \Delta\text{id},F(W,X)Y\rangle}"', from=1-2, to=2-2]
                                          \arrow["{((\mathbf r_F)_{W,X})_Y}", curve={height=-100pt}, from=1-2, to=3-2]
      \end{tikzcd}
    \end{equation}
    Let $g:X'\to X$ and let $u\in F(W,X')Y$. Chasing elements we find
    \begin{align*}
      (q_X\circ\aa{\Delta\id, F(W,X)Y}\circ w(X'))(g,u)
      &= (q_X\circ\aa{\Delta\id, F(W,X)Y})((F(W,g)Y)u) \\
      &= q_X(\id, (F(W,g)Y)u)\\
      &= q_{X'}(g,u)
    \end{align*}
    where the last step follows from the cowedge property of the coend. Thus
    (\ref{eq:r_iso_proof}) commutes, making (\ref{eq:strong_presheaf_r}) into a
    cowedge morphism. By universality of the coend it follows that this must
    be an isomorphism.
  \end{proof}
\end{lemma}

\section{Induced relative pseudomonad structure}\label{sec:relative_pseudomonads}

Strong monads are special monads. For our generalisation to make sense, strong
$J$-pseudomonads ought to be special relative pseudomonads over $J$. While this
relationship is not as obvious as in the 1-categorical case, the structure
\ref{def:prestrong_inclusion_pseudomonad_structure} already resembles that of a
relative pseudomonad. This observation makes it straightforward to see how the
former gives rise to the latter.

\begin{definition}\label{def:induced_relative_pseudomonad_structure}
  Let $T$ be a strong $J$-pseudomonad. The \emph{relative pseudomonad structure induced
  by $T$} consists of
  \begin{itemize}
    \item for all $X,Y\in\bicat{J}$, the extension functor $\extend{\rr{-}}{*}$ as in
      \ref{eq:relative_pseudomonad_extension};
    \item for all 1-cells $f:JX\to TY$ and $g:JY\to TZ$, the 2-cell
      \begin{align*}
        \bicell m_{f,g}=\bicell c_{f,g\lambda}\inv\lambda_{TX}
      \end{align*}
      as in the diagram
      \begin{equation}
        % https://q.uiver.app/?q=WzAsNSxbNiwwLCJUWSJdLFswLDAsIlRYIl0sWzEsMCwiMVxcdGltZXMgVFgiXSxbMywxLCJUWiJdLFs1LDAsIjFcXHRpbWVzIFRZIl0sWzEsMiwiXFxsYW1iZGFeey0xfSIsMl0sWzIsMywiKChnXFxsYW1iZGEpXlxcZGFnZ2VyKDFcXHRpbWVzIGYpKV5cXGRhZ2dlciIsMl0sWzQsMywiKGdcXGxhbWJkYSleXFxkYWdnZXIiXSxbMCw0LCJcXGxhbWJkYV57LTF9Il0sWzIsNCwiMVxcdGltZXMgZl4qIl0sWzEsMCwiZl4qIiwwLHsiY3VydmUiOi01fV0sWzEsMywiKGdeKmYpXioiLDIseyJjdXJ2ZSI6NX1dLFswLDMsImdeKiIsMCx7ImN1cnZlIjotNX1dLFs2LDcsIlxcbWF0aGJmIGMiLDAseyJzaG9ydGVuIjp7InNvdXJjZSI6MjAsInRhcmdldCI6MjB9fV1d
        \begin{tikzcd}
          TX & {1\times TX} &&&& {1\times TY} & TY \\
             &&& TZ
             \arrow["{\lambda^{-1}}"', from=1-1, to=1-2]
             \arrow[""{name=0, anchor=center, inner sep=0}, "{((g\lambda)^\dagger(1\times f))^\dagger}"', from=1-2, to=2-4]
             \arrow[""{name=1, anchor=center, inner sep=0}, "{(g\lambda)^\dagger}", from=1-6, to=2-4]
             \arrow["{\lambda^{-1}}", from=1-7, to=1-6]
             \arrow["{1\times f^*}", from=1-2, to=1-6]
             \arrow["{f^*}", curve={height=-30pt}, from=1-1, to=1-7]
             \arrow["{(g^*f)^*}"', curve={height=30pt}, from=1-1, to=2-4]
             \arrow["{g^*}", curve={height=-30pt}, from=1-7, to=2-4]
             \arrow["{\mathbf c}", shorten <=13pt, shorten >=13pt, Rightarrow, from=0, to=1]
        \end{tikzcd}
      \end{equation}
    \item for all $f:JX\to TY$, the 2-cell
      \begin{align*}
        \bicell e_f = \mathbf r_{f\lambda}\inv\lambda_{JX}
      \end{align*}
      as in the diagram
      \begin{equation}
        % https://q.uiver.app/?q=WzAsNSxbMCwwLCJKWCJdLFsyLDEsIlRZIl0sWzQsMCwiVFgiXSxbMywwLCIxXFx0aW1lcyBUWCJdLFsxLDAsIjFcXHRpbWVzIEpYIl0sWzAsMiwiXFxldGEiLDAseyJjdXJ2ZSI6LTV9XSxbMiwzLCJcXGxhbWJkYV57LTF9Il0sWzMsMSwiKGZcXGxhbWJkYSleXFxkYWdnZXIiXSxbMCw0LCJcXGxhbWJkYV57LTF9IiwyXSxbNCwzLCIxXFx0aW1lcyBcXGV0YSJdLFs0LDEsImZcXGxhbWJkYSIsMl0sWzIsMSwiZl4qIiwwLHsiY3VydmUiOi0zfV0sWzAsMSwiZiIsMix7ImN1cnZlIjozfV0sWzEwLDcsIlxcbWF0aGJmIHIiLDAseyJzaG9ydGVuIjp7InNvdXJjZSI6MjAsInRhcmdldCI6MjB9fV1d
        \begin{tikzcd}
          JX & {1\times JX} && {1\times TX} & TX \\
             && TY
             \arrow["\eta", curve={height=-30pt}, from=1-1, to=1-5]
             \arrow["{\lambda^{-1}}", from=1-5, to=1-4]
             \arrow[""{name=0, anchor=center, inner sep=0}, "{(f\lambda)^\dagger}", from=1-4, to=2-3]
             \arrow["{\lambda^{-1}}"', from=1-1, to=1-2]
             \arrow["{1\times \eta}", from=1-2, to=1-4]
             \arrow[""{name=1, anchor=center, inner sep=0}, "f\lambda"', from=1-2, to=2-3]
             \arrow["{f^*}", curve={height=-18pt}, from=1-5, to=2-3]
             \arrow["f"', curve={height=18pt}, from=1-1, to=2-3]
             \arrow["{\mathbf r}", shorten <=7pt, shorten >=7pt, Rightarrow, from=1, to=0]
        \end{tikzcd}
      \end{equation}
    \item for all $X\in\bicat{J}$, the 2-cell
      \begin{align*}
        \bicell t_X=\mathbf l_{\eta_X}\inv\lambda_{TX}
      \end{align*}
      as in the diagram
      \begin{equation}
        % https://q.uiver.app/?q=WzAsMyxbMCwwLCJUWCJdLFsyLDIsIlRYIl0sWzIsMCwiMVxcdGltZXMgVFgiXSxbMCwxLCJcXGV0YV57VFxcbGFtYmRhfSIsMix7ImN1cnZlIjoyfV0sWzAsMiwiXFxsYW1iZGFeey0xfSJdLFsyLDEsIihcXGV0YVxcbGFtYmRhKV5UIiwyLHsiY3VydmUiOjN9XSxbMiwxLCJcXGxhbWJkYSIsMCx7ImN1cnZlIjotM31dLFs1LDYsIlxcbWF0aGJmIGwiLDAseyJzaG9ydGVuIjp7InNvdXJjZSI6MjAsInRhcmdldCI6MjB9fV1d
        \begin{tikzcd}
          TX && {1\times TX} \\
          \\
             && TX
             \arrow["{\eta^{T\lambda}}"', curve={height=12pt}, from=1-1, to=3-3]
             \arrow["{\lambda^{-1}}", from=1-1, to=1-3]
             \arrow[""{name=0, anchor=center, inner sep=0}, "{(\eta\lambda)^T}"', curve={height=18pt}, from=1-3, to=3-3]
             \arrow[""{name=1, anchor=center, inner sep=0}, "\lambda", curve={height=-18pt}, from=1-3, to=3-3]
             \arrow["{\mathbf l}", shorten <=7pt, shorten >=7pt, Rightarrow, from=0, to=1]
        \end{tikzcd}
      \end{equation}
  \end{itemize}
\end{definition}

When considering the presheaf construction, the induced relative pseudomonad structure
consists of several isomorphisms that may be familiar from coend calculus.
While we are not concerned with the details, it is worth investigating the structure
nonetheless:

\begin{example}
  For the presheaf construction, we obtain the following relative pseudomonad structure:
  \begin{enumerate}
    \item for all $F,G:\scat X\to\widehat{\scat Y}$, $\phi : F\Rightarrow G$,
      and $P\in\widehat{\scat X}$,
      \begin{align*}
        \extend{F}{*}(P) = \int^X PX\times FX(-), \hs
        \rr{\extend{\phi}{*}}_P = \int^X PX\times \phi_X;
      \end{align*}
    \item for all $X\in\scat X$, $\eta_{\scat X}(X) = \Hom\rr{-,X}$;
    \item for all $F:\scat X\to\widehat{\scat Y}$ and $X\in\scat X$, there is a
      natural isomorphism with components
      \begin{align*}
        \rr{\bicell e_F}_X:FX\to\int^{X'} \Hom(X',X)\times FX';
      \end{align*}
    \item for all $F:\scat X\to\widehat{\scat Y}$, $G:\scat Y\to\widehat{\scat
      Z}$, and $P\in\widehat{\scat X}$, $(\bicell m_{F,G})_P$ is a natural
      isomorphism with components
      \begin{align*}
        \int^X PX\times\int^Y (FX)Y\times (GY)Z
        \to\int^Y \rr{\int^X PX\times (FX)Y}\times (GY)Z;
      \end{align*}
    \item for all $P\in\widehat{\scat X}$, $\bicell t$ has as components isomorphisms
      \begin{align*}
        \rr{\rr{\bicell t_{\scat X}}_P}_X: \int^{X'} \Hom(X,X')\times PX'\to PX
      \end{align*}
  \end{enumerate}
\end{example}

\section{Prestrong axioms}

A prestrong $J$-pseudomonad should induce a relative pseudomonad over $J$. It is therefore
not surprising that the conditions which we impose are entirely analogous to those
stated in~\cite{fiore2017}.

\begin{definition}\label{def:prestrong_inclusion_pseudomonad_axioms}
  A \emph{prestrong $J$-pseudomonad} is a prestrong $J$-pseudomonad structure such that
  \begin{enumerate}
    \item $\bicell r_f$ is natural in $f$;
    \item $\bicell c_{f,g,i}$ is natural in $f$, $g$, and $i$;
    \item for all $f:W\times JX\to TY$,
      \begin{equation}\label{eq:strong_pseudomonad_square}
        % https://q.uiver.app/?q=WzAsNCxbMCwwLCJmXlxcZGFnZ2VyIl0sWzAsMSwiZl5cXGRhZ2dlciJdLFsyLDAsIihmXlxcZGFnZ2VyKDFcXHRpbWVzXFxldGEpKV5cXGRhZ2dlciJdLFsyLDEsImZeXFxkYWdnZXIoMVxcdGltZXNcXGV0YV4qKSJdLFswLDIsIlxcbWF0aGJmIHIiXSxbMiwzLCJcXG1hdGhiZiBjIl0sWzMsMSwiZl5cXGRhZ2dlcigxXFx0aW1lc1xcbWF0aGJmIGxcXGxhbWJkYV57LTF9KSJdLFswLDEsIiIsMix7ImxldmVsIjoyLCJzdHlsZSI6eyJoZWFkIjp7Im5hbWUiOiJub25lIn19fV1d
        \begin{tikzcd}
          {f^\dagger} && {(f^\dagger(1\times\eta))^\dagger} \\
          {f^\dagger} && {f^\dagger(1\times\eta^*)}
          \arrow["{\mathbf r}", from=1-1, to=1-3]
          \arrow["{\mathbf c}", from=1-3, to=2-3]
          \arrow["{f^\dagger(1\times\mathbf l\lambda^{-1})}", from=2-3, to=2-1]
          \arrow[Rightarrow, no head, from=1-1, to=2-1]
        \end{tikzcd}
      \end{equation}
    \item for all suitable 1-cells $f,g,h,i,j$,
      \begin{equation}\label{eq:strong_pseudomonad_house}
        % https://q.uiver.app/?q=WzAsNSxbMSwwLCIoKGheXFxkYWdnZXIoalxcdGltZXMgZykpXlxcZGFnZ2VyKGlcXHRpbWVzIGYpKV5cXGRhZ2dlciJdLFswLDEsIihoXlxcZGFnZ2VyKGppXFx0aW1lcyBnXipmKSleXFxkYWdnZXIiXSxbMCwzLCJoXlxcZGFnZ2VyKGppXFx0aW1lcyhnXipmKV5cXGRhZ2dlcikiXSxbMiwxLCIoaF5cXGRhZ2dlcihqXFx0aW1lcyBnKSleXFxkYWdnZXIoaVxcdGltZXMgZl4qKSJdLFsyLDMsImheXFxkYWdnZXIoamlcXHRpbWVzIGdeKmZeKikiXSxbMSwyLCJcXG1hdGhiZiBjIiwxXSxbMCwzLCJcXG1hdGhiZiBjIiwxXSxbMiw0LCJoXlxcZGFnZ2VyKGppXFx0aW1lc1xcbWF0aGJmIGNcXGxhbWJkYV57LTF9KSIsMV0sWzMsNCwiXFxtYXRoYmYgYyhpXFx0aW1lcyBmXiopIiwxXSxbMCwxLCIoXFxtYXRoYmYgYyhpXFx0aW1lcyBmKSleXFxkYWdnZXIiLDFdXQ==
        \begin{tikzcd}
  & {((h^\dagger(j\times g))^\dagger(i\times f))^\dagger} \\
          {(h^\dagger(ji\times g^*f))^\dagger} && {(h^\dagger(j\times g))^\dagger(i\times f^*)} \\
          \\
          {h^\dagger(ji\times(g^*f)^\dagger)} && {h^\dagger(ji\times g^*f^*)}
          \arrow["{\mathbf c}"{description}, from=2-1, to=4-1]
          \arrow["{\mathbf c}"{description}, from=1-2, to=2-3]
          \arrow["{h^\dagger(ji\times\mathbf c\lambda^{-1})}"{description}, from=4-1, to=4-3]
          \arrow["{\mathbf c(i\times f^*)}"{description}, from=2-3, to=4-3]
          \arrow["{(\mathbf c(i\times f))^\dagger}"{description}, from=1-2, to=2-1]
        \end{tikzcd}
      \end{equation}
  \end{enumerate}
\end{definition}

Now that we have stated some axioms, we need to make sure that they are sensible. The first
step is to verify that the presheaf construction remains a model.

\begin{proposition}
  The prestrong presheaf construction in \ref{ex:prestrong_presheaves} is a prestrong $J$-pseudomonad.
  \begin{proof}
    We verify the axioms:
    \begin{enumerate}
      \item We note that
        \begin{equation*}
          % https://q.uiver.app/?q=WzAsNixbMCwwLCJGKFcsWClZIl0sWzEsMCwiRyhXLFgpWSJdLFswLDIsIlxcdGV4dHtIb219KFgsWClcXHRpbWVzIEYoVyxYKVkiXSxbMSwyLCJcXHRleHR7SG9tfShYLFgpXFx0aW1lcyBHKFcsWClZIl0sWzAsNCwiXFxpbnRee1gnfVxcdGV4dHtIb219KFgnLFgpXFx0aW1lcyBGKFcsWCcpWSJdLFsxLDQsIlxcaW50XntYJ30gXFx0ZXh0e0hvbX0oWCcsWClcXHRpbWVzIEYoVyxYJylZIl0sWzAsMSwiXFxsZWZ0KFxccGhpX3tXLFh9XFxyaWdodClfWSIsMV0sWzIsMywiXFx0ZXh0e0hvbX0oWCxYKVxcdGltZXMgXFxsZWZ0KFxccGhpX3tXLFh9XFxyaWdodClfWSIsMix7ImN1cnZlIjoyfV0sWzAsMiwiXFxsYW5nbGUgXFxEZWx0YVxcdGV4dHtpZH0sRihXLFgpWVxccmFuZ2xlIiwxXSxbMSwzLCJcXGxhbmdsZSBcXERlbHRhXFx0ZXh0e2lkfSxHKFcsWClZXFxyYW5nbGUiLDFdLFsyLDQsInEiLDFdLFszLDUsInEiLDFdLFs0LDUsIlxcaW50XntYJ31cXHRleHR7SG9tfShYJyxYKVxcdGltZXMgXFxsZWZ0KFxccGhpX3tXLFh9XFxyaWdodClfWSIsMix7ImN1cnZlIjoyfV0sWzEsNSwiXFxtYXRoYmYgciIsMSx7ImN1cnZlIjotNX1dLFswLDQsIlxcbWF0aGJmIHIiLDEseyJjdXJ2ZSI6NX1dLFs2LDcsIlxcbGFuZ2xlXFxEZWx0YVxcdGV4dHtpZH0sLVxccmFuZ2xlXFx0ZXh0ey1uYXR9IiwxLHsic2hvcnRlbiI6eyJzb3VyY2UiOjIwLCJ0YXJnZXQiOjIwfSwic3R5bGUiOnsiYm9keSI6eyJuYW1lIjoibm9uZSJ9LCJoZWFkIjp7Im5hbWUiOiJub25lIn19fV0sWzcsMTIsIiIsMSx7InNob3J0ZW4iOnsic291cmNlIjoyMCwidGFyZ2V0IjoyMH0sInN0eWxlIjp7ImJvZHkiOnsibmFtZSI6Im5vbmUifSwiaGVhZCI6eyJuYW1lIjoibm9uZSJ9fX1dLFsxMCwxNCwiXFxtYXRoYmYgclxcdGV4dHstZGVmfSIsMSx7InNob3J0ZW4iOnsic291cmNlIjoyMH0sInN0eWxlIjp7ImJvZHkiOnsibmFtZSI6Im5vbmUifSwiaGVhZCI6eyJuYW1lIjoibm9uZSJ9fX1dLFsxMywxMSwiXFxtYXRoYmYgclxcdGV4dHstZGVmfSIsMSx7InNob3J0ZW4iOnsidGFyZ2V0IjoyMH0sInN0eWxlIjp7ImJvZHkiOnsibmFtZSI6Im5vbmUifSwiaGVhZCI6eyJuYW1lIjoibm9uZSJ9fX1dXQ==
          \begin{tikzcd}
            {F(W,X)Y} & {G(W,X)Y} \\
            \\
            {\text{Hom}(X,X)\times F(W,X)Y} & {\text{Hom}(X,X)\times G(W,X)Y} \\
            \\
            {\int^{X'}\text{Hom}(X',X)\times F(W,X')Y} & {\int^{X'} \text{Hom}(X',X)\times F(W,X')Y}
            \arrow[""{name=0, anchor=center, inner sep=0}, "{\left(\phi_{W,X}\right)_Y}"{description}, from=1-1, to=1-2]
            \arrow[""{name=1, anchor=center, inner sep=0}, "{\text{Hom}(X,X)\times \left(\phi_{W,X}\right)_Y}"', curve={height=20pt}, from=3-1, to=3-2]
            \arrow["{\langle \Delta\text{id},F(W,X)Y\rangle}"{description}, from=1-1, to=3-1]
            \arrow["{\langle \Delta\text{id},G(W,X)Y\rangle}"{description}, from=1-2, to=3-2]
            \arrow[""{name=2, anchor=center, inner sep=0}, "q"{description}, from=3-1, to=5-1]
            \arrow[""{name=3, anchor=center, inner sep=0}, "q"{description}, from=3-2, to=5-2]
            \arrow[""{name=4, anchor=center, inner sep=0}, "{\int^{X'}\text{Hom}(X',X)\times \left(\phi_{W,X}\right)_Y}"', curve={height=20pt}, from=5-1, to=5-2]
            \arrow[""{name=5, anchor=center, inner sep=0}, "{\mathbf r}"{description}, curve={height=-100pt}, from=1-2, to=5-2]
            \arrow[""{name=6, anchor=center, inner sep=0}, "{\mathbf r}"{description}, curve={height=100pt}, from=1-1, to=5-1]
            \arrow["{\langle\Delta\text{id},-\rangle\text{-nat}}"{description}, draw=none, from=0, to=1]
            \arrow[draw=none, from=1, to=4]
            \arrow["{\mathbf r\text{-def}}"{description}, draw=none, from=2, to=6]
            \arrow["{\mathbf r\text{-def}}"{description}, draw=none, from=5, to=3]
          \end{tikzcd}
        \end{equation*}
        commutes due to (\ref{eq:coend_natural_transformation}) on the right. This shows naturality
        of $\bicell r$.
      \item Naturality of $\bicell c$ follows from naturality of the
        underlying isomorphism.\footnote{
          Full diagram of naturality of $\bicell c_{F,G,I}$ in $F$ and $G$ in
        \href{https://q.uiver.app/?q=WzAsMTIsWzAsMCwiXFxpbnReWCBQWFxcdGltZXNcXGludF5ZKEZYKVlcXHRpbWVzIEcoSVYsWSlaIl0sWzUsMCwiXFxpbnReWVxcbGVmdChcXGludF5YIFBYXFx0aW1lcyAoRlgpWVxccmlnaHQpXFx0aW1lcyBHKElWLFkpWiJdLFswLDksIlxcaW50XlggUFhcXHRpbWVzXFxpbnReWShIWClZXFx0aW1lcyBLKElWLFkpWiJdLFs1LDksIlxcaW50XllcXGxlZnQoXFxpbnReWCBQWFxcdGltZXMoSFgpWVxccmlnaHQpXFx0aW1lcyBLKElWLFkpWiJdLFsxLDEsIlBYXFx0aW1lc1xcaW50XlkoRlgpWVxcdGltZXMgRyhJVixZKVoiXSxbNCwxLCJcXGxlZnQoXFxpbnReWCBQWFxcdGltZXMoRlgpWVxccmlnaHQpXFx0aW1lcyBHKElWLFkpWiJdLFsxLDgsIlBYXFx0aW1lc1xcaW50XlkoSFgpWVxcdGltZXMgSyhJVixZKVoiXSxbNCw4LCJcXGxlZnQoXFxpbnReWCBQWFxcdGltZXMoSFgpWVxccmlnaHQpXFx0aW1lcyBLKElWLFkpWiJdLFsyLDIsIlBYXFx0aW1lcygoRlgpWVxcdGltZXMgRyhJVixZKVopIl0sWzIsNywiUFhcXHRpbWVzKChIWClZXFx0aW1lcyBLKElWLFkpWikiXSxbMyw3LCIoUFhcXHRpbWVzIChIWClZKVxcdGltZXMgSyhJVixZKVoiXSxbMywyLCIoUFhcXHRpbWVzKEZYKVkpXFx0aW1lcyBHKElWLFkpWiJdLFsyLDMsIlxcbWF0aGJmIGMiXSxbMCwxLCJcXG1hdGhiZiBjIl0sWzAsMiwiXFxpbnReWCBQWFxcdGltZXNcXGludF5ZKFxccGhpX1gpX1lcXHRpbWVzKFxccHNpX3tJVixZfSlfWiIsMV0sWzcsMywicSIsMV0sWzYsMiwicSIsMV0sWzQsMCwicSIsMV0sWzUsMSwicSIsMl0sWzksMTAsIlxcY29uZyIsMSx7ImN1cnZlIjotMn1dLFs4LDExLCJcXGNvbmciLDEseyJjdXJ2ZSI6Mn1dLFs4LDQsIlBYXFx0aW1lcyBxIiwxXSxbMTEsNSwicVxcdGltZXMgRyhJVixZKVoiLDFdLFs5LDYsIlBYXFx0aW1lcyBxIiwxXSxbMTAsNywicVxcdGltZXMgSyhJVixZKVoiLDFdLFsxLDMsIlxcaW50XllcXGxlZnQoXFxpbnReWCBQWFxcdGltZXMgKFxccGhpX1gpX1lcXHJpZ2h0KVxcdGltZXMoXFxwc2lfe0lWLFl9KV9aIiwxXSxbOCw5LCJQWFxcdGltZXMoKFxccGhpX1gpX1lcXHRpbWVzKFxccHNpX3tJVixZfSlfWikiLDFdLFsxMSwxMCwiKFBYXFx0aW1lcyhcXHBoaV9YKV9ZKVxcdGltZXMoXFxwc2lfe0lWLFl9KV9aIiwxXV0=}{quiver}.}
      \item We note that the following commutes:
        \begin{equation*}
          % https://q.uiver.app/?q=WzAsNixbMCwwLCJQWFxcdGltZXMgRihXLFgpWSJdLFsxLDAsIlBYXFx0aW1lc1xcaW50XntYJ31cXHRleHR7SG9tfShYJyxYKVxcdGltZXMgRihXLFgnKVkiXSxbMSwyLCJQWFxcdGltZXMoXFx0ZXh0e0hvbX0oWCxYKVxcdGltZXMgRihXLFgpWSkiXSxbMSw1LCJcXGxlZnQoXFxpbnReWCBQWFxcdGltZXNcXHRleHR7SG9tfShYJyxYKVxccmlnaHQpXFx0aW1lcyBGKFcsWClZIl0sWzEsMywiKFBYXFx0aW1lc1xcdGV4dHtIb219KFgsWCkpXFx0aW1lcyBGKFcsWClZIl0sWzAsNSwiUFhcXHRpbWVzIEYoVyxYKVkiXSxbMiwxLCJQWFxcdGltZXMgcSIsMSx7ImN1cnZlIjozfV0sWzAsMSwiUFhcXHRpbWVzXFxtYXRoYmYgciIsMSx7ImN1cnZlIjotM31dLFswLDIsIlBYXFx0aW1lc1xcbGFuZ2xlXFxEZWx0YVxcdGV4dHtpZH0sRihXLFgpWVxccmFuZ2xlIiwxXSxbNCwzLCJxXFx0aW1lcyBGKFcsWClZIiwxLHsiY3VydmUiOi0zfV0sWzIsNCwiXFxhbHBoYSIsMV0sWzUsMywiXFxtYXRoYmYgbFxcdGltZXMgRihXLFgnKVkiLDEseyJjdXJ2ZSI6M31dLFswLDUsIiIsMSx7ImxldmVsIjoyLCJzdHlsZSI6eyJoZWFkIjp7Im5hbWUiOiJub25lIn19fV0sWzUsNCwiXFxsYW5nbGUgUFgsXFxEZWx0YVxcdGV4dHtpZH1cXHJhbmdsZVxcdGltZXMgRihXLFgpWSIsMV0sWzgsMSwiXFxtYXRoYmYgclxcdGV4dHstZGVmfSIsMSx7InNob3J0ZW4iOnsic291cmNlIjoyMH0sInN0eWxlIjp7ImJvZHkiOnsibmFtZSI6Im5vbmUifSwiaGVhZCI6eyJuYW1lIjoibm9uZSJ9fX1dLFsxMywzLCJcXG1hdGhiZiByXFx0ZXh0ey1kZWZ9IiwxLHsic2hvcnRlbiI6eyJzb3VyY2UiOjIwfSwic3R5bGUiOnsiYm9keSI6eyJuYW1lIjoibm9uZSJ9LCJoZWFkIjp7Im5hbWUiOiJub25lIn19fV1d
          \begin{tikzcd}
            {PX\times F(W,X)Y} & {PX\times\int^{X'}\text{Hom}(X',X)\times F(W,X')Y} \\
            \\
                               & {PX\times(\text{Hom}(X,X)\times F(W,X)Y)} \\
                               & {(PX\times\text{Hom}(X,X))\times F(W,X)Y} \\
                               \\
            {PX\times F(W,X)Y} & {\left(\int^X PX\times\text{Hom}(X',X)\right)\times F(W,X)Y}
            \arrow["{PX\times q}"{description}, curve={height=18pt}, from=3-2, to=1-2]
            \arrow["{PX\times\mathbf r}"{description}, curve={height=-18pt}, from=1-1, to=1-2]
            \arrow[""{name=0, anchor=center, inner sep=0}, "{PX\times\langle\Delta\text{id},F(W,X)Y\rangle}"{description}, from=1-1, to=3-2]
            \arrow["{q\times F(W,X)Y}"{description}, curve={height=-18pt}, from=4-2, to=6-2]
            \arrow["\alpha"{description}, from=3-2, to=4-2]
            \arrow["{\mathbf l\times F(W,X')Y}"{description}, curve={height=18pt}, from=6-1, to=6-2]
            \arrow[Rightarrow, no head, from=1-1, to=6-1]
            \arrow[""{name=1, anchor=center, inner sep=0}, "{\langle PX,\Delta\text{id}\rangle\times F(W,X)Y}"{description}, from=6-1, to=4-2]
            \arrow["{\mathbf r\text{-def}}"{description}, draw=none, from=0, to=1-2]
            \arrow["{\mathbf r\text{-def}}"{description}, draw=none, from=1, to=6-2]
          \end{tikzcd}
        \end{equation*}
        Postcomposition with the appropriate canonical morphisms yields
        (\ref{eq:strong_pseudomonad_square}).\footnote{
          Full diagram for proof of (\ref{eq:strong_pseudomonad_square}) in
          \href{https://q.uiver.app/?q=WzAsMTAsWzEsMSwiUFhcXHRpbWVzIEYoVyxYKVkiXSxbMiwxLCJQWFxcdGltZXNcXGludF57WCd9XFx0ZXh0e0hvbX0oWCcsWClcXHRpbWVzIEYoVyxYJylZIl0sWzIsMywiUFhcXHRpbWVzKFxcdGV4dHtIb219KFgsWClcXHRpbWVzIEYoVyxYKVkpIl0sWzIsNiwiXFxsZWZ0KFxcaW50XlggUFhcXHRpbWVzXFx0ZXh0e0hvbX0oWCcsWClcXHJpZ2h0KVxcdGltZXMgRihXLFgpWSJdLFsyLDQsIihQWFxcdGltZXNcXHRleHR7SG9tfShYLFgpKVxcdGltZXMgRihXLFgpWSJdLFsxLDYsIlBYXFx0aW1lcyBGKFcsWClZIl0sWzMsNywiXFxpbnRee1gnfSBcXGxlZnQoXFxpbnReWCBQWFxcdGltZXNcXHRleHR7SG9tfShYJyxYKVxccmlnaHQpXFx0aW1lcyBGKFcsWCcpWSJdLFszLDAsIlxcaW50XlhQWFxcdGltZXNcXGludF57WCd9XFx0ZXh0e0hvbX0oWCcsWClcXHRpbWVzIEYoVyxYJylZIl0sWzAsMCwiXFxpbnReWCBQWFxcdGltZXMgRihXLFgpWSJdLFswLDcsIlxcaW50XlggUFhcXHRpbWVzIEYoVyxYKVkiXSxbMiwxLCJQWFxcdGltZXMgcSIsMSx7ImN1cnZlIjo1fV0sWzAsMSwiUFhcXHRpbWVzXFxtYXRoYmYgciIsMSx7ImN1cnZlIjotM31dLFswLDIsIlBYXFx0aW1lc1xcbGFuZ2xlXFxEZWx0YSBYLEYoVyxYKVlcXHJhbmdsZSIsMV0sWzQsMywicVxcdGltZXMgRihXLFgpWSIsMSx7ImN1cnZlIjotNX1dLFsyLDQsIlxcYWxwaGEiLDFdLFs1LDMsIlxcbWF0aGJmIGxcXHRpbWVzIEYoVyxYJylZIiwxLHsiY3VydmUiOjN9XSxbMCw1LCIiLDEseyJsZXZlbCI6Miwic3R5bGUiOnsiaGVhZCI6eyJuYW1lIjoibm9uZSJ9fX1dLFs1LDQsIlxcbGFuZ2xlIFBYLFxcRGVsdGEgWFxccmFuZ2xlXFx0aW1lcyBGKFcsWClZIiwxXSxbMyw2LCJxIiwxXSxbNyw2LCJcXG1hdGhiZiBjIiwxXSxbMSw3LCJxIiwxXSxbOCw3LCJcXGludF5YUFhcXHRpbWVzXFxtYXRoYmYgciIsMV0sWzAsOCwicSIsMV0sWzYsOSwiXFxpbnRee1gnfVxcbWF0aGJmIGxeey0xfVxcdGltZXMgRihXLFgnKSIsMV0sWzgsOSwiIiwwLHsibGV2ZWwiOjIsInN0eWxlIjp7ImhlYWQiOnsibmFtZSI6Im5vbmUifX19XSxbNSw5LCJxIiwxXV0=}
        {quiver}.}
      \item The proof of (\ref{eq:strong_pseudomonad_house}) is similar to the above.\footnote{
          Full diagram for proof of (\ref{eq:strong_pseudomonad_house}) in \href{https://q.uiver.app/?q=WzAsMjAsWzAsMCwiXFxpbnReVyBQV1xcdGltZXMgXFxpbnReWChGVylYXFx0aW1lcyBcXGludF5ZIChHWClZXFx0aW1lcyBIKEpJVixZKVoiXSxbNywwLCJcXGludF5XIFBXXFx0aW1lcyBcXGludF5ZXFxsZWZ0KFxcaW50XlgoRlcpWFxcdGltZXMgKEdYKVlcXHJpZ2h0KVxcdGltZXMgSChKSVYsWSlaIl0sWzcsNSwiXFxpbnReWVxcbGVmdChcXGludF5XIFBXXFx0aW1lc1xcbGVmdChcXGludF5YKEZXKVhcXHRpbWVzIChHWClZXFxyaWdodClcXHJpZ2h0KVxcdGltZXMgSChKSVYsWSlaIl0sWzcsMTAsIlxcaW50XllcXGxlZnQoXFxpbnReWFxcbGVmdChcXGludF5XIFBXXFx0aW1lcyhGVylYXFxyaWdodClcXHRpbWVzIChHWClZXFxyaWdodClcXHRpbWVzIEgoSklWLFkpWiJdLFsxLDEsIlBXXFx0aW1lc1xcaW50XlgoRlcpWFxcdGltZXNcXGludF5ZKEdYKVlcXHRpbWVzIEgoSklWLFkpWiJdLFsyLDIsIlBXXFx0aW1lcyBcXGxlZnQoKEZXKVhcXHRpbWVzIFxcaW50XlkoR1gpWVxcdGltZXMgSChKSVYsWSlaXFxyaWdodCkiXSxbMCwxMCwiXFxpbnReWFxcbGVmdChcXGludF5XIFBXXFx0aW1lcyAoRlcpWFxccmlnaHQpXFx0aW1lcyBcXGludF5ZIChHWClZXFx0aW1lcyBIKEpJVixZKVoiXSxbMSw5LCJcXGxlZnQoXFxpbnReVyBQV1xcdGltZXMgKEZXKVhcXHJpZ2h0KVxcdGltZXMgXFxpbnReWSAoR1gpWVxcdGltZXMgSChKSVYsWSlaIl0sWzIsOCwiXFxsZWZ0KFBXXFx0aW1lcyAoRlcpWFxccmlnaHQpXFx0aW1lc1xcaW50XlkoR1gpWVxcdGltZXMgSChKSVYsWSlaIl0sWzQsMywiUFdcXHRpbWVzXFxsZWZ0KFxcbGVmdCgoRldYXFx0aW1lcyAoR1gpWVxccmlnaHQpXFx0aW1lcyBIKEpJVixZKVpcXHJpZ2h0KSJdLFs2LDEsIlBXXFx0aW1lcyBcXGludF5ZXFxsZWZ0KFxcaW50XlgoRlcpWFxcdGltZXMgKEdYKVlcXHJpZ2h0KVxcdGltZXMgSChKSVYsWSlaIl0sWzUsMiwiUFdcXHRpbWVzIFxcbGVmdChcXGxlZnQoXFxpbnReWChGVylYXFx0aW1lcyAoR1gpWVxccmlnaHQpXFx0aW1lcyBIKEpJVixZKVpcXHJpZ2h0KSJdLFszLDMsIlBXXFx0aW1lc1xcbGVmdCgoRlcpWFxcdGltZXNcXGxlZnQoKEdYKVlcXHRpbWVzIEgoSklWLFkpWlxccmlnaHQpXFxyaWdodCkiXSxbNiw5LCJcXGxlZnQoXFxpbnReWFxcbGVmdChcXGludF5XIFBXXFx0aW1lcyhGVylYXFxyaWdodClcXHRpbWVzIChHWClZXFxyaWdodClcXHRpbWVzIEgoSklWLFkpWiJdLFs1LDgsIlxcbGVmdChcXGxlZnQoXFxpbnReVyBQV1xcdGltZXMoRlcpWFxccmlnaHQpXFx0aW1lcyAoR1gpWVxccmlnaHQpXFx0aW1lcyBIKEpJVixZKVoiXSxbNCw3LCJcXGxlZnQoXFxsZWZ0KFBXXFx0aW1lcyhGVylYXFxyaWdodClcXHRpbWVzIChHWClZXFxyaWdodClcXHRpbWVzIEgoSklWLFkpWiJdLFs2LDUsIlxcbGVmdChcXGludF5XIFBXXFx0aW1lc1xcbGVmdChcXGludF5YKEZXKVhcXHRpbWVzIChHWClZXFxyaWdodClcXHJpZ2h0KVxcdGltZXMgSChKSVYsWSlaIl0sWzUsNSwiXFxsZWZ0KFBXXFx0aW1lc1xcbGVmdChcXGludF5YKEZXKVhcXHRpbWVzIChHWClZXFxyaWdodClcXHJpZ2h0KVxcdGltZXMgSChKSVYsWSlaIl0sWzQsNSwiXFxsZWZ0KFBXXFx0aW1lc1xcbGVmdCgoRlcpWFxcdGltZXMgKEdYKVlcXHJpZ2h0KVxccmlnaHQpXFx0aW1lcyBIKEpJVixZKVoiXSxbMyw3LCIoUFdcXHRpbWVzIChGVylYKVxcdGltZXMoKEdYKVlcXHRpbWVzIEgoSklWLFkpWikiXSxbMCwxLCJcXGludF5XIFBXXFx0aW1lc1xcbWF0aGJmIGMiLDFdLFsxLDIsIlxcbWF0aGJmIGMiLDFdLFsyLDMsIlxcaW50XllcXG1hdGhiZiBjXFx0aW1lcyBIKEpJVixZKVoiLDFdLFs0LDAsInEiXSxbNSw0LCJQV1xcdGltZXMgcSJdLFs2LDMsIlxcaW50XllcXG1hdGhiZiBjXFx0aW1lcyBIKEpJVixZKVoiLDFdLFswLDYsIlxcbWF0aGJmIGMiLDFdLFs3LDYsInEiXSxbOCw3LCJxIl0sWzUsOCwiXFxjb25nIiwxXSxbMTAsMSwicSIsMV0sWzExLDEwLCJQV1xcdGltZXMgcSIsMV0sWzksMTEsIlBXXFx0aW1lcyAocVxcdGltZXMgSChKSVYsWSlaKSIsMV0sWzQsMTAsIlBXXFx0aW1lc1xcbWF0aGJmIGMiLDFdLFsxMiw1LCJQV1xcdGltZXMoKEZXKVhcXHRpbWVzIHEpIiwxXSxbMTIsOSwiXFxjb25nIiwxLHsiY3VydmUiOi0yfV0sWzE1LDE0LCIocVxcdGltZXMoR1gpWSlcXHRpbWVzIEgoSklWLFkpWiIsMV0sWzE0LDEzLCJxXFx0aW1lcyBIKEpJVixZKVoiLDFdLFsxMywzLCJxIiwxXSxbMTgsMTcsIihQV1xcdGltZXMgcSlcXHRpbWVzIEgoSklWLFkpWiIsMSx7ImN1cnZlIjotM31dLFsxNywxNiwicVxcdGltZXMgSChKSVYsWSlaIiwxLHsiY3VydmUiOi0zfV0sWzE2LDIsInEiLDFdLFsxOCwxNSwiXFxjb25nIiwxXSxbMTYsMTMsIlxcbWF0aGJmIGMiLDFdLFsxMSwxNywiXFxjb25nIiwxXSxbMTksOCwiKFBXXFx0aW1lcyAoRlcpWClcXHRpbWVzIHEiLDFdLFsxOSwxNSwiXFxjb25nIiwxLHsiY3VydmUiOi0yfV0sWzEyLDE5LCJcXGNvbmciLDFdLFs5LDE4LCJcXGNvbmciLDFdLFs3LDEzLCJcXG1hdGhiZiBjIiwxXV0=}
        {quiver}.}
    \end{enumerate}
  \end{proof}
\end{proposition}

Only a little more work is required to make sure that a prestrong $J$-pseudomonad
does indeed correspond to a relative pseudomonad over $J$.

\begin{theorem}\label{thm:prestrong_inclusion_pseudomonads_are_relative_pseudomonads}
  Let $T$ be a strong $J$-pseudomonad. Then the relative pseudomonad structure induced
  by $T$ is a relative pseudomonad in the sense of~\cite[Definition 3.1]{fiore2017}.
  \begin{proof}
    We have the obvious correspondence between the structure in~\cite{fiore2017} and
    \ref{def:induced_relative_pseudomonad_structure}:
    $\extend{\rr{-}}{*}_{X,Y}$ is identical,
    $i_X$ is $\eta_X$,
    $\mu_{g,f}$ is $\bicell m_{f,g}$,
    $\eta_f$ is $\bicell e_f$,
    $\theta_X$ is $\bicell t_X$.
    We now verify the axioms:
    \begin{itemize}
      \item naturality of $\bicell e$ is just naturality of $\bicell r$;
      \item naturality of $\bicell m$ is just naturality of $\bicell c$;
      \item \cite[{(3.2)}]{fiore2017} holds by postcomposing (\ref{eq:strong_pseudomonad_square})
        with $\inv\lambda$ as in the diagram
        \begin{equation*}
          % https://q.uiver.app/?q=WzAsOCxbMCwwLCJmXioiXSxbNCwwLCIoZl4qXFxldGEpXioiXSxbNCwzLCJmXipcXGV0YV4qIl0sWzAsMywiZl4qIl0sWzEsMSwiKGZcXGxhbWJkYSleXFxkYWdnZXIiXSxbMywxLCIoKGZcXGxhbWJkYSleXFxkYWdnZXIoMVxcdGltZXNcXGV0YSkpXlxcZGFnZ2VyIl0sWzMsMiwiKGZcXGxhbWJkYSleXFxkYWdnZXIoMVxcdGltZXMoXFxldGFcXGxhbWJkYSleXFxkYWdnZXIpIl0sWzEsMiwiKGZcXGxhbWJkYSleXFxkYWdnZXIiXSxbMCwxLCIoXFxtYXRoYmYgZV9mKV4qIl0sWzEsMiwiXFxtYXRoYmYgbSJdLFswLDMsIiIsMix7ImxldmVsIjoyLCJzdHlsZSI6eyJoZWFkIjp7Im5hbWUiOiJub25lIn19fV0sWzIsMywiZl4qXFxtYXRoYmYgdCJdLFs3LDMsIi1cXGNpcmNcXGxhbWJkYV57LTF9IiwxXSxbMCw0LCItXFxjaXJjXFxsYW1iZGEiLDFdLFsxLDUsIi1cXGNpcmNcXGxhbWJkYSIsMV0sWzIsNiwiLVxcY2lyY1xcbGFtYmRhIiwxXSxbNSw2LCJcXG1hdGhiZiBjIl0sWzQsNSwiXFxtYXRoYmYgciJdLFs2LDcsIihmXFxsYW1iZGEpXlxcZGFnZ2VyKDFcXHRpbWVzXFxtYXRoYmYgbCkiXSxbNCw3LCIiLDEseyJsZXZlbCI6Miwic3R5bGUiOnsiaGVhZCI6eyJuYW1lIjoibm9uZSJ9fX1dXQ==
          \begin{tikzcd}
            {f^*} &&&& {(f^*\eta)^*} \\
                  & {(f\lambda)^\dagger} && {((f\lambda)^\dagger(1\times\eta))^\dagger} \\
                  & {(f\lambda)^\dagger} && {(f\lambda)^\dagger(1\times(\eta\lambda)^\dagger)} \\
            {f^*} &&&& {f^*\eta^*}
            \arrow["{(\mathbf e_f)^*}", from=1-1, to=1-5]
            \arrow["{\mathbf m}", from=1-5, to=4-5]
            \arrow[Rightarrow, no head, from=1-1, to=4-1]
            \arrow["{f^*\mathbf t}", from=4-5, to=4-1]
            \arrow["{-\circ\lambda^{-1}}"{description}, from=3-2, to=4-1]
            \arrow["{-\circ\lambda}"{description}, from=1-1, to=2-2]
            \arrow["{-\circ\lambda}"{description}, from=1-5, to=2-4]
            \arrow["{-\circ\lambda}"{description}, from=4-5, to=3-4]
            \arrow["{\mathbf c}", from=2-4, to=3-4]
            \arrow["{\mathbf r}", from=2-2, to=2-4]
            \arrow["{(f\lambda)^\dagger(1\times\mathbf l)}", from=3-4, to=3-2]
            \arrow[Rightarrow, no head, from=2-2, to=3-2]
          \end{tikzcd}
        \end{equation*}
      \item \cite[{(3.1)}]{fiore2017} holds by postcomposing (\ref{eq:strong_pseudomonad_house}) with
        $\lambda^{-1}$ just like above.
    \end{itemize}
  \end{proof}
\end{theorem}

Thus we may conclude that the definitions that we have stated so far fit in
with already established work.

\section{Induced pseudofunctor}

Before we move on to the strong structure of a $J$-pseudomonad, we take some time to
appreciate what we have developed so far. Relative monads induce functors~\cite{altenkirch2015}.
It is therefore desirable that relative pseudomonads induce pseudofunctors. This was
not explicitly proven in~\cite{fiore2017}. We will show how to obtain the pseudofunctor
induced by a prestrong $J$-pseudomonad.

Fix a prestrong $J$-pseudomonad $T$. Analogous to \ref{def:pseudofunctor}, we define the following:

\begin{definition}
  The \emph{pseudofunctor structure induced by $T$} consists of
  \begin{enumerate}
    \item for all $X\in\bicat{J}$, the object $TX\in\bicat{C}$;
    \item for all $X,Y\in\bicat{J}$, the functor $T_{X,Y}$ is the composite
      \begin{equation}
        % https://q.uiver.app/?q=WzAsNCxbMCwwLCJcXHRleHR7SG9tfVtYLFldIl0sWzAsMSwiXFx0ZXh0e0hvbX1bSlgsSlldIl0sWzIsMSwiXFx0ZXh0e0hvbX1bSlgsVFldIl0sWzIsMCwiXFx0ZXh0e0hvbX1bVFgsVFldIl0sWzIsMywiKC0pXntUXFxsYW1iZGF9IiwyXSxbMSwyLCJcXGV0YVxcY2lyYy0iLDJdLFswLDEsIkpfe1gsWX0iLDJdLFswLDMsIlRfe1gsWX0iXV0=
        \begin{tikzcd}
          {\text{Hom}[X,Y]} && {\text{Hom}[TX,TY]} \\
          {\text{Hom}[JX,JY]} && {\text{Hom}[JX,TY]}
          \arrow["{(-)^{T\lambda}}"', from=2-3, to=1-3]
          \arrow["{\eta\circ-}"', from=2-1, to=2-3]
          \arrow["{J_{X,Y}}"', from=1-1, to=2-1]
          \arrow["{T_{X,Y}}", from=1-1, to=1-3]
        \end{tikzcd}
      \end{equation}
    \item for all $X\in\bicat{J}$, the 2-cell $\bicell i_X = \inv{\bicell l_X}\inv\lambda$;
    \item for all $f:X\to Y$ and $g:Y\to Z$ in $\bicat{J}$, the 2-cell
      \begin{align*}
        \bicell d_{f,g} = \bicell m_{\eta Jf,\eta Jg}\bullet\extend{\rr{\bicell e_{\eta Jg}Jf}}{T\lambda}
      \end{align*}
      as in the diagram
      \begin{equation}
        % https://q.uiver.app/?q=WzAsNCxbMCwwLCJUWCJdLFs3LDAsIlRZIl0sWzcsMiwiVFoiXSxbMCwyLCJUWiJdLFswLDEsIihcXGV0YSBKZileKiIsMV0sWzEsMiwiKFxcZXRhIEpnKV4qIiwxXSxbMCwyLCIoKFxcZXRhIEpnKV4qXFxldGEgSmYpXioiLDFdLFszLDIsIiIsMSx7ImxldmVsIjoyLCJzdHlsZSI6eyJoZWFkIjp7Im5hbWUiOiJub25lIn19fV0sWzAsMywiKFxcZXRhIEpnSmYpXioiLDFdLFswLDEsIlRmIiwwLHsiY3VydmUiOi01fV0sWzEsMiwiVGciLDAseyJjdXJ2ZSI6LTV9XSxbMCwzLCJUKGdmKSIsMix7ImN1cnZlIjo1fV0sWzYsMSwiXFxtYXRoYmYgbSIsMix7InNob3J0ZW4iOnsic291cmNlIjozMCwidGFyZ2V0IjozMH19XSxbMyw2LCIoXFxtYXRoYmYgZUpmKV4qIiwwLHsic2hvcnRlbiI6eyJzb3VyY2UiOjMwLCJ0YXJnZXQiOjMwfX1dXQ==
        \begin{tikzcd}
          TX &&&&&&& TY \\
          \\
          TZ &&&&&&& TZ
          \arrow["{(\eta Jf)^*}"{description}, from=1-1, to=1-8]
          \arrow["{(\eta Jg)^*}"{description}, from=1-8, to=3-8]
          \arrow[""{name=0, anchor=center, inner sep=0}, "{((\eta Jg)^*\eta Jf)^*}"{description}, from=1-1, to=3-8]
          \arrow[Rightarrow, no head, from=3-1, to=3-8]
          \arrow["{(\eta JgJf)^*}"{description}, from=1-1, to=3-1]
          \arrow["Tf", curve={height=-30pt}, from=1-1, to=1-8]
          \arrow["Tg", curve={height=-30pt}, from=1-8, to=3-8]
          \arrow["{T(gf)}"', curve={height=30pt}, from=1-1, to=3-1]
          \arrow["{\mathbf m}"', shorten <=31pt, shorten >=31pt, Rightarrow, from=0, to=1-8]
          \arrow["{(\mathbf eJf)^*}", shorten <=31pt, shorten >=31pt, Rightarrow, from=3-1, to=0]
        \end{tikzcd}
      \end{equation}
  \end{enumerate}
\end{definition}

\begin{example}
  The prestrong presheaf construction induces a pseudofunctor structure
  $\widehat{-} : \biCat\to\biCAT$ as described in \ref{ex:presheaf_pseudofunctor}.
\end{example}

Similar to the induced relative pseudomonad structure, we can prove a general
statement and no additional work is required to show that the presheaf
construction induces a pseudofunctor:

\begin{proposition}\label{prop:induced_pseudofunctor}
  The pseudofunctor structure induced by $T$ is a pseudofunctor.
  \begin{proof}
    We verify the axioms:
    \begin{enumerate}
      \item We have the commuting diagram
        \begin{equation}
          % https://q.uiver.app/?q=WzAsMTAsWzAsMCwiVChoZ2YpIl0sWzAsMiwiVChoZylUZiJdLFswLDUsIlRoIFRnIFRmIl0sWzIsMCwiVGggVChnZikiXSxbMCw0LCIoVGhcXGV0YSBKZyleKlRmIl0sWzAsMSwiKFQoaGcpXFxldGEgSmYpXntUXFxsYW1iZGF9Il0sWzEsMCwiKFRoXFxldGEgSihnZikpXioiXSxbMiw1LCJUaCgoXFxldGEgSmcpXipcXGV0YSBKZileKiJdLFsxLDMsIigoXFxldGEgSmgpXiooXFxldGEgSmcpXipcXGV0YSBKZileKiJdLFsxLDEsIigoVGhcXGV0YSBKZyleKlxcZXRhIEpmKV4qIl0sWzEsNCwiKFxcbWF0aGJmIGUgSmcpXipUZiIsMl0sWzQsMiwiXFxtYXRoYmYgbVRmIiwyXSxbMCw1LCIoXFxtYXRoYmYgZUpmKV4qIiwyXSxbNSwxLCJcXG1hdGhiZiBtIiwyXSxbMCw2LCIoXFxtYXRoYmYgZUooZ2YpKV4qIl0sWzYsMywiXFxtYXRoYmYgbSJdLFszLDcsIlRoKFxcbWF0aGJmIGVKZileKiIsMV0sWzcsMiwiKFxcZXRhIEpoKV4qXFxtYXRoYmYgbSJdLFs2LDgsIigoXFxldGEgSmgpXipcXG1hdGhiZiBlIEpmKV4qIiwxLHsiY3VydmUiOi01fV0sWzgsNywiXFxtYXRoYmYgbSJdLFs5LDQsIlxcbWF0aGJmIG0iLDJdLFs1LDksIigoXFxtYXRoYmYgZUpnKV4qIFxcZXRhIEpmKV4qIiwxLHsiY3VydmUiOjN9XSxbOSw4LCIoXFxtYXRoYmYgbVxcZXRhIEpmKV4qIiwxXSxbNiw5LCIoXFxtYXRoYmYgZSBKZileKiIsMl0sWzIsOCwiKFxccmVme2VxOnN0cm9uZ19wc2V1ZG9tb25hZF9ob3VzZX0pIiwxLHsic3R5bGUiOnsiYm9keSI6eyJuYW1lIjoibm9uZSJ9LCJoZWFkIjp7Im5hbWUiOiJub25lIn19fV0sWzEyLDIzLCJcXG1hdGhiZiBlXFx0ZXh0ey1uYXR9IiwyLHsic2hvcnRlbiI6eyJzb3VyY2UiOjIwLCJ0YXJnZXQiOjIwfSwic3R5bGUiOnsiYm9keSI6eyJuYW1lIjoibm9uZSJ9LCJoZWFkIjp7Im5hbWUiOiJub25lIn19fV0sWzQsMjEsIlxcbWF0aGJmIG5cXHRleHR7LW5hdH0iLDEseyJzaG9ydGVuIjp7InRhcmdldCI6MjB9LCJzdHlsZSI6eyJib2R5Ijp7Im5hbWUiOiJub25lIn0sImhlYWQiOnsibmFtZSI6Im5vbmUifX19XSxbMTUsMTYsIlxcbWF0aGJmIG1cXHRleHR7LW5hdH0iLDEseyJzaG9ydGVuIjp7InNvdXJjZSI6MjAsInRhcmdldCI6MjB9LCJzdHlsZSI6eyJib2R5Ijp7Im5hbWUiOiJub25lIn0sImhlYWQiOnsibmFtZSI6Im5vbmUifX19XV0=
          \begin{tikzcd}
            {T(hgf)} & {(Th\eta J(gf))^*} & {Th T(gf)} \\
            {(T(hg)\eta Jf)^{T\lambda}} & {((Th\eta Jg)^*\eta Jf)^*} \\
            {T(hg)Tf} \\
                                        & {((\eta Jh)^*(\eta Jg)^*\eta Jf)^*} \\
                                        {(Th\eta Jg)^*Tf} \\
            {Th Tg Tf} && {Th((\eta Jg)^*\eta Jf)^*}
            \arrow["{(\mathbf e Jg)^*Tf}"', from=3-1, to=5-1]
            \arrow["{\mathbf mTf}"', from=5-1, to=6-1]
            \arrow[""{name=0, anchor=center, inner sep=0}, "{(\mathbf eJf)^*}"', from=1-1, to=2-1]
            \arrow["{\mathbf m}"', from=2-1, to=3-1]
            \arrow["{(\mathbf eJ(gf))^*}", from=1-1, to=1-2]
            \arrow[""{name=1, anchor=center, inner sep=0}, "{\mathbf m}", from=1-2, to=1-3]
            \arrow[""{name=2, anchor=center, inner sep=0}, "{Th(\mathbf eJf)^*}"{description}, from=1-3, to=6-3]
            \arrow["{(\eta Jh)^*\mathbf m}", from=6-3, to=6-1]
            \arrow["{((\eta Jh)^*\mathbf e Jf)^*}"{description}, curve={height=-70pt}, from=1-2, to=4-2]
            \arrow["{\mathbf m}", from=4-2, to=6-3]
            \arrow["{\mathbf m}"', from=2-2, to=5-1]
            \arrow[""{name=3, anchor=center, inner sep=0}, "{((\mathbf eJg)^* \eta Jf)^*}"{description}, curve={height=18pt}, from=2-1, to=2-2]
            \arrow["{(\mathbf m\eta Jf)^*}"{description}, from=2-2, to=4-2]
            \arrow[""{name=4, anchor=center, inner sep=0}, "{(\mathbf e Jf)^*}"', from=1-2, to=2-2]
            \arrow["{(\ref{eq:strong_pseudomonad_house})}"{description}, draw=none, from=6-1, to=4-2]
            \arrow["{\mathbf e\text{-nat}}"', draw=none, from=0, to=4]
            \arrow["{\mathbf n\text{-nat}}"{description}, draw=none, from=5-1, to=3]
            \arrow["{\mathbf m\text{-nat}}"{description}, draw=none, from=1, to=2]
          \end{tikzcd}
        \end{equation}
        where the remaining face follows from~\cite[Lemma 3.2 (i)]{fiore2017}.
        We have shown \ref{eq:pseudofunctor_coherence_associativity}.
      \item The conditions \ref{eq:pseudofunctor_coherence_identity} are just~\cite[(3.1)]{fiore2017}.
    \end{enumerate}
  \end{proof}
\end{proposition}

\section{Strong structure}

We are now ready for our final step towards the definition of a strong relative
pseudomonad. As we are not be able to prove all the results that we would have
liked, we provide some additional insight into how one might come up with the
definition of a strong $J$-pseudomonad structure.

Firstly, we notice that the prestrong structure already gives rise to a
1-cell $X\times TY\to T(X\times Y)$ by extending the unit $\eta_{X\times Y}$. We will
therefore think of $\eta^\dagger$ as the strength. Secondly, we observe that of the four
structural 2-cells in~\cite[Definitions 8 and 9]{saville2023} only one involves repeated
applications of the object map. The others are thus easily translated. To avoid the
repeated applications in the problematic case, we take inspiration from the definition
of a strong relative monad in~\cite{tarmo}. The result is the rather unintuitive
family of invertible 2-cells $\bicell q$ which will allow us to construct the usual
pentagon in the case where $J$ is the identity.

\begin{definition}\label{def:strong_inclusion_pseudomonad_structure}
  A \emph{strong $J$-pseudomonad structure} consists of
  \begin{enumerate}
    \item a prestrong $J$-pseudomonad $T$;
    \item for all $X,Y,Z\in\bicat{J}$, an invertible 2-cell
      \begin{equation}\label{eq:strong_pseudomonad_p}
        % https://q.uiver.app/?q=WzAsNSxbMCwwLCIoSlhcXHRpbWVzIEpZKVxcdGltZXMgVFoiXSxbMCwxLCJKWFxcdGltZXMoSllcXHRpbWVzIFRaKSJdLFsyLDAsIlQoKFhcXHRpbWVzIFkpXFx0aW1lcyBaKSJdLFsyLDEsIlQoWFxcdGltZXMoWVxcdGltZXMgWikpIl0sWzEsMSwiSlhcXHRpbWVzIFQoWVxcdGltZXMgWikiXSxbMSw0LCJcXGV0YV5cXGRhZ2dlciIsMl0sWzQsMywiXFxldGFeXFxkYWdnZXIiLDJdLFswLDEsIlxcYWxwaGEiLDJdLFswLDIsIlxcZXRhXlxcZGFnZ2VyIl0sWzIsMywiVFxcYWxwaGEiXSxbNyw5LCJcXG1hdGhiZiBwX3tYLFksWn0iLDEseyJzaG9ydGVuIjp7InNvdXJjZSI6MjAsInRhcmdldCI6MjB9fV1d
        \begin{tikzcd}
          {(JX\times JY)\times TZ} && {T((X\times Y)\times Z)} \\
          {JX\times(JY\times TZ)} & {JX\times T(Y\times Z)} & {T(X\times(Y\times Z))}
          \arrow["{\eta^\dagger}"', from=2-1, to=2-2]
          \arrow["{\eta^\dagger}"', from=2-2, to=2-3]
          \arrow[""{name=0, anchor=center, inner sep=0}, "\alpha"', from=1-1, to=2-1]
          \arrow["{\eta^\dagger}", from=1-1, to=1-3]
          \arrow[""{name=1, anchor=center, inner sep=0}, "T\alpha", from=1-3, to=2-3]
          \arrow["{\mathbf p_{X,Y,Z}}"{description}, shorten <=29pt, shorten >=29pt, Rightarrow, from=0, to=1]
        \end{tikzcd}
      \end{equation}
    \item for all invertible 2-cells
      \begin{equation*}
        % https://q.uiver.app/?q=WzAsMyxbMCwwLCJKV1xcdGltZXMgSlgiXSxbMiwwLCJKV1xcdGltZXMgVFkiXSxbMSwxLCJUKFdcXHRpbWVzIFkpIl0sWzEsMiwiXFxldGFeXFxkYWdnZXIiXSxbMCwxLCJKV1xcdGltZXMgZiJdLFswLDIsImciLDJdLFs1LDMsIlxcbWF0aGJmIHUiLDAseyJzaG9ydGVuIjp7InNvdXJjZSI6MjAsInRhcmdldCI6MjB9fV1d
        \begin{tikzcd}
          {JW\times JX} && {JW\times TY} \\
                        & {T(W\times Y)}
                        \arrow[""{name=0, anchor=center, inner sep=0}, "{\eta^\dagger}", from=1-3, to=2-2]
                        \arrow["{JW\times f}", from=1-1, to=1-3]
                        \arrow[""{name=1, anchor=center, inner sep=0}, "g"', from=1-1, to=2-2]
                        \arrow["{\mathbf u}", shorten <=10pt, shorten >=10pt, Rightarrow, from=1, to=0]
        \end{tikzcd}
      \end{equation*}
      an invertible 2-cell
      \begin{equation}\label{eq:strong_pseudomonad_q}
        % https://q.uiver.app/?q=WzAsNCxbMCwwLCJKV1xcdGltZXMgVFgiXSxbMywwLCJUKFdcXHRpbWVzIFgpIl0sWzAsMSwiSldcXHRpbWVzIFRZIl0sWzMsMSwiVChXXFx0aW1lcyBZKSJdLFswLDEsIlxcZXRhXlxcZGFnZ2VyIl0sWzIsMywiXFxldGFeXFxkYWdnZXIiLDJdLFsxLDMsImdeKiJdLFswLDIsIkpXXFx0aW1lcyBmXioiLDJdLFs3LDYsIlxcbWF0aGJmIHFfe1xcbWF0aGJmIHV9IiwxLHsic2hvcnRlbiI6eyJzb3VyY2UiOjIwLCJ0YXJnZXQiOjIwfX1dXQ==
        \begin{tikzcd}
          {JW\times TX} &&& {T(W\times X)} \\
          {JW\times TY} &&& {T(W\times Y)}
          \arrow["{\eta^\dagger}", from=1-1, to=1-4]
          \arrow["{\eta^\dagger}"', from=2-1, to=2-4]
          \arrow[""{name=0, anchor=center, inner sep=0}, "{g^*}", from=1-4, to=2-4]
          \arrow[""{name=1, anchor=center, inner sep=0}, "{JW\times f^*}"', from=1-1, to=2-1]
          \arrow["{\mathbf q_{\mathbf u}}"{description}, shorten <=23pt, shorten >=23pt, Rightarrow, from=1, to=0]
        \end{tikzcd}
      \end{equation}
    \item for all $f:JW\times JX\to TY$ in $\bicat{C}$, an invertible 2-cell
      \begin{equation}\label{eq:strong_pseudomonad_s}
        % https://q.uiver.app/?q=WzAsMyxbMCwwLCJKV1xcdGltZXMgVFgiXSxbMiwxLCJUWSJdLFs0LDAsIlQoV1xcdGltZXMgWCkiXSxbMCwyLCJcXGV0YV5cXGRhZ2dlciJdLFsyLDEsImZeKiJdLFswLDEsImZeXFxkYWdnZXIiLDJdLFs1LDQsIlxcbWF0aGJmIHNfZiIsMCx7InNob3J0ZW4iOnsic291cmNlIjoyMCwidGFyZ2V0IjoyMH19XV0=
        \begin{tikzcd}
          {JW\times TX} &&&& {T(W\times X)} \\
                        && TY
                        \arrow["{\eta^\dagger}", from=1-1, to=1-5]
                        \arrow[""{name=0, anchor=center, inner sep=0}, "{f^*}", from=1-5, to=2-3]
                        \arrow[""{name=1, anchor=center, inner sep=0}, "{f^\dagger}"', from=1-1, to=2-3]
                        \arrow["{\mathbf s_f}", shorten <=13pt, shorten >=13pt, Rightarrow, from=1, to=0]
        \end{tikzcd}
      \end{equation}
  \end{enumerate}
\end{definition}

Let us now verify that it is indeed possible to obtain such a structure for the
presheaf construction. Of particular interest to us is the structural 2-cells
$\bicell q_{\bicell u}$ because they are different to all the others that we
have seen so far.

\begin{example}
  We extend the prestrong presheaf construction as follows:
  \begin{enumerate}
    \item for all $X\in\scat X$, $Y\in\scat Y$, and $P\in\widehat{\scat Z}$,
      $\rr{\bicell p_{\scat X,\scat Y,\scat Z}}_{X,Y,P}$ is the isomorphism of coends
      \begin{align*}
        &\int^{X',Y',Z'}\rr{\int^Z PZ\times\Hom(((X',Y'),Z'),((X,Y),Z))} \times \Hom(-,(X',(Y',Z'))) \\
        &\cong \int^{Y',Z'}\rr{\int^Z PZ\times\Hom((Y',Z'),(Y,Z))} \times \Hom(-,(X,(Y',Z')))
      \end{align*}
      which may be obtained by composition of $\bicell l$ and $\bicell c$;\footnote{
        Construction of $\bicell p$ in \href{https://q.uiver.app/?q=WzAsMTEsWzQsMCwiXFxpbnRee1gnLFknLFonfVxcbGVmdChcXGludF5aIFBaXFx0aW1lc1xcdGV4dHtIb219KCgoWCcsWScpLFonKSwoKFgsWSksWikpXFxyaWdodClcXHRpbWVzXFx0ZXh0e0hvbX0oLSxYJywoWScsWicpKSJdLFs0LDYsIlxcaW50XntZJyxaJ31cXGxlZnQoXFxpbnReWiBQWlxcdGltZXNcXHRleHR7SG9tfSgoWScsWicpLChZLFopKVxccmlnaHQpXFx0aW1lc1xcdGV4dHtIb219KC0sKFgsKFknLFonKSkpIl0sWzIsMCwiXFxpbnRee1gnfVxcbGVmdChcXGludF5aIFBaXFx0aW1lc1xcdGV4dHtIb219KCgoWCcsWScpLFonKSwoKFgsWSksWikpXFxyaWdodClcXHRpbWVzXFx0ZXh0e0hvbX0oLSwoWCcsKFknLFonKSkpIl0sWzIsNiwiXFxsZWZ0KFxcaW50XlpQWlxcdGltZXNcXHRleHR7SG9tfSgoWScsWicpLChZLFopKVxccmlnaHQpXFx0aW1lc1xcdGV4dHtIb219KC0sKFgsKFknLFonKSkpIl0sWzIsMiwiXFxpbnReWiBQWlxcdGltZXNcXGludF57WCd9XFx0ZXh0e0hvbX0oKChYJyxZJyksWicpLCgoWCxZKSxaKSlcXHRpbWVzXFx0ZXh0e0hvbX0oLSwoWCcsKFknLFonKSkpIl0sWzAsNiwiXFxsZWZ0KFBaXFx0aW1lc1xcdGV4dHtIb219KChZJyxaJyksKFksWikpXFxyaWdodClcXHRpbWVzXFx0ZXh0e0hvbX0oLSwoWCwoWScsWicpKSkiXSxbMCwyLCJQWlxcdGltZXNcXGludF57WCd9XFx0ZXh0e0hvbX0oKChYJyxZJyksWicpLCgoWCxZKSxaKSlcXHRpbWVzXFx0ZXh0e0hvbX0oLSwoWCcsKFknLFonKSkpIl0sWzIsMywiXFxpbnReWiBQWlxcdGltZXNcXGludF57WCd9KFxcdGV4dHtIb219KChZJyxaJyksKFksWikpXFx0aW1lc1xcdGV4dHtIb219KC0sKFgnLChZJyxaJykpKSlcXHRpbWVzXFx0ZXh0e0hvbX0oWCcsWCkiXSxbMiw1LCJcXGludF5aUFpcXHRpbWVzKFxcdGV4dHtIb219KChZJyxaJyksKFksWikpXFx0aW1lc1xcdGV4dHtIb219KC0sKFgsKFknLFonKSkpKSJdLFswLDMsIlBaXFx0aW1lc1xcaW50XntYJ31cXHRleHR7SG9tfSgoWScsWicpLChZLFopKVxcdGltZXNcXHRleHR7SG9tfSgtLChYJywoWScsWicpKSlcXHRpbWVzXFx0ZXh0e0hvbX0oWCcsWCkiXSxbMCw1LCJQWlxcdGltZXMoXFx0ZXh0e0hvbX0oKFknLFonKSwoWSxaKSlcXHRpbWVzXFx0ZXh0e0hvbX0oLSwoWCwoWScsWicpKSkpIl0sWzAsMSwiXFxtYXRoYmYgcCJdLFszLDEsInEiLDJdLFsyLDAsInEiXSxbMiw0LCJcXG1hdGhiZiBjXnstMX0iLDJdLFs2LDQsInEiXSxbNSwzLCJxXFx0aW1lc1xcdGV4dHtIb219KC0sKFgsKFknLFonKSkpIiwyXSxbNCw3LCJcXGNvbmciLDFdLFs3LDgsIlxcaW50XlpQWlxcdGltZXNcXG1hdGhiZiBsXnstMX0iLDFdLFs4LDMsIlxcY29uZyIsMV0sWzEwLDgsInEiLDFdLFs5LDcsInEiLDFdLFsxMCw1LCJcXGNvbmciLDFdLFs5LDEwLCJQWlxcdGltZXNcXG1hdGhiZiBsXnstMX0iLDFdLFs2LDksIlxcY29uZyIsMV1d}
      {quiver}}
    \item for all natural isomorphisms with components
      \begin{align*}
        \bicell u_{W,X} : G(W,X)\cong \int^Y (FX)Y\times\Hom(-,(W,Y))
      \end{align*}
      we have the natural isomorphism $\bicell q_{\bicell u}$ whose components are themselves
      natural isomorphisms
      \begin{align*}
        &\int^Y \rr{\int^X PX\times(FX)Y}\times\Hom\rr{-,(W,Y)} \\
        &\cong \int^{W',X'}\rr{\int^X PX\times\Hom((W',X'),(W,X))}\times G(W',X')(-)
      \end{align*}
      which are obtained by composing $\mathbf c$, $\mathbf l$, and  $\mathbf u$;\footnote{
        Construction of $\bicell q$ in \href{
        https://q.uiver.app/?q=WzAsMTAsWzMsMCwiXFxpbnReWVxcbGVmdChcXGludF5YIFBYXFx0aW1lcyhGWClZXFxyaWdodClcXHRpbWVzXFx0ZXh0e0hvbX0oLSwoVyxZKSkiXSxbMyw0LCJcXGludF57VycsWCd9XFxsZWZ0KFxcaW50XlggUFhcXHRpbWVzXFx0ZXh0e0hvbX0oKFcnLFgnKSwoVyxYKSlcXHJpZ2h0KVxcdGltZXMgRyhXJyxYJykoLSkiXSxbMiwwLCJcXGludF5YIFBYIFxcdGltZXNcXGludF5ZKEZYKVlcXHRpbWVzXFx0ZXh0e0hvbX0oLSwoVyxZKSkiXSxbMiw0LCJcXGludF5YIFBYXFx0aW1lc1xcaW50XntXJyxYJ31cXHRleHR7SG9tfSgoVycsWCcpLChXLFgpKVxcdGltZXMgRyhXJyxYJykoLSkiXSxbMCwwLCJQWCBcXHRpbWVzXFxpbnReWShGWClZXFx0aW1lc1xcdGV4dHtIb219KC0sKFcsWSkpIl0sWzAsNCwiUFhcXHRpbWVzXFxpbnRee1cnLFgnfVxcdGV4dHtIb219KChXJyxYJyksKFcsWCkpXFx0aW1lcyBHKFcnLFgnKSgtKSJdLFswLDEsIlBYXFx0aW1lcyBHKFcsWCkoLSkiXSxbMCwzLCJQWFxcdGltZXNcXGludF57VycsWCd9RyhXJyxYJykoLSlcXHRpbWVzXFx0ZXh0e0hvbX0oKFcnLFgnKSwoVyxYKSkiXSxbMiwxLCJcXGludF5YUFhcXHRpbWVzIEcoVyxYKSgtKSJdLFsyLDMsIlxcaW50XlhQWFxcdGltZXNcXGludF57VycsWCd9RyhXJyxYJykoLSlcXHRpbWVzXFx0ZXh0e0hvbX0oKFcnLFgnKSwoVyxYKSkiXSxbMCwxLCIoXFxtYXRoYmYgcV97XFxtYXRoYmYgdX0pX3tXLFB9Il0sWzMsMSwiXFxtYXRoYmYgYyIsMV0sWzAsMiwiXFxtYXRoYmYgY157LTF9IiwxXSxbNSwzLCJxIiwxXSxbNCwyLCJxIiwxXSxbNCw2LCJQWFxcdGltZXNcXG1hdGhiZiB1IiwxXSxbNiw3LCJQWFxcdGltZXNcXG1hdGhiZiBsIiwxXSxbNyw1LCJQWFxcdGltZXNcXGludFxcZ2FtbWEiLDFdLFs2LDgsInEiLDFdLFs3LDksInEiLDFdLFs5LDMsIlxcaW50XlggUFhcXHRpbWVzIFxcaW50XFxnYW1tYSIsMV0sWzIsOCwiXFxpbnReWCBQWFxcdGltZXNcXG1hdGhiZiB1IiwxXSxbOCw5LCJcXGludF5YIFBYXFx0aW1lc1xcbWF0aGJmIGwiLDFdXQ==}
      {quiver}.}
    \item for all $F:\scat W\times\scat X\to \widehat{\scat Y}$, the natural isomorphism
      $\bicell s_F$ has as components further natural isomorphisms
      \begin{align*}
        &\int^X PX\times F(W,X)(-)\\
        &\cong \int^X PX\times \int^{W',X'} F(W',X')(-)\times\Hom((W',X'),(W,X)) \\
        &\cong \int^X PX\times \int^{W',X'} \Hom((W',X'),(W,X))\times F(W',X')(-) \\
        &\cong \int^{W',X'}\rr{\int^X PX\times\Hom((W',X'),(W,X))}\times F(W',X')(-).
      \end{align*}
  \end{enumerate}
\end{example}

\section{Towards an induced strong pseudomonad}\label{sec:induced_strong_pseudomonad}

The structure \ref{def:strong_inclusion_pseudomonad_structure} requires some further
axioms to be useful. Ideally, we would like for the structural 2-cells
to be coherent. However, proving such a result may be very difficult. A more attainable
goal is to show that, in the case where the inclusion is the identity, we obtain a strong
pseudomonad.

We are not able to state any further axioms or prove any of the results above. Instead
we are going to outline how one can obtain the structure of a strong pseudomonad.
We begin by constructing the strength and proceed by adding the structural 2-cells that
promote it to a strength of the induced pseudofunctor and subsequently the induced
pseudomonad.

Fix a strong $J$-pseudomonad structure $T$.

We have already discussed how to obtain the 1-cells $JW\times TX\to T(W\times X)$ that
resemble the components of a pseudonatural transformation. We are now able to extend this
structure to include the naturality 2-cell. Combining these, we obtain the structure
of a pseudonatural transformation.

\begin{definition}\label{def:induced_strength}
  The \emph{strength induced by $T$} consists of
  \begin{enumerate}
    \item for all $W,X\in\bicat{J}$, the 1-cell
      \begin{equation}\label{eq:strong_inclusion_pseudomonad_induced_strength}
        \sigma_{W,X} = \extend{\rr{\eta_{W\times X}}}{\dagger} : JW\times TX\to T(W\times X)
      \end{equation}
    \item for all $f:W\to W'$ and $g:X\to X'$ in $\bicat{J}$, the naturality 2-cell
      \begin{align*}
        \bicell n_{f,g} = \mathbf c\bullet(\mathbf rJ(f\times g))^\dagger\bullet\inv{\mathbf s}
      \end{align*}
      as in the diagram
      \begin{equation}
        % https://q.uiver.app/?q=WzAsNixbMCwyLCJKV1xcdGltZXMgVFgiXSxbMCw1LCJKVydcXHRpbWVzIFRYJyJdLFszLDAsIlQoV1xcdGltZXMgWCkiXSxbMywzLCJUKFcnXFx0aW1lcyBYJykiXSxbMCwwLCJKV1xcdGltZXMgVFgiXSxbMyw1LCJUKFcnXFx0aW1lcyBYJykiXSxbMiwzLCJUKGZcXHRpbWVzIGcpIiwxXSxbMCwxLCJKZlxcdGltZXMgVGciLDFdLFsxLDUsIlxcZXRhXlxcZGFnZ2VyIiwxXSxbMyw1LCIiLDEseyJsZXZlbCI6Miwic3R5bGUiOnsiaGVhZCI6eyJuYW1lIjoibm9uZSJ9fX1dLFs0LDIsIlxcZXRhXlxcZGFnZ2VyIiwxXSxbNCwzLCIoXFxldGEgKEpmXFx0aW1lcyBKZykpXlxcZGFnZ2VyIiwxXSxbMCw1LCIoXFxldGFeXFxkYWdnZXIgKEpmXFx0aW1lc1xcZXRhIEpnKSleXFxkYWdnZXIiLDFdLFs0LDAsIiIsMSx7ImxldmVsIjoyLCJzdHlsZSI6eyJoZWFkIjp7Im5hbWUiOiJub25lIn19fV0sWzExLDEyLCIoXFxtYXRoYmYgciAoSmZcXHRpbWVzIEpnKSleXFxkYWdnZXIiLDEseyJzaG9ydGVuIjp7InNvdXJjZSI6MjAsInRhcmdldCI6MjB9fV0sWzEyLDgsIlxcbWF0aGJmIGMiLDEseyJzaG9ydGVuIjp7InNvdXJjZSI6MjAsInRhcmdldCI6MjB9fV0sWzEwLDExLCJcXG1hdGhiZiBzXnstMX0iLDEseyJzaG9ydGVuIjp7InNvdXJjZSI6MzAsInRhcmdldCI6MzB9fV1d
        \begin{tikzcd}
          {JW\times TX} &&& {T(W\times X)} \\
          \\
          {JW\times TX} \\
                        &&& {T(W'\times X')} \\
                        \\
          {JW'\times TX'} &&& {T(W'\times X')}
          \arrow["{T(f\times g)}"{description}, from=1-4, to=4-4]
          \arrow["{Jf\times Tg}"{description}, from=3-1, to=6-1]
          \arrow[""{name=0, anchor=center, inner sep=0}, "{\eta^\dagger}"{description}, from=6-1, to=6-4]
          \arrow[Rightarrow, no head, from=4-4, to=6-4]
          \arrow[""{name=1, anchor=center, inner sep=0}, "{\eta^\dagger}"{description}, from=1-1, to=1-4]
          \arrow[""{name=2, anchor=center, inner sep=0}, "{(\eta (Jf\times Jg))^\dagger}"{description}, from=1-1, to=4-4]
          \arrow[""{name=3, anchor=center, inner sep=0}, "{(\eta^\dagger (Jf\times\eta Jg))^\dagger}"{description}, from=3-1, to=6-4]
          \arrow[Rightarrow, no head, from=1-1, to=3-1]
          \arrow["{(\mathbf r (Jf\times Jg))^\dagger}"{description}, shorten <=9pt, shorten >=9pt, Rightarrow, from=2, to=3]
          \arrow["{\mathbf c}"{description}, shorten <=6pt, shorten >=6pt, Rightarrow, from=3, to=0]
          \arrow["{\mathbf s^{-1}}"{description}, shorten <=10pt, shorten >=10pt, Rightarrow, from=1, to=2]
        \end{tikzcd}
      \end{equation}
  \end{enumerate}
\end{definition}

With the axioms we have included so far, we are not able to prove that this is indeed
a pseudonatural transformation. Further, we have not been able to identify any intuitive or obvious axioms that
would achieve this goal.\footnote{For future reference, \href{https://q.uiver.app/?q=WzAsMjEsWzAsMCwiVChnXFx0aW1lcyBnJylUKGZcXHRpbWVzIGYnKVxcc2lnbWEiXSxbMCw3LCJUKGdmXFx0aW1lcyBnJ2YnKVxcc2lnbWEiXSxbMTQsNywiXFxzaWdtYShKKGdmKVxcdGltZXMgVChnJ2YnKSkiXSxbMTQsMCwiXFxzaWdtYShKKGdmKVxcdGltZXMgVGcnVGYnKSJdLFs3LDAsIlQoZ1xcdGltZXMgZycpXFxzaWdtYShKZlxcdGltZXMgVGYnKSJdLFsxLDYsIihcXGV0YSBKKGdmXFx0aW1lcyBnJ2YnKSlee1RcXGxhbWJkYX1cXGV0YV5UIl0sWzEsMSwiKFxcZXRhIEooZ1xcdGltZXMgZycpKV57VFxcbGFtYmRhfShcXGV0YSBKKGZcXHRpbWVzIGYnKSlee1RcXGxhbWJkYX1cXGV0YV5UIl0sWzEzLDEsIlxcZXRhXlQoSihnZilcXHRpbWVzIChcXGV0YSBKZylee1RcXGxhbWJkYX0oXFxldGEgSmYpXntUXFxsYW1iZGF9KSJdLFsxMyw2LCJcXGV0YV5UKEooZ2YpXFx0aW1lcyhcXGV0YSBKKGcnZicpKV57VFxcbGFtYmRhfSkiXSxbNywxLCIoXFxldGEgSihnXFx0aW1lcyBnJykpXntUXFxsYW1iZGF9XFxldGFeVChKZlxcdGltZXMgKFxcZXRhIEpmJylee1RcXGxhbWJkYX0pIl0sWzUsNiwiKFxcZXRhIEooZ2ZcXHRpbWVzIGcnZicpKV57VH0iXSxbOSw2LCIoXFxldGFeVChKKGdmKVxcdGltZXMoXFxldGEgSihnJ2YnKSkpKV5UIl0sWzMsMSwiKFxcZXRhIEooZ1xcdGltZXMgZycpKV57VFxcbGFtYmRhfShcXGV0YSBKKGZcXHRpbWVzIGYnKSleVCJdLFs1LDEsIihcXGV0YSBKKGdcXHRpbWVzIGcnKSlee1RcXGxhbWJkYX0oXFxldGFeVChKZlxcdGltZXMgXFxldGEgSmYnKSleVCJdLFs5LDEsIihcXGV0YSBKKGdcXHRpbWVzIGcnKSleVChKZlxcdGltZXMgKFxcZXRhIEpmJylee1RcXGxhbWJkYX0pIl0sWzExLDEsIihcXGV0YV5UKEpnXFx0aW1lcyBcXGV0YSBKZycpKV5UKEpmXFx0aW1lcyBcXGV0YSBKZicpXntUXFxsYW1iZGF9Il0sWzEzLDMsIlxcZXRhXlQoSihnZilcXHRpbWVzKChcXGV0YSBKZylee1RcXGxhbWJkYX1cXGV0YSBKZilee1RcXGxhbWJkYX0pIl0sWzEsNCwiKChcXGV0YSBKKGdcXHRpbWVzIGcnKSlee1RcXGxhbWJkYX1cXGV0YSBKKGZcXHRpbWVzIGYnKSlee1RcXGxhbWJkYX1cXGV0YV5UIl0sWzMsNCwiKChcXGV0YSBKKGdcXHRpbWVzIGcnKSlee1RcXGxhbWJkYX1cXGV0YSBKKGZcXHRpbWVzIGYnKSlee1R9Il0sWzYsNCwiKChcXGV0YSBKKGdcXHRpbWVzIGcnKSlee1RcXGxhbWJkYX1cXGV0YV5UIChKZlxcdGltZXMgXFxldGEgSmYnKSlee1R9Il0sWzksNCwiKChcXGV0YSBKKGdcXHRpbWVzIGcnKSleVChKZlxcdGltZXMgXFxldGEgSmYnKSleVCJdLFs0LDMsIlxcbWF0aGJmIG4oSmZcXHRpbWVzIFRmJykiLDFdLFswLDQsIlQoZ1xcdGltZXMgZycpXFxtYXRoYmYgbiIsMV0sWzAsMSwiXFxtYXRoYmYgZF57LTF9XFxzaWdtYSIsMV0sWzMsMiwiXFxzaWdtYShKKGdmKVxcdGltZXMgXFxtYXRoYmYgZF57LTF9KSIsMV0sWzEsMiwiXFxtYXRoYmYgbiIsMV0sWzAsNl0sWzEsNV0sWzMsN10sWzgsMl0sWzUsMTAsIlxcbWF0aGJmIHNeey0xfSIsMV0sWzEwLDExLCIoXFxtYXRoYmYgcihKKGdmKVxcdGltZXMgSihnJ2YnKSkpXlQiLDFdLFsxMSw4LCJcXG1hdGhiZiBjIiwxXSxbNiwxMiwiKFxcZXRhIEooZ1xcdGltZXMgZycpKV5UXFxtYXRoYmYgc157LTF9IiwxLHsiY3VydmUiOi0zfV0sWzEyLDEzLCIoXFxldGEgSihnXFx0aW1lcyBnJykpXntUXFxsYW1iZGF9KFxcbWF0aGJmIHIoSmZcXHRpbWVzIEpmJykpXlQiLDEseyJjdXJ2ZSI6LTN9XSxbMTMsOSwiKFxcZXRhIEooZ1xcdGltZXMgZycpKV57VFxcbGFtYmRhfVxcbWF0aGJmIGMiLDFdLFs5LDE0LCJcXG1hdGhiZiBzXnstMX0oSmZcXHRpbWVzIChcXGV0YSBKZicpXntUXFxsYW1iZGF9KSIsMSx7ImN1cnZlIjotM31dLFsxNCwxNSwiKFxcbWF0aGJmIHIoSmdcXHRpbWVzIEpnJykpXlQoSmZcXHRpbWVzIFxcZXRhIEpmJylee1RcXGxhbWJkYX0iLDEseyJjdXJ2ZSI6LTN9XSxbMTUsNywiXFxtYXRoYmYgYyhKZlxcdGltZXMgXFxldGEgSmYnKV57VFxcbGFtYmRhfSIsMSx7ImN1cnZlIjotM31dLFs2LDE3LCJcXG1hdGhiZiBtXnstMX1cXGV0YV5UIiwxXSxbMTcsNSwiKFxcbWF0aGJmIGVeey0xfSBKKGZcXHRpbWVzIGYnKSlee1RcXGxhbWJkYX1cXGV0YV5UIiwxXSxbNywxNiwiXFxldGFeVChKKGdmKVxcdGltZXMgXFxtYXRoYmYgbV57LTF9KSIsMV0sWzE2LDgsIlxcZXRhXlQoSihnZilcXHRpbWVzIChcXG1hdGhiZiBlXnstMX1KZilee1RcXGxhbWJkYX0pIiwxXSxbMTcsMTgsIlxcbWF0aGJmIHNeey0xfSIsMV0sWzE4LDEwLCIoXFxtYXRoYmYgZV57LTF9IEooZlxcdGltZXMgZicpKV5UIiwxXSxbNCw5XSxbMTgsMTksIigoXFxldGEgSihnXFx0aW1lcyBnJykpXntUXFxsYW1iZGF9XFxtYXRoYmYgckooZlxcdGltZXMgZicpKV5UIiwxXSxbMTksMjAsIihcXG1hdGhiZiBzXnstMX0oSmZcXHRpbWVzIFxcZXRhIEpmJykpXlQiLDFdLFsxMCwyMCwiKFxcbWF0aGJmIHJeey0xfSBKKGZcXHRpbWVzIGYnKSleVCIsMV0sWzE0LDIwLCJcXG1hdGhiZiBjXnstMX0iLDFdLFs0MCw0NCwiXFxtYXRoYmYgc1xcdGV4dHstbmF0fSIsMSx7InNob3J0ZW4iOnsic291cmNlIjoyMCwidGFyZ2V0IjoyMH0sInN0eWxlIjp7ImJvZHkiOnsibmFtZSI6Im5vbmUifSwiaGVhZCI6eyJuYW1lIjoibm9uZSJ9fX1dXQ==}
{the diagram} that we require to commute}.

\begin{example}
  For the strong presheaf construction, the strength is given by
  \begin{enumerate}
    \item the functor $\sigma:\scat W\times\widehat{\scat X}\to\widehat{\scat W\times\scat X}$
      acting on objects by
      \begin{align*}
        \sigma(W,P) = \int^X PX \times \Hom(-,(W,X));
      \end{align*}
    \item the natural transformation $\bicell n_{F,G}:JF\times \widehat{G}\to \widehat{F\times G}$
      which has as components natural isomorphisms
      \begin{align*}
        &\int^{X',Y'}\rr{\int^Y PY\times\Hom((X',Y'),(X,Y))}\times\Hom(-,(FX',GY'))\\
        &\cong \int^{Y'}\rr{\int^Y PY\times\Hom(Y',GY)}\times\Hom(-,(FX,Y')).
      \end{align*}
  \end{enumerate}
\end{example}

In order for a strong $J$-pseudomonad to induce a strong pseudomonad in the
sense of~\cite{saville2023}, we require $J$ to be the identity. We describe how
one obtains the appropriate structure:

\begin{definition}\label{def:induced_strong_pseudomonad}
  Let $J:\bicat C\to\bicat C$ be the identity and $T$ a strong $J$-pseudomonad structure.
  The \emph{strong pseudomonad structure induced by $T$} consists of
  \begin{enumerate}
    \item the strength \ref{def:induced_strength};
    \item for all $X\in\bicat{C}$, the invertible 2-cell $\bicell x_X=\bicell s_{\eta\lambda}\bullet \bicell l_X$;
    \item for all $X,Y,Z\in\bicat{C}$, the invertible 2-cell $\bicell y_{X,Y,Z}=\bicell p_{X,Y,Z}$;
    \item for all $X,Y\in\bicat{C}$, the invertible 2-cell $\bicell w_{X,Y}$ given by
      \begin{equation}
        % https://q.uiver.app/?q=WzAsNixbMCwwLCJXXFx0aW1lcyBUXjJYIl0sWzAsMSwiVChXXFx0aW1lcyBUWCkiXSxbMCwzLCJUXjIoV1xcdGltZXMgWCkiXSxbNywwLCJXXFx0aW1lcyBUWCJdLFs3LDEsIlQoV1xcdGltZXMgWCkiXSxbNywzLCJUKFdcXHRpbWVzIFgpIl0sWzAsMywiWFxcdGltZXMgXFx0ZXh0e2lkfV4qIiwxXSxbMyw0LCJcXGV0YV5cXGRhZ2dlciIsMV0sWzAsMSwiXFxldGFeXFxkYWdnZXIiLDFdLFsyLDUsIlxcdGV4dHtpZH1eKiIsMV0sWzQsNSwiIiwxLHsibGV2ZWwiOjIsInN0eWxlIjp7ImhlYWQiOnsibmFtZSI6Im5vbmUifX19XSxbMSw1LCIoXFxsYW1iZGFeXFxkYWdnZXIoMVxcdGltZXNcXGV0YVxcZXRhXlxcZGFnZ2VyKSleXFxkYWdnZXJcXGxhbWJkYV57LTF9IiwxXSxbMSwyLCIoXFxldGFcXGV0YV5cXGRhZ2dlcileKiIsMV0sWzEsNCwiKFxcZXRhXlxcZGFnZ2VyKV4qIiwxXSxbMTEsMTAsIihcXG1hdGhiZiByXnstMX0oMVxcdGltZXNcXGV0YV5cXGRhZ2dlcikpXlxcZGFnZ2VyXFxsYW1iZGFeey0xfSIsMCx7InNob3J0ZW4iOnsic291cmNlIjozMCwidGFyZ2V0IjozMH19XSxbMTIsMTEsIlxcbWF0aGJmIGNeey0xfSIsMix7InNob3J0ZW4iOnsic291cmNlIjozMCwidGFyZ2V0IjozMH19XSxbOCw3LCJcXG1hdGhiZiBxXnstMX0iLDEseyJzaG9ydGVuIjp7InNvdXJjZSI6MjAsInRhcmdldCI6MjB9fV1d
        \begin{tikzcd}
          {W\times T^2X} &&&&&&& {W\times TX} \\
          {T(W\times TX)} &&&&&&& {T(W\times X)} \\
          \\
          {T^2(W\times X)} &&&&&&& {T(W\times X)}
          \arrow["{X\times \text{id}^*}"{description}, from=1-1, to=1-8]
          \arrow[""{name=0, anchor=center, inner sep=0}, "{\eta^\dagger}"{description}, from=1-8, to=2-8]
          \arrow[""{name=1, anchor=center, inner sep=0}, "{\eta^\dagger}"{description}, from=1-1, to=2-1]
          \arrow["{\text{id}^*}"{description}, from=4-1, to=4-8]
          \arrow[""{name=2, anchor=center, inner sep=0}, Rightarrow, no head, from=2-8, to=4-8]
          \arrow[""{name=3, anchor=center, inner sep=0}, "{(\lambda^\dagger(1\times\eta\eta^\dagger))^\dagger\lambda^{-1}}"{description}, from=2-1, to=4-8]
          \arrow[""{name=4, anchor=center, inner sep=0}, "{(\eta\eta^\dagger)^*}"{description}, from=2-1, to=4-1]
          \arrow["{(\eta^\dagger)^*}"{description}, from=2-1, to=2-8]
          \arrow["{(\mathbf r^{-1}(1\times\eta^\dagger))^\dagger\lambda^{-1}}", shorten <=37pt, shorten >=37pt, Rightarrow, from=3, to=2]
          \arrow["{\mathbf c^{-1}}"', shorten <=37pt, shorten >=37pt, Rightarrow, from=4, to=3]
          \arrow["{\mathbf q^{-1}}"{description}, shorten <=49pt, shorten >=49pt, Rightarrow, from=1, to=0]
        \end{tikzcd}
      \end{equation}
    \item for all $X,Y\in\bicat{C}$, the invertible 2-cell $\bicell z_{X,Y}=\inv{\bicell r}_\eta$.
  \end{enumerate}
\end{definition}

Beyond the previously mentioned pseudonaturality, it is straightforward to postulate
axioms that ensure the coherence axioms for strong pseudofunctors are satisfied.
This is because none of these coherence axioms contain repeated applications of the
pseudofunctor. Thus the notion generalises from endofunctors to arbitrary pseudofunctors.



\chapter{Evaluation}

It is now time to reflect on what we have and have not achieved. We begin
by highlighting two alternative approaches that we could have taken to solve
the problem. We then assess the quality of our results. Finally, we focus
on the presentation of these results and the compromises involved.

\section{Small presheaves instead of relative pseudomonads}\label{sec:small_presheaves}

We worked towards extending synthetic measure theory to admit the presheaf
construction as a model by generalising to relative pseudomonads. This has made
it very difficult to develop the required theory but almost trivial to establish
the model.

While we believe that this path can lead to success, it may not be the simplest.
In the early stages of this project we were aiming to restrict $\biCAT$ to a
suitable cartesian closed 2-category on which the presheaf construction is a
pseudomonad. We tried to work with small presheaves on locally presentable
categories but this failed. After spending a significant amount of time with
this approach, we decided that it would be safer to follow the potentially
longer route of relative pseudomonads as it would allow us to make some progress
rightaway.

This does not mean, however, that it is impossible to restrict the presheaf
construction and the underlying category appropriately so that generalising
synthetic measure theory to pseudomonads is sufficient. In a sense, these
considerations are likely to lead to an entirely new set of problems:
formulating the theory would become signficantly easier, but constructing a
non-trivial model would be difficult.

\section{Strength as a natural transformation}

The next thing that has to be criticised is our choice of changing the extension
operator to incorporate strength, rather than adding the strength to a relative
pseudomonad as a suitable pseudonatural transformation. Whether this approach is
fruitful remains to be seen. However, there are several notable disadvantages.

Firstly, it requires a certain amount of reinventing the wheel. Rather than
sticking with the already established theory of relative pseudomonads, we had to
rewrite the definition entirely. While the result is similar, a lot of time and
effort went into making everything consistent. Given that we have not been able to
reap the benefits of the new structure, it is unclear whether this detour will
eventually pay off.

Secondly, the new structure is in many ways less elegant than usual relative
pseudomonads. For example, have a look a the 2-cell families $\bicell c$ and
$\bicell m$ in the case of the presheaf construction. The former involves
additional parts that are not required for the latter. This makes it more
difficult to distinguish important details from the overall noise.

\section{Quality of the definitions}

Inspecting where the complexity is inspires confidence. This is because
the largest diagrams arise whenever coends are involved. This is deceptive,
however. Showing coend-related results has, for the most part, been a
straightforward mechanical task. The vast majority of work went into the three
core definitions \ref{def:prestrong_inclusion_pseudomonad_structure},
\ref{def:prestrong_inclusion_pseudomonad_axioms}, and
\ref{def:strong_inclusion_pseudomonad_structure}. The fact that all the
developments related to our theory, in particular the proofs and the induced
structures, are short and succinct suggests that we have indeed developed a
suitable language to reason about strong relative pseudomonads.

\section{Presentation}

The complexity of the expressions has repeatedly led to diagrams whose width was
several times what could fit the page. We have therefore not been able to
include as much detail as we would have liked. This means that some proofs may
be harder to follow than is appropriate. In any case, we do not expect anyone to
be able to reconstruct every step of our developments without some pen and
paper.

This problem is not new to category theorists. There have been several
approaches to deal with large diagrams that usually require even more notational
shortcuts. See \cite{marmolejo2013} and \cite{saville2023} for some related
examples. While this would have allowed us to condense more information onto the
page, it would have also meant hiding a signficiant amount of complexity and
thus required long explanations as to what is going on. We are doubtful whether
this would have been possible within the scope of this report.

To aid the reader we have decided to include quiver links to some particularly
large diagrams. There are several reasons why this is not a solution that can be
relied upon, though. Firstly, there is the technical problem that the content of
those links is not strictly part of the report. Secondly, the quiver server
will, eventually, go offline. If we were to rely on the service, then a
significant part of the content would be lost. Of course, the latter problem
may be solved by installing quiver locally.

Machine verified proofs may be another way to maintain rigour while improving
the presentation. This way we would be able to hide some technical details in the
comforting knowledge that everything has been made to work as intended.
Those who are interested would be welcome to read the corresponding source
code. While this is in a sense the optimal solution it is also idealistic:
formalising our work like this would vastly increase our time investment.
This would be a particularly risky bet given that we have not been able
to show that things are going to work out in the end.

\section{Target audience}

This report is supposed to be targeted towards undergraduate students. Meeting
this requirement has been difficult. While there are a few courses that define
categories, functors, and natural transformations, category theory is not taught
at the university in its own right. This has led us to take some major
shortcuts. For example, we would have liked to investigate our attempt at
restricting our model as described in \ref{sec:small_presheaves}. Unfortunately,
this would have required us to define locally presentable categories from the
ground up which is impossible with the space that we are given.

While this report may not provide a comprehensive introduction to category
theory, it still offers valuable insights and contributions that can be
understood by readers with varying degrees of knowledge in the subject. To
fully comprehend the technical details of the report, only a basic understanding of
category theory is required. As a result, undergraduate students with some
familiarity with natural transformations should be able to understand almost
all the technical details of our work. The only exceptions are more technical
arguments such as the existence of coends in $\Set$, which have to be taken for
granted. However, our explanations make it possible for readers without prior
knowledge to follow the main ideas and results presented.

\chapter{Future work}

Given that we have not been able to establish the presheaf construction as a
model of a generalised synthetic measure theory, a lot of future work remains.

The first step should be to make sure that the induced strength is a
pseudonatural transformation. It is hard to say how difficult a problem this
would be, but more axioms related to the interplay of the strong structural
2-cells $\bicell u$ and $\bicell s$ with their prestrong counterparts are
required. Such axioms will certainly prove useful for later developments.

Secondly, there is the problem of postulating axioms so that the induced strong
pseudomonad satisfies the coherence conditions of a strong pseudomonad in the
sense of \cite{saville2023}. This has the potential to be a lot of work: there
are five axioms for which there are no obvious counterparts. The unusual 2-cell
$\bicell q$ in \ref{def:strong_inclusion_pseudomonad_structure} is a good
example of the ingenuity that might be required to make this work. It is
safe to say that the notion of strength does not lend itself to being
generalised to relative monads.

Thirdly, once strong relative pseudomonads have been defined it is time for the
most interesting part. Generalising synthetic measure theory. Studying the
measure theory arising from the presheaf construction should be satisfying in
its own right. After all, this is the motivation for this whole project. How
difficult of a task this will end up being depends on the quality of the
definitions obtained in the previous steps.

Finally, formalising our theory in a theorem proving language seems like a
logical extension. The axioms required to make a complete definition of a strong
relative pseudomonad are bound to be complex and the addition of synthetic
measure theory will not improve the situation. This means that it will become
even harder to verify the correctness of the statements involved.


\bibliographystyle{plain}
\bibliography{sources}

%\appendix

%\chapter{Symmetric monoidal categories}

\section{Definition}\label{sec:symmetric_monoidal_categories}

Monoidal categories formalise a general way of combining objects and morphisms between them
in a way that captures the behaviour of many mathematical operations such as categorical
products (cf.~\ref{sec:cartesian_closed_categories_are_symmetric_monoidal}, categorical
coproducts, and the tensor product.

\begin{definition}\label{def:monoidal_category}
  A monoidal category consists of
  \begin{enumerate}
    \item a category $\cat{C}$;
    \item an object $I\in\cat{C}$;
    \item a bifunctor $\otimes:\cat{C}\times\cat{C}\to\cat{C}$;
    \item a natural isomorphism given by arrows
      \begin{align*}
        \alpha:\rr{X\otimes Y}\otimes Z\to X\otimes\rr{Y\otimes Z}
      \end{align*}
      for $X,Y,Z\in\cat{C}$;
    \item a natural isomorphism given by arrows
      \begin{align*}
        \lambda:I\otimes X\to X
      \end{align*}
      for $X\in\cat{C}$;
    \item a natural isomorphism given by arrows
      \begin{align*}
        \rho:X\otimes I\to X
      \end{align*}
      for $X\in\cat{C}$
  \end{enumerate}
  such that, for all $W,X,Y,Z\in\cat{C}$ the following commute:
  \begin{equation}
    \label{eq:monoidal_pentagon}
    % https://q.uiver.app/?q=WzAsNSxbMSwwLCIoV1xcb3RpbWVzIFgpXFxvdGltZXMoWVxcb3RpbWVzIFopIl0sWzAsMSwiKChXXFxvdGltZXMgWClcXG90aW1lcyBZKVxcb3RpbWVzIFoiXSxbMCwzLCIoV1xcb3RpbWVzIChYXFxvdGltZXMgWSkpXFxvdGltZXMgWiJdLFsyLDMsIldcXG90aW1lcygoWFxcb3RpbWVzIFkpXFxvdGltZXMgWikiXSxbMiwxLCJXXFxvdGltZXMgKFhcXG90aW1lcyAoWVxcb3RpbWVzIFopIl0sWzEsMCwiXFxhbHBoYV97V1xcb3RpbWVzIFgsWSxafSJdLFsxLDIsIlxcYWxwaGFfe1csWCxZfVxcb3RpbWVzIDFfWCIsMl0sWzIsMywiXFxhbHBoYV97VyxYXFxvdGltZXMgWSxafSIsMl0sWzMsNCwiMV9XXFxvdGltZXMgXFxhbHBoYV97WCxZLFp9IiwyXSxbMCw0LCJcXGFscGhhX3tXLFgsWVxcb3RpbWVzIFp9Il1d
    \begin{tikzcd}
                & {(W\otimes X)\otimes(Y\otimes Z)} \\
      {((W\otimes X)\otimes Y)\otimes Z} && {W\otimes (X\otimes (Y\otimes Z)} \\
      \\
      {(W\otimes (X\otimes Y))\otimes Z} && {W\otimes((X\otimes Y)\otimes Z)}
      \arrow["{\alpha_{W\otimes X,Y,Z}}", from=2-1, to=1-2]
      \arrow["{\alpha_{W,X,Y}\otimes X}"', from=2-1, to=4-1]
      \arrow["{\alpha_{W,X\otimes Y,Z}}"', from=4-1, to=4-3]
      \arrow["{W\otimes \alpha_{X,Y,Z}}"', from=4-3, to=2-3]
      \arrow["{\alpha_{W,X,Y\otimes Z}}", from=1-2, to=2-3]
    \end{tikzcd}
  \end{equation}
  \begin{equation}
    \label{eq:monoidal_triangle}
    % https://q.uiver.app/?q=WzAsMyxbMCwwLCIoWFxcb3RpbWVzIDEpXFxvdGltZXMgWSJdLFsyLDAsIlhcXG90aW1lcygxXFxvdGltZXMgWSkiXSxbMSwxLCJYXFxvdGltZXMgWSJdLFswLDEsIlxcYWxwaGFfe1gsMSxZfSJdLFsxLDIsIjFfWFxcb3RpbWVzXFxsYW1iZGFfWSJdLFswLDIsIlxccmhvX1hcXG90aW1lcyAxX1kiLDJdXQ==
    \begin{tikzcd}
      {(X\otimes I)\otimes Y} && {X\otimes(I\otimes Y)} \\
                              & {X\otimes Y}
                              \arrow["{\alpha_{X,I,Y}}", from=1-1, to=1-3]
                              \arrow["{X\otimes\lambda_Y}", from=1-3, to=2-2]
                              \arrow["{\rho_X\otimes Y}"', from=1-1, to=2-2]
    \end{tikzcd}
  \end{equation}
\end{definition}

In many cases the monoidal product $\otimes$ is commutative up to a canonical isomorphism.
This is captured by the notion of a symmetric monoidal category.

\begin{definition}
  A symmetric monoidal category consists of
  \begin{enumerate}
    \item a monoidal category $(\cat{C},I,\otimes,\alpha,\lambda,\rho)$;
    \item a natural isomorphism given by the arrows
      \begin{align*}
        \gamma_{X,Y}:X\otimes Y\to Y\otimes X
      \end{align*}
      for $X,Y\in\cat{C}$
  \end{enumerate}
  such that, for all $X,Y,Z\in\cat{C}$, the following commutes:
  \begin{equation}\label{eq:symmetric_monoidal_hexagon}
    % https://q.uiver.app/?q=WzAsNixbMCwwLCIoWFxcb3RpbWVzIFkpXFxvdGltZXMgWiJdLFs2LDAsIihZXFxvdGltZXMgWClcXG90aW1lcyBaIl0sWzAsMiwiWFxcb3RpbWVzKFlcXG90aW1lcyBaKSJdLFswLDQsIihZXFxvdGltZXMgWilcXG90aW1lcyBYIl0sWzYsNCwiWVxcb3RpbWVzKFpcXG90aW1lcyBYKSJdLFs2LDIsIllcXG90aW1lcyhYXFxvdGltZXMgWikiXSxbMCwyLCJcXGFscGhhX3tYXFxvdGltZXMgWSxafSIsMl0sWzIsMywiXFxnYW1tYV97WCxZXFxvdGltZXMgWn0iLDJdLFswLDEsIlxcZ2FtbWFfe1gsWX1cXG90aW1lcyAxX1oiXSxbMSw1LCJcXGFscGhhX3tZXFxvdGltZXMgWCxafSJdLFs1LDQsIjFfWVxcb3RpbWVzIFxcZ2FtbWFfe1gsWn0iXSxbMyw0LCJcXGFscGhhX3tZXFxvdGltZXMgWixYfSIsMl1d
    \begin{tikzcd}
      {(X\otimes Y)\otimes Z} &&&&&& {(Y\otimes X)\otimes Z} \\
      \\
      {X\otimes(Y\otimes Z)} &&&&&& {Y\otimes(X\otimes Z)} \\
      \\
      {(Y\otimes Z)\otimes X} &&&&&& {Y\otimes(Z\otimes X)}
      \arrow["{\alpha_{X\otimes Y,Z}}"', from=1-1, to=3-1]
      \arrow["{\gamma_{X,Y\otimes Z}}"', from=3-1, to=5-1]
      \arrow["{\gamma_{X,Y}\otimes Z}", from=1-1, to=1-7]
      \arrow["{\alpha_{Y\otimes X,Z}}", from=1-7, to=3-7]
      \arrow["{Y\otimes \gamma_{X,Z}}", from=3-7, to=5-7]
      \arrow["{\alpha_{Y\otimes Z,X}}"', from=5-1, to=5-7]
    \end{tikzcd}
  \end{equation}
\end{definition}

All the previously mentioned monoidal categories are in fact symmetric monoidal.
In particular, in any category with products $X\times Y\cong Y\times X$, in any
category with coproducts $X+Y\cong Y+ X$, and for all vector spaces $U,V$,
$U\otimes V\cong V\otimes U$. Moroever, these isomorphisms are natural and coherent
in the sense made precise above.

\section{Cartesian categories are symmetric monoidal}\label{sec:cartesian_closed_categories_are_symmetric_monoidal}

Let $\cat C$ be a category with a choice of terminal object $1\in\cat C$
and, for all $X,Y\in{\cat C}$, a binary product $X\times Y\in\cat C$
with the projection morphisms
\begin{align}
  \label{eq:monoidal_projection1}
  \pi^1_{X,Y} : X \times Y \to X \\
  \label{eq:monoidal_projection2}
  \pi^2_{X,Y} : X \times Y \to Y
\end{align}

\begin{definition}\label{def:monoidal_tensor_product}
  Define a bifunctor structure
  \begin{align*}
    \otimes : \cat C \times \cat C \to \cat C
  \end{align*}
  as follows: for all $X,X',Y,Y'\in{\cat{C}}$, let $X\otimes X'$ be the
  product $X\times X'$ in $\cat C$, and, for each pair of arrows
  $f:X\to Y$, $g:X'\to Y'$, let
  \begin{align*}
    f \otimes g : X \otimes X' \to Y \otimes Y'
  \end{align*}
  be the unique arrow such that both squares in the following commute:
  \begin{equation}\label{eq:monoidal_tensor_product}
    \begin{tikzcd}[row sep=huge, column sep=huge]
      X \arrow{d}{f} &
      X \otimes X'
      \arrow{l}[swap]{\pi^1_{X,X'}}
      \arrow{r}{\pi^2_{X,X'}}
      \arrow{d}{f\otimes g} &
      X' \arrow{d}{g}\\
      Y &
      Y \otimes Y'
      \arrow{l}[swap]{\pi^1_{Y,Y'}}
      \arrow{r}{\pi^2_{Y, Y'}} &
      Y'
    \end{tikzcd}
  \end{equation}
\end{definition}

\begin{lemma}
  $\otimes : \cat C \times \cat C \to \cat C$ is a bifunctor.
  Furthermore, the transformations $\pi^1,\pi^2$ given by the maps
  (\ref{eq:monoidal_projection1}) and (\ref{eq:monoidal_projection2}), respectively,
  are natural.
  \begin{proof}
    Consider arrows
    \begin{align*}
      f : X & \to Y, & f': X' & \to Y', \\
      g : Y & \to Z, & g': Y' & \to Z'.
    \end{align*}
    in $\cat{C}$. We then note that the following commutes:
    \begin{center}
      \begin{tikzcd}[row sep=huge, column sep=huge]
        X \arrow{d}{f} &
        X \otimes X'
        \arrow{l}[swap]{\pi^1_{X, X'}}
        \arrow{r}{\pi^2_{X, X'}}
        \arrow{d}{f\otimes f'} &
        X' \arrow{d}{f'}\\
        Y \arrow{d}{g}&
        Y \otimes Y'
        \arrow{l}[swap]{\pi^1_{Y, Y'}}
        \arrow{r}{\pi^2_{Y, Y'}}
        \arrow{d}{g\otimes g'} &
        Y' \arrow{d}{g'} \\
        Z &
        Z \otimes Z'
        \arrow{l}[swap]{\pi^1_{Z, Z'}}
        \arrow{r}{\pi^2_{Z, Z}} &
        Z'
      \end{tikzcd}
    \end{center}
    In particular, $(g\otimes g')\circ(f\otimes f')$ makes the outer diagram
    commute. By~\ref{eq:monoidal_tensor_product}, we then have
    \begin{align*}
      (g\otimes g')\circ(f\otimes f') = (g\circ f)\otimes(g'\circ f').
    \end{align*}
    Finally, the following commutes trivially:
    \begin{equation*}
      \begin{tikzcd}[row sep=huge, column sep=huge]
        X
        \arrow{d}{X} &
        X \otimes X'
        \arrow{l}[swap]{\pi^1_{X,X'}}
        \arrow{d}{{X\otimes X'}}
        \arrow{r}{\pi^2_{X,X'}} &
        X'
        \arrow{d}{{X'}} \\
        X &
        X \otimes X'
        \arrow{l}[swap]{\pi^1_{X,X'}}
        \arrow{r}{\pi^2_{X,X'}} &
        X
      \end{tikzcd}
    \end{equation*}
    I.e. $1_X \otimes 1_{X'} = 1_{X\otimes X'}$.
    Thus $\otimes:\cat C\times\cat C\to\cat C$ is a bifunctor.
  \end{proof}
\end{lemma}

\begin{definition}[Associator]\label{def:monoidal_associator}
  For all $X,Y,Z\in\cat C$, define
  \begin{align*}
    \alpha_{X,Y,Z} : (X \otimes Y) \otimes Z \to X \otimes (Y \otimes Z)
  \end{align*}
  to be the unique morphism that makes the following commute:
  \begin{equation}\label{eq:monoidal_a}
    \begin{tikzcd}[row sep=huge, column sep=huge]
      X\otimes Y \arrow{d}{\pi^1_{X,Y}} &
      \rr{X\otimes Y}\otimes Z
      \arrow{l}[swap]{\pi^1_{X\otimes Y,Z}}
      \arrow{d}{\alpha_{X,Y,Z}}
      \arrow{dr}{\pi^2_{X,Y}\otimes Z} \\
      X &
      X\otimes\rr{Y\otimes Z}
      \arrow{l}[swap]{\pi^1_{X,Y\otimes Z}}
      \arrow{r}{\pi^2_{X,Y\otimes Z}} &
      Y\otimes Z
    \end{tikzcd}
  \end{equation}
\end{definition}

\begin{lemma}
  The transformation given by the maps
  $\alpha_{X,Y,Z}:(X\otimes Y)\otimes Z \to X\otimes(Y\otimes Z)$
  forms a natural isomorphism in all three arguments.
  \begin{proof}
    For all $X,Y,Z\in\cat C$, define
    \begin{align*}
      \inv \alpha_{X,Y,Z}:X\otimes\rr{Y\otimes Z}\to\rr{X\otimes Y}\otimes Z
    \end{align*}
    to be the unique morphism that makes the following commute:
    \begin{equation}\label{eq:monoidal_inv_a}
      \begin{tikzcd}[row sep=huge, column sep=huge]
                &
                X \otimes\rr{Y\otimes Z}
                \arrow{dl}[swap]{X\otimes\pi^1_{Y,Z}}
                \arrow{d}{\inv \alpha_{X,Y,Z}}
        \arrow{r}{\pi^2_{X,Y\otimes Z}} &
        Y\otimes Z
        \arrow{d}{\pi^2_{Y,Z}}\\
        X\otimes Y &
        \rr{X\otimes Y}\otimes Z
        \arrow{l}[swap]{\pi^1_{X\otimes Y,Z}}
        \arrow{r}{\pi^2_{X\otimes Y,Z}} &
        Z
      \end{tikzcd}
    \end{equation}
    We intend to show
    \begin{equation}\label{eq:monoidal_a_iso}
      \begin{tikzcd}
                &&&& {(X\otimes Y)\otimes Z} \\
                \\
                {X \otimes (Y \otimes Z)} \\
                &&&&&& {} \\
                &&&& {(X\otimes Y)\otimes Z}
                \arrow["{\alpha_{X,Y,Z}}"', bend right=30, from=1-5, to=3-1]
                \arrow["{\inv\alpha_{X,Y,Z}}"', bend right=30, from=3-1, to=5-5]
                \arrow["{{(X\otimes Y)\otimes Z}}", bend left=30, from=1-5, to=5-5]
      \end{tikzcd}
    \end{equation}
    Firstly, this diagram commutes:
    \begin{equation}\label{eq:monoidal_a_iso_pi1_pi1}
      % https://q.uiver.app/?q=WzAsNyxbMCwxXSxbMCwyLCJYXFxvdGltZXMoWVxcb3RpbWVzIFopIl0sWzQsMCwiKFhcXG90aW1lcyBZKVxcb3RpbWVzIFoiXSxbNCw0LCIoWFxcb3RpbWVzIFkpXFxvdGltZXMgWiJdLFsyLDIsIlgiXSxbNSwyXSxbNCwyLCJYXFxvdGltZXMgWSJdLFsyLDMsIjFfeyhYXFxvdGltZXMgWSlcXG90aW1lcyBafSIsMCx7ImN1cnZlIjotNX1dLFsyLDEsIlxcYWxwaGFfe1gsWSxafSIsMix7ImN1cnZlIjo1fV0sWzEsMywiXFxhbHBoYV57LTF9X3tYLFksWn0iLDIseyJjdXJ2ZSI6NX1dLFsyLDYsIlxccGleMV97WFxcb3RpbWVzIFksWn0iLDJdLFsxLDQsIlxccGleMV97WCxZXFxvdGltZXMgWn0iLDJdLFs2LDQsIlxccGleMV97WCxZfSJdLFsxLDYsIjFfWFxcb3RpbWVzXFxwaV4xX3tZLFp9IiwwLHsiY3VydmUiOi0zfV0sWzMsNiwiXFxwaV4xX3tYXFxvdGltZXMgWSxafSJdLFs0LDEzLCJcXGV4cGxhaW57ZXE6bW9ub2lkYWwtdGVuc29yX3Byb2R1Y3R9IiwxLHsic2hvcnRlbiI6eyJ0YXJnZXQiOjIwfSwic3R5bGUiOnsiYm9keSI6eyJuYW1lIjoibm9uZSJ9LCJoZWFkIjp7Im5hbWUiOiJub25lIn19fV0sWzYsOSwiXFxleHBsYWlue2VxOm1vbm9pZGFsLWludl9hfSIsMSx7InNob3J0ZW4iOnsidGFyZ2V0IjoyMH0sInN0eWxlIjp7ImJvZHkiOnsibmFtZSI6Im5vbmUifSwiaGVhZCI6eyJuYW1lIjoibm9uZSJ9fX1dLFsyLDEzLCJcXGV4cGxhaW57ZXE6bW9ub2lkYWwtYX0iLDEseyJzaG9ydGVuIjp7InRhcmdldCI6MjB9LCJzdHlsZSI6eyJib2R5Ijp7Im5hbWUiOiJub25lIn0sImhlYWQiOnsibmFtZSI6Im5vbmUifX19XV0=
      \begin{tikzcd}
                &&&& {(X\otimes Y)\otimes Z} \\
                {} \\
        {X\otimes(Y\otimes Z)} && X && {X\otimes Y} & {} \\
        \\
                               &&&& {(X\otimes Y)\otimes Z}
                               \arrow["{{(X\otimes Y)\otimes Z}}", bend left=30, from=1-5, to=5-5]
                               \arrow["{\alpha_{X,Y,Z}}"', bend right=30, from=1-5, to=3-1]
                               \arrow[""{name=0, anchor=center, inner sep=0}, "{\alpha^{-1}_{X,Y,Z}}"', bend right=30, from=3-1, to=5-5]
                               \arrow["{\pi^1_{X\otimes Y,Z}}"', from=1-5, to=3-5]
                               \arrow["{\pi^1_{X,Y\otimes Z}}"', from=3-1, to=3-3]
                               \arrow["{\pi^1_{X,Y}}", from=3-5, to=3-3]
                               \arrow[""{name=1, anchor=center, inner sep=0}, "{X\otimes\pi^1_{Y,Z}}", bend left=18, from=3-1, to=3-5]
                               \arrow["{\pi^1_{X\otimes Y,Z}}", from=5-5, to=3-5]
                               \arrow["{\explain{eq:monoidal_tensor_product}}"{description}, Rightarrow, draw=none, from=3-3, to=1]
                               \arrow["{\explain{eq:monoidal_inv_a}}"{description}, Rightarrow, draw=none, from=3-5, to=0]
                               \arrow["{\explain{eq:monoidal_a}}"{description}, Rightarrow, draw=none, from=1-5, to=1]
      \end{tikzcd}
    \end{equation}
    Secondly, we have this commutative diagram:
    \begin{equation}\label{eq:monoidal_a_iso_pi1_pi2}
      % https://q.uiver.app/?q=WzAsOSxbMCwxXSxbMCwyLCJYXFxvdGltZXMoWVxcb3RpbWVzIFopIl0sWzQsMCwiKFhcXG90aW1lcyBZKVxcb3RpbWVzIFoiXSxbNCw0LCIoWFxcb3RpbWVzIFkpXFxvdGltZXMgWiJdLFszLDMsIlhcXG90aW1lcyBZIl0sWzMsMiwiWSJdLFsyLDEsIllcXG90aW1lcyBaIl0sWzIsMl0sWzUsMl0sWzIsMywiMV97KFhcXG90aW1lcyBZKVxcb3RpbWVzIFp9IiwwLHsiY3VydmUiOi01fV0sWzIsMSwiXFxhbHBoYV97WCxZLFp9IiwyLHsiY3VydmUiOjV9XSxbMSwzLCJcXGFscGhhXnstMX1fe1gsWSxafSIsMix7ImN1cnZlIjo1fV0sWzIsNCwiXFxwaV4xX3tYXFxvdGltZXMgWSxafSJdLFszLDQsIlxccGleMV97WFxcb3RpbWVzIFksWn0iXSxbMSw2LCJcXHBpXjJfe1gsWVxcb3RpbWVzIFp9Il0sWzYsNSwiXFxwaV4xX3tZLFp9Il0sWzQsNSwiXFxwaV4yX3tYLFl9Il0sWzEsNCwiMV9YXFxvdGltZXMgXFxwaV4xX3tZLFp9Il0sWzIsNiwiXFxwaV4yX3tYLFl9XFxvdGltZXMgMV9aIl0sWzYsMTcsIlxcZXhwbGFpbntlcTptb25vaWRhbC10ZW5zb3JfcHJvZHVjdH0iLDEseyJzaG9ydGVuIjp7InRhcmdldCI6MjB9LCJzdHlsZSI6eyJib2R5Ijp7Im5hbWUiOiJub25lIn0sImhlYWQiOnsibmFtZSI6Im5vbmUifX19XSxbMTIsNiwiXFxleHBsYWlue2VxOm1vbm9pZGFsLXRlbnNvcl9wcm9kdWN0fSIsMSx7InNob3J0ZW4iOnsic291cmNlIjoyMH0sInN0eWxlIjp7ImJvZHkiOnsibmFtZSI6Im5vbmUifSwiaGVhZCI6eyJuYW1lIjoibm9uZSJ9fX1dLFs0LDksIiIsMSx7InNob3J0ZW4iOnsic291cmNlIjo0MCwidGFyZ2V0Ijo0MH0sInN0eWxlIjp7ImJvZHkiOnsibmFtZSI6Im5vbmUifSwiaGVhZCI6eyJuYW1lIjoibm9uZSJ9fX1dLFs0LDExLCJcXGV4cGxhaW57ZXE6bW9ub2lkYWwtaW52X2F9IiwxLHsic2hvcnRlbiI6eyJ0YXJnZXQiOjIwfSwic3R5bGUiOnsiYm9keSI6eyJuYW1lIjoibm9uZSJ9LCJoZWFkIjp7Im5hbWUiOiJub25lIn19fV0sWzYsMTAsIlxcZXhwbGFpbntlcTptb25vaWRhbC1hfSIsMSx7InNob3J0ZW4iOnsidGFyZ2V0IjoyMH0sInN0eWxlIjp7ImJvZHkiOnsibmFtZSI6Im5vbmUifSwiaGVhZCI6eyJuYW1lIjoibm9uZSJ9fX1dXQ==
      \begin{tikzcd}
                &&&& {(X\otimes Y)\otimes Z} \\
        {} && {Y\otimes Z} \\
        {X\otimes(Y\otimes Z)} && {} & Y && {} \\
                               &&& {X\otimes Y} \\
                               &&&& {(X\otimes Y)\otimes Z}
                               \arrow[""{name=0, anchor=center, inner sep=0}, "{{(X\otimes Y)\otimes Z}}", bend left=30, from=1-5, to=5-5]
                               \arrow[""{name=1, anchor=center, inner sep=0}, "{\alpha_{X,Y,Z}}"', bend right=30, from=1-5, to=3-1]
                               \arrow[""{name=2, anchor=center, inner sep=0}, "{\alpha^{-1}_{X,Y,Z}}"', bend right=30, from=3-1, to=5-5]
                               \arrow[""{name=3, anchor=center, inner sep=0}, "{\pi^1_{X\otimes Y,Z}}", from=1-5, to=4-4]
                               \arrow["{\pi^1_{X\otimes Y,Z}}", from=5-5, to=4-4]
                               \arrow["{\pi^2_{X,Y\otimes Z}}", from=3-1, to=2-3]
                               \arrow["{\pi^1_{Y,Z}}", from=2-3, to=3-4]
                               \arrow["{\pi^2_{X,Y}}", from=4-4, to=3-4]
                               \arrow[""{name=4, anchor=center, inner sep=0}, "{X\otimes \pi^1_{Y,Z}}", from=3-1, to=4-4]
                               \arrow["{\pi^2_{X,Y}\otimes Z}", from=1-5, to=2-3]
                               \arrow["{\explain{eq:monoidal_tensor_product}}"{description}, Rightarrow, draw=none, from=2-3, to=4]
                               \arrow["{\explain{eq:monoidal_tensor_product}}"{description}, Rightarrow, draw=none, from=3, to=2-3]
                               \arrow[Rightarrow, draw=none, from=4-4, to=0]
                               \arrow["{\explain{eq:monoidal_inv_a}}"{description}, Rightarrow, draw=none, from=4-4, to=2]
                               \arrow["{\explain{eq:monoidal_a}}"{description}, Rightarrow, draw=none, from=2-3, to=1]
      \end{tikzcd}
    \end{equation}
    Thus we conclude that the following commutes:
    \begin{equation}
      \label{eq:monoidal_a_iso_pi1}
      % https://q.uiver.app/?q=WzAsNSxbMCwxXSxbMCwyLCJYXFxvdGltZXMoWVxcb3RpbWVzIFopIl0sWzQsMCwiKFhcXG90aW1lcyBZKVxcb3RpbWVzIFoiXSxbNCw0LCIoWFxcb3RpbWVzIFkpXFxvdGltZXMgWiJdLFszLDIsIlhcXG90aW1lcyBZIl0sWzIsMywiMV97KFhcXG90aW1lcyBZKVxcb3RpbWVzIFp9IiwwLHsiY3VydmUiOi01fV0sWzIsMSwiXFxhbHBoYV97WCxZLFp9IiwyLHsiY3VydmUiOjV9XSxbMSwzLCJcXGFscGhhXnstMX1fe1gsWSxafSIsMix7ImN1cnZlIjo1fV0sWzEsNCwiMV9YXFxvdGltZXMgXFxwaV4xX3tZLFp9Il0sWzIsNCwiXFxwaV4xX3tYXFxvdGltZXMgWSxafSJdLFszLDQsIlxccGleMV97WFxcb3RpbWVzIFksWn0iXV0=
      \begin{tikzcd}
                &&&& {(X\otimes Y)\otimes Z} \\
                {} \\
        {X\otimes(Y\otimes Z)} &&& {X\otimes Y} \\
        \\
                               &&&& {(X\otimes Y)\otimes Z}
                               \arrow["{{(X\otimes Y)\otimes Z}}", bend left=30, from=1-5, to=5-5]
                               \arrow["{\alpha_{X,Y,Z}}"', bend right=30, from=1-5, to=3-1]
                               \arrow["{\alpha^{-1}_{X,Y,Z}}"', bend right=30, from=3-1, to=5-5]
                               \arrow["{\pi^1_{X\otimes Y,Z}}", from=1-5, to=3-4]
                               \arrow["{\pi^1_{X\otimes Y,Z}}", from=5-5, to=3-4]
      \end{tikzcd}
    \end{equation}
    Similarly, we have the diagram:
    \begin{equation}
      \label{eq:monoidal_a_iso_pi2}
      \begin{tikzcd}
                &&&& {(X\otimes Y)\otimes Z} \\
                \\
        {X \otimes (Y \otimes Z)} && {Y \otimes Z} & Z \\
                                  &&&&&& {} \\
                                  &&&& {(X\otimes Y)\otimes Z}
                                  \arrow["{\alpha_{X,Y,Z}}"', bend right=30, from=1-5, to=3-1]
                                  \arrow["{\inv\alpha_{X,Y,Z}}"', bend right=30, from=3-1, to=5-5]
                                  \arrow["{{(X\otimes Y)\otimes Z}}", bend left=30, from=1-5, to=5-5]
                                  \arrow["{\pi^2_{X,Y\otimes Z}}"', from=3-1, to=3-3]
                                  \arrow["{\pi^2_{Y,Z}}"', from=3-3, to=3-4]
                                  \arrow["{\pi^2_{X\otimes Y,Z}}", from=1-5, to=3-4]
                                  \arrow["{\pi^2_{X,Y}\otimes Z}"', bend right=12, from=1-5, to=3-3]
                                  \arrow["{\pi^2_{X\otimes Y,Z}}"', from=5-5, to=3-4]
      \end{tikzcd}
    \end{equation}
    Combining (\ref{eq:monoidal_a_iso_pi1}) and (\ref{eq:monoidal_a_iso_pi2}),
    (\ref{eq:monoidal_a_iso}) commutes. We omit the proof that
    $\alpha_{X,Y,Z}\circ\inv\alpha_{X,Y,Z}=1_{(X\otimes Y)\otimes Z}$
    and conclude that $\alpha_{X,Y,Z}$ is an isomorphism.

    To show naturality, we require that the following commutes:
    \begin{equation}
      \label{eq:monoidal_a-naturality}
      \begin{tikzcd}[row sep=huge, column sep=huge]
        \rr{X\otimes Y}\otimes Z
        \arrow{r}{\alpha_{X,Y,Z}}
        \arrow{d}{\rr{f\otimes g}\otimes h}&
        X \otimes \rr{Y\otimes Z}
        \arrow{d}{f \otimes \rr{g \otimes h}} \\
        \rr{X'\otimes Y'}\otimes Z'
        \arrow{r}{\alpha_{X',Y',Z'}} &
        X' \otimes \rr{Y' \otimes Z'}
      \end{tikzcd}
    \end{equation}

    Firstly, we consider the following diagram:
    \begin{equation*}
      \begin{tikzcd}
        {(X\otimes Y)\otimes Z} &&&& {(X'\otimes Y')\otimes Z'} \\
                                & {X\otimes Y} && {X'\otimes Y'} \\
                                \\
                                & X && {X'} \\
        {X\otimes(Y\otimes Z)} &&&& {X'\otimes(Y'\otimes Z')}
        \arrow["{\alpha_{X,Y,Z}}"', from=1-1, to=5-1]
        \arrow["{\alpha_{X',Y',Z'}}", from=1-5, to=5-5]
        \arrow["{f\otimes(g\otimes h)}"', from=5-1, to=5-5]
        \arrow["{(f\otimes g)\otimes h}", from=1-1, to=1-5]
        \arrow["{\pi^1_{X,Y}}"', from=2-2, to=4-2]
        \arrow["f"', from=4-2, to=4-4]
        \arrow["{\pi^1_{X',Y'}}", from=2-4, to=4-4]
        \arrow["{\pi^1_{X'\otimes Y',Z'}}", from=1-5, to=2-4]
        \arrow["{\pi^1_{X\otimes Y,Z}}"', from=1-1, to=2-2]
        \arrow["{\pi^1_{X,Y\otimes Z}}", from=5-1, to=4-2]
        \arrow["{\pi^1_{X',Y'\otimes Z'}}"', from=5-5, to=4-4]
        \arrow["{f\otimes g}", from=2-2, to=2-4]
      \end{tikzcd}
    \end{equation*}
    Here the left and right rectangles commute by~\ref{def:monoidal_associator}.
    Then remaining three rectangles then commute due to naturality of
    $\pi^1$.
    Secondly, we consider
    \begin{equation*}
      \begin{tikzcd}
        {(X\otimes Y)\otimes Z} &&&& {(X'\otimes Y')\otimes Z'} \\
        \\
                                & {Y\otimes Z} && {Y'\otimes Z'} \\
                                \\
        {X\otimes(Y\otimes Z)} &&&& {X'\otimes(Y'\otimes Z')}
        \arrow["{\alpha_{X,Y,Z}}"', from=1-1, to=5-1]
        \arrow["{\alpha_{X',Y',Z'}}", from=1-5, to=5-5]
        \arrow["{f\otimes(g\otimes h)}"', from=5-1, to=5-5]
        \arrow["{(f\otimes g)\otimes h}", from=1-1, to=1-5]
        \arrow["{\pi^2_{X,Y\otimes Z}}"', from=5-1, to=3-2]
        \arrow["{g\otimes h}"', from=3-2, to=3-4]
        \arrow["{\pi^2_{X',Y'\otimes Z'}}", from=5-5, to=3-4]
        \arrow["{\pi^2_{X,Y}\otimes Z}", from=1-1, to=3-2]
        \arrow["{\pi^2_{X',Y'}\otimes {Z'}}"', from=1-5, to=3-4]
      \end{tikzcd}
    \end{equation*}
    Here the two triangles are just the triangles in (\ref{eq:monoidal_a}).
    The rectangles commute due to functoriality of $\otimes$ and naturality of
    $\pi^2$.

    It follows that (\ref{eq:monoidal_a-naturality}) commutes, i.e. $\alpha_{X,Y,Z}$ is natural
    in all three arguments.
  \end{proof}
\end{lemma}

\begin{definition}[Unitors]
  For each $X\in\cat C$, define maps
  \begin{align*}
    \lambda_X = \pi^2_{1,X} : 1 \otimes X & \to X \\
    \rho_X = \pi^1_{X,1}: X \otimes 1     & \to X
  \end{align*}
\end{definition}

\begin{lemma}\label{eq:monoidal_lambda}
  The transformation $\lambda$  given by the maps $\lambda_X:1\otimes X\to X$ is a natural isomorphism.
  \begin{proof}
    For each $X\in{\cat C}$, define $\inv\lambda_X:X\to 1\otimes X$
    to be the unique arrow that makes the following commute:
    \begin{center}
      \begin{tikzcd}[row sep=huge, column sep=huge]
                &
                X
                \arrow{dl}{}
                \arrow{d}{\inv\lambda_X}
                \arrow{dr}{X}
                &
                \\
        1 &
        1 \otimes X
        \arrow{l}{\pi^1_{1,X}}
        \arrow{r}[swap]{\pi^2_{1,X}}&
        X
      \end{tikzcd}
    \end{center}
    We note that $1\in{\cat C}$ is terminal so $\pi^1_{1,X}$
    is the unique arrow $1\otimes X\to 1$. Further, $\pi^2_{1,X}=\lambda_X$
    so $\lambda_X\circ\inv\lambda_X = 1_X$. Similarly, it can be shown that
    $\inv\lambda_X\circ\lambda_X = 1_{1\otimes X}$.
    Thus $\lambda_X$ is an isomorphism.

    Noting $\lambda_X=\pi^2_{1,X}$ and $\lambda_Y=\pi^2_{1,X}$ it follows
    from \ref{eq:monoidal_tensor_product} that the following commutes:
    \begin{center}
      \begin{tikzcd}[row sep=huge, column sep=huge]
        1\otimes X
        \arrow{d}{1\otimes f}
        \arrow{r}{\lambda_X} &
        X
        \arrow{d}{f}\\
        1\otimes Y
        \arrow{r}{\lambda_Y} &
        Y
      \end{tikzcd}
    \end{center}
    Thus $\lambda$ is a natural isomorphism.
  \end{proof}
\end{lemma}

\begin{lemma}\label{eq:monoidal_rho}
  The transformation $\rho$ given by the maps $\rho:X\otimes 1\to X$ is a natural transformation.
  \begin{proof}
    Analogous to~\ref{eq:monoidal_lambda}.
  \end{proof}
\end{lemma}

\begin{lemma}\label{eq:monoidal}
  $(\cat C, \otimes, 1, \alpha, \lambda, \rho)$ forms a monoidal category.
  \begin{proof}
    Let $W,X,Y,Z\in\cat C$. We intend to show \ref{eq:monoidal_triangle}.
    Firstly, we consider the following diagram:
    \begin{equation}
      \label{eq:monoidal_triangle_pi1}
      % https://q.uiver.app/?q=WzAsNSxbMCwwLCIoWFxcb3RpbWVzIDEpXFxvdGltZXMgWSJdLFs2LDAsIlhcXG90aW1lcygxXFxvdGltZXMgWSkiXSxbMywzLCJYXFxvdGltZXMgWSJdLFs0LDEsIlgiXSxbMiwxLCJYXFxvdGltZXMgMSJdLFswLDEsIlxcYWxwaGFfe1gsMSxZfSJdLFsxLDIsIjFfWFxcb3RpbWVzXFxsYW1iZGFfWSIsMCx7ImN1cnZlIjotNX1dLFswLDIsIlxccmhvX1hcXG90aW1lcyAxX1kiLDIseyJjdXJ2ZSI6NX1dLFsyLDMsIlxccGleMV97WCxZfSJdLFsxLDMsIlxccGleMV97WCwxXFxvdGltZXMgWX0iXSxbMCw0LCJcXHBpXjFfe1hcXG90aW1lcyAxLCBZfSJdLFs0LDMsIlxccGleMV97WCwxfSJdXQ==
      \begin{tikzcd}
        {(X\otimes 1)\otimes Y} &&&&&& {X\otimes(1\otimes Y)} \\
                                && {X\otimes 1} && X \\
                                \\
                                &&& {X\otimes Y}
                                \arrow["{\alpha_{X,1,Y}}", from=1-1, to=1-7]
                                \arrow["{X\otimes\lambda_Y}", bend left=30, from=1-7, to=4-4]
                                \arrow["{\rho_X\otimes Y}"', bend right=30, from=1-1, to=4-4]
                                \arrow["{\pi^1_{X,Y}}", from=4-4, to=2-5]
                                \arrow["{\pi^1_{X,1\otimes Y}}", from=1-7, to=2-5]
                                \arrow["{\pi^1_{X\otimes 1, Y}}", from=1-1, to=2-3]
                                \arrow["{\pi^1_{X,1}}", from=2-3, to=2-5]
      \end{tikzcd}
    \end{equation}
    Here the top rectangle is just the rectangle in (\ref{eq:monoidal_a}) and the
    remaining commute due to naturality of $\pi^1$, noting the definition of $\rho_X$.
    Secondly, we have:
    \begin{equation}
      \label{eq:monoidal_triangle_pi2}
      % https://q.uiver.app/?q=WzAsNSxbMCwwLCIoWFxcb3RpbWVzIDEpXFxvdGltZXMgWSJdLFs2LDAsIlhcXG90aW1lcygxXFxvdGltZXMgWSkiXSxbMywzLCJYXFxvdGltZXMgWSJdLFsyLDIsIlkiXSxbNSwxLCIxXFxvdGltZXMgWSJdLFswLDEsIlxcYWxwaGFfe1gsMSxZfSJdLFsxLDIsIjFfWFxcb3RpbWVzXFxsYW1iZGFfWSIsMCx7ImN1cnZlIjotNX1dLFswLDIsIlxccmhvX1hcXG90aW1lcyAxX1kiLDIseyJjdXJ2ZSI6NX1dLFsxLDQsIlxccGleMl97WCwxXFxvdGltZXMgWX0iXSxbMiwzLCJcXHBpXjJfe1gsWX0iXSxbMCwzLCJcXHBpXjJfe1hcXG90aW1lcyAxLFl9Il0sWzQsMywiXFxwaV4yX3sxLFl9Il0sWzAsNCwiXFxwaV4yX3tYLDF9XFxvdGltZXMgMV9ZIl1d
      \begin{tikzcd}
        {(X\otimes 1)\otimes Y} &&&&&& {X\otimes(1\otimes Y)} \\
                                &&&&& {1\otimes Y} \\
                                && Y \\
                                &&& {X\otimes Y}
                                \arrow["{\alpha_{X,1,Y}}", from=1-1, to=1-7]
                                \arrow["{X\otimes\lambda_Y}", bend left=30, from=1-7, to=4-4]
                                \arrow["{\rho_X\otimes Y}"', bend right=30, from=1-1, to=4-4]
                                \arrow["{\pi^2_{X,1\otimes Y}}", from=1-7, to=2-6]
                                \arrow["{\pi^2_{X,Y}}", from=4-4, to=3-3]
                                \arrow["{\pi^2_{X\otimes 1,Y}}", from=1-1, to=3-3]
                                \arrow["{\pi^2_{1,Y}}", from=2-6, to=3-3]
                                \arrow["{\pi^2_{X,1}\otimes Y}", from=1-1, to=2-6]
      \end{tikzcd}
    \end{equation}
    Her the top top triangle is just the triangle in (\ref{eq:monoidal_a}) and the remaining
    commute due to naturality of $\pi^2$, once again noting the definition of $\lambda_Y$.
    Combining (\ref{eq:monoidal_triangle_pi1}) and (\ref{eq:monoidal_triangle_pi2}) we find
    that (\ref{eq:monoidal_triangle}) holds.

    We intend to show \ref{eq:monoidal_pentagon}.
    Firstly, we consider
    \begin{equation}
      \label{eq:monoidal_pentagon_pi1}
      % https://q.uiver.app/?q=WzAsOSxbMSwwLCIoV1xcb3RpbWVzIFgpXFxvdGltZXMoWVxcb3RpbWVzIFopIl0sWzAsMSwiKChXXFxvdGltZXMgWClcXG90aW1lcyBZKVxcb3RpbWVzIFoiXSxbMCw0LCIoV1xcb3RpbWVzIChYXFxvdGltZXMgWSkpXFxvdGltZXMgWiJdLFszLDQsIldcXG90aW1lcygoWFxcb3RpbWVzIFkpXFxvdGltZXMgWikiXSxbMywxLCJXXFxvdGltZXMgKFhcXG90aW1lcyAoWVxcb3RpbWVzIFopIl0sWzEsMiwiKFdcXG90aW1lcyBYKVxcb3RpbWVzIFkiXSxbMSwzLCJXXFxvdGltZXMoWFxcb3RpbWVzIFkpIl0sWzIsMywiVyJdLFsyLDIsIldcXG90aW1lcyBYIl0sWzEsMCwiXFxhbHBoYV97V1xcb3RpbWVzIFgsWSxafSJdLFsxLDIsIlxcYWxwaGFfe1csWCxZfVxcb3RpbWVzIFgiLDJdLFsyLDMsIlxcYWxwaGFfe1csWFxcb3RpbWVzIFksWn0iLDJdLFszLDQsIldcXG90aW1lcyBcXGFscGhhX3tYLFksWn0iLDJdLFswLDQsIlxcYWxwaGFfe1csWCxZXFxvdGltZXMgWn0iXSxbNSw4LCJcXHBpXjFfe1dcXG90aW1lcyBYLFl9Il0sWzYsNywiXFxwaV4xX3tXLFhcXG90aW1lcyBZfSJdLFs4LDcsIlxccGleMV97VyxYfSIsMl0sWzUsNiwiXFxhbHBoYV97VyxYLFl9IiwyXSxbMSw1LCJcXHBpXjFfeyhXXFxvdGltZXMgWClcXG90aW1lcyBZLFp9IiwyXSxbMiw2LCJcXHBpXjFfe1dcXG90aW1lcyhYXFxvdGltZXMgWSksWn0iXSxbMCw4LCJcXHBpXjFfe1dcXG90aW1lcyBYLFlcXG90aW1lcyBafSIsMl0sWzQsNywiXFxwaV4xX3tXLFhcXG90aW1lcyhZXFxvdGltZXMgWil9IiwyXSxbMyw3LCJcXHBpXjFfe1csKFhcXG90aW1lcyBZKVxcb3RpbWVzIFp9IiwyXV0=
      \begin{tikzcd}
  & {(W\otimes X)\otimes(Y\otimes Z)} \\
        {((W\otimes X)\otimes Y)\otimes Z} &&& {W\otimes (X\otimes (Y\otimes Z)} \\
                                           & {(W\otimes X)\otimes Y} & {W\otimes X} \\
                                           & {W\otimes(X\otimes Y)} & W \\
        {(W\otimes (X\otimes Y))\otimes Z} &&& {W\otimes((X\otimes Y)\otimes Z)}
        \arrow["{\alpha_{W\otimes X,Y,Z}}", from=2-1, to=1-2]
        \arrow["{\alpha_{W,X,Y}\otimes X}"', from=2-1, to=5-1]
        \arrow["{\alpha_{W,X\otimes Y,Z}}"', from=5-1, to=5-4]
        \arrow["{W\otimes \alpha_{X,Y,Z}}"', from=5-4, to=2-4]
        \arrow["{\alpha_{W,X,Y\otimes Z}}", from=1-2, to=2-4]
        \arrow["{\pi^1_{W\otimes X,Y}}", from=3-2, to=3-3]
        \arrow["{\pi^1_{W,X\otimes Y}}", from=4-2, to=4-3]
        \arrow["{\pi^1_{W,X}}"', from=3-3, to=4-3]
        \arrow["{\alpha_{W,X,Y}}"', from=3-2, to=4-2]
        \arrow["{\pi^1_{(W\otimes X)\otimes Y,Z}}"', from=2-1, to=3-2]
        \arrow["{\pi^1_{W\otimes(X\otimes Y),Z}}", from=5-1, to=4-2]
        \arrow["{\pi^1_{W\otimes X,Y\otimes Z}}"', from=1-2, to=3-3]
        \arrow["{\pi^1_{W,X\otimes(Y\otimes Z)}}"', from=2-4, to=4-3]
        \arrow["{\pi^1_{W,(X\otimes Y)\otimes Z}}"', from=5-4, to=4-3]
      \end{tikzcd}
    \end{equation}
    Here the rectangle on the left and the triangle on the right commute due to
    naturality of $\pi^1$. All the remaining rectangles are just the rectangle in
    (\ref{eq:monoidal_a}). Thus (\ref{eq:monoidal_pentagon_pi1}) commutes, i.e.
    \begin{align*}
      \pi^1_{W,X\otimes\rr{Y\otimes Z}} \circ W\otimes \alpha_{X,Y,Z} \circ \alpha_{W,X\otimes Y,Z} \circ \alpha_{W,X,Y}\otimes Z
      = \pi^1_{W,X\otimes\rr{Y\otimes Z}} \circ \alpha_{W,X,Y\otimes Z} \circ \alpha_{W\otimes X,Y,Z}
    \end{align*}
    Similarly, we consider
    \begin{equation}
      \label{eq:monoidal_pentagon_pi2}
      % https://q.uiver.app/?q=WzAsNyxbMiwwLCIoV1xcb3RpbWVzIFgpXFxvdGltZXMoWVxcb3RpbWVzIFopIl0sWzAsMSwiKChXXFxvdGltZXMgWClcXG90aW1lcyBZKVxcb3RpbWVzIFoiXSxbMCw0LCIoV1xcb3RpbWVzIChYXFxvdGltZXMgWSkpXFxvdGltZXMgWiJdLFs0LDQsIldcXG90aW1lcygoWFxcb3RpbWVzIFkpXFxvdGltZXMgWikiXSxbNCwxLCJXXFxvdGltZXMgKFhcXG90aW1lcyAoWVxcb3RpbWVzIFopIl0sWzMsMiwiWFxcb3RpbWVzKFlcXG90aW1lcyBaKSJdLFsyLDMsIihYXFxvdGltZXMgWSlcXG90aW1lcyBaIl0sWzEsMCwiXFxhbHBoYV97V1xcb3RpbWVzIFgsWSxafSJdLFsxLDIsIlxcYWxwaGFfe1csWCxZfVxcb3RpbWVzIDFfWCIsMl0sWzIsMywiXFxhbHBoYV97VyxYXFxvdGltZXMgWSxafSIsMl0sWzMsNCwiMV9XXFxvdGltZXMgXFxhbHBoYV97WCxZLFp9IiwyXSxbMCw0LCJcXGFscGhhX3tXLFgsWVxcb3RpbWVzIFp9Il0sWzIsNiwiXFxwaV4yX3tXLFhcXG90aW1lcyBZfVxcb3RpbWVzIDFfWiJdLFszLDYsIlxccGleMl97WFxcb3RpbWVzIFksIFp9IiwyXSxbNiw1LCJcXGFscGhhX3tYLFksWn0iLDJdLFsxLDYsIihcXHBpXjJfe1csWH1cXG90aW1lcyAxX1kpXFxvdGltZXMgMV9aIl0sWzAsNSwiXFxwaV4yX3tXLFh9XFxvdGltZXMgKDFfWVxcb3RpbWVzIDFfWikiLDJdLFs0LDUsIlxccGleMl97VyxYXFxvdGltZXMoWVxcb3RpbWVzIFopfSIsMl1d
      \begin{tikzcd}
                && {(W\otimes X)\otimes(Y\otimes Z)} \\
        {((W\otimes X)\otimes Y)\otimes Z} &&&& {W\otimes (X\otimes (Y\otimes Z)} \\
                                           &&& {X\otimes(Y\otimes Z)} \\
                                           && {(X\otimes Y)\otimes Z} \\
        {(W\otimes (X\otimes Y))\otimes Z} &&&& {W\otimes((X\otimes Y)\otimes Z)}
        \arrow["{\alpha_{W\otimes X,Y,Z}}", from=2-1, to=1-3]
        \arrow["{\alpha_{W,X,Y}\otimes X}"', from=2-1, to=5-1]
        \arrow["{\alpha_{W,X\otimes Y,Z}}"', from=5-1, to=5-5]
        \arrow["{W\otimes \alpha_{X,Y,Z}}"', from=5-5, to=2-5]
        \arrow["{\alpha_{W,X,Y\otimes Z}}", from=1-3, to=2-5]
        \arrow["{\pi^2_{W,X\otimes Y}\otimes Z}", from=5-1, to=4-3]
        \arrow["{\pi^2_{X\otimes Y, Z}}"', from=5-5, to=4-3]
        \arrow["{\alpha_{X,Y,Z}}"', from=4-3, to=3-4]
        \arrow["{(\pi^2_{W,X}\otimes Y)\otimes Z}", from=2-1, to=4-3]
        \arrow["{\pi^2_{W,X}\otimes ({Y\otimes Z})}"', from=1-3, to=3-4]
        \arrow["{\pi^2_{W,X\otimes(Y\otimes Z)}}"', from=2-5, to=3-4]
      \end{tikzcd}
    \end{equation}
    Here the leftmost triangle is just the triangle in (\ref{eq:monoidal_a}).
    The rectangle at the top follows due to naturality of $\alpha$ and the rest
    commutes due to naturality of $\pi^2$. Thus (\ref{eq:monoidal_pentagon_pi2})
    commutes.

    Combining (\ref{eq:monoidal_pentagon_pi1}) and (\ref{eq:monoidal_pentagon_pi2}) we
    find that (\ref{eq:monoidal_pentagon}) holds.
  \end{proof}
\end{lemma}

\begin{definition}[Braiding]\label{def:cartesian_braiding}
  For all $X,Y\in\cat C$, define
  \begin{align*}
    \gamma : X \otimes Y \to Y \otimes X
  \end{align*}
  to be the unique arrow that makes the following commute:
  \begin{equation}
    \label{eq:monoidal_y}
    \begin{tikzcd}[row sep=huge, column sep=huge]
            &
            X \otimes Y
            \arrow{dl}[swap]{\pi^2_{X,Y}}
            \arrow{d}{\gamma_{X,Y}}
            \arrow{dr}{\pi^1_{X,Y}}\\
      Y &
      Y \otimes X
      \arrow{l}{\pi^1_{Y,X}}
      \arrow{r}[swap]{\pi^2_{Y,X}} &
      X
    \end{tikzcd}
  \end{equation}
\end{definition}

\begin{proposition}
  \label{eq:monoidal-braiding}
  The transformation given by the maps $\gamma_{X,Y}: X\otimes Y\to Y\otimes X$ is a natural isomorphism in both
  arguments.
  \begin{proof}
    Let $X,X',Y,Y'\in\cat C$.
    The fact that $\gamma$ is an isomorphism is immediate as there exists
    a unique morphism
    \begin{align}
      \label{eq:monoidal_inv_y_y}
      \inv\gamma_{X,Y}=\gamma_{Y,X}:Y\otimes X\to X\otimes Y
    \end{align}
    that makes the following commute:
    \begin{equation*}
      \label{eq:monoidal_inv_y}
      \begin{tikzcd}[row sep=huge, column sep=huge]
                &
                Y \otimes X
                \arrow{dl}[swap]{\pi^2_{Y,X}}
                \arrow{d}{\inv\gamma_{X,Y}}
                \arrow{dr}{\pi^1_{Y,X}}\\
        X &
        X \otimes Y
        \arrow{l}{\pi^1_{X,Y}}
        \arrow{r}[swap]{\pi^2_{X,Y}} &
        Y
      \end{tikzcd}
    \end{equation*}
    Using~\ref{eq:monoidal_y} it is then straightforward to show
    $\inv\gamma_{X,Y}\circ\gamma_{X,Y} = {X\otimes Y}$ and
    $\gamma_{X,Y}\circ\inv\gamma_{X,Y} = {Y\otimes X}$.

    Further, let $f:X\to X'$ and $g:Y\to Y'$. Then the following two
    diagrams commute:
    \begin{equation*}
      % https://q.uiver.app/?q=WzAsNixbMCwwLCJYXFxvdGltZXMgWSJdLFszLDAsIlgnXFxvdGltZXMgWSciXSxbMywyLCJZJ1xcb3RpbWVzIFgnIl0sWzAsMiwiWVxcb3RpbWVzIFgiXSxbMSwxLCJYIl0sWzIsMSwiWCciXSxbMCwxLCJmXFxvdGltZXMgZyJdLFszLDIsImdcXG90aW1lcyBmIiwyXSxbMSwyLCJcXHVwc2lsb25fe1gnLFknfSJdLFswLDMsIlxcdXBzaWxvbl97WCxZfSIsMl0sWzAsNCwiXFxwaV4xX3tYLFl9Il0sWzQsNSwiZiJdLFsyLDUsIlxccGleMl97WScsWCd9Il0sWzMsNCwiXFxwaV4yX3tZLFh9IiwyXSxbMSw1LCJcXHBpXjFfe1gnLFknfSIsMl0sWzQsOSwiXFxleHBsYWlue2VxOm1vbm9pZGFsLXl9IiwxLHsic2hvcnRlbiI6eyJzb3VyY2UiOjIwLCJ0YXJnZXQiOjIwfSwic3R5bGUiOnsiYm9keSI6eyJuYW1lIjoibm9uZSJ9LCJoZWFkIjp7Im5hbWUiOiJub25lIn19fV0sWzgsNSwiXFxleHBsYWlue2VxOm1vbm9pZGFsLXl9IiwxLHsic2hvcnRlbiI6eyJzb3VyY2UiOjIwfSwic3R5bGUiOnsiYm9keSI6eyJuYW1lIjoibm9uZSJ9LCJoZWFkIjp7Im5hbWUiOiJub25lIn19fV0sWzcsMTEsIlxcZXhwbGFpbntlcTptb25vaWRhbC10ZW5zb3JfcHJvZHVjdH0iLDEseyJzaG9ydGVuIjp7InNvdXJjZSI6MjAsInRhcmdldCI6MjB9LCJzdHlsZSI6eyJib2R5Ijp7Im5hbWUiOiJub25lIn0sImhlYWQiOnsibmFtZSI6Im5vbmUifX19XSxbNiwxMSwiXFxleHBsYWlue2VxOm1vbm9pZGFsLXRlbnNvcl9wcm9kdWN0fSIsMSx7InNob3J0ZW4iOnsic291cmNlIjoyMCwidGFyZ2V0IjoyMH0sInN0eWxlIjp7ImJvZHkiOnsibmFtZSI6Im5vbmUifSwiaGVhZCI6eyJuYW1lIjoibm9uZSJ9fX1dXQ==
      \begin{tikzcd}
        {X\otimes Y} &&& {X'\otimes Y'} \\
                     & X & {X'} \\
        {Y\otimes X} &&& {Y'\otimes X'}
        \arrow[""{name=0, anchor=center, inner sep=0}, "{f\otimes g}", from=1-1, to=1-4]
        \arrow[""{name=1, anchor=center, inner sep=0}, "{g\otimes f}"', from=3-1, to=3-4]
        \arrow[""{name=2, anchor=center, inner sep=0}, "{\gamma_{X',Y'}}", from=1-4, to=3-4]
        \arrow[""{name=3, anchor=center, inner sep=0}, "{\gamma_{X,Y}}"', from=1-1, to=3-1]
        \arrow["{\pi^1_{X,Y}}", from=1-1, to=2-2]
        \arrow[""{name=4, anchor=center, inner sep=0}, "f", from=2-2, to=2-3]
        \arrow["{\pi^2_{Y',X'}}", from=3-4, to=2-3]
        \arrow["{\pi^2_{Y,X}}"', from=3-1, to=2-2]
        \arrow["{\pi^1_{X',Y'}}"', from=1-4, to=2-3]
        \arrow["{\explain{eq:monoidal_y}}"{description}, Rightarrow, draw=none, from=2-2, to=3]
        \arrow["{\explain{eq:monoidal_y}}"{description}, Rightarrow, draw=none, from=2, to=2-3]
        \arrow["{\explain{eq:monoidal_tensor_product}}"{description}, Rightarrow, draw=none, from=1, to=4]
        \arrow["{\explain{eq:monoidal_tensor_product}}"{description}, Rightarrow, draw=none, from=0, to=4]
      \end{tikzcd}
    \end{equation*}
    \begin{equation*}
      % https://q.uiver.app/?q=WzAsNixbMCwwLCJYXFxvdGltZXMgWSJdLFszLDAsIlgnXFxvdGltZXMgWSciXSxbMywyLCJZJ1xcb3RpbWVzIFgnIl0sWzAsMiwiWVxcb3RpbWVzIFgiXSxbMSwxLCJZIl0sWzIsMSwiWSciXSxbMCwxLCJmXFxvdGltZXMgZyJdLFszLDIsImdcXG90aW1lcyBmIiwyXSxbMSwyLCJcXHVwc2lsb25fe1gnLFknfSJdLFswLDMsIlxcdXBzaWxvbl97WCxZfSIsMl0sWzMsNCwiXFxwaV4xX3tZLFh9IiwyXSxbMCw0LCJcXHBpXjFfe1gsWX0iXSxbNCw1LCJnIl0sWzIsNSwiXFxwaV4xX3tZJyxYJ30iXSxbMSw1LCJcXHBpXjJfe1gnLFknfSIsMl0sWzYsMTIsIlxcZXhwbGFpbntlcTptb25vaWRhbC10ZW5zb3JfcHJvZHVjdH0iLDEseyJzaG9ydGVuIjp7InNvdXJjZSI6MjAsInRhcmdldCI6MjB9LCJzdHlsZSI6eyJib2R5Ijp7Im5hbWUiOiJub25lIn0sImhlYWQiOnsibmFtZSI6Im5vbmUifX19XSxbNywxMiwiXFxleHBsYWlue2VxOm1vbm9pZGFsLXRlbnNvcl9wcm9kdWN0fSIsMSx7InNob3J0ZW4iOnsic291cmNlIjoyMCwidGFyZ2V0IjoyMH0sInN0eWxlIjp7ImJvZHkiOnsibmFtZSI6Im5vbmUifSwiaGVhZCI6eyJuYW1lIjoibm9uZSJ9fX1dLFs4LDUsIlxcZXhwbGFpbntlcTptb25vaWRhbC15fSIsMSx7InNob3J0ZW4iOnsic291cmNlIjoyMH0sInN0eWxlIjp7ImJvZHkiOnsibmFtZSI6Im5vbmUifSwiaGVhZCI6eyJuYW1lIjoibm9uZSJ9fX1dLFs5LDQsIlxcZXhwbGFpbntlcTptb25vaWRhbC15fSIsMSx7InNob3J0ZW4iOnsic291cmNlIjoyMH0sInN0eWxlIjp7ImJvZHkiOnsibmFtZSI6Im5vbmUifSwiaGVhZCI6eyJuYW1lIjoibm9uZSJ9fX1dXQ==
      \begin{tikzcd}
        {X\otimes Y} &&& {X'\otimes Y'} \\
                     & Y & {Y'} \\
        {Y\otimes X} &&& {Y'\otimes X'}
        \arrow[""{name=0, anchor=center, inner sep=0}, "{f\otimes g}", from=1-1, to=1-4]
        \arrow[""{name=1, anchor=center, inner sep=0}, "{g\otimes f}"', from=3-1, to=3-4]
        \arrow[""{name=2, anchor=center, inner sep=0}, "{\gamma_{X',Y'}}", from=1-4, to=3-4]
        \arrow[""{name=3, anchor=center, inner sep=0}, "{\gamma_{X,Y}}"', from=1-1, to=3-1]
        \arrow["{\pi^1_{Y,X}}"', from=3-1, to=2-2]
        \arrow["{\pi^2_{X,Y}}", from=1-1, to=2-2]
        \arrow[""{name=4, anchor=center, inner sep=0}, "g", from=2-2, to=2-3]
        \arrow["{\pi^1_{Y',X'}}", from=3-4, to=2-3]
        \arrow["{\pi^2_{X',Y'}}"', from=1-4, to=2-3]
        \arrow["{\explain{eq:monoidal_tensor_product}}"{description}, Rightarrow, draw=none, from=0, to=4]
        \arrow["{\explain{eq:monoidal_tensor_product}}"{description}, Rightarrow, draw=none, from=1, to=4]
        \arrow["{\explain{eq:monoidal_y}}"{description}, Rightarrow, draw=none, from=2, to=2-3]
        \arrow["{\explain{eq:monoidal_y}}"{description}, Rightarrow, draw=none, from=3, to=2-2]
      \end{tikzcd}
    \end{equation*}
    Thus the outer diagram
    \begin{equation*}
      % https://q.uiver.app/?q=WzAsNCxbMCwwLCJYXFxvdGltZXMgWSJdLFszLDAsIlgnXFxvdGltZXMgWSciXSxbMywyLCJZJ1xcb3RpbWVzIFgnIl0sWzAsMiwiWVxcb3RpbWVzIFgiXSxbMCwxLCJmXFxvdGltZXMgZyJdLFszLDIsImdcXG90aW1lcyBmIiwyXSxbMSwyLCJcXHVwc2lsb25fe1gnLFknfSJdLFswLDMsIlxcdXBzaWxvbl97WCxZfSIsMl1d
      \begin{tikzcd}
        {X\otimes Y} &&& {X'\otimes Y'} \\
        \\
        {Y\otimes X} &&& {Y'\otimes X'}
        \arrow["{f\otimes g}", from=1-1, to=1-4]
        \arrow["{g\otimes f}"', from=3-1, to=3-4]
        \arrow["{\gamma_{X',Y'}}", from=1-4, to=3-4]
        \arrow["{\gamma_{X,Y}}"', from=1-1, to=3-1]
      \end{tikzcd}
    \end{equation*}
    commutes, showing naturality.
  \end{proof}
\end{proposition}


\begin{proposition}
  \label{eq:monoidal-symmetric}
  $(\cat C, \otimes, 1, a, \lambda, \rho, \gamma)$ forms a symmetric monoidal category.
  \begin{proof}
    We have the unit axiom
    \begin{equation*}
      % https://q.uiver.app/?q=WzAsMyxbMCwwLCJYXFxvdGltZXMgMSJdLFsyLDAsIjFcXG90aW1lcyBYIl0sWzEsMSwiWCJdLFswLDIsIlxccmhvX1giLDJdLFsxLDIsIlxcbGFtYmRhX1giXSxbMCwxLCJcXHVwc2lsb25fe1gsMX0iXSxbMiw1LCJcXGV4cGxhaW57ZXE6bW9ub2lkYWwteX0iLDEseyJzaG9ydGVuIjp7InRhcmdldCI6MjB9LCJzdHlsZSI6eyJib2R5Ijp7Im5hbWUiOiJub25lIn0sImhlYWQiOnsibmFtZSI6Im5vbmUifX19XV0=
      \begin{tikzcd}
        {X\otimes 1} && {1\otimes X} \\
                     & X
                     \arrow["{\rho_X}"', from=1-1, to=2-2]
                     \arrow["{\lambda_X}", from=1-3, to=2-2]
                     \arrow[""{name=0, anchor=center, inner sep=0}, "{\gamma_{X,1}}", from=1-1, to=1-3]
                     \arrow["{\explain{eq:monoidal_y}}"{description}, Rightarrow, draw=none, from=2-2, to=0]
      \end{tikzcd}
    \end{equation*}
    and the inverse axiom
    \begin{equation*}
      % https://q.uiver.app/?q=WzAsMyxbMCwwLCJYXFxvdGltZXMgWSJdLFsyLDAsIlhcXG90aW1lcyBZIl0sWzEsMSwiWVxcb3RpbWVzIFgiXSxbMCwxLCIxX3tYXFxvdGltZXMgWX0iXSxbMCwyLCJcXHVwc2lsb25fe1gsWX0iLDJdLFsyLDEsIlxcdXBzaWxvbl97WSxYfSIsMl0sWzIsMywiXFxleHBsYWlue2VxOm1vbm9pZGFsLXl9IiwxLHsic2hvcnRlbiI6eyJ0YXJnZXQiOjIwfSwic3R5bGUiOnsiYm9keSI6eyJuYW1lIjoibm9uZSJ9LCJoZWFkIjp7Im5hbWUiOiJub25lIn19fV1d
      \begin{tikzcd}
        {X\otimes Y} && {X\otimes Y} \\
                     & {Y\otimes X}
                     \arrow[""{name=0, anchor=center, inner sep=0}, "{{X\otimes Y}}", from=1-1, to=1-3]
                     \arrow["{\gamma_{X,Y}}"', from=1-1, to=2-2]
                     \arrow["{\gamma_{Y,X}}"', from=2-2, to=1-3]
                     \arrow["{\explain{eq:monoidal_inv_y_y}}"{description}, Rightarrow, draw=none, from=2-2, to=0]
      \end{tikzcd}
    \end{equation*}
    Now we find that the following commutes:
    \begin{equation*}
      % https://q.uiver.app/?q=WzAsMTAsWzAsMCwiKFhcXG90aW1lcyBZKVxcb3RpbWVzIFoiXSxbNiwwLCIoWVxcb3RpbWVzIFgpXFxvdGltZXMgWiJdLFswLDIsIlhcXG90aW1lcyhZXFxvdGltZXMgWikiXSxbMCw0LCIoWVxcb3RpbWVzIFopXFxvdGltZXMgWCJdLFs2LDQsIllcXG90aW1lcyhaXFxvdGltZXMgWCkiXSxbNiwyLCJZXFxvdGltZXMoWFxcb3RpbWVzIFopIl0sWzIsMywiWVxcb3RpbWVzIFoiXSxbNCwzLCJZIl0sWzQsMSwiWVxcb3RpbWVzIFgiXSxbMiwxLCJYXFxvdGltZXMgWSJdLFswLDIsIlxcYWxwaGFfe1hcXG90aW1lcyBZLFp9IiwyXSxbMiwzLCJcXHVwc2lsb25fe1gsWVxcb3RpbWVzIFp9IiwyXSxbMCwxLCJcXHVwc2lsb25fe1gsWX1cXG90aW1lcyAxX1oiXSxbMSw1LCJcXGFscGhhX3tZXFxvdGltZXMgWCxafSJdLFs1LDQsIjFfWVxcb3RpbWVzIFxcdXBzaWxvbl97WCxafSJdLFszLDQsIlxcYWxwaGFfe1lcXG90aW1lcyBaLFh9IiwyXSxbMCw2LCJcXHBpXjJfe1gsWX1cXG90aW1lcyAxX1oiXSxbMCw5LCJcXHBpXjFfe1hcXG90aW1lcyBZLFp9Il0sWzEsOCwiXFxwaV4xX3tZXFxvdGltZXMgWCxafSJdLFs5LDgsIlxcdXBzaWxvbl97WCxZfSJdLFs4LDcsIlxccGleMV97WSxYfSJdLFs5LDcsIlxccGleMl97WCxZfSJdLFs2LDcsIlxccGleMV97WSxafSJdLFs1LDcsIlxccGleMV97WSxYXFxvdGltZXMgWn0iXSxbNCw3LCJcXHBpXjFfe1ksWlxcb3RpbWVzIFh9Il0sWzMsNiwiXFxwaV4xX3tZXFxvdGltZXMgWixYfSIsMl0sWzIsNiwiXFxwaV4yX3tYLFlcXG90aW1lcyBafSJdLFsyMywxOCwiXFxleHBsYWlue2VxOm1vbm9pZGFsLWF9IiwxLHsic2hvcnRlbiI6eyJzb3VyY2UiOjIwLCJ0YXJnZXQiOjIwfSwic3R5bGUiOnsiYm9keSI6eyJuYW1lIjoibm9uZSJ9LCJoZWFkIjp7Im5hbWUiOiJub25lIn19fV0sWzE5LDEyLCJcXGV4cGxhaW5uYXR7XFxwaV4xfSIsMSx7InNob3J0ZW4iOnsic291cmNlIjoyMCwidGFyZ2V0IjoyMH0sInN0eWxlIjp7ImJvZHkiOnsibmFtZSI6Im5vbmUifSwiaGVhZCI6eyJuYW1lIjoibm9uZSJ9fX1dLFs3LDE0LCJcXGV4cGxhaW5uYXR7XFxwaV4xfSIsMSx7InNob3J0ZW4iOnsidGFyZ2V0IjoyMH0sInN0eWxlIjp7ImJvZHkiOnsibmFtZSI6Im5vbmUifSwiaGVhZCI6eyJuYW1lIjoibm9uZSJ9fX1dLFsxNiwyMSwiXFxleHBsYWlubmF0e1xccGleMX0iLDEseyJzaG9ydGVuIjp7InNvdXJjZSI6MjAsInRhcmdldCI6MjB9LCJzdHlsZSI6eyJib2R5Ijp7Im5hbWUiOiJub25lIn0sImhlYWQiOnsibmFtZSI6Im5vbmUifX19XSxbMiwxNiwiXFxleHBsYWlue2VxOm1vbm9pZGFsLWF9IiwxLHsic2hvcnRlbiI6eyJ0YXJnZXQiOjIwfSwic3R5bGUiOnsiYm9keSI6eyJuYW1lIjoibm9uZSJ9LCJoZWFkIjp7Im5hbWUiOiJub25lIn19fV0sWzExLDYsIlxcZXhwbGFpbmRlZntcXHVwc2lsb259IiwxLHsic2hvcnRlbiI6eyJzb3VyY2UiOjIwfSwic3R5bGUiOnsiYm9keSI6eyJuYW1lIjoibm9uZSJ9LCJoZWFkIjp7Im5hbWUiOiJub25lIn19fV0sWzE1LDIyLCJcXGV4cGxhaW57ZXE6bW9ub2lkYWwtYX0iLDEseyJzaG9ydGVuIjp7InNvdXJjZSI6MjAsInRhcmdldCI6MjB9LCJzdHlsZSI6eyJib2R5Ijp7Im5hbWUiOiJub25lIn0sImhlYWQiOnsibmFtZSI6Im5vbmUifX19XSxbMjEsOCwiXFxleHBsYWluZGVme1xcdXBzaWxvbn0iLDEseyJzaG9ydGVuIjp7InNvdXJjZSI6MjB9LCJzdHlsZSI6eyJib2R5Ijp7Im5hbWUiOiJub25lIn0sImhlYWQiOnsibmFtZSI6Im5vbmUifX19XV0=
      \begin{tikzcd}
        {(X\otimes Y)\otimes Z} &&&&&& {(Y\otimes X)\otimes Z} \\
                                && {X\otimes Y} && {Y\otimes X} \\
        {X\otimes(Y\otimes Z)} &&&&&& {Y\otimes(X\otimes Z)} \\
                               && {Y\otimes Z} && Y \\
        {(Y\otimes Z)\otimes X} &&&&&& {Y\otimes(Z\otimes X)}
        \arrow["{\alpha_{X\otimes Y,Z}}"', from=1-1, to=3-1]
        \arrow[""{name=0, anchor=center, inner sep=0}, "{\gamma_{X,Y\otimes Z}}"', from=3-1, to=5-1]
        \arrow[""{name=1, anchor=center, inner sep=0}, "{\gamma_{X,Y}\otimes Z}", from=1-1, to=1-7]
        \arrow["{\alpha_{Y\otimes X,Z}}", from=1-7, to=3-7]
        \arrow[""{name=2, anchor=center, inner sep=0}, "{Y\otimes \gamma_{X,Z}}", from=3-7, to=5-7]
        \arrow[""{name=3, anchor=center, inner sep=0}, "{\alpha_{Y\otimes Z,X}}"', from=5-1, to=5-7]
        \arrow[""{name=4, anchor=center, inner sep=0}, "{\pi^2_{X,Y}\otimes Z}", from=1-1, to=4-3]
        \arrow["{\pi^1_{X\otimes Y,Z}}", from=1-1, to=2-3]
        \arrow[""{name=5, anchor=center, inner sep=0}, "{\pi^1_{Y\otimes X,Z}}", from=1-7, to=2-5]
        \arrow[""{name=6, anchor=center, inner sep=0}, "{\gamma_{X,Y}}", from=2-3, to=2-5]
        \arrow["{\pi^1_{Y,X}}", from=2-5, to=4-5]
        \arrow[""{name=7, anchor=center, inner sep=0}, "{\pi^2_{X,Y}}", from=2-3, to=4-5]
        \arrow[""{name=8, anchor=center, inner sep=0}, "{\pi^1_{Y,Z}}", from=4-3, to=4-5]
        \arrow[""{name=9, anchor=center, inner sep=0}, "{\pi^1_{Y,X\otimes Z}}", from=3-7, to=4-5]
        \arrow["{\pi^1_{Y,Z\otimes X}}", from=5-7, to=4-5]
        \arrow["{\pi^1_{Y\otimes Z,X}}"', from=5-1, to=4-3]
        \arrow["{\pi^2_{X,Y\otimes Z}}", from=3-1, to=4-3]
        \arrow["{\explain{eq:monoidal_a}}"{description}, draw=none, from=9, to=5]
        \arrow["{\explainnat{\pi^1}}"{description}, draw=none, from=6, to=1]
        \arrow["{\explainnat{\pi^1}}"{description}, draw=none, from=4-5, to=2]
        \arrow["{\explainnat{\pi^1}}"{description}, draw=none, from=4, to=7]
        \arrow["{\explain{eq:monoidal_a}}"{description}, draw=none, from=3-1, to=4]
        \arrow["{\explaindef{\gamma}}"{description}, draw=none, from=0, to=4-3]
        \arrow["{\explain{eq:monoidal_a}}"{description}, draw=none, from=3, to=8]
        \arrow["{\explaindef{\gamma}}"{description}, draw=none, from=7, to=2-5]
      \end{tikzcd}
    \end{equation*}
    Now we note that
    \begin{equation*}
      % https://q.uiver.app/?q=WzAsNSxbMCwwLCIoWFxcb3RpbWVzIFkpXFxvdGltZXMgWiJdLFs2LDAsIlhcXG90aW1lcyhZXFxvdGltZXMgWikiXSxbMywzLCJYXFxvdGltZXMgWiJdLFszLDEsIlgiXSxbMiwxLCJYXFxvdGltZXMgWSJdLFswLDEsIlxcYWxwaGFfe1hcXG90aW1lcyBZLFp9Il0sWzEsMiwiMV9YXFxvdGltZXMgXFxwaV4yX3tZLFp9IiwwLHsiY3VydmUiOi01fV0sWzAsMiwiXFxwaV4xX3tYLFl9XFxvdGltZXMgMV9aIiwyLHsiY3VydmUiOjV9XSxbMiwzLCJcXHBpXjFfe1gsWn0iXSxbMSwzLCJcXHBpXjFfe1gsWVxcb3RpbWVzIFp9Il0sWzAsNCwiXFxwaV4xX3tYXFxvdGltZXMgWSxafSJdLFs0LDMsIlxccGleMV97WCxZfSJdLFszLDYsIlxcZXhwbGFpbm5hdHtcXHBpXjF9IiwxLHsic2hvcnRlbiI6eyJ0YXJnZXQiOjIwfSwic3R5bGUiOnsiYm9keSI6eyJuYW1lIjoibm9uZSJ9LCJoZWFkIjp7Im5hbWUiOiJub25lIn19fV0sWzUsMTEsIlxcZXhwbGFpbmRlZntcXGFscGhhfSIsMSx7InNob3J0ZW4iOnsic291cmNlIjoyMCwidGFyZ2V0IjoyMH0sInN0eWxlIjp7ImJvZHkiOnsibmFtZSI6Im5vbmUifSwiaGVhZCI6eyJuYW1lIjoibm9uZSJ9fX1dLFs3LDgsIlxcZXhwbGFpbm5hdHtcXHBpXjF9IiwxLHsic2hvcnRlbiI6eyJzb3VyY2UiOjIwLCJ0YXJnZXQiOjIwfSwic3R5bGUiOnsiYm9keSI6eyJuYW1lIjoibm9uZSJ9LCJoZWFkIjp7Im5hbWUiOiJub25lIn19fV1d
      \begin{tikzcd}
        {(X\otimes Y)\otimes Z} &&&&&& {X\otimes(Y\otimes Z)} \\
                                && {X\otimes Y} & X \\
                                \\
                                &&& {X\otimes Z}
                                \arrow[""{name=0, anchor=center, inner sep=0}, "{\alpha_{X\otimes Y,Z}}", from=1-1, to=1-7]
                                \arrow[""{name=1, anchor=center, inner sep=0}, "{X\otimes \pi^2_{Y,Z}}", bend left=30, from=1-7, to=4-4]
                                \arrow[""{name=2, anchor=center, inner sep=0}, "{\pi^1_{X,Y}\otimes Z}"', bend right=30, from=1-1, to=4-4]
                                \arrow[""{name=3, anchor=center, inner sep=0}, "{\pi^1_{X,Z}}", from=4-4, to=2-4]
                                \arrow["{\pi^1_{X,Y\otimes Z}}", from=1-7, to=2-4]
                                \arrow["{\pi^1_{X\otimes Y,Z}}", from=1-1, to=2-3]
                                \arrow[""{name=4, anchor=center, inner sep=0}, "{\pi^1_{X,Y}}", from=2-3, to=2-4]
                                \arrow["{\explainnat{\pi^1}}"{description}, draw=none, from=2-4, to=1]
                                \arrow["{\explaindef{\alpha}}"{description}, draw=none, from=0, to=4]
                                \arrow["{\explainnat{\pi^1}}"{description}, draw=none, from=2, to=3]
      \end{tikzcd}
    \end{equation*}
    and
    \begin{equation*}
      % https://q.uiver.app/?q=WzAsNSxbMCwwLCIoWFxcb3RpbWVzIFkpXFxvdGltZXMgWiJdLFs2LDAsIlhcXG90aW1lcyhZXFxvdGltZXMgWikiXSxbMywzLCJYXFxvdGltZXMgWiJdLFszLDEsIloiXSxbNCwxLCJZXFxvdGltZXMgWiJdLFswLDEsIlxcYWxwaGFfe1hcXG90aW1lcyBZLFp9Il0sWzEsMiwiMV9YXFxvdGltZXMgXFxwaV4yX3tZLFp9IiwwLHsiY3VydmUiOi01fV0sWzAsMiwiXFxwaV4xX3tYLFl9XFxvdGltZXMgMV9aIiwyLHsiY3VydmUiOjV9XSxbMCwzLCJcXHBpXjJfe1hcXG90aW1lcyBZLFp9Il0sWzEsNCwiXFxwaV4yX3tYLFlcXG90aW1lcyBafSIsMl0sWzIsMywiXFxwaV4yX3tYLFp9IiwyXSxbNCwzLCJcXHBpXjJfe1ksWn0iXSxbNywzLCJcXGV4cGxhaW5uYXR7XFxwaV4yfSIsMSx7InNob3J0ZW4iOnsic291cmNlIjoyMH0sInN0eWxlIjp7ImJvZHkiOnsibmFtZSI6Im5vbmUifSwiaGVhZCI6eyJuYW1lIjoibm9uZSJ9fX1dLFs1LDExLCJcXGV4cGxhaW5kZWZ7XFxpbnZcXGFscGhhfSIsMSx7InNob3J0ZW4iOnsic291cmNlIjoyMCwidGFyZ2V0IjoyMH0sInN0eWxlIjp7ImJvZHkiOnsibmFtZSI6Im5vbmUifSwiaGVhZCI6eyJuYW1lIjoibm9uZSJ9fX1dLFs2LDEwLCJcXGV4cGxhaW5uYXR7XFxwaV4yfSIsMSx7InNob3J0ZW4iOnsic291cmNlIjoyMH0sInN0eWxlIjp7ImJvZHkiOnsibmFtZSI6Im5vbmUifSwiaGVhZCI6eyJuYW1lIjoibm9uZSJ9fX1dXQ==
      \begin{tikzcd}
        {(X\otimes Y)\otimes Z} &&&&&& {X\otimes(Y\otimes Z)} \\
                                &&& Z & {Y\otimes Z} \\
                                \\
                                &&& {X\otimes Z}
                                \arrow[""{name=0, anchor=center, inner sep=0}, "{\alpha_{X\otimes Y,Z}}", from=1-1, to=1-7]
                                \arrow[""{name=1, anchor=center, inner sep=0}, "{X\otimes \pi^2_{Y,Z}}", bend left=30, from=1-7, to=4-4]
                                \arrow[""{name=2, anchor=center, inner sep=0}, "{\pi^1_{X,Y}\otimes Z}"', bend right=30, from=1-1, to=4-4]
                                \arrow["{\pi^2_{X\otimes Y,Z}}", from=1-1, to=2-4]
                                \arrow["{\pi^2_{X,Y\otimes Z}}"', from=1-7, to=2-5]
                                \arrow[""{name=3, anchor=center, inner sep=0}, "{\pi^2_{X,Z}}"', from=4-4, to=2-4]
                                \arrow[""{name=4, anchor=center, inner sep=0}, "{\pi^2_{Y,Z}}", from=2-5, to=2-4]
                                \arrow["{\explainnat{\pi^2}}"{description}, draw=none, from=2, to=2-4]
                                \arrow["{\explaindef{\inv\alpha}}"{description}, draw=none, from=0, to=4]
                                \arrow["{\explainnat{\pi^2}}"{description}, draw=none, from=1, to=3]
      \end{tikzcd}
    \end{equation*}
    commute. Thus we have
    \begin{equation}
      \label{eq:monoidal_symmetric_triangle}
      % https://q.uiver.app/?q=WzAsMyxbMCwwLCIoWFxcb3RpbWVzIFkpXFxvdGltZXMgWiJdLFs2LDAsIlhcXG90aW1lcyhZXFxvdGltZXMgWikiXSxbMywzLCJYXFxvdGltZXMgWiJdLFswLDEsIlxcYWxwaGFfe1hcXG90aW1lcyBZLFp9Il0sWzEsMiwiMV9YXFxvdGltZXMgXFxwaV4yX3tZLFp9Il0sWzAsMiwiXFxwaV4xX3tYLFl9XFxvdGltZXMgMV9aIiwyXV0=
      \begin{tikzcd}
        {(X\otimes Y)\otimes Z} &&&&&& {X\otimes(Y\otimes Z)} \\
        \\
        \\
                                &&& {X\otimes Z}
                                \arrow["{\alpha_{X\otimes Y,Z}}", from=1-1, to=1-7]
                                \arrow["{X\otimes \pi^2_{Y,Z}}", from=1-7, to=4-4]
                                \arrow["{\pi^1_{X,Y}\otimes Z}"', from=1-1, to=4-4]
      \end{tikzcd}
    \end{equation}
    Using this, we show the following:
    \begin{equation*}
      % https://q.uiver.app/?q=WzAsOCxbMCwwLCIoWFxcb3RpbWVzIFkpXFxvdGltZXMgWiJdLFs2LDAsIihZXFxvdGltZXMgWClcXG90aW1lcyBaIl0sWzAsMiwiWFxcb3RpbWVzKFlcXG90aW1lcyBaKSJdLFswLDQsIihZXFxvdGltZXMgWilcXG90aW1lcyBYIl0sWzYsNCwiWVxcb3RpbWVzKFpcXG90aW1lcyBYKSJdLFs2LDIsIllcXG90aW1lcyhYXFxvdGltZXMgWikiXSxbNCwzLCJaXFxvdGltZXMgWCJdLFs0LDIsIlhcXG90aW1lcyBaIl0sWzAsMiwiXFxhbHBoYV97WFxcb3RpbWVzIFksWn0iLDJdLFsyLDMsIlxcZ2FtbWFfe1gsWVxcb3RpbWVzIFp9IiwyXSxbMCwxLCJcXGdhbW1hX3tYLFl9XFxvdGltZXMgMV9aIl0sWzEsNSwiXFxhbHBoYV97WVxcb3RpbWVzIFgsWn0iXSxbNSw0LCIxX1lcXG90aW1lcyBcXGdhbW1hX3tYLFp9Il0sWzMsNCwiXFxhbHBoYV97WVxcb3RpbWVzIFosWH0iLDJdLFszLDYsIlxccGleMl97WSxafVxcb3RpbWVzIDFfWCJdLFs0LDYsIlxccGleMl97WSxaXFxvdGltZXMgWH0iLDJdLFs1LDcsIlxccGleMl97WSxYXFxvdGltZXMgWn0iXSxbNyw2LCJcXGdhbW1hX3tYLFp9Il0sWzEsNywiXFxwaV4yX3tZLFp9XFxvdGltZXMgMV9aIiwyXSxbMiw3LCIxX1hcXG90aW1lcyBcXHBpXjJfe1ksWn0iXSxbMCw3LCJcXHBpXjFfe1gsWX1cXG90aW1lcyAxX1oiXSxbMTMsNiwiXFxleHBsYWluZGVme1xcYWxwaGF9IiwxLHsic2hvcnRlbiI6eyJzb3VyY2UiOjIwfSwic3R5bGUiOnsiYm9keSI6eyJuYW1lIjoibm9uZSJ9LCJoZWFkIjp7Im5hbWUiOiJub25lIn19fV0sWzIsMTQsIlxcZXhwbGFpbmRlZntcXGdhbW1hfSIsMSx7InNob3J0ZW4iOnsidGFyZ2V0IjoyMH0sInN0eWxlIjp7ImJvZHkiOnsibmFtZSI6Im5vbmUifSwiaGVhZCI6eyJuYW1lIjoibm9uZSJ9fX1dLFsxNywxMiwiXFxleHBsYWlubmF0e1xccGleMn0iLDEseyJzaG9ydGVuIjp7InNvdXJjZSI6MjAsInRhcmdldCI6MjB9LCJzdHlsZSI6eyJib2R5Ijp7Im5hbWUiOiJub25lIn0sImhlYWQiOnsibmFtZSI6Im5vbmUifX19XSxbMTgsNSwiXFxleHBsYWluZGVme1xcYWxwaGF9IiwxLHsic2hvcnRlbiI6eyJzb3VyY2UiOjIwfSwic3R5bGUiOnsiYm9keSI6eyJuYW1lIjoibm9uZSJ9LCJoZWFkIjp7Im5hbWUiOiJub25lIn19fV0sWzcsMTAsIlxcZXhwbGFpbmRlZntcXGdhbW1hfSIsMSx7InNob3J0ZW4iOnsidGFyZ2V0IjoyMH0sInN0eWxlIjp7ImJvZHkiOnsibmFtZSI6Im5vbmUifSwiaGVhZCI6eyJuYW1lIjoibm9uZSJ9fX1dLFsyMCwyLCJcXGV4cGxhaW57ZXE6bW9ub2lkYWwtc3ltbWV0cmljLXRyaWFuZ2xlfSIsMSx7InNob3J0ZW4iOnsic291cmNlIjoyMH0sInN0eWxlIjp7ImJvZHkiOnsibmFtZSI6Im5vbmUifSwiaGVhZCI6eyJuYW1lIjoibm9uZSJ9fX1dXQ==
      \begin{tikzcd}
        {(X\otimes Y)\otimes Z} &&&&&& {(Y\otimes X)\otimes Z} \\
        \\
        {X\otimes(Y\otimes Z)} &&&& {X\otimes Z} && {Y\otimes(X\otimes Z)} \\
                               &&&& {Z\otimes X} \\
        {(Y\otimes Z)\otimes X} &&&&&& {Y\otimes(Z\otimes X)}
        \arrow["{\alpha_{X\otimes Y,Z}}"', from=1-1, to=3-1]
        \arrow["{\gamma_{X,Y\otimes Z}}"', from=3-1, to=5-1]
        \arrow[""{name=0, anchor=center, inner sep=0}, "{\gamma_{X,Y}\otimes Z}", from=1-1, to=1-7]
        \arrow["{\alpha_{Y\otimes X,Z}}", from=1-7, to=3-7]
        \arrow[""{name=1, anchor=center, inner sep=0}, "{Y\otimes \gamma_{X,Z}}", from=3-7, to=5-7]
        \arrow[""{name=2, anchor=center, inner sep=0}, "{\alpha_{Y\otimes Z,X}}"', from=5-1, to=5-7]
        \arrow[""{name=3, anchor=center, inner sep=0}, "{\pi^2_{Y,Z}\otimes X}", from=5-1, to=4-5]
        \arrow["{\pi^2_{Y,Z\otimes X}}"', from=5-7, to=4-5]
        \arrow["{\pi^2_{Y,X\otimes Z}}", from=3-7, to=3-5]
        \arrow[""{name=4, anchor=center, inner sep=0}, "{\gamma_{X,Z}}", from=3-5, to=4-5]
        \arrow[""{name=5, anchor=center, inner sep=0}, "{\pi^2_{Y,Z}\otimes Z}"', from=1-7, to=3-5]
        \arrow["{X\otimes \pi^2_{Y,Z}}", from=3-1, to=3-5]
        \arrow[""{name=6, anchor=center, inner sep=0}, "{\pi^1_{X,Y}\otimes Z}", from=1-1, to=3-5]
        \arrow["{\explaindef{\alpha}}"{description}, draw=none, from=2, to=4-5]
        \arrow["{\explaindef{\gamma}}"{description}, draw=none, from=3-1, to=3]
        \arrow["{\explainnat{\pi^2}}"{description}, draw=none, from=4, to=1]
        \arrow["{\explaindef{\alpha}}"{description}, draw=none, from=5, to=3-7]
        \arrow["{\explaindef{\gamma}}"{description}, draw=none, from=3-5, to=0]
        \arrow["{\explain{eq:monoidal_symmetric_triangle}}"{description}, draw=none, from=6, to=3-1]
      \end{tikzcd}
    \end{equation*}
    Thus \ref{eq:symmetric_monoidal_hexagon} holds.
  \end{proof}
\end{proposition}

\section{Symmetric monoidal 2-categories}\label{sec:symmetric_monoidal_2categories}

\begin{definition}
  A symmetric monoidal 2-category consists of
  \begin{enumerate}
    \item a 2-category $\bicat C$;
    \item a 2-functor $\otimes : \bicat C\times\bicat C\to\bicat C$;
    \item an object $I\in\bicat C$;
    \item a 2-natural transformations $\alpha,\lambda,\rho,\gamma$ with components
      \begin{align*}
        \alpha_{X,Y,Z}:(X\otimes Y)\otimes Z &\to X\otimes(Y\otimes Z),\\
        \lambda_X:I\otimes X &\to X,\\
        \rho_X:X\otimes I &\to X,\\
        \gamma_{X,Y}:X\otimes Y &\to Y\otimes X
      \end{align*}
  \end{enumerate}
  such that
  \begin{enumerate}
    \item for all $W,X,Y,Z\in\bicat C$,
      \begin{equation}
        % https://q.uiver.app/?q=WzAsNSxbMCwwLCIoKFdcXG90aW1lcyBYKVxcb3RpbWVzIFkpXFxvdGltZXMgWiJdLFsxLDAsIihXXFxvdGltZXMgKFhcXG90aW1lcyBZKSlcXG90aW1lcyBaIl0sWzIsMCwiV1xcb3RpbWVzICgoWFxcb3RpbWVzIFkpXFxvdGltZXMgWikiXSxbMiwxLCJXXFxvdGltZXMoWFxcb3RpbWVzKFlcXG90aW1lcyBaKSkiXSxbMCwxLCIoV1xcb3RpbWVzIFgpXFxvdGltZXMoWVxcb3RpbWVzIFopIl0sWzAsNCwiXFxhbHBoYSIsMl0sWzAsMSwiXFxhbHBoYVxcb3RpbWVzIFoiXSxbMSwyLCJcXGFscGhhIl0sWzQsMywiXFxhbHBoYSIsMl0sWzIsMywiV1xcb3RpbWVzIFxcYWxwaGEiXV0=
        \begin{tikzcd}
          {((W\otimes X)\otimes Y)\otimes Z} & {(W\otimes (X\otimes Y))\otimes Z} & {W\otimes ((X\otimes Y)\otimes Z)} \\
          {(W\otimes X)\otimes(Y\otimes Z)} && {W\otimes(X\otimes(Y\otimes Z))}
          \arrow["\alpha"', from=1-1, to=2-1]
          \arrow["{\alpha\otimes Z}", from=1-1, to=1-2]
          \arrow["\alpha", from=1-2, to=1-3]
          \arrow["\alpha"', from=2-1, to=2-3]
          \arrow["{W\otimes \alpha}", from=1-3, to=2-3]
        \end{tikzcd}
      \end{equation}
    \item for all $X,Y\in\bicat C$,
      \begin{equation}
        % https://q.uiver.app/?q=WzAsMyxbMCwwLCIoWFxcb3RpbWVzIEkpXFxvdGltZXMgWSJdLFsyLDAsIlhcXG90aW1lcyhJXFxvdGltZXMgWSkiXSxbMSwxLCJYXFxvdGltZXMgWSJdLFswLDEsIlxcYWxwaGEiXSxbMCwyLCJcXHJob1xcb3RpbWVzIFkiLDJdLFsxLDIsIlhcXG90aW1lc1xcbGFtYmRhIl1d
        \begin{tikzcd}
          {(X\otimes I)\otimes Y} && {X\otimes(I\otimes Y)} \\
                                  & {X\otimes Y}
                                  \arrow["\alpha", from=1-1, to=1-3]
                                  \arrow["{\rho\otimes Y}"', from=1-1, to=2-2]
                                  \arrow["X\otimes\lambda", from=1-3, to=2-2]
        \end{tikzcd}
      \end{equation}
    \item for all $X,Y\in\bicat C$,
      \begin{equation}
        % https://q.uiver.app/?q=WzAsMyxbMCwwLCIoSVxcb3RpbWVzIFgpXFxvdGltZXMgWSJdLFsyLDAsIklcXG90aW1lcyAoWFxcb3RpbWVzIFkpIl0sWzEsMSwiWFxcb3RpbWVzIFkiXSxbMCwyLCJcXGxhbWJkYVxcb3RpbWVzIFkiLDJdLFsxLDIsIlxcbGFtYmRhIl0sWzAsMSwiXFxhbHBoYSJdXQ==
        \begin{tikzcd}
          {(I\otimes X)\otimes Y} && {I\otimes (X\otimes Y)} \\
                                  & {X\otimes Y}
                                  \arrow["{\lambda\otimes Y}"', from=1-1, to=2-2]
                                  \arrow["\lambda", from=1-3, to=2-2]
                                  \arrow["\alpha", from=1-1, to=1-3]
        \end{tikzcd}
      \end{equation}
    \item for all $X,Y\in\bicat C$,
      \begin{equation}
        % https://q.uiver.app/?q=WzAsMyxbMCwwLCIoWFxcb3RpbWVzIFkpXFxvdGltZXMgSSJdLFsyLDAsIlhcXG90aW1lcyhZXFxvdGltZXMgSSkiXSxbMSwxLCJYXFxvdGltZXMgWSJdLFswLDIsIlxccmhvIiwyXSxbMSwyLCJYXFxvdGltZXMgXFxyaG8iXSxbMCwxLCJcXGFscGhhIl1d
        \begin{tikzcd}
          {(X\otimes Y)\otimes I} && {X\otimes(Y\otimes I)} \\
                                  & {X\otimes Y}
                                  \arrow["\rho"', from=1-1, to=2-2]
                                  \arrow["{X\otimes \rho}", from=1-3, to=2-2]
                                  \arrow["\alpha", from=1-1, to=1-3]
        \end{tikzcd}
      \end{equation}
  \end{enumerate}
\end{definition}



\end{document}
