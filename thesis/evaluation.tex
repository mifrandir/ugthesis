\chapter{Evaluation}

It is now time to reflect on what we have and have not achieved. We begin
by highlighting two alternative approaches that we could have taken to solve
the problem. We then assess the quality of our results. Finally, we focus
on the presentation of these results and the compromises involved.

\section{Small presheaves instead of relative pseudomonads}\label{sec:small_presheaves}

We worked towards extending synthetic measure theory to admit the presheaf
construction as a model by generalising to relative pseudomonads. This has made
it very difficult to develop the required theory but almost trivial to establish
the model.

While we believe that this path can lead to success, it may not be the
simplest. In the early stages of this project we were aiming to restrict
$\biCAT$ to a suitable cartesian closed 2-category on which the presheaf
construction is a pseudomonad. We tried to work with small presheaves on
locally presentable categories but this failed. After spending a significant
amount of time with this approach, we decided to follow the potentially longer
route of relative pseudomonads as it would allow us to make some progress
rightaway.

This does not mean, however, that it is impossible to restrict the presheaf
construction and the underlying category appropriately so that generalising
synthetic measure theory to pseudomonads is sufficient. In a sense, these
considerations are likely to lead to an entirely new set of problems:
formulating the theory would become signficantly easier, but constructing a
non-trivial model would be difficult.

\section{Strength as a natural transformation}

The next thing that has to be criticised is our choice of changing the extension
operator to incorporate strength, rather than adding the strength to a relative
pseudomonad as a suitable pseudonatural transformation. Whether this approach is
fruitful remains to be seen. However, there are several notable disadvantages.

Firstly, it requires a certain amount of reinventing the wheel. Rather than
sticking with the already established theory of relative pseudomonads, we had to
rewrite the definition entirely. While the result is similar, a lot of time and
effort went into making everything consistent. Given that we have not been able to
reap the benefits of the new structure, it is unclear whether this detour will
eventually pay off.

Secondly, the new structure is in many ways less elegant than usual relative
pseudomonads. For example, have a look a the 2-cell families $\bicell c$ and
$\bicell m$ in the case of the presheaf construction. The former involves
additional parts that are not required for the latter. This makes it more
difficult to distinguish important details from the overall noise.

\section{Quality of the definitions}

Inspecting where the complexity is inspires confidence. This is because
the largest diagrams arise whenever coends are involved. This is deceptive,
however. Showing coend-related results has, for the most part, been a
straightforward mechanical task. The vast majority of work went into the three
core definitions \ref{def:prestrong_inclusion_pseudomonad_structure},
\ref{def:prestrong_inclusion_pseudomonad_axioms}, and
\ref{def:strong_inclusion_pseudomonad_structure}. The fact that all the
developments related to our theory, in particular the proofs and the induced
structures, are short and succinct suggests that we have indeed developed a
suitable language to reason about strong relative pseudomonads.

\section{Presentation}

The complexity of the expressions has repeatedly led to diagrams whose width was
several times what could fit the page. We have therefore not been able to
include as much detail as we would have liked. This means that some proofs may
be harder to follow than is appropriate. In any case, we do not expect anyone to
be able to reconstruct every step of our developments without some pen and
paper.

This problem is not new to category theorists. There have been several
approaches to deal with large diagrams that usually require even more notational
shortcuts. See \cite{marmolejo2013} and \cite{saville2023} for some related
examples. While this would have allowed us to condense more information onto the
page, it would have also meant hiding a signficiant amount of complexity and
thus required long explanations as to what is going on. We are doubtful whether
this would have been possible within the scope of this report.

To aid the reader we have decided to include quiver links to some particularly
large diagrams. There are several reasons why this is not a solution that can be
relied upon, though. Firstly, there is the technical problem that the content of
those links is not strictly part of the report. Secondly, the quiver server
will, eventually, go offline. If we were to rely on the service, then a
significant part of the content would be lost. Of course, the latter problem
may be solved by installing quiver locally.

Machine verified proofs may be another way to maintain rigour while improving
the presentation. This way we would be able to hide some technical details in the
comforting knowledge that everything has been made to work as intended.
Those who are interested would be welcome to read the corresponding source
code. While this is in a sense the optimal solution it is also idealistic:
formalising our work like this would vastly increase our time investment.
This would be a particularly risky bet given that we have not been able
to show that things are going to work out in the end.

\section{Target audience}

This report is supposed to be targeted towards undergraduate students. Meeting
this requirement has been difficult. While there are a few courses that define
categories, functors, and natural transformations, category theory is not taught
at the university in its own right. This has led us to take some major
shortcuts. For example, we would have liked to investigate our attempt at
restricting our model as described in \ref{sec:small_presheaves}. Unfortunately,
this would have required us to define locally presentable categories from the
ground up which is impossible with the space that we are given.

While this report may not provide a comprehensive introduction to category
theory, it still offers valuable insights and contributions that can be
understood by readers with varying degrees of knowledge in the subject. To
fully comprehend the technical details of the report, only a basic understanding of
category theory is required. As a result, undergraduate students with some
familiarity with natural transformations should be able to understand almost
all the technical details of our work. The only exceptions are more technical
arguments such as the existence of coends in $\Set$, which have to be taken for
granted. However, our explanations make it possible for readers without prior
knowledge to follow the main ideas and results presented.
