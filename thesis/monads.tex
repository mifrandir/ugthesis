\chapter{The family of monads}

We give definitions of various types of monads which are referenced throughout the report,
adapted to our notation and needs. For strong monads, we restrict our attention to cartesian
categories (rather than general monoidal ones) as this is the only case we are interested in.
Similarly, we only consider pseudomonads between 2-categories.

These definitions are complete except for one: we omit the coherence
conditions of strong pseudofunctors and strong pseudomonads as writing
them down is very involved and they are not too relevant for this report.


\section{Monads}

This section is based on \cite{marmolejo2010}.
Fix a category $\cat{C}$.

\begin{definition}[see \ref{def:monad}]
  Let $\cat{C}$ be a category. A \emph{monad $T$ on $\cat{C}$} consists of
  \begin{enumerate}
    \item for each $X\in\cat{C}$, an object $TX\in\cat{C}$;
    \item for each $X\in\cat{C}$, an morphism $\eta_X : X \to TX$;
    \item for all $X,Y\in\cat{C}$, a map
      \begin{align*}
        \extend{\rr{-}}{*}_{X,Y}:\Hom\rr{X,TY}\to \Hom\rr{TX,TY}
      \end{align*}
  \end{enumerate}
  such that, for all $X,Y,Z\in{\cat{C}}$, $f:X\to TY$, and $g:Y\to TZ$,
  the following commute:
  \begin{equation}
    % https://q.uiver.app/?q=WzAsMixbMCwwLCJUWCJdLFsyLDAsIlRYIl0sWzAsMSwiXFxldGFeKiIsMix7ImN1cnZlIjozfV0sWzAsMSwiIiwwLHsiY3VydmUiOi0zLCJsZXZlbCI6Miwic3R5bGUiOnsiaGVhZCI6eyJuYW1lIjoibm9uZSJ9fX1dXQ==
    \begin{tikzcd}
      TX && TX
      \arrow["{\eta^*}"', curve={height=18pt}, from=1-1, to=1-3]
      \arrow[curve={height=-18pt}, Rightarrow, no head, from=1-1, to=1-3]
    \end{tikzcd}
    \hs
    % https://q.uiver.app/?q=WzAsMyxbMCwwLCJUWCJdLFsyLDAsIlRZIl0sWzIsMiwiVFoiXSxbMCwxLCJmXioiXSxbMSwyLCJnXioiXSxbMCwyLCIoZ14qZileKiIsMl1d
    \begin{tikzcd}
      TX && TY \\
      \\
         && TZ
         \arrow["{f^*}", from=1-1, to=1-3]
         \arrow["{g^*}", from=1-3, to=3-3]
         \arrow["{(g^*f)^*}"', from=1-1, to=3-3]
    \end{tikzcd}
    \hs
    % https://q.uiver.app/?q=WzAsMyxbMCwwLCJUWCJdLFsyLDAsIlRZIl0sWzIsMiwiVFoiXSxbMCwxLCJmXioiXSxbMSwyLCJnXioiXSxbMCwyLCIoZ14qZileKiIsMl1d
    \begin{tikzcd}
      TX && TY \\
      \\
         && TZ
         \arrow["{f^*}", from=1-1, to=1-3]
         \arrow["{g^*}", from=1-3, to=3-3]
         \arrow["{(g^*f)^*}"', from=1-1, to=3-3]
    \end{tikzcd}
  \end{equation}
\end{definition}

\section{Relative monads}

This section is based on \cite{altenkirch2015}. Fix a category $\cat{C}$.

\begin{definition}
  Let $\cat{J},\cat{C}$ be categories and let $J:\cat{J}\to\cat{C}$ be a functor.
  A \emph{relative monad $T$ over $J$} consists of
  \begin{enumerate}
    \item for each $X\in\cat{J}$, an object $TX\in\cat{C}$;
    \item for each $X\in\cat{J}$, an morphism $\eta_X : JX \to TX$;
    \item for all $X,Y\in\cat{J}$, a map
      \begin{align*}
        \extend{\rr{-}}{*}_{X,Y}:\Hom\rr{JX,TY}\to \Hom\rr{TX,TY}
      \end{align*}
  \end{enumerate}
  such that, for all $X,Y,Z\in{\cat{J}}$, $f:JX\to TY$, and $g:JY\to TZ$,
  the following commute:
  \begin{equation}
    % https://q.uiver.app/?q=WzAsMixbMCwwLCJUWCJdLFsyLDAsIlRYIl0sWzAsMSwiXFxldGFeKiIsMix7ImN1cnZlIjozfV0sWzAsMSwiIiwwLHsiY3VydmUiOi0zLCJsZXZlbCI6Miwic3R5bGUiOnsiaGVhZCI6eyJuYW1lIjoibm9uZSJ9fX1dXQ==
    \begin{tikzcd}
      TX && TX
      \arrow["{\eta^*}"', curve={height=18pt}, from=1-1, to=1-3]
      \arrow[curve={height=-18pt}, Rightarrow, no head, from=1-1, to=1-3]
    \end{tikzcd}
    \hs
    % https://q.uiver.app/?q=WzAsMyxbMCwwLCJKWCJdLFsyLDAsIlRZIl0sWzIsMiwiVFkiXSxbMCwxLCJcXGV0YSJdLFsxLDIsImZeKiJdLFswLDIsImYiLDJdXQ==
    \begin{tikzcd}
      JX && TY \\
      \\
         && TY
         \arrow["\eta", from=1-1, to=1-3]
         \arrow["{f^*}", from=1-3, to=3-3]
         \arrow["f"', from=1-1, to=3-3]
    \end{tikzcd}
    \hs
    % https://q.uiver.app/?q=WzAsMyxbMCwwLCJUWCJdLFsyLDAsIlRZIl0sWzIsMiwiVFoiXSxbMCwxLCJmXioiXSxbMSwyLCJnXioiXSxbMCwyLCIoZ14qZileKiIsMl1d
    \begin{tikzcd}
      TX && TY \\
      \\
         && TZ
         \arrow["{f^*}", from=1-1, to=1-3]
         \arrow["{g^*}", from=1-3, to=3-3]
         \arrow["{(g^*f)^*}"', from=1-1, to=3-3]
    \end{tikzcd}
  \end{equation}
\end{definition}

\section{Pseudomonads}\label{sec:pseudomonads}

Similar to monads, we adapt the no-iteration version of pseudomonads from \cite{marmolejo2010}.
We take inspiration from \cite{fiore2017} where some significant simplifications were
made to the definition.
We reconstruct some of the usual structure (as seen in \cite{saville2023}) because we will require it in \ref{sec:strong_pseudomonads}.

Fix a 2-category $\bicat{C}$.

\begin{definition}
  A \emph{pseudomonad $T$ on $\bicat{C}$} consists of
  \begin{enumerate}
    \item for every $X\in\bicat{C}$, an object $TX\in\bicat{C}$;
    \item for every $X,Y\in\bicat{C}$, an extension functor
      \begin{equation}
        (-)^* : \Hom\bb{X,TY}\to\Hom\bb{TX,TY};
      \end{equation}
    \item for every $X\in\bicat{C}$, a 1-cell $\eta_X : X\to TX$;
    \item for every $X\in\bicat{C}$, an invertible 2-cell
      $\bicell t_X : \eta^* \Rightarrow \id$;
    \item for all morphisms $f:X\to TY$ and $g:Y\to TZ$ in $\bicat{C}$,
      an invertible 2-cell
      \begin{equation}
        % https://q.uiver.app/?q=WzAsMyxbMCwwLCJUWCJdLFsyLDAsIlRZIl0sWzEsMSwiVFoiXSxbMSwyLCJnXioiXSxbMCwxLCJmXioiXSxbMCwyLCIoZ14qZileKiIsMl0sWzUsMywiXFxtYXRoYmYgbV97ZixnfSIsMCx7InNob3J0ZW4iOnsic291cmNlIjoyMCwidGFyZ2V0IjoyMH19XV0=
        \begin{tikzcd}
          TX && TY \\
             & TZ
             \arrow[""{name=0, anchor=center, inner sep=0}, "{g^*}", from=1-3, to=2-2]
             \arrow["{f^*}", from=1-1, to=1-3]
             \arrow[""{name=1, anchor=center, inner sep=0}, "{(g^*f)^*}"', from=1-1, to=2-2]
             \arrow["{\mathbf m_{f,g}}", shorten <=6pt, shorten >=6pt, Rightarrow, from=1, to=0]
        \end{tikzcd}
      \end{equation}
    \item for every morphism $f:X\to TY$ in $\bicat{C}$, an invertible 2-cell
      \begin{equation}
        % https://q.uiver.app/?q=WzAsMyxbMCwwLCJYIl0sWzIsMCwiVFgiXSxbMSwxLCJUWSJdLFsxLDIsImZeKiJdLFswLDEsIlxcZXRhIl0sWzAsMiwiZiIsMl0sWzUsMywiXFxtYXRoYmYgZV9mIiwwLHsic2hvcnRlbiI6eyJzb3VyY2UiOjIwLCJ0YXJnZXQiOjIwfX1dXQ==
        \begin{tikzcd}
          X && TX \\
            & TY
            \arrow[""{name=0, anchor=center, inner sep=0}, "{f^*}", from=1-3, to=2-2]
            \arrow["\eta", from=1-1, to=1-3]
            \arrow[""{name=1, anchor=center, inner sep=0}, "f"', from=1-1, to=2-2]
            \arrow["{\mathbf e_f}", shorten <=6pt, shorten >=6pt, Rightarrow, from=1, to=0]
        \end{tikzcd}
      \end{equation}
  \end{enumerate}
  such that
  \begin{itemize}
    \item $\bicell m$ is natural in both arguments;
    \item $\bicell e$ is natural;
    \item for all $f:X\to TY$,
      \begin{equation}
        % https://q.uiver.app/?q=WzAsNCxbMCwxLCJmXioiXSxbMiwxLCJmXipcXGV0YV4qIl0sWzIsMCwiKGZeKlxcZXRhKV4qIl0sWzAsMCwiZl4qIl0sWzEsMCwiZl4qXFxtYXRoYmYgdCJdLFsyLDEsIlxcbWF0aGJmIG0iXSxbMywyLCJcXG1hdGhiZiBlXioiXSxbMCwzLCIiLDIseyJsZXZlbCI6Miwic3R5bGUiOnsiaGVhZCI6eyJuYW1lIjoibm9uZSJ9fX1dXQ==
        \begin{tikzcd}
          {f^*} && {(f^*\eta)^*} \\
          {f^*} && {f^*\eta^*}
          \arrow["{f^*\mathbf t}", from=2-3, to=2-1]
          \arrow["{\mathbf m}", from=1-3, to=2-3]
          \arrow["{\mathbf e^*}", from=1-1, to=1-3]
          \arrow[Rightarrow, no head, from=2-1, to=1-1]
        \end{tikzcd}
      \end{equation}
    \item for 1-cells $f:W\to TX$, $g:X\to TY$, and $h:Y\to TZ$,
      \begin{equation}
        % https://q.uiver.app/?q=WzAsNSxbMSwwLCIoKGheKiBnKV4qIGYpXioiXSxbMCwxLCIoaF4qZ14qZileKiJdLFsyLDEsIihoXipnKV4qZl4qIl0sWzAsMiwiaF4qKGdeKmYpXioiXSxbMiwyLCJoXipnXipmXioiXSxbMCwyLCJcXG1hdGhiZiBtIl0sWzAsMSwiKFxcbWF0aGJmIG0gZileKiIsMl0sWzMsNCwiaF4qXFxtYXRoYmYgbSIsMl0sWzEsMywiXFxtYXRoYmYgbSIsMl0sWzIsNCwiXFxtYXRoYmYgbSBmXioiXV0=
        \begin{tikzcd}
  & {((h^* g)^* f)^*} \\
          {(h^*g^*f)^*} && {(h^*g)^*f^*} \\
          {h^*(g^*f)^*} && {h^*g^*f^*}
          \arrow["{\mathbf m}", from=1-2, to=2-3]
          \arrow["{(\mathbf m f)^*}"', from=1-2, to=2-1]
          \arrow["{h^*\mathbf m}"', from=3-1, to=3-3]
          \arrow["{\mathbf m}"', from=2-1, to=3-1]
          \arrow["{\mathbf m f^*}", from=2-3, to=3-3]
        \end{tikzcd}
      \end{equation}
  \end{itemize}
\end{definition}

Let $T$ be a pseudomonad on $\bicat{C}$.

\begin{enumerate}
  \item For each $X\in\bicat{C}$, we have the 2-cell $\bicell a_X$
    given by
    \begin{equation}
      % https://q.uiver.app/?q=WzAsMyxbMCwzLCJUWCJdLFsyLDAsIlReMlgiXSxbNCwzLCJUWCJdLFswLDEsIlRcXGV0YSIsMCx7ImN1cnZlIjotNX1dLFsxLDIsIlxcdGV4dHtpZH1eKiIsMSx7ImN1cnZlIjotNX1dLFswLDIsIiIsMSx7ImN1cnZlIjo1LCJsZXZlbCI6Miwic3R5bGUiOnsiaGVhZCI6eyJuYW1lIjoibm9uZSJ9fX1dLFswLDIsIihcXHRleHR7aWR9XipcXGV0YVxcZXRhKV4qIiwwLHsiY3VydmUiOi01fV0sWzAsMiwiXFxldGEiLDFdLFswLDEsIihcXGV0YVxcZXRhKV4qIl0sWzEsNiwiXFxtYXRoYmYgbV57LTF9IiwxLHsic2hvcnRlbiI6eyJ0YXJnZXQiOjIwfX1dLFs2LDcsIihcXG1hdGhiZiBlXnstMX1cXGV0YSleKiIsMSx7InNob3J0ZW4iOnsic291cmNlIjoyMCwidGFyZ2V0IjoyMH19XSxbNyw1LCJcXG1hdGhiZiB0IiwwLHsic2hvcnRlbiI6eyJzb3VyY2UiOjIwLCJ0YXJnZXQiOjIwfX1dXQ==
      \begin{tikzcd}
  && {T^2X} \\
  \\
  \\
        TX &&&& TX
        \arrow["T\eta", curve={height=-30pt}, from=4-1, to=1-3]
        \arrow["{\text{id}^*}"{description}, curve={height=-30pt}, from=1-3, to=4-5]
        \arrow[""{name=0, anchor=center, inner sep=0}, curve={height=30pt}, Rightarrow, no head, from=4-1, to=4-5]
        \arrow[""{name=1, anchor=center, inner sep=0}, "{(\text{id}^*\eta\eta)^*}", curve={height=-30pt}, from=4-1, to=4-5]
        \arrow[""{name=2, anchor=center, inner sep=0}, "\eta"{description}, from=4-1, to=4-5]
        \arrow["{(\eta\eta)^*}", from=4-1, to=1-3]
        \arrow["{\mathbf m^{-1}}"{description}, shorten >=8pt, Rightarrow, from=1-3, to=1]
        \arrow["{(\mathbf e^{-1}\eta)^*}"{description}, shorten <=4pt, shorten >=4pt, Rightarrow, from=1, to=2]
        \arrow["{\mathbf t}", shorten <=4pt, shorten >=4pt, Rightarrow, from=2, to=0]
      \end{tikzcd}
    \end{equation}
  \item For each $X\in\bicat{C}$, we have the 2-cell $\bicell b_X$
    given by
    \begin{equation}
      % https://q.uiver.app/?q=WzAsNixbMCwwLCJUXjMgWCJdLFswLDMsIlReMlgiXSxbNCwyLCJUWCJdLFs0LDAsIlReMlgiXSxbNCwzLCJUWCJdLFswLDEsIlReM1giXSxbMCwzLCJcXHRleHR7aWR9XioiXSxbMywyLCJcXHRleHR7aWR9XioiXSxbMCwyLCIoXFx0ZXh0e2lkfV4qKV4qIiwxXSxbMSw0LCJcXHRleHR7aWR9XioiLDJdLFs1LDEsIlQoXFx0ZXh0e2lkfV4qKSIsMl0sWzUsNCwiKFxcdGV4dHtpZH1eKlxcZXRhXFx0ZXh0e2lkfV4qKV4qIiwxXSxbMiw0LCIiLDEseyJsZXZlbCI6Miwic3R5bGUiOnsiaGVhZCI6eyJuYW1lIjoibm9uZSJ9fX1dLFswLDUsIiIsMSx7ImxldmVsIjoyLCJzdHlsZSI6eyJoZWFkIjp7Im5hbWUiOiJub25lIn19fV0sWzgsMywiXFxtYXRoYmYgbSIsMSx7InNob3J0ZW4iOnsic291cmNlIjoyMCwidGFyZ2V0IjoyMH19XSxbMSwxMSwiXFxtYXRoYmYgbV57LTF9IiwxLHsic2hvcnRlbiI6eyJzb3VyY2UiOjIwLCJ0YXJnZXQiOjIwfX1dLFsxMSw4LCIoXFxtYXRoYmYgZV57LTF9XFx0ZXh0e2lkfV4qKV4qIiwxLHsic2hvcnRlbiI6eyJzb3VyY2UiOjIwLCJ0YXJnZXQiOjIwfX1dXQ==
      \begin{tikzcd}
        {T^3 X} &&&& {T^2X} \\
        {T^3X} \\
                &&&& TX \\
        {T^2X} &&&& TX
        \arrow["{\text{id}^*}", from=1-1, to=1-5]
        \arrow["{\text{id}^*}", from=1-5, to=3-5]
        \arrow[""{name=0, anchor=center, inner sep=0}, "{(\text{id}^*)^*}"{description}, from=1-1, to=3-5]
        \arrow["{\text{id}^*}"', from=4-1, to=4-5]
        \arrow["{T(\text{id}^*)}"', from=2-1, to=4-1]
        \arrow[""{name=1, anchor=center, inner sep=0}, "{(\text{id}^*\eta\text{id}^*)^*}"{description}, from=2-1, to=4-5]
        \arrow[Rightarrow, no head, from=3-5, to=4-5]
        \arrow[Rightarrow, no head, from=1-1, to=2-1]
        \arrow["{\mathbf m}"{description}, shorten <=11pt, shorten >=11pt, Rightarrow, from=0, to=1-5]
        \arrow["{\mathbf m^{-1}}"{description}, shorten <=11pt, shorten >=11pt, Rightarrow, from=4-1, to=1]
        \arrow["{(\mathbf e^{-1}\text{id}^*)^*}"{description}, shorten <=4pt, shorten >=4pt, Rightarrow, from=1, to=0]
      \end{tikzcd}
    \end{equation}
\end{enumerate}

\section{Relative pseudomonads}

This section is based on \cite{fiore2017}.

Fix a pseudofunctor $J:\bicat{J}\to\bicat{C}$ between 2-categories.
Strong pseudomonads and premonoidal bicategories
\begin{definition}
  A \emph{relative pseudomonad $T$ over $J$} consists of
  \begin{enumerate}
    \item for every $X\in\bicat{J}$, an object $TX\in\bicat{C}$;
    \item for every $X,Y\in\bicat{J}$, an extension functor
      \begin{equation}
        (-)^* : \Hom\bb{JX,TY}\to\Hom\bb{TX,TY};
      \end{equation}
    \item for every $X\in\bicat{C}$, a 1-cell $\eta_X : JX\to TX$;
    \item for every $X\in\bicat{C}$, an invertible 2-cell
      $\bicell t_X : \eta^* \Rightarrow \id$;
    \item for all morphisms $f:X\to TY$ and $g:Y\to TZ$ in $\bicat{C}$,
      an invertible 2-cell
      \begin{equation}
        % https://q.uiver.app/?q=WzAsMyxbMCwwLCJUWCJdLFsyLDAsIlRZIl0sWzEsMSwiVFoiXSxbMSwyLCJnXioiXSxbMCwxLCJmXioiXSxbMCwyLCIoZ14qZileKiIsMl0sWzUsMywiXFxtYXRoYmYgbV97ZixnfSIsMCx7InNob3J0ZW4iOnsic291cmNlIjoyMCwidGFyZ2V0IjoyMH19XV0=
        \begin{tikzcd}
          TX && TY \\
             & TZ
             \arrow[""{name=0, anchor=center, inner sep=0}, "{g^*}", from=1-3, to=2-2]
             \arrow["{f^*}", from=1-1, to=1-3]
             \arrow[""{name=1, anchor=center, inner sep=0}, "{(g^*f)^*}"', from=1-1, to=2-2]
             \arrow["{\mathbf m_{f,g}}", shorten <=6pt, shorten >=6pt, Rightarrow, from=1, to=0]
        \end{tikzcd}
      \end{equation}
    \item for every morphism $f:X\to TY$ in $\bicat{C}$, an invertible 2-cell
      \begin{equation}
        % https://q.uiver.app/?q=WzAsMyxbMCwwLCJYIl0sWzIsMCwiVFgiXSxbMSwxLCJUWSJdLFsxLDIsImZeKiJdLFswLDEsIlxcZXRhIl0sWzAsMiwiZiIsMl0sWzUsMywiXFxtYXRoYmYgZV9mIiwwLHsic2hvcnRlbiI6eyJzb3VyY2UiOjIwLCJ0YXJnZXQiOjIwfX1dXQ==
        \begin{tikzcd}
          X && TX \\
            & TY
            \arrow[""{name=0, anchor=center, inner sep=0}, "{f^*}", from=1-3, to=2-2]
            \arrow["\eta", from=1-1, to=1-3]
            \arrow[""{name=1, anchor=center, inner sep=0}, "f"', from=1-1, to=2-2]
            \arrow["{\mathbf e_f}", shorten <=6pt, shorten >=6pt, Rightarrow, from=1, to=0]
        \end{tikzcd}
      \end{equation}
  \end{enumerate}
  such that
  \begin{itemize}
    \item $\bicell m$ is natural in both arguments;
    \item $\bicell e$ is natural;
    \item for all $f:X\to TY$,
      \begin{equation}
        % https://q.uiver.app/?q=WzAsNCxbMCwxLCJmXioiXSxbMiwxLCJmXipcXGV0YV4qIl0sWzIsMCwiKGZeKlxcZXRhKV4qIl0sWzAsMCwiZl4qIl0sWzEsMCwiZl4qXFxtYXRoYmYgdCJdLFsyLDEsIlxcbWF0aGJmIG0iXSxbMywyLCJcXG1hdGhiZiBlXioiXSxbMCwzLCIiLDIseyJsZXZlbCI6Miwic3R5bGUiOnsiaGVhZCI6eyJuYW1lIjoibm9uZSJ9fX1dXQ==
        \begin{tikzcd}
          {f^*} && {(f^*\eta)^*} \\
          {f^*} && {f^*\eta^*}
          \arrow["{f^*\mathbf t}", from=2-3, to=2-1]
          \arrow["{\mathbf m}", from=1-3, to=2-3]
          \arrow["{\mathbf e^*}", from=1-1, to=1-3]
          \arrow[Rightarrow, no head, from=2-1, to=1-1]
        \end{tikzcd}
      \end{equation}
    \item for 1-cells $f:JW\to TX$, $g:JX\to TY$, and $h:JY\to TZ$,
      \begin{equation}
        % https://q.uiver.app/?q=WzAsNSxbMSwwLCIoKGheKiBnKV4qIGYpXioiXSxbMCwxLCIoaF4qZ14qZileKiJdLFsyLDEsIihoXipnKV4qZl4qIl0sWzAsMiwiaF4qKGdeKmYpXioiXSxbMiwyLCJoXipnXipmXioiXSxbMCwyLCJcXG1hdGhiZiBtIl0sWzAsMSwiKFxcbWF0aGJmIG0gZileKiIsMl0sWzMsNCwiaF4qXFxtYXRoYmYgbSIsMl0sWzEsMywiXFxtYXRoYmYgbSIsMl0sWzIsNCwiXFxtYXRoYmYgbSBmXioiXV0=
        \begin{tikzcd}
  & {((h^* g)^* f)^*} \\
          {(h^*g^*f)^*} && {(h^*g)^*f^*} \\
          {h^*(g^*f)^*} && {h^*g^*f^*}
          \arrow["{\mathbf m}", from=1-2, to=2-3]
          \arrow["{(\mathbf m f)^*}"', from=1-2, to=2-1]
          \arrow["{h^*\mathbf m}"', from=3-1, to=3-3]
          \arrow["{\mathbf m}"', from=2-1, to=3-1]
          \arrow["{\mathbf m f^*}", from=2-3, to=3-3]
        \end{tikzcd}
      \end{equation}
  \end{itemize}
\end{definition}

\section{Strong monads}

This section is based on \cite{kock1970}. Fix a cartesian category $\cat{C}$.

\begin{definition}\label{def:strong_monad}
  A \emph{strength} for a monad $T$ on $\cat{C}$ is a natural transformation $\sigma$
  with components
  \begin{equation}
    \sigma_{X,Y} : X\times TY \to T(X\times Y)
  \end{equation}
  such that
  \begin{enumerate}
    \item for all $X\in\cat{C}$,
      \begin{equation}
        % https://q.uiver.app/?q=WzAsMyxbMCwwLCIxXFx0aW1lcyBUWCJdLFsyLDAsIlQoMVxcdGltZXMgWCkiXSxbMiwyLCJUWCJdLFswLDIsIlxcbGFtYmRhIiwyXSxbMSwyLCJUXFxsYW1iZGEiXSxbMCwxLCJcXHNpZ21hIl1d
        \begin{tikzcd}
          {1\times TX} && {T(1\times X)} \\
          \\
                       && TX
                       \arrow["\lambda"', from=1-1, to=3-3]
                       \arrow["T\lambda", from=1-3, to=3-3]
                       \arrow["\sigma", from=1-1, to=1-3]
        \end{tikzcd}
      \end{equation}
    \item for all $X,Y,Z\in\cat{C}$,
      \begin{equation}
        % https://q.uiver.app/?q=WzAsNSxbMCwwLCIoWFxcdGltZXMgWSlcXHRpbWVzIFRaIl0sWzQsMCwiVCgoWFxcdGltZXMgWSlcXHRpbWVzIFopIl0sWzQsMSwiVChYXFx0aW1lcyhZXFx0aW1lcyBaKSkiXSxbMiwxLCJYXFx0aW1lcyBUKFlcXHRpbWVzIFopIl0sWzAsMSwiWFxcdGltZXMgKFlcXHRpbWVzIFRaKSJdLFswLDQsIlxcYWxwaGEiLDJdLFsxLDIsIlRcXGFscGhhIl0sWzMsMiwiXFxzaWdtYSIsMl0sWzQsMywiWFxcdGltZXNcXHNpZ21hIiwyXSxbMCwxLCJcXHNpZ21hIl1d
        \begin{tikzcd}
          {(X\times Y)\times TZ} &&&& {T((X\times Y)\times Z)} \\
          {X\times (Y\times TZ)} && {X\times T(Y\times Z)} && {T(X\times(Y\times Z))}
          \arrow["\alpha"', from=1-1, to=2-1]
          \arrow["T\alpha", from=1-5, to=2-5]
          \arrow["\sigma"', from=2-3, to=2-5]
          \arrow["X\times\sigma"', from=2-1, to=2-3]
          \arrow["\sigma", from=1-1, to=1-5]
        \end{tikzcd}
      \end{equation}
    \item for all $X,Y\in\cat{C}$,
      \begin{equation}
        % https://q.uiver.app/?q=WzAsMyxbMCwwLCJYXFx0aW1lcyBZIl0sWzIsMCwiWFxcdGltZXMgVFkiXSxbMiwyLCJUKFhcXHRpbWVzIFkpIl0sWzAsMSwiWFxcdGltZXNcXGV0YSJdLFswLDIsIlxcZXRhIiwyXSxbMSwyLCJcXHNpZ21hIl1d
        \begin{tikzcd}
          {X\times Y} && {X\times TY} \\
          \\
                      && {T(X\times Y)}
                      \arrow["X\times\eta", from=1-1, to=1-3]
                      \arrow["\eta"', from=1-1, to=3-3]
                      \arrow["\sigma", from=1-3, to=3-3]
        \end{tikzcd}
      \end{equation}
    \item for all $X,Y\in\cat{C}$,
      \begin{equation}
        % https://q.uiver.app/?q=WzAsNSxbMCwwLCJYXFx0aW1lcyBUXjJZIl0sWzIsMCwiVChYXFx0aW1lcyBUWSkiXSxbNCwwLCJUXjIgKFhcXHRpbWVzIFkpIl0sWzQsMSwiVChYXFx0aW1lcyBZKSJdLFswLDEsIlhcXHRpbWVzIFRZIl0sWzQsMywiXFxzaWdtYSIsMl0sWzIsMywiXFx0ZXh0e2lkfV4qIl0sWzEsMiwiVFxcc2lnbWEiXSxbMCw0LCJYXFx0aW1lcyBcXHRleHR7aWR9XioiLDJdLFswLDEsIlxcc2lnbWEiXV0=
        \begin{tikzcd}
          {X\times T^2Y} && {T(X\times TY)} && {T^2 (X\times Y)} \\
          {X\times TY} &&&& {T(X\times Y)}
          \arrow["\sigma"', from=2-1, to=2-5]
          \arrow["{\text{id}^*}", from=1-5, to=2-5]
          \arrow["T\sigma", from=1-3, to=1-5]
          \arrow["{X\times \text{id}^*}"', from=1-1, to=2-1]
          \arrow["\sigma", from=1-1, to=1-3]
        \end{tikzcd}
      \end{equation}
  \end{enumerate}
  A \emph{strong monad} is a monad that is equipped with a strength.
\end{definition}

\begin{definition}
  The \emph{costrength} of a strong monad $T$ is the natural
  transformation $\tau$ given by components
  \begin{equation}
    % https://q.uiver.app/?q=WzAsNCxbMCwwLCJUWFxcdGltZXMgWSJdLFsyLDAsIlQoWFxcdGltZXMgWSkiXSxbMCwxLCJZXFx0aW1lcyBUWCJdLFsyLDEsIlQoWVxcdGltZXMgWCkiXSxbMiwzLCJcXHNpZ21hIiwyXSxbMywxLCJUXFxnYW1tYSIsMl0sWzAsMiwiXFxnYW1tYSIsMl0sWzAsMSwiXFx0YXUiXV0=
    \begin{tikzcd}
      {TX\times Y} && {T(X\times Y)} \\
      {Y\times TX} && {T(Y\times X)}
      \arrow["\sigma"', from=2-1, to=2-3]
      \arrow["T\gamma"', from=2-3, to=1-3]
      \arrow["\gamma"', from=1-1, to=2-1]
      \arrow["\tau", from=1-1, to=1-3]
    \end{tikzcd}
  \end{equation}
\end{definition}

\begin{definition}
  A strong monad $T$ on $\cat{C}$ is commutative if, for all $X,Y\in\cat{C}$,
  \begin{equation}
    % https://q.uiver.app/?q=WzAsNixbMCwwLCJUWFxcdGltZXMgVFkiXSxbMiwwLCJUKFhcXHRpbWVzIFRZKSJdLFswLDEsIlQoVFhcXHRpbWVzIFkpIl0sWzIsMSwiVF4yKFhcXHRpbWVzIFkpIl0sWzQsMSwiVChYXFx0aW1lcyBZKSJdLFs0LDAsIlReMihYXFx0aW1lcyBZKSJdLFswLDIsIlxcc2lnbWEiLDJdLFsyLDMsIlRcXHRhdSIsMl0sWzMsNCwiXFx0ZXh0e2lkfV4qIiwyXSxbNSw0LCJcXHRleHR7aWR9XioiXSxbMSw1LCJUXFxzaWdtYSJdLFswLDEsIlxcdGF1Il1d
    \begin{tikzcd}
      {TX\times TY} && {T(X\times TY)} && {T^2(X\times Y)} \\
      {T(TX\times Y)} && {T^2(X\times Y)} && {T(X\times Y)}
      \arrow["\sigma"', from=1-1, to=2-1]
      \arrow["T\tau"', from=2-1, to=2-3]
      \arrow["{\text{id}^*}"', from=2-3, to=2-5]
      \arrow["{\text{id}^*}", from=1-5, to=2-5]
      \arrow["T\sigma", from=1-3, to=1-5]
      \arrow["\tau", from=1-1, to=1-3]
    \end{tikzcd}
  \end{equation}
\end{definition}

\section{Strong relative monads}

This section is based on \cite{tarmo}. Fix cartesian categories $\cat{J}$ and $\cat{C}$
and a monoidal functor $(J,\varepsilon,\mu)$ from $\cat{J}$ to $\cat{C}$. I.e.
$J:\cat{J}\to\cat{C}$ is a functor and $\varepsilon$ and $\mu$ are natural transformations
with components
\begin{align*}
  \varepsilon_* : 1 \to J1, \hs \mu_{X,Y} : JX\times JY \to J(X\times Y)
\end{align*}
following coherence conditions.

\begin{definition}
  A \emph{strong relative monad $T$ over $(J,\varepsilon,\mu)$} consists of
  \begin{enumerate}
    \item a relative monad $T$ over $J$;
    \item a natural transformation with components
      \begin{align*}
        \sigma_{X,Y} : JX \times TY \to T(X\times Y)
      \end{align*}
  \end{enumerate}
  such that
  \begin{enumerate}
    \item for all $X\in\cat{J}$,
      \begin{equation}
        % https://q.uiver.app/?q=WzAsNCxbMCwwLCIxXFx0aW1lcyBUWCJdLFsyLDAsIkoxXFx0aW1lcyBUWCJdLFswLDEsIlRYIl0sWzIsMSwiVCgxXFx0aW1lcyBYKSJdLFsxLDMsIlxcc2lnbWEiXSxbMCwxLCJcXHZhcmVwc2lsb25cXHRpbWVzIFRYIl0sWzAsMiwiXFxsYW1iZGEiLDJdLFszLDIsIlRcXGxhbWJkYSJdXQ==
        \begin{tikzcd}
          {1\times TX} && {J1\times TX} \\
          TX && {T(1\times X)}
          \arrow["\sigma", from=1-3, to=2-3]
          \arrow["{\varepsilon\times TX}", from=1-1, to=1-3]
          \arrow["\lambda"', from=1-1, to=2-1]
          \arrow["T\lambda", from=2-3, to=2-1]
        \end{tikzcd}
      \end{equation}
    \item for all $X,Y\in\cat{J}$,
      \begin{equation}
        % https://q.uiver.app/?q=WzAsNCxbMCwwLCJKWFxcdGltZXMgSlkiXSxbMCwxLCJKWFxcdGltZXMgVFkiXSxbMiwxLCJUKFhcXHRpbWVzIFkpIl0sWzIsMCwiSihYXFx0aW1lcyBZKSJdLFszLDIsIlxcZXRhIl0sWzEsMiwiXFxzaWdtYSIsMl0sWzAsMSwiSlhcXHRpbWVzIFxcZXRhIiwyXSxbMCwzLCJcXG11Il1d
        \begin{tikzcd}
          {JX\times JY} && {J(X\times Y)} \\
          {JX\times TY} && {T(X\times Y)}
          \arrow["\eta", from=1-3, to=2-3]
          \arrow["\sigma"', from=2-1, to=2-3]
          \arrow["{JX\times \eta}"', from=1-1, to=2-1]
          \arrow["\mu", from=1-1, to=1-3]
        \end{tikzcd}
      \end{equation}
    \item for all $X,Y,Z\in\cat{J}$,
      \begin{equation}
        % https://q.uiver.app/?q=WzAsNixbMCwxLCJKWFxcdGltZXMoSllcXHRpbWVzIFRaKSJdLFsyLDEsIkpYXFx0aW1lcyBUKFlcXHRpbWVzIFopIl0sWzQsMSwiVChYXFx0aW1lcyAoWVxcdGltZXMgWikpIl0sWzAsMCwiKEpYXFx0aW1lcyBKWSlcXHRpbWVzIFRaIl0sWzQsMCwiVCgoWFxcdGltZXMgWSlcXHRpbWVzIFopIl0sWzIsMCwiSihYXFx0aW1lcyBZKVxcdGltZXMgVFoiXSxbMywwLCJcXGFscGhhIiwyXSxbMCwxLCJKWFxcdGltZXMgXFxzaWdtYSIsMl0sWzEsMiwiXFxzaWdtYSIsMl0sWzQsMiwiVFxcYWxwaGEiXSxbNSw0LCJcXHNpZ21hIl0sWzMsNSwiXFxtdVxcdGltZXMgVFoiXV0=
        \begin{tikzcd}
          {(JX\times JY)\times TZ} && {J(X\times Y)\times TZ} && {T((X\times Y)\times Z)} \\
          {JX\times(JY\times TZ)} && {JX\times T(Y\times Z)} && {T(X\times (Y\times Z))}
          \arrow["\alpha"', from=1-1, to=2-1]
          \arrow["{JX\times \sigma}"', from=2-1, to=2-3]
          \arrow["\sigma"', from=2-3, to=2-5]
          \arrow["T\alpha", from=1-5, to=2-5]
          \arrow["\sigma", from=1-3, to=1-5]
          \arrow["{\mu\times TZ}", from=1-1, to=1-3]
        \end{tikzcd}
      \end{equation}
    \item for all morphisms $f:JY\to TY'$ and $g:J(X\times Y)\to T(X\times Y')$ such that
      \begin{equation}
        % https://q.uiver.app/?q=WzAsNCxbMCwwLCJKWFxcdGltZXMgSlkiXSxbMCwxLCJKWFxcdGltZXMgVFknIl0sWzIsMSwiVChYXFx0aW1lcyBZJykiXSxbMiwwLCJKKFhcXHRpbWVzIFkpIl0sWzEsMiwiXFxzaWdtYSIsMl0sWzAsMSwiSlhcXHRpbWVzIGYiLDJdLFszLDIsImciXSxbMCwzLCJcXG11Il1d
        \begin{tikzcd}
          {JX\times JY} && {J(X\times Y)} \\
          {JX\times TY'} && {T(X\times Y')}
          \arrow["\sigma"', from=2-1, to=2-3]
          \arrow["{JX\times f}"', from=1-1, to=2-1]
          \arrow["g", from=1-3, to=2-3]
          \arrow["\mu", from=1-1, to=1-3]
        \end{tikzcd}
      \end{equation}
      we have
      \begin{equation}
        % https://q.uiver.app/?q=WzAsNCxbMCwwLCJKWFxcdGltZXMgVFkiXSxbMCwxLCJKWFxcdGltZXMgVFknIl0sWzIsMSwiVChYXFx0aW1lcyBZJykiXSxbMiwwLCJUKFhcXHRpbWVzIFkpIl0sWzEsMiwiXFxzaWdtYSIsMl0sWzAsMSwiSlhcXHRpbWVzIGZeKiIsMl0sWzAsMywiXFxzaWdtYSJdLFszLDIsImdeKiJdXQ==
        \begin{tikzcd}
          {JX\times TY} && {T(X\times Y)} \\
          {JX\times TY'} && {T(X\times Y')}
          \arrow["\sigma"', from=2-1, to=2-3]
          \arrow["{JX\times f^*}"', from=1-1, to=2-1]
          \arrow["\sigma", from=1-1, to=1-3]
          \arrow["{g^*}", from=1-3, to=2-3]
        \end{tikzcd}
      \end{equation}
  \end{enumerate}
\end{definition}

\section{Strong pseudofunctors}

This section is based on the unpublished \cite{saville2023}.
For this chapter only, we will write products as juxtaposition
and functor application as subscripts, i.e. $T(X\times TY)$ becomes $T_{XT_Y}$.

Fix a cartesian 2-category $\bicat{C}$.

\begin{definition}
  A \emph{strength for a pseudofunctor $T:\bicat{C}\to\bicat{C}$} consists of
  \begin{enumerate}
    \item a pseudonatural transformation with components
      \begin{equation}
        \sigma_{X,Y} : XT_Y \to T_{XY};
      \end{equation}
    \item for all $X\in\bicat{C}$, an invertible 2-cell
      \begin{equation}
        % https://q.uiver.app/?q=WzAsMyxbMCwwLCIxVF9YIl0sWzIsMCwiVF97MVh9Il0sWzEsMSwiVFgiXSxbMSwyLCJUX1xcbGFtYmRhIl0sWzAsMSwiXFxzaWdtYSJdLFswLDIsIlxcbGFtYmRhIiwyXSxbMyw1LCJcXG1hdGhiZiB4X1giLDIseyJzaG9ydGVuIjp7InNvdXJjZSI6MjAsInRhcmdldCI6MjB9fV1d
        \begin{tikzcd}
          {1T_X} && {T_{1X}} \\
                 & TX
                 \arrow[""{name=0, anchor=center, inner sep=0}, "{T_\lambda}", from=1-3, to=2-2]
                 \arrow["\sigma", from=1-1, to=1-3]
                 \arrow[""{name=1, anchor=center, inner sep=0}, "\lambda"', from=1-1, to=2-2]
                 \arrow["{\mathbf x_X}"', shorten <=6pt, shorten >=6pt, Rightarrow, from=0, to=1]
        \end{tikzcd}
      \end{equation}
    \item for all $X,Y,Z\in\bicat{C}$, an invertible 2-cell
      \begin{equation}
        % https://q.uiver.app/?q=WzAsNSxbMCwwLCIoWFkpVF9aIl0sWzAsMSwiWChZVF9aKSJdLFsyLDEsIlhUX3tZWn0iXSxbNCwxLCJUX3tYKFlaKX0iXSxbNCwwLCJUX3soWFkpWn0iXSxbMCw0LCJcXHNpZ21hIl0sWzEsMiwiWFxcc2lnbWEiLDJdLFsyLDMsIlxcc2lnbWEiLDJdLFs0LDMsIlRfXFxhbHBoYSJdLFswLDEsIlxcYWxwaGEiLDJdLFs5LDgsIlxcbWF0aGJmIHlfe1gsWSxafSIsMSx7InNob3J0ZW4iOnsic291cmNlIjoyMCwidGFyZ2V0IjoyMH19XV0=
        \begin{tikzcd}
          {(XY)T_Z} &&&& {T_{(XY)Z}} \\
          {X(YT_Z)} && {XT_{YZ}} && {T_{X(YZ)}}
          \arrow["\sigma", from=1-1, to=1-5]
          \arrow["X\sigma"', from=2-1, to=2-3]
          \arrow["\sigma"', from=2-3, to=2-5]
          \arrow[""{name=0, anchor=center, inner sep=0}, "{T_\alpha}", from=1-5, to=2-5]
          \arrow[""{name=1, anchor=center, inner sep=0}, "\alpha"', from=1-1, to=2-1]
          \arrow["{\mathbf y_{X,Y,Z}}"{description}, shorten <=29pt, shorten >=29pt, Rightarrow, from=1, to=0]
        \end{tikzcd}
      \end{equation}
  \end{enumerate}
  following certain coherence conditions.
\end{definition}

\section{Strong pseudomonads}\label{sec:strong_pseudomonads}

This section is based on \cite{kock1970}. Fix a cartesian 2-category $\bicat{C}$.

\begin{definition}
  A \emph{strength} for a psuedomonad $T$ on $\cat{C}$ consists of
  \begin{enumerate}
    \item a strength $\sigma$ for the underlying pseudofunctor;
    \item for all $X,Y\in\bicat{C}$, an invertible 2-cell
      \begin{equation}
        % https://q.uiver.app/?q=WzAsNSxbMCwwLCJYVF4yX1kiXSxbMCwxLCJUX3tYVF9ZfSJdLFs0LDAsIlhUX1kiXSxbNCwxLCJUX3tYWX0iXSxbMiwxLCJUXjJfe1hZfSJdLFswLDIsIlhcXHRleHR7aWR9XioiXSxbMiwzLCJcXHNpZ21hIl0sWzQsMywiXFx0ZXh0e2lkfV4qIiwyXSxbMSw0LCJUX1xcc2lnbWEiLDJdLFswLDEsIlxcc2lnbWEiLDJdLFs5LDYsIlxcbWF0aGJmIHdfe1gsWX0iLDEseyJzaG9ydGVuIjp7InNvdXJjZSI6MjAsInRhcmdldCI6MjB9fV1d
        \begin{tikzcd}
          {XT^2_Y} &&&& {XT_Y} \\
          {T_{XT_Y}} && {T^2_{XY}} && {T_{XY}}
          \arrow["{X\text{id}^*}", from=1-1, to=1-5]
          \arrow[""{name=0, anchor=center, inner sep=0}, "\sigma", from=1-5, to=2-5]
          \arrow["{\text{id}^*}"', from=2-3, to=2-5]
          \arrow["{T_\sigma}"', from=2-1, to=2-3]
          \arrow[""{name=1, anchor=center, inner sep=0}, "\sigma"', from=1-1, to=2-1]
          \arrow["{\mathbf w_{X,Y}}"{description}, shorten <=26pt, shorten >=26pt, Rightarrow, from=1, to=0]
        \end{tikzcd}
      \end{equation}
    \item for all $X,Y\in\bicat{C}$, an invertible 2-cell
      \begin{equation}
        % https://q.uiver.app/?q=WzAsMyxbMCwwLCJYWSJdLFs0LDAsIlhUX1kiXSxbMiwxLCJUX3tYWX0iXSxbMCwyLCJcXGV0YSIsMl0sWzAsMSwiWFxcZXRhIl0sWzEsMiwiXFxzaWdtYSJdLFszLDUsIlxcbWF0aGJmIHpfe1gsWX0iLDAseyJzaG9ydGVuIjp7InNvdXJjZSI6MjAsInRhcmdldCI6MjB9fV1d
        \begin{tikzcd}
          XY &&&& {XT_Y} \\
             && {T_{XY}}
             \arrow[""{name=0, anchor=center, inner sep=0}, "\eta"', from=1-1, to=2-3]
             \arrow["X\eta", from=1-1, to=1-5]
             \arrow[""{name=1, anchor=center, inner sep=0}, "\sigma", from=1-5, to=2-3]
             \arrow["{\mathbf z_{X,Y}}", shorten <=13pt, shorten >=13pt, Rightarrow, from=0, to=1]
        \end{tikzcd}
      \end{equation}
  \end{enumerate}
  following certain coherence conditions.
\end{definition}

