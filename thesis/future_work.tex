\chapter{Future work}

Given that we have not been able to establish the presheaf construction as a
model of a generalised synthetic measure theory, a lot of future work remains.

The first step should be to make sure that the induced strength is a
pseudonatural transformation. It is hard to say how difficult a problem this
would be, but more axioms related to the interplay of the strong structural
2-cells $\bicell u$ and $\bicell s$ with their prestrong counterparts are
required. Such axioms will certainly prove useful for later developments.

Secondly, there is the problem of postulating axioms so that the induced strong
pseudomonad satisfies the coherence conditions of a strong pseudomonad in the
sense of \cite{saville2023}. This has the potential to be a lot of work: there
are five axioms for which there are no obvious counterparts. The unusual 2-cell
$\bicell q$ in \ref{def:strong_inclusion_pseudomonad_structure} is a good
example of the ingenuity that might be required to make this work. It is
safe to say that the notion of strength does not lend itself to being
generalised to relative monads.

Thirdly, once strong relative pseudomonads have been defined it is time for the
most interesting part. Generalising synthetic measure theory. Studying the
measure theory arising from the presheaf construction should be satisfying in
its own right. After all, this is the motivation for this whole project. How
difficult of a task this will end up being depends on the quality of the
definitions obtained in the previous steps.

Finally, formalising our theory in a theorem proving language seems like a
logical extension. The axioms required to make a complete definition of a strong
relative pseudomonad are bound to be complex and the addition of synthetic
measure theory will not improve the situation. This means that it will become
even harder to verify the correctness of the statements involved.
