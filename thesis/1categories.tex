\chapter{1-categories}\label{sec:synthetic_measure_theory_and_presheaves}

There are two purely categorical notions of integration that we are interested
in: the integral arising from Kock's synthetic measure theory~\cite{kock2011}
and coends as an integral of certain functors.

We begin this chapter by revisiting some introductory category theory. We then
proceed to define commutative monads, the backbone of synthetic measure theory.
After that, we shift our attention to coends, outlining in what sense the coend
of a functor may be thought of as an integral.

\section{Categories, functors, and natural transformations}\label{sec:categories}

This section presents standard constructions from the literature such as~\cite{maclane1997}.

Recall that a category consists of objects, morphisms between objects, and a
composition operation that is associative and has identities. We call a
category small if its collection of objects is a set as opposed to a proper
class.

Let $\cat{C}$ be a category. For objects $X,Y\in\cat{C}$ we denote by
$\Hom_{\cat{C}}(X,Y)$ the class of all morphisms $X\to Y$ and by $\id_X$ the
identity $X\to X$. Moreover, we will drop the composition $\circ$ and
subscripts whenever it is convenient to do so. Such notational conventions will
prove particularly useful when considering larger structures later on.

One of the most important categories is \Set{} whose objects are sets and whose
morphisms are functions. Composition is just the usual composition of functions
and the identities are the usual identity functions.

Each category $\cat{C}$ has a dual $\catop{C}$ called the opposite category.
The objects of $\catop{C}$ are exactly the objects of $\cat{C}$ but the
morphisms $Y\to X$ in $\catop{C}$ are exactly the morphisms $X\to Y$ in $\cat{C}$.
Composition in $\catop{C}$ is the same as in $\cat{C}$.

Given two categories $\cat{C}$ and $\cat{D}$, we have the product category
$\cat{C}\times\cat{D}$ which has as objects all pairs $(X,Y)$ with $X\in\cat{C}$
and $Y\in\cat{D}$ and hom-sets $\Hom((X,Y),(X',Y'))=\Hom(X,Y)\times\Hom(X',Y')$.
Composition is defined pointwise so the identites are $(\id,\id)$.

Functors are morphisms between categories. They act on objects and morphisms in
a way that preserves identities and distributes over composition. One
particularly important functor is $\Hom : \scatop{C}\times\scat{C}\to\Set$,
where $\scatop C$ is the opposite category. It takes a pairs of objects in a small
category to its hom-set and the action on morphisms is given by composition:
$\Hom(f,g)(h) = ghf$. We may also fix an object $X\in\scat{C}$ to obtain the
functors
\begin{align*}
  \Hom(X,-):\scat C\to\Set, \hs \Hom(-,X):\scatop C\to\Set.
\end{align*}

Let $F:\cat{C}\to\cat{D}$ and $G:\cat{D}\to\cat{E}$ be functors. The composite
$GF:\cat{C}\to\cat{E}$ is defined as $GF = G(F-)$. We thus have the category
$\CAT$ of all categories and all functors between them and its full subcategory
$\Cat$ of small categories.

Let $F,G:\cat{C}\to\cat{D}$ be parallel functors. A natural transformation $\phi
: F \Rightarrow G$ consists of morphisms $\phi_X : FX \to GX$ in $\cat{D}$, for
all $X\in\cat{C}$, that satisfy the naturality condition: for all $f:X\to Y$ in
$\cat{C}$, $\phi_Y\circ Ff = Gf\circ \phi_X$.

We can compose natural transformations $F\Rightarrow G$ and $G\Rightarrow H$ by
composing the components. Thus, for all categories $\cat{C}$ and $\cat{D}$, we
have the functor category $\bb{\cat{C},\cat{D}}$ with functors
$\cat{C}\to\cat{D}$ as objects and natural transformations as morphisms.

Consider functors $F,G:\cat{C}\to\bb{\cat{D},\cat{E}}$. A natural
transformation $\phi : F\Rightarrow G$ has components $\phi_C : FC\to GC$ in
$\bb{\cat{D},\cat{E}}$. This means that each $\phi_C$ is itself a natural
transformation with components $(\phi_C)_D : (FC)D \to (GC)D$ in
$\cat{E}$. Such functors and natural transformations will play a major role in
later chapters. We will make sure to keep our notation precise to assist the
reader in peeling back the various layers of indirection.

\section{Cartesian structure}\label{sec:cartesian_categories}

Now recall that the product of objects $X,Y\in\cat{C}$ is an object $X\times Y$
together with morphisms $\pi_1 : X\times Y \to X$ and $\pi_2 : X\times Y \to
Y$. It is universal in the sense that, for all other pairs of morphisms $f:W\to
X$ and $g:W\to Y$, there is a unique morphism $\aa{f,g}:W\to X\times Y$ such
that the following commutes:
\begin{equation}\label{eq:product_diagram}
  % https://q.uiver.app/?q=WzAsNCxbMiwwLCJXIl0sWzIsMSwiWFxcdGltZXMgWSJdLFswLDEsIlgiXSxbNCwxLCJZIl0sWzEsMywiXFxwaV8yIiwyXSxbMSwyLCJcXHBpXzEiXSxbMCwxLCJcXGxhbmdsZSBmLGdcXHJhbmdsZSIsMV0sWzAsMiwiZiIsMl0sWzAsMywiZyJdXQ==
  \begin{tikzcd}
  && W \\
    X && {X\times Y} && Y
    \arrow["{\pi_2}"', from=2-3, to=2-5]
    \arrow["{\pi_1}", from=2-3, to=2-1]
    \arrow["{\langle f,g\rangle}"{description}, from=1-3, to=2-3]
    \arrow["f"', from=1-3, to=2-1]
    \arrow["g", from=1-3, to=2-5]
  \end{tikzcd}
\end{equation}

We say that a diagram commutes if every two paths from the same source to the
same sink are equal. We will make heavy use of commutative diagrams, both as
axioms that we postulate and as statements that we prove. In the latter case,
the commutativity may be left implicit. Unfortunately, diagrams tend to get
much more complex in later chapters. Therefore it will not be possible to write
down every proof in full detail. We will, however, occasionally include links
to quiver~\cite{quiver}, a graphical editor for commutative diagrams. For
example, the diagram \ref{eq:product_diagram} corresponds to \href{
  https://q.uiver.app/?q=WzAsNCxbMiwwLCJXIl0sWzIsMSwiWFxcdGltZXMgWSJdLFswLDEsIlgiXSxbNCwxLCJZIl0sWzEsMywiXFxwaV8yIiwyXSxbMSwyLCJcXHBpXzEiXSxbMCwxLCJcXGxhbmdsZSBmLGdcXHJhbmdsZSIsMV0sWzAsMiwiZiIsMl0sWzAsMywiZyJdXQ==
}{this quiver link}. The purpose of these links is twofold: On the one hand
they should serve as a useful guide through the less obvious proofs and on the
other we hope that they may serve as resource for future reference. However, we
believe that this report is entirely self-contained and therefore the reader is
not required to engage with the quiver links whatsoever.

We now take the product of morphisms $f:X\to X'$ and $g:Y\to Y'$ to be the
morphism $f\times g:X\times X'\to Y\times Y'$ given by $f\times
g=\aa{f\pi_1,g\pi_2}$. Thus we have a functor $\cat{C}\times\cat{C}\to\cat{C}$.
It is important to realises that, in general, products are not unique.
Fortunately, they are unique \emph{up to canonical isomorphism}. Therefore
specifying such a product functor amounts to choosing a product for each pair
of objects.

As the product functor depends on the choice of products, we do not expect it to
be associative or commutative in the strict sense. However, there are natural
isomorphisms with components
\begin{align}\label{eq:product_alpha_gamma}
  \alpha_{X,Y,Z} : (X\times Y)\times Z \cong X\times (Y\times Z), \hs
  \gamma_{X,Y} : X\times Y \cong Y\times X.
\end{align}
It is worth pointing out that $\gamma$ is its own inverse. That is,
$\inv\gamma_{X,Y}=\gamma_{Y,X}$.

An object $1\in\cat{C}$ is terminal if, for all $X\in\cat{C}$, there is a unique
morphism $X\to 1$. It turns out that terminal objects are units for
multiplication. That is, there is a natural isomorphism with components
\begin{align*}
  \lambda_X : 1\times X \cong X.
\end{align*}

Combining this with $\gamma$ from (\ref{eq:product_alpha_gamma}) yields the
right unitor $\rho=\lambda\gamma:X\times 1\cong X$.

To tie all of this together, a cartesian category is a category with a choice of
all (binary) products and a distinguished terminal object.

\begin{example}
  Two particular cartesian structures are going to be of interest to us: on $\Set$
  and on $\CAT$ (and thereby $\Cat$). In $\Set$, we take the product of two sets
  to be the cartesian product with the usual projections $(x,y)\mapsto x$ and
  $(x,y)\mapsto y$. The terminal object $1\in\Set$ is a distinguished singleton
  $1=\cc{*}$. In $\CAT$, the product of categories $\cat{C}$ and $\cat{D}$ is the
  product category $\cat{C}\times\cat{D}$ with the obvious projection functors and
  the terminal category $\mathbb 1\in\CAT$ is a distinguished category with a
  single object and a single (identity) morphism. Note that, for small categories
  $\scat C$ and $\scat D$, $\scat C\times\scat D$ is small so $\Cat$ inherits the
  cartesian structure.
\end{example}

\section{Monads}\label{sec:monads}

Monads are central to the study of category theory. They arise naturally in practical settings,
e.g.\ to model computation with side effects, as well as for theoretical
purposes, e.g.\ monad algebras as a generalisation of algebraic theories.

We adapt the no-iteration definition of a monad~\cite{marmolejo2010}
because it more easily generalises to relative monads~\cite{altenkirch2015}.

\begin{definition}\label{def:monad}
  Let $\cat{C}$ be a category. A \emph{monad $T$ on $\cat{C}$} consists of
  \begin{enumerate}
    \item for all $X\in\cat{C}$, an object $TX\in\cat{C}$;
    \item for all $X\in\cat{C}$, a morphism $\eta_X : X \to TX$ in $\cat{C}$;
    \item for all $X,Y\in\cat{C}$, a map
      \begin{align*}
        \extend{\rr{-}}{*}_{X,Y}:\Hom\rr{X,TY}\to \Hom\rr{TX,TY}
      \end{align*}
  \end{enumerate}
  such that, for all $X,Y,Z\in{\cat{C}}$, $f:X\to TY$, and $g:Y\to TZ$,
  the following commute:
  \begin{equation}
    % https://q.uiver.app/?q=WzAsMixbMCwwLCJUWCJdLFsyLDAsIlRYIl0sWzAsMSwiXFxldGFeKiIsMix7ImN1cnZlIjozfV0sWzAsMSwiIiwwLHsiY3VydmUiOi0zLCJsZXZlbCI6Miwic3R5bGUiOnsiaGVhZCI6eyJuYW1lIjoibm9uZSJ9fX1dXQ==
    \begin{tikzcd}
      TX && TX
      \arrow["{\eta^*}"', curve={height=18pt}, from=1-1, to=1-3]
      \arrow[curve={height=-18pt}, Rightarrow, no head, from=1-1, to=1-3]
    \end{tikzcd}
    \hs
    % https://q.uiver.app/?q=WzAsMyxbMCwwLCJYIl0sWzIsMCwiVFkiXSxbMiwyLCJUWSJdLFswLDEsIlxcZXRhIl0sWzEsMiwiZl4qIl0sWzAsMiwiZiIsMl1d
    \begin{tikzcd}
      X && TY \\
      \\
        && TY
        \arrow["\eta", from=1-1, to=1-3]
        \arrow["{f^*}", from=1-3, to=3-3]
        \arrow["f"', from=1-1, to=3-3]
    \end{tikzcd}
    \hs
    % https://q.uiver.app/?q=WzAsMyxbMCwwLCJUWCJdLFsyLDAsIlRZIl0sWzIsMiwiVFoiXSxbMCwxLCJmXioiXSxbMSwyLCJnXioiXSxbMCwyLCIoZ14qZileKiIsMl1d
    \begin{tikzcd}
      TX && TY \\
      \\
         && TZ
         \arrow["{f^*}", from=1-1, to=1-3]
         \arrow["{g^*}", from=1-3, to=3-3]
         \arrow["{(g^*f)^*}"', from=1-1, to=3-3]
    \end{tikzcd}
  \end{equation}
\end{definition}

The double line in the first diagram refers to the identity $TX\to TX$.

Let us now investigate the powerset monad which we touched on in chapter \ref{sec:introduction}.

\begin{example}
  The powerset construction $\P$ takes each set $X$ to its powerset $\P X$.
  We add a monad structure like so: the unit map takes elements
  to singletons, i.e. $\eta\rr{x}=\cc{x}$. The extension of a
  function $f:X\to\P Y$ takes $A\in\P X$ to the union
  $\extend{f}{T}\rr{A}=\bigcup \cc{f\rr{x} : x \in A}$.
  This is known as the powerset monad.
\end{example}

Every monad $T$ on $\cat{C}$ gives rise to an endofunctor $\cat{C}\to\cat{C}$
with object map $X\mapsto TX$ and morphism map $f\mapsto\extend{\rr{\eta
f}}{T}$. We abuse notation and denote this endofunctor simply by $T$ and write
$Tf:TX\to TY$, as usual. In the case of the powerset monad, the functorial
action takes a function $f:X\to Y$ to the direct image map $A \mapsto
\cc{f\rr{x} : x \in A}$.

\section{Commutative monads}\label{sec:commutative_monads}

Given that a monad is just a monoid in the category of
endofunctors~\cite{maclane1997}, one might expect that the commutativity of a
monad is related to the commutativity of the corresponding monoid. This is not
the case. The commutativity required for synthetic measure theory refers to the
strength of a monad. For our purposes, a monad is strong if it interacts well
with the cartesian structure of the corresponding category.

Fix a cartesian category $\cat{C}$.
A strength $\sigma$ for a monad $T$ on $\cat{C}$ is a natural
transformation with components
\begin{align*}
  \sigma_{X,Y}:X\times TY \to T\rr{X\times Y}
\end{align*}
following certain conditions. A strong monad is a monad $T$ equipped with a
strength $\sigma$.

Such a strong monad is then called commutative~\cite{kock1970}, if the costrength
\begin{align*}
  \tau_{X,Y}:TX\times Y\to T(X\times Y)
\end{align*}
given by the composite
\begin{equation}
  % https://q.uiver.app/?q=WzAsNCxbMCwwLCJUWFxcdGltZXMgWSJdLFsyLDAsIlQoWFxcdGltZXMgWSkiXSxbMCwxLCJZXFx0aW1lcyBUWCJdLFsyLDEsIlQoWVxcdGltZXMgWCkiXSxbMiwzLCJcXHNpZ21hIiwyXSxbMywxLCJUXFxnYW1tYSIsMl0sWzAsMiwiXFxnYW1tYSIsMl0sWzAsMSwiXFx0YXUiXV0=
  \begin{tikzcd}
    {TX\times Y} && {T(X\times Y)} \\
    {Y\times TX} && {T(Y\times X)}
    \arrow["\sigma"', from=2-1, to=2-3]
    \arrow["T\gamma"', from=2-3, to=1-3]
    \arrow["\gamma"', from=1-1, to=2-1]
    \arrow["\tau", from=1-1, to=1-3]
  \end{tikzcd}
\end{equation}
makes the diagram below commute:

\begin{equation}\label{eq:commutative_monad}
  % https://q.uiver.app/?q=WzAsNixbMCwwLCJUWFxcdGltZXMgVFkiXSxbMiwwLCJUKFRYXFx0aW1lcyBZKSJdLFs0LDAsIlReMihYXFx0aW1lcyBZKSJdLFs0LDEsIlQoWFxcdGltZXMgWSkiXSxbMiwxLCJUXjIoWFxcdGltZXMgWSkiXSxbMCwxLCJUKFhcXHRpbWVzIFRZKSJdLFswLDUsIlxcdGF1IiwyXSxbMCwxLCJcXHNpZ21hIl0sWzEsMiwiVFxcdGF1Il0sWzUsNCwiVFxcc2lnbWEiLDJdLFs0LDMsIlxcdGV4dHtpZH1eKiIsMl0sWzIsMywiXFx0ZXh0e2lkfV4qIl1d
  \begin{tikzcd}
    {TX\times TY} && {T(TX\times Y)} && {T^2(X\times Y)} \\
    {T(X\times TY)} && {T^2(X\times Y)} && {T(X\times Y)}
    \arrow["\tau"', from=1-1, to=2-1]
    \arrow["\sigma", from=1-1, to=1-3]
    \arrow["T\tau", from=1-3, to=1-5]
    \arrow["T\sigma"', from=2-1, to=2-3]
    \arrow["{\text{id}^*}"', from=2-3, to=2-5]
    \arrow["{\text{id}^*}", from=1-5, to=2-5]
  \end{tikzcd}
\end{equation}

\begin{example}
  In the case of $\Set$, the product is the usual cartesian product of sets. The powerset
  monad is a commutative monad with strength and costrength given by
  $\sigma\rr{x,B}=\cc{x}\times B$ and $\tau\rr{A,y}=A\times\cc{y}$, respectively.
\end{example}

\section{Relative monads}

This section closely follows~\cite{altenkirch2015}.

In section~\ref{sec:monads}, we chose to present the no-iteration definition of
a monad because it avoids repeated applications of the object map. This now
leads to an obvious generalisation of monads that are functors
$\cat{J}\to\cat{C}$ rather than endofunctors. We require another functor
$\cat{J}\to\cat{C}$ to relate objects in $\cat{J}$ to those in the image of
the relative monad. The resulting definition is remarkably similar to~\ref{def:monad}.

\begin{definition}\label{def:relative_monad}
  Let $J:\cat{J}\to\cat{C}$ be a functor. A \emph{relative monad $T$ over $J$}
  consists of
  \begin{enumerate}
    \item for all $X\in\cat{J}$, an object $TX\in\cat{C}$;
    \item for all $X\in\cat{J}$, a morphism $\eta_X : JX \to TX$ in $\cat{C}$;
    \item for all $X,Y\in\cat{J}$, a map
      \begin{align*}
        \extend{\rr{-}}{*}_{X,Y}:\Hom\rr{JX,TY}\to \Hom\rr{TX,TY}
      \end{align*}
  \end{enumerate}
  such that, for all $X,Y,Z\in{\cat{J}}$, $f:JX\to TY$, and $g:JY\to TZ$,
  the following commute:
  \begin{equation}
    % https://q.uiver.app/?q=WzAsMixbMCwwLCJUWCJdLFsyLDAsIlRYIl0sWzAsMSwiXFxldGFeKiIsMix7ImN1cnZlIjozfV0sWzAsMSwiIiwwLHsiY3VydmUiOi0zLCJsZXZlbCI6Miwic3R5bGUiOnsiaGVhZCI6eyJuYW1lIjoibm9uZSJ9fX1dXQ==
    \begin{tikzcd}
      TX && TX
      \arrow["{\eta^*}"', curve={height=18pt}, from=1-1, to=1-3]
      \arrow[curve={height=-18pt}, Rightarrow, no head, from=1-1, to=1-3]
    \end{tikzcd}
    \hs
    % https://q.uiver.app/?q=WzAsMyxbMCwwLCJKWCJdLFsyLDAsIlRZIl0sWzIsMiwiVFkiXSxbMCwxLCJcXGV0YSJdLFsxLDIsImZeKiJdLFswLDIsImYiLDJdXQ==
    \begin{tikzcd}
      JX && TY \\
      \\
         && TY
         \arrow["\eta", from=1-1, to=1-3]
         \arrow["{f^*}", from=1-3, to=3-3]
         \arrow["f"', from=1-1, to=3-3]
    \end{tikzcd}
    \hs
    % https://q.uiver.app/?q=WzAsMyxbMCwwLCJUWCJdLFsyLDAsIlRZIl0sWzIsMiwiVFoiXSxbMCwxLCJmXioiXSxbMSwyLCJnXioiXSxbMCwyLCIoZ14qZileKiIsMl1d
    \begin{tikzcd}
      TX && TY \\
      \\
         && TZ
         \arrow["{f^*}", from=1-1, to=1-3]
         \arrow["{g^*}", from=1-3, to=3-3]
         \arrow["{(g^*f)^*}"', from=1-1, to=3-3]
    \end{tikzcd}
  \end{equation}
\end{definition}

\begin{example}
  Let $T$ be a monad on $\cat{C}$. Then $T$ is a relative monad over the identity
  $\cat{C}\to\cat{C}$. Moreover, for any functor $J:\cat{J}\to\cat{C}$, we have a
  relative monad $T'$ over $J$ with object map $T'X = TJX$.
\end{example}

Our motivation for presenting this definition is the presheaf construction. It
behaves like a relative monad $\Cat\to\CAT$ but fails to satisfy the axioms in
the strict sense because coends, which we shall introduce next, are unique only
up to isomorphism.

While our definition of a monad is easy to generalise to the relative case, the
same cannot be said for commutative monads. The axioms of strong and
commutative monads involve repeated applications of the object map (see
\ref{eq:commutative_monad}). We therefore have to find alternative conditions
to impose. Fortunately,~\cite{tarmo} already contains the definition of a
strong relative moand that we will make use of. However, we will encounter
similar problems when identifying suitable coherence conditions in chapter~
\ref{sec:strong_relative_pseudomonads}.

\section{Coends}\label{sec:coends}

Let us now turn our attention to the categorical integral that we are hoping to
capture in a model of synthetic measure theory. There are two dual notions of an
integral of a functor $\catop{C}\times\cat{C}\to\cat{D}$: ends and coends. We will
focus on the latter because coends arise naturally when considering the Yoneda
embedding, one of the most fundamental constructions in category theory.

We begin by defining a more general structure:

\begin{definition}\label{def:cowedge}
  Let $F:\catop{C}\times\cat{C}\to\cat{D}$ be a functor. A \emph{cowedge of $F$} consists of
  \begin{enumerate}
    \item an object $W\in\cat{D}$;
    \item for all $X\in\cat{C}$, a morphism $w_X:F\rr{X,X}\to W$
  \end{enumerate}
  such that, for all $f:X\to Y$ in $\cat{C}$,
  \begin{equation}\label{eq:cowedge_property}
    % https://q.uiver.app/?q=WzAsNCxbMiwxLCJXIl0sWzAsMSwiRihYLFgpIl0sWzIsMCwiRihZLFkpIl0sWzAsMCwiRihZLFgpIl0sWzEsMCwiZV9YIiwyXSxbMiwwLCJlX1kiXSxbMywyLCJGKFksZikiXSxbMywxLCJGKGYsWCkiLDJdXQ==
    \begin{tikzcd}
      {F(Y,X)} && {F(Y,Y)} \\
      {F(X,X)} && W
      \arrow["{w_X}"', from=2-1, to=2-3]
      \arrow["{w_Y}", from=1-3, to=2-3]
      \arrow["{F(Y,f)}", from=1-1, to=1-3]
      \arrow["{F(f,X)}"', from=1-1, to=2-1]
    \end{tikzcd}
  \end{equation}
  Let $(W,w_X)$ and $(W',w'_X)$ be cowedges of $F$. A morphism $h:W\to W'$ is a morphism of
  cowedges $(W,w_X)\to(W',w'_X)$ if, for all $X\in\cat{C}$, $hw_X=w'_X$.
\end{definition}

Examples of cowedges that are both interesting and specific are hard to come by.
What the following lacks in specificity it makes up for in importance.

\begin{example}\label{ex:cowedge}
  Let $P:\catop{C}\to\Set$ be a functor and $W\in\cat{C}$ an object. Consider the functor
  $\catop{C}\times\cat{C}\to\Set$ given by
  \begin{align*}
    Y,X \mapsto PY \times \Hom(W,X).
  \end{align*}
  and, for each $X\in\cat{C}$, the function
  \begin{align*}
    w_X : PX \times \Hom(W,X) \to PW
  \end{align*}
  given by $x,f\mapsto (Pf)(x)$. Now fix $f:W\to X$, $g:X\to Y$, and $y\in PY$.
  Then
  \begin{align*}
    w(y,gf) = P(gf)(y) = PfPg(y) = w(Pg(y), f).
  \end{align*}
  I.e. the diagram
  \begin{equation*}
    % https://q.uiver.app/?q=WzAsNCxbMCwwLCJQWVxcdGltZXNcXHRleHR7SG9tfShXLFgpIl0sWzAsMSwiUFlcXHRpbWVzXFx0ZXh0e0hvbX0oVyxZKSJdLFszLDAsIlBYXFx0aW1lc1xcdGV4dHtIb219KFcsWCkiXSxbMywxLCJQVyJdLFswLDIsIlBnXFx0aW1lc1xcdGV4dHtIb219KFcsWCkiXSxbMCwxLCJQWVxcdGltZXNcXHRleHR7SG9tfShXLGcpIiwyXSxbMSwzLCJ3X1kiLDJdLFsyLDMsIndfWiJdXQ==
    \begin{tikzcd}
      {PY\times\text{Hom}(W,X)} &&& {PX\times\text{Hom}(W,X)} \\
      {PY\times\text{Hom}(W,Y)} &&& PW
      \arrow["{Pg\times\text{Hom}(W,X)}", from=1-1, to=1-4]
      \arrow["{PY\times\text{Hom}(W,g)}"', from=1-1, to=2-1]
      \arrow["{w_Y}"', from=2-1, to=2-4]
      \arrow["{w_Z}", from=1-4, to=2-4]
    \end{tikzcd}
  \end{equation*}
  commutes. Thus $(PW, w_X)$ is a cowedge.
\end{example}

We notice that, for any cowedge $(W,w_X)$, postcomposition with a morphism
$f:W\to W'$ yields another cowedge $(W',fw_X)$, making $f$ into a cowedge
homomorphism. Now a coend is an initial cowedge:

\begin{definition}\label{def:coend}
  Let $F:\catop{C}\times\cat{C}\to\cat{D}$ be a functor. A \emph{coend of $F$}
  is a cowedge $(E,e_X)$ of $F$ such that, for all cowedges $(W,w_X)$ of $F$,
  there exists a unique cowedge homomorphism $(E,e_X)\to(W,w_X)$.
\end{definition}

Coends are special colimits (see~\cite[Remark 1.2.3]{loregian2015}). So, while
there is no guarantee that a particular functor has a coend, any two coends of
the same functor are canonically isomorphic. This allows us to fix a choice of
coends without loss of generality, just as we discussed for products.

Observe a useful property of coends: consider functors
$F,G:\catop{C}\times\cat{C}\to\cat{D}$, a natural transformation
$\phi:F\Rightarrow G$ between them, and assume that we have chosen coends of $F$
and $G$. By precomposing the coend diagram of $G$ with $\phi$, we obtain a
cowedge of $F$. Thus, for each choice of coends, there exists a unique cowedge
homomorphism induced by $\phi$.

Thus a choice of coends assigns objects to functors $\catop{C}\times\cat{C}\to\cat{D}$
and morphisms to natural transformations between such functors. This
observations suggests that one might want to think of coends as a functor
$\bb{\catop{C}\times\cat{C},\cat{D}}\to\cat{D}$. There is one problem, however:
just like measure theoretic integrals, statements about coends rely on their
existence in the first place. Fortunately, all coends in this thesis will be of
functors of the form $\scatop{C}\times\scat{C}\to\Set$ where $\scat C$ is
small. As $\Set$ is cocomplete, all such coends exist and are indeed
functorial. It is possible to construct coends in \Set{} explicitly by taking
disjoint unions and quotienting them by a particular equivalence relation.
Unfortunately, the resulting construction is far from being intuitive so we
will not investigate it further.

\begin{notation}
  The notation that arises from the formal definition of coends does not scale
  well to more complicated calculations. We improve it as follows:
  \begin{itemize}
    \item For a functor $F:\catop{C}\times\cat{C}\to\cat{D}$, we will denote the
      object and morphism of a chosen coend by
      \begin{align}\label{eq:coend_integral_notation}
        \int^{C} F(C,C) \hs\text{and}\hs q_X:F(X,X)\to\int^C F(C,C).
      \end{align}
      Here the variable $C$ is bound by the integral. That is, the choice of
      symbol is arbitrary.
    \item For functors $F,G:\catop{C}\times\cat{C}\to\cat{D}$, a natural
      transformation $\phi: F\Rightarrow G$, and choices of coends, we denote
      the morphism of cowedges induced by $\phi$ by
      \begin{equation}\label{eq:coend_natural_transformation}
        \int^C \phi_{C,C} : \int^C F(C,C) \to \int^C G(C,C).
      \end{equation}
  \end{itemize}
\end{notation}

We already saw an example of a coend in \ref{ex:cowedge}. We establish the
universality by constructing a cowedge homomorphism to the chosen coend.

\begin{example}\label{ex:coend}
  The cowedge in \ref{ex:cowedge} is a coend. To see this fix a choice of coend and
  consider the morphism $h$ given by the composite
  \begin{equation}
    % https://q.uiver.app/?q=WzAsMyxbMCwwLCJQVyJdLFsyLDEsIlxcaW50XntDfSBQQ1xcdGltZXNcXHRleHR7SG9tfShXLEMpIl0sWzIsMCwiUFdcXHRpbWVzXFx0ZXh0e0hvbX0oVyxXKSJdLFswLDIsIlxcbGFuZ2xlIFBXLFxcRGVsdGFcXHRleHR7aWR9XFxyYW5nbGUiXSxbMiwxLCJxIl0sWzAsMSwiaCIsMl1d
    \begin{tikzcd}
      PW && {PW\times\text{Hom}(W,W)} \\
         && {\int^{C} PC\times\text{Hom}(W,C)}
         \arrow["{\langle PW,\Delta\text{id}\rangle}", from=1-1, to=1-3]
         \arrow["q", from=1-3, to=2-3]
         \arrow["h"', from=1-1, to=2-3]
    \end{tikzcd}
  \end{equation}
  Here $\Delta\id$ is the constant function $x\mapsto \id$.
  Now let $x\in PX$ and $f:W\to X$. Then
  \begin{align*}
    (h w)(x,f) = h(Pf(x))a = q(Pf(x),\id) = q(x,f).
  \end{align*}
  where the last equality holds due to~\ref{eq:cowedge_property}.
  Thus the diagram
  \begin{equation*}
    % https://q.uiver.app/?q=WzAsMyxbMSwwLCJQVyJdLFsxLDEsIlxcaW50XkMgUENcXHRpbWVzXFx0ZXh0e0hvbX0oVyxDKSJdLFswLDAsIlBYXFx0aW1lc1xcdGV4dHtIb219KFcsWCkiXSxbMCwxLCJoIl0sWzIsMCwidyJdLFsyLDEsInEiLDJdXQ==
    \begin{tikzcd}
      {PX\times\text{Hom}(W,X)} & PW \\
                                & {\int^C PC\times\text{Hom}(W,C)}
                                \arrow["h", from=1-2, to=2-2]
                                \arrow["w", from=1-1, to=1-2]
                                \arrow["q"', from=1-1, to=2-2]
    \end{tikzcd}
  \end{equation*}
  commutes, showing that $h$ is a cowedge homomorphism. By universality of the coend we conclude
  that we have an isomorphism
  \begin{align*}
    h:PW\cong \int^C PC\times\Hom(W,C).
  \end{align*}
  This is sometimes referred to as the co-Yoneda lemma.
\end{example}

\section{Properties of coends}

We are now in a position to make precise the functoriality of the coend:

\begin{lemma}\label{lemma:functoriality_of_coends}
  Let $\scat C$ be a small category.
  Every choice of coends for all functors $\scatop{C}\times\scat{C}\to\Set$ induces
  a functor $\bb{\scatop{C}\times\scat{C},\Set}\to\Set$ given by
  \begin{align*}
    F \mapsto \int^C F(C,C), \hs \phi \mapsto \int^C \phi_{C,C}.
  \end{align*}
  \begin{proof}
    Fix $F,G,H:\scatop{C}\times\scat{C}\to\Set$. Consider natural
    transformations $\phi:F\Rightarrow G$ and $\psi:G\Rightarrow H$. Now
    \begin{align*}
      \rr{\int^C \psi_{C,C}}\rr{\int^C \phi_{C,C}} = \int^C (\psi\phi)_{C,C}
    \end{align*}
    because both sides are cowedge homomorphisms
    \begin{align*}
      \int^C F(C,C) \to \int^C H(C,C).
    \end{align*}
    By universality of the coend on the left, this is unique.

    Preservation of identities follows by a similar argument.
  \end{proof}
\end{lemma}

While it is not at all obvious in what sense coends describe integration, we observe
that they tend to behave like integrals in the analytic sense. For example, we have the
following result about scalar multiplication:

\begin{lemma}\label{lemma:scalar_multiplication_of_coends}
  Let $F:\catop{C}\times\cat{C}\to\cat{D}$ be a functor where $\cat{D}$ is cartesian,
  and let $D\in\cat{D}$. For any choice of coend of $F$, there is a canonical isomorphism
  \begin{align*}
    D\times \int^{C} F(C,C) \cong \int^{C} D\times F(C,C).
  \end{align*}
  \begin{proof}
    We have a functor $D\times -:\cat{D}\to\cat{D}$ with inverse $\pi_2 \circ -$.
    In particular, $D\times -$ is cocontinuous. By~\cite[Theorem 1.2.7]{loregian2015},
    we have
    \begin{align*}
      (D\times -)\rr{\int^C F(C,C)} \cong \int^C (D\times -)F(C,C),
    \end{align*}
    as required.
  \end{proof}
\end{lemma}

Due to the canonical isomorphism of products $A\times B\cong B\times A$,
we immediately have a similar statement for scalar multiplication on the right.

The final property that relates coends to conventional integrals is
the Fubini rule:

\begin{theorem}[{\cite[Theorem 1.3.1]{loregian2015}}]\label{thm:fubini}
  Let $F:\catop{B}\times\cat{B}\times\catop{C}\times\cat{C}\to\cat{D}$ be a functor.
  Then there are canonical isomorphisms
  \begin{align*}
    \int^{B}\int^{C} F(B,B,C,C)
    \cong \int^{B,C} F(B,B,C,C)
    \cong \int^{C}\int^{B} F(B,B,C,C).
  \end{align*}
  That is, if one of the coends above exists, so do the other and there are
  unique isomorphisms of cowedges between them.
\end{theorem}

