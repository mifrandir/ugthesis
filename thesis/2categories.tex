\chapter{2-categories}\label{sec:2categories}

To account for the possibility of axioms holding only up to a specified
isomorphism, we need to consider 2-categories. We avoid confusion by referring
to  the usual categories as 1-categories. Since 2-categorical notions tend to
be straightforward generalisations of their 1-categorical counterparts, many of
the developments in this chapter are going to seem unremarkable. The difficulty
lies in choosing which axioms we allow to hold only up to a coherent
isomorphism. This type of problem rarely arises in 1-category theory but is a
central to the study of 2-categories.

\begin{notation}
  It will be convenient to fix some notational conventions for the remainder of this report.
  Unless otherwise indicated,
  \begin{itemize}
    \item curly upper-case letters denote 1-categories, e.g. $\cat{C}$;
    \item blackboard bold upper-case letters denote small 1-categories, e.g. $\scat C$;
    \item plain upper-case letters denote objects, e.g. $X$.
  \end{itemize}
\end{notation}


\section{Definition}

The idea of a 2-category is straightforward. In mathematics, it has proven useful
to replace equality by different notions of isomorphisms. Commutative diagrams in
1-categories are nothing but equalities between morphisms, so we would like to study
isomorphisms between morphisms instead. Thus 2-categories have morphisms between morphisms.
To disambiguate we call morphisms between objects 1-cells and morphisms between morphisms
2-cells.

The concept of morphisms between morphisms should be familiar already. The 1-category
\CAT{} has functors as morphisms and we know that natural transformations are just
morphisms between functors. We will make use of this and extend \CAT{} to a 2-category
\biCAT{}.

While the additional structure brings with it a lot of freedom, it comes with a price.
Firstly, 2-categorical concepts contain more data that needs to be specified and
more axioms that need to be verified. This makes 2-category theory more complex.
Secondly, due to this additional complexity, it is not always desirable to state axioms up to
isomorphism. Thus, one has to make an informed choice about the degree of strictness
required at each step along the way.

Even when defining a 2-category itself, strictness plays a role: it is possible to
demand that the axioms hold only up to isomorphism. This consideration leads to weak
2-categories or bicategories. (see~\cite{leinster1998}) We note that the weakness of the former
has nothing to do with the strength of a monad as in~\ref{sec:commutative_monads}.
We do not require the full generality of bicategories and are thus going
to restrict our attention to (strict) 2-categories:

\begin{definition}\label{def:2category}
  A \emph{2-category} $\bicat{C}$ consists of \begin{enumerate}
    \item a class of objects $\Obj_{\bicat{C}}$;
    \item for all objects $X,Y$, a 1-category $\Hom_{\bicat C}\bb{X,Y}$ with
      composition $\bullet$ whose objects $f:X\to Y$ are called 1-cells and
      whose morphisms $\bicell u:f\Rightarrow g$ are called 2-cells;
    \item for all $X\in\bicat{C}$, an identiy 1-cell $\text{id}_X:X\to X$;
    \item for all $X,Y,Z\in\bicat{C}$, a composition functor
      \begin{equation}
        \label{eq:bicategory_composition}
        \circ_{X,Y,Z}:\Hom_{\bicat{C}}\bb{Y,Z}\times\Hom_{\bicat{C}}\bb{X,Y}
        \to\Hom_{\bicat{C}}\bb{X,Z};
      \end{equation}
  \end{enumerate}
  such that the following hold:
  \begin{enumerate}
    \item for all composable 1-cells $f,g,h$,
      $h\circ\rr{g\circ f}=\rr{h\circ g}\circ f$;
    \item for all $f:X\to Y$, $\id_Y\circ f = f = f\circ \id_X$.
  \end{enumerate}
\end{definition}

We note that there are two ways to compose 2-cells:
\begin{enumerate}
  \item For 1-cells $f,g,h:X\to Y$ and 2-cells $\bicell u: f\Rightarrow g$,
    $\bicell v: g\Rightarrow h$ we have the vertical composite
    $\bicell v\bullet\bicell u: f\Rightarrow h$ given by the composition in $\Hom\bb{X,Y}$. In a diagram:
    \begin{equation*}
      % https://q.uiver.app/?q=WzAsMixbMCwwLCJYIl0sWzIsMCwiWSJdLFswLDEsImYiLDAseyJjdXJ2ZSI6LTV9XSxbMCwxLCJoIiwyLHsiY3VydmUiOjV9XSxbMCwxLCJnIiwxXSxbMiw0LCJcXG1hdGhiZiB1IiwxLHsic2hvcnRlbiI6eyJzb3VyY2UiOjIwLCJ0YXJnZXQiOjIwfX1dLFs0LDMsIlxcbWF0aGJmIHYiLDEseyJzaG9ydGVuIjp7InNvdXJjZSI6MjAsInRhcmdldCI6MjB9fV1d
      \begin{tikzcd}
        X && Y
        \arrow[""{name=0, anchor=center, inner sep=0}, "f", curve={height=-40pt}, from=1-1, to=1-3]
        \arrow[""{name=1, anchor=center, inner sep=0}, "h"', curve={height=40pt}, from=1-1, to=1-3]
        \arrow[""{name=2, anchor=center, inner sep=0}, "g"{description}, from=1-1, to=1-3]
        \arrow["{\mathbf u}"{description}, shorten <=4pt, shorten >=4pt, Rightarrow, from=0, to=2]
        \arrow["{\mathbf v}"{description}, shorten <=4pt, shorten >=4pt, Rightarrow, from=2, to=1]
      \end{tikzcd}
    \end{equation*}
  \item For 1-cells $f,g:X\to Y$, $h,k:Y\to Z$ and 2-cells $\bicell
    u:f\Rightarrow g$, $\bicell v:h\Rightarrow k$ we have the horizontal
    composite $\bicell v\circ\bicell u:h\circ f\Rightarrow k\circ g$ given by
    the composition functor $\circ_{X,Y,Z}$. In a diagram:
    \begin{equation*}
      % https://q.uiver.app/?q=WzAsMyxbMCwwLCJYIl0sWzIsMCwiWSJdLFs0LDAsIloiXSxbMCwxLCJmIiwwLHsiY3VydmUiOi00fV0sWzAsMSwiZyIsMix7ImN1cnZlIjo0fV0sWzEsMiwiaCIsMCx7ImN1cnZlIjotNH1dLFsxLDIsImsiLDIseyJjdXJ2ZSI6NH1dLFszLDQsIlxcbWF0aGJmIHUiLDEseyJzaG9ydGVuIjp7InNvdXJjZSI6MjAsInRhcmdldCI6MjB9fV0sWzUsNiwiXFxtYXRoYmYgdiIsMSx7InNob3J0ZW4iOnsic291cmNlIjoyMCwidGFyZ2V0IjoyMH19XV0=
      \begin{tikzcd}
        X && Y && Z
        \arrow[""{name=0, anchor=center, inner sep=0}, "f", curve={height=-24pt}, from=1-1, to=1-3]
        \arrow[""{name=1, anchor=center, inner sep=0}, "g"', curve={height=24pt}, from=1-1, to=1-3]
        \arrow[""{name=2, anchor=center, inner sep=0}, "h", curve={height=-24pt}, from=1-3, to=1-5]
        \arrow[""{name=3, anchor=center, inner sep=0}, "k"', curve={height=24pt}, from=1-3, to=1-5]
        \arrow["{\mathbf u}"{description}, shorten <=6pt, shorten >=6pt, Rightarrow, from=0, to=1]
        \arrow["{\mathbf v}"{description}, shorten <=6pt, shorten >=6pt, Rightarrow, from=2, to=3]
      \end{tikzcd}
    \end{equation*}
\end{enumerate}

\begin{notation}
  We adopt similar conventions as for 1-categories:
  \begin{itemize}
    \item $X\in\bicat{C}$ means $X\in\Obj_{\bicat{C}}$;
    \item We drop the 1-cell composition and therefore the horizontal
      composition of 2-cells. Thus $\bicell v\bicell u:hf\Rightarrow kg$ means
      $\bicell v\circ\bicell u:h\circ f\Rightarrow k\circ g$.
    \item To avoid unnecessary subscripts we identify objects and 1-cells with their
      respective identities. Thus we might write $h\bicell uX:hfX\Rightarrow hgX$
      to mean
      \begin{align*}
        \textbf{id}_h\circ\bicell u\circ\textbf{id}_{\id_X}:h\circ f\circ\id_X\Rightarrow h\circ g\circ\id_X.
      \end{align*}
      Note how we identify $\textbf{id}_{\id_X}$ with $\id_X$ and thus with $X$.
  \end{itemize}
\end{notation}


\begin{example}
  The most important 2-category that we will consider is $\biCAT$ which has as
  objects all categories, as 1-cells all functors, and as 2-cells all natural transformations.
  That is, for all $\cat{C},\cat{D}\in\biCAT$, $\Hom\bb{\cat{C},\cat{D}}$ is the functor
  category $\bb{\cat{C},\cat{D}}$. Composition of 1-cells is the usual composition of functors.
\end{example}

Given two 2-categories, we obtain the product 2-category which is entirely analogous to the
product 1-category:

\begin{definition}\label{def:product_2category}
  Let $\bicat{C},\bicat{D}$ be 2-categories. The \emph{product 2-category} $\bicat{C}\times\bicat{D}$ has
  \begin{enumerate}
    \item as objects all pairs $(X,X')$ with $X\in\bicat{C}$ and $X'\in\bicat{D}$;
    \item for all objects $(X,X')$ and $(Y,Y')$, the hom-category
      \begin{align*}
        \Hom\bb{(X,Y),(X',Y')} = \Hom\bb{X,Y}\times\Hom\bb{X',Y'};
      \end{align*}
    \item for all objects $(X,X')$, the identity 1-cell $(\text{id}_X,\text{id}_{X'})$;
    \item the composition functor given by pointwise composition in the respective
      2-categories, i.e. $(g,g')(f,f')=(gf,g'f')$.
  \end{enumerate}
\end{definition}

\section{Cartesian structure}\label{sec:bicartesian_2categories}

Kock's commutative monads are strong with respect to the cartesian structure
of a 1-category. In order to formalise commutative relative pseudomonads, we require
a similar structure on 2-categories. We are fortunate in that our main 2-category
of interest, \biCAT, admits strict products. That is, the required 1-cell equations
hold up to equality and not merely up to isomorphism.

While it is straightforward to define larger product structures
(e.g.~\cite{saville2020}), the notion of strength of a monad only
requires binary products and a terminal object which serves as the
identity.

The standard definition of a 1-categorical product that we outlined in
\ref{sec:cartesian_categories} requires an alternative formulation
before we may generalise it elegantly. We invite the reader to verify
that adapting the following definition to 1-categories in the obvious
way does indeed yields a product in the usual sense.

\begin{definition}\label{def:bicategories_products}
  A \emph{cartesian structure for a 2-category $\bicat{C}$} consists of
  \begin{enumerate}
    \item an object $1\in\bicat{C}$;
    \item for all $W\in\bicat{C}$, an isomorphism of categories
      $\Hom\bb{W,1}\cong\mathbb 1$;
    \item for all $X,Y\in\bicat{C}$:
      \begin{enumerate}
        \item an object $X\times Y\in\bicat{C}$;
        \item 1-cells $\pi_1:X\times Y\to X$ and $\pi_2:X\times Y\to Y$ called projections;
        \item for all $W\in\bicat{C}$, an isomorphism of categories
          \begin{equation}\label{eq:cartesian_isomorphism}
            % https://q.uiver.app/?q=WzAsMixbMCwwLCJcXHRleHR7SG9tfVtXLFhcXHRpbWVzIFldIl0sWzIsMCwiXFx0ZXh0e0hvbX1bVyxYXVxcdGltZXNcXHRleHR7SG9tfVtXLFldIl0sWzAsMSwiKFxccGlfMVxcY2lyYy0sXFxwaV8yXFxjaXJjIC0pIiwwLHsiY3VydmUiOi0zfV0sWzEsMCwiXFxsYW5nbGUtXFxyYW5nbGUiLDAseyJjdXJ2ZSI6LTN9XSxbMiwzLCJcXGNvbmciLDEseyJzaG9ydGVuIjp7InNvdXJjZSI6MjAsInRhcmdldCI6MjB9LCJzdHlsZSI6eyJib2R5Ijp7Im5hbWUiOiJub25lIn0sImhlYWQiOnsibmFtZSI6Im5vbmUifX19XV0=
            \begin{tikzcd}
              {\text{Hom}[W,X\times Y]} && {\text{Hom}[W,X]\times\text{Hom}[W,Y]}
              \arrow[""{name=0, anchor=center, inner sep=0}, "{(\pi_1\circ-,\pi_2\circ -)}", curve={height=-18pt}, from=1-1, to=1-3]
              \arrow[""{name=1, anchor=center, inner sep=0}, "{\langle-\rangle}", curve={height=-18pt}, from=1-3, to=1-1]
              \arrow["\cong"{description}, draw=none, from=0, to=1]
            \end{tikzcd}
          \end{equation}
          The functor $\aa{-}$ is called tupling.
      \end{enumerate}
  \end{enumerate}
\end{definition}

\begin{example}
  Let $\cat{C},\cat{D}$ be 1-categories. It is known that the product of categories
  $\cat{C}\times\cat{D}$ is a product in $\CAT$. Let $\pi_1,\pi_2$ denote the corresponding
  projections in $\CAT$. These are 1-cells in $\biCAT$. Define the tupling functor by
  \begin{align*}
    \aa{F,G} &=  \rr{F-,G-} \\
    \aa{\phi,\psi}_X &= \rr{\phi_X,\psi_X}
  \end{align*}
  By using the universal properties of the 1-categorical product it is straightforward to
  verify that this is indeed an inverse to $\rr{\pi_1\circ-,\pi_2,\circ -}$. Thus the
  1-categorical cartesian structure of $\CAT$ extends to a 2-categorical cartesian
  structure of $\biCAT$.
\end{example}

\section{Pseudofunctors}\label{sec:inclusions}

Category theory is in many ways about studying morphisms rather than objects.
We are therefore interested in defining the notion of a morphism between 2-categories which we
call pseudofunctors.
Just as a functor specifies where objects and morphisms are mapped to,
a pseudofunctor is a map on objects, 1-cells, and 2-cells.

There is an important difference, however: pseudofunctors are not the
most strict morphism between 2-categories. In particular, we allow for
the possibility that the functor axioms hold only up to a canonical
isomorphism. This is achieved by specifying these isomorphisms as part
of the structure: preservation of identities is witnessed by $\bicell i$
and distributivity is witnessed by $\bicell d$.

Now that the functoriality axioms are part of the structure, the axioms of a
pseudofunctor serve a different purpose entirely. While the structural 2-cells
ensure that certain 1-cells are isomorphic, we want to avoid the possibility of
deriving multiple different isomorphisms between the same 1-cells from this
structure. This is referred to as coherence. While proving coherence directly
is hard in general, many structures come with a sufficient set of conditions
that lead to coherence.

\begin{definition}\label{def:pseudofunctor}
  Let $\bicat{C},\bicat{D}$ be 2-categories. Then a \emph{pseudofunctor $F:\bicat{C}\to\bicat{D}$} consists of
  \begin{enumerate}
    \item for all $X\in\bicat{C}$, an object $FX\in\bicat{D}$;
    \item for all $X,Y\in{\bicat{C}}$, a functor
      $F_{X,Y}:\Hom\bb{X,Y}\to\Hom\bb{FX,FY}$;
    \item for all $X\in\bicat{C}$, an invertible 2-cell
      \begin{equation}\label{eq:pseudofunctor_identity}
        % https://q.uiver.app/?q=WzAsMixbMCwwLCJGWCJdLFswLDEsIkZYIl0sWzAsMSwiIiwwLHsiY3VydmUiOi01LCJsZXZlbCI6Miwic3R5bGUiOnsiaGVhZCI6eyJuYW1lIjoibm9uZSJ9fX1dLFswLDEsIkZfe1gsWH0oXFx0ZXh0e2lkfV9YKSIsMix7ImN1cnZlIjo1fV0sWzMsMiwiXFxiaWNlbGwgaV9YIiwwLHsic2hvcnRlbiI6eyJzb3VyY2UiOjIwLCJ0YXJnZXQiOjIwfX1dXQ==
        \begin{tikzcd}
          FX \\
          FX
          \arrow[""{name=0, anchor=center, inner sep=0}, curve={height=-30pt}, Rightarrow, no head, from=1-1, to=2-1]
          \arrow[""{name=1, anchor=center, inner sep=0}, "{F_{X,X}(\text{id}_X)}"', curve={height=30pt}, from=1-1, to=2-1]
          \arrow["{\bicell i_X}", shorten <=12pt, shorten >=12pt, Rightarrow, from=1, to=0]
        \end{tikzcd}
      \end{equation}
    \item for all $f:X\to Y$ and $g:Y\to Z$ in $\bicat{C}$, an invertible 2-cell
      \begin{equation}\label{eq:pseudofunctor_distributivity}
        % https://q.uiver.app/?q=WzAsMyxbMCwwLCJGWCJdLFsyLDAsIkZZIl0sWzEsMSwiRloiXSxbMCwxLCJGZiJdLFsxLDIsIkZnIl0sWzAsMiwiRihnZikiLDJdLFs1LDQsIlxcbWF0aGJmIGRfe2YsZ30iLDAseyJzaG9ydGVuIjp7InNvdXJjZSI6MjAsInRhcmdldCI6MjB9fV1d
        \begin{tikzcd}
          FX && FY \\
             & FZ
             \arrow["Ff", from=1-1, to=1-3]
             \arrow[""{name=0, anchor=center, inner sep=0}, "Fg", from=1-3, to=2-2]
             \arrow[""{name=1, anchor=center, inner sep=0}, "{F(gf)}"', from=1-1, to=2-2]
             \arrow["{\mathbf d_{f,g}}", shorten <=6pt, shorten >=6pt, Rightarrow, from=1, to=0]
        \end{tikzcd}
      \end{equation}
  \end{enumerate}
  such that
  \begin{enumerate}
    \item for all composable 1-cells $f,g,h$,
      \begin{equation}\label{eq:pseudofunctor_coherence_associativity}
        % https://q.uiver.app/?q=WzAsNCxbMCwwLCJGKGhnZikiXSxbMCwxLCJGKGhnKSBGZiJdLFsxLDAsIkZoRihnZikiXSxbMSwxLCJGaEZnRmYiXSxbMCwxLCJcXG1hdGhiZiBkIiwyXSxbMCwyLCJcXG1hdGhiZiBkIl0sWzIsMywiRmhcXG1hdGhiZiBkIl0sWzEsMywiXFxtYXRoYmYgZEZmIiwyXV0=
        \begin{tikzcd}
          {F(hgf)} & {FhF(gf)} \\
          {F(hg) Ff} & FhFgFf
          \arrow["{\mathbf d}"', from=1-1, to=2-1]
          \arrow["{\mathbf d}", from=1-1, to=1-2]
          \arrow["{Fh\mathbf d}", from=1-2, to=2-2]
          \arrow["{\mathbf dFf}"', from=2-1, to=2-2]
        \end{tikzcd}
      \end{equation}
    \item for all 1-cells $f$,
      \begin{equation}\label{eq:pseudofunctor_coherence_identity}
        % https://q.uiver.app/?q=WzAsNCxbMCwwLCJGZkYoXFx0ZXh0e2lkfSkiXSxbMCwxLCJGKGZcXHRleHR7aWR9KSJdLFsyLDAsIkZmXFx0ZXh0e2lkfSJdLFsyLDEsIkZmIl0sWzAsMSwiXFxtYXRoYmYgZF57LTF9IiwyXSxbMCwyLCJGZlxcbWF0aGJmIGkiXSxbMiwzLCIiLDIseyJsZXZlbCI6Miwic3R5bGUiOnsiaGVhZCI6eyJuYW1lIjoibm9uZSJ9fX1dLFsxLDMsIiIsMCx7ImxldmVsIjoyLCJzdHlsZSI6eyJoZWFkIjp7Im5hbWUiOiJub25lIn19fV1d
        \begin{tikzcd}
          {FfF(\text{id})} && {Ff\text{id}} \\
          {F(f\text{id})} && Ff
          \arrow["{\mathbf d^{-1}}"', from=1-1, to=2-1]
          \arrow["{Ff\mathbf i}", from=1-1, to=1-3]
          \arrow[Rightarrow, no head, from=1-3, to=2-3]
          \arrow[Rightarrow, no head, from=2-1, to=2-3]
        \end{tikzcd}\hs
        % https://q.uiver.app/?q=WzAsNCxbMCwwLCJGKFxcdGV4dHtpZH0pRmYiXSxbMCwxLCJGKFxcdGV4dHtpZH1mKSJdLFsyLDAsIlxcdGV4dHtpZH1GZiJdLFsyLDEsIkZmIl0sWzAsMSwiXFxtYXRoYmYgZF57LTF9IiwyXSxbMCwyLCJcXG1hdGhiZiBpIEZmIl0sWzIsMywiIiwyLHsibGV2ZWwiOjIsInN0eWxlIjp7ImhlYWQiOnsibmFtZSI6Im5vbmUifX19XSxbMSwzLCIiLDAseyJsZXZlbCI6Miwic3R5bGUiOnsiaGVhZCI6eyJuYW1lIjoibm9uZSJ9fX1dXQ==
        \begin{tikzcd}
          {F(\text{id})Ff} && {\text{id}Ff} \\
          {F(\text{id}f)} && Ff
          \arrow["{\mathbf d^{-1}}"', from=1-1, to=2-1]
          \arrow["{\mathbf i Ff}", from=1-1, to=1-3]
          \arrow[Rightarrow, no head, from=1-3, to=2-3]
          \arrow[Rightarrow, no head, from=2-1, to=2-3]
        \end{tikzcd}
      \end{equation}
  \end{enumerate}
  A 2-functor is a pseudofunctor whose 2-cells $\bicell i$ and $\bicell d$ are identities.
\end{definition}

As we will be dealing with at most one pseudofunctor with non-identity 2-cells, our
notation is unambiguous. One 2-functor that will be of interest is the product 2-functor
that allows us to take products not just of objects but also of 1-cells and 2-cells:

\begin{example}\label{ex:product_2functor}
  Analogously to the 1-categorical case (see \ref{sec:cartesian_categories}),
  we observe that taking binary products in a cartesian 2-category yields a
  2-functor $\bicat{C}\times\bicat{C}\to\bicat{C}$, given by the following
  structure:
  \begin{enumerate}
    \item for all $(X,Y)\in\bicat{C}\times\bicat{C}$, we have the object $X\times Y\in\bicat{C}$;
    \item for all $X,X',Y,Y'\in\bicat{C}$, the functor on hom-categories is given
      by the composite
      \begin{equation}
        % https://q.uiver.app/?q=WzAsMyxbMCwwLCJcXHRleHR7SG9tfVtYLFldXFx0aW1lc1xcdGV4dHtIb219W1gnLFknXSJdLFsyLDEsIlxcdGV4dHtIb219W1hcXHRpbWVzIFgnLFlcXHRpbWVzIFknXSJdLFsyLDAsIlxcdGV4dHtIb219W1hcXHRpbWVzIFgnLFldXFx0aW1lc1xcdGV4dHtIb219W1hcXHRpbWVzIFgnLFknXSJdLFsyLDEsIlxcbGFuZ2xlLV9rXFxyYW5nbGUiXSxbMCwyLCIoLVxcY2lyY1xccGlfMSwtXFxjaXJjXFxwaV8yKSJdLFswLDEsIihcXHRpbWVzKV97KFgsWCcpLChZLFknKX0iLDJdXQ==
        \begin{tikzcd}
          {\text{Hom}[X,Y]\times\text{Hom}[X',Y']} && {\text{Hom}[X\times X',Y]\times\text{Hom}[X\times X',Y']} \\
                                                   && {\text{Hom}[X\times X',Y\times Y']}
                                                   \arrow["{\langle-_k\rangle}", from=1-3, to=2-3]
                                                   \arrow["{(-\circ\pi_1,-\circ\pi_2)}", from=1-1, to=1-3]
                                                   \arrow["{(\times)_{(X,X'),(Y,Y')}}"', from=1-1, to=2-3]
        \end{tikzcd}
      \end{equation}
  \end{enumerate}
  We need to verify that \ref{eq:pseudofunctor_identity} and \ref{eq:pseudofunctor_distributivity}
  are identities. We verify, for all $X,X'\in\bicat{C}$,
  \begin{align*}
    \text{id}_X\times \text{id}_{X'} = \aa{\pi_1,\pi_2} = \text{id}_{X\times X'}
  \end{align*}
  where the last equality follows from the isomorphism
  (\ref{eq:cartesian_isomorphism}). Similarly, consider
  $(f,f'):(X,X')\to(Y,Y')$ and $(g,g'):(Y,Y')\to(Z,Z')$ in
  $\bicat{C}\times\bicat{C}$. In the diagram
  \begin{equation}
    % https://q.uiver.app/?q=WzAsOSxbMiwwLCJYXFx0aW1lcyBYJyJdLFsyLDEsIllcXHRpbWVzIFknIl0sWzIsMiwiWlxcdGltZXMgWiciXSxbNCwyLCJaJyJdLFs0LDEsIlknIl0sWzQsMCwiWCciXSxbMCwwLCJYIl0sWzAsMSwiWSJdLFswLDIsIloiXSxbNyw4LCJnIiwyXSxbNiw3LCJmIiwyXSxbNSw0LCJmJyJdLFs0LDMsImcnIl0sWzEsMiwiXFxsYW5nbGUgZ1xccGlfMSxnJ1xccGlfMlxccmFuZ2xlIiwyXSxbMCwxLCJcXGxhbmdsZSBmXFxwaV8xLGYnXFxwaV8yXFxyYW5nbGUiLDJdLFswLDYsIlxccGlfMSIsMl0sWzAsNSwiXFxwaV8yIl0sWzEsNywiXFxwaV8xIiwyXSxbMSw0LCJcXHBpXzIiXSxbMiw4LCJcXHBpXzEiLDJdLFsyLDMsIlxccGlfMiJdXQ==
    \begin{tikzcd}
      X && {X\times X'} && {X'} \\
      Y && {Y\times Y'} && {Y'} \\
      Z && {Z\times Z'} && {Z'}
      \arrow["g"', from=2-1, to=3-1]
      \arrow["f"', from=1-1, to=2-1]
      \arrow["{f'}", from=1-5, to=2-5]
      \arrow["{g'}", from=2-5, to=3-5]
      \arrow["{\langle g\pi_1,g'\pi_2\rangle}"', from=2-3, to=3-3]
      \arrow["{\langle f\pi_1,f'\pi_2\rangle}"', from=1-3, to=2-3]
      \arrow["{\pi_1}"', from=1-3, to=1-1]
      \arrow["{\pi_2}", from=1-3, to=1-5]
      \arrow["{\pi_1}"', from=2-3, to=2-1]
      \arrow["{\pi_2}", from=2-3, to=2-5]
      \arrow["{\pi_1}"', from=3-3, to=3-1]
      \arrow["{\pi_2}", from=3-3, to=3-5]
    \end{tikzcd}
  \end{equation}
  each tile commutes by (\ref{eq:cartesian_isomorphism}). Now notice $gf\times g'f' =
  \aa{gf\pi_1,g'f'\pi_2}$. The claim then follows by the universal property of
  the product.
\end{example}

The presheaf construction gives rise to a pseudofunctor. We will show this in
the next chapter (see \ref{prop:induced_pseudofunctor}). For now, let us
briefly investigate what this structure looks like without spelling out the
details:

\begin{example}\label{ex:presheaf_pseudofunctor}
  Consider the pseudofunctor structure $\widehat{-} : \biCat\to\biCAT$:
  \begin{enumerate}
    \item for all $\scat X\in\biCat$, $\widehat{\scat X}=\bb{\scatop X,\Set}$;
    \item for all $F:\scat X\to\scat Y$ and $P\in\widehat{\scat X}$,
      \begin{align*}
        \widehat{F}P = \int^X PX\times\Hom\rr{-,FX};
      \end{align*}
    \item the natural isomorphism $\bicell i$ has as components natural isomorphisms
      \begin{align*}
        \int^X PX\times\Hom(-,X) \cong P;
      \end{align*}
    \item for all composable functors $F$ and $G$, the natural isomorphism $\bicell d_{F,G}$
      has as components isomorphisms
      \begin{align*}
        \int^X PX\times\Hom(-,(GF)X) \cong \int^Y\rr{\int^X PX\times\Hom(Y,FX)}\times\Hom(-,GY).
      \end{align*}
  \end{enumerate}
\end{example}

There are two functors corresponding to a particular relative monad: on the one
hand there is the functor that the monad is relative to and on the other we
have the functor induced by the monad~\cite{altenkirch2015}. Similarly, a
relative pseudomonad has two corresponding pseudofunctors. It is for this
reason that we require two degrees of strictness: while the inclusion
$\biCat\to\biCAT$ is a 2-functor, the pseudofunctor induced by the presheaf
construction is not. To limit complexity, we aim to make use of the strictness
of the former while allowing for the non-strictness of the latter.

\section{Inclusions of cartesian 2-categories}

Before we can endow the presheaf construction with a strength, we need to
capture some of the additional properties of the inclusion. It only makes sense
to define a strong pseudomonad relative to a 2-functor $\bicat J\to\bicat C$ if
both its domain and codomain are cartesian and the cartesian structures agree.
We could treat this in full generality by considering monoidal pseudofunctors,
similar to~\cite{tarmo}. However, the inclusion $\biCat\to\biCAT$ preserves
products and terminal objects we are able to impose and satisfy much stricter
conditions.

An inclusion functor is injective on objects and morphisms. Similarly, we demand an
inclusion pseudofunctor be injective on objects, 1-cells, and 2-cells. Moreover,
we require inclusions to preserve the cartesian structure.

\begin{definition}
  An \emph{inclusion of cartesian 2-categories} consists of
  \begin{enumerate}
    \item cartesian 2-categories $\bicat{J}$ and $\bicat{C}$;
    \item a 2-functor $J:\bicat{J}\to\bicat{C}$;
  \end{enumerate}
  such that
  \begin{enumerate}
    \item $J$ is injective on objects, 1-cells, and 2-cells;
    \item for all $X,Y\in\bicat{J}$,
      \begin{align*}
        J1=1,\hs
        J\rr{X\times Y}=JX\times JY,\hs
        J(\pi_1) = \pi_1,\hs J(\pi_2)=\pi_2.
      \end{align*}
  \end{enumerate}
\end{definition}

\begin{example}
  Define $\biCat$ as the 2-category with objects all small categories and, for all
  $\scat{C},\scat D\in\biCat$, the hom-category $\Hom\bb{\scat C,\scat D}=\bb{\scat C,\scat D}$.
  We now have the obvious inclusion $J:\biCat\to\biCAT$ that is the identity
  on objects and hom-categories. Define the cartesian structure
  on $\biCat$ in the same way that we defined it for $\biCAT$. This is justified as the
  product of small categories is itself small. Thus $J$ forms an inclusion of cartesian
  2-categories.
\end{example}

\section{Pseudonatural transformations}

Naturally, we have morphisms between pseudofunctors. The naturality of a
pseudonatural transformation is witnessed by a structural 2-cell which obeys a
coherence axiom. Once again, we have two levels of strictness. It is worth
noting, however, that these are not related to the strictness of the underlying
pseudofunctors.

\begin{definition}
  Let $F,G:\bicat{C}\to\bicat{D}$ be pseudofunctors. A \emph{pseudonatural transformation}
  $\phi:F\Rightarrow G$ consists of
  \begin{enumerate}
    \item for all $X\in\bicat C$, a 1-cell $\phi_X : FX\to GX$;
    \item for all $f:X\to Y$ in $\bicat C$, a naturality 2-cell
      \begin{equation}\label{eq:naturality_2cell}
        % https://q.uiver.app/?q=WzAsNCxbMCwwLCJGWCJdLFsyLDAsIkZZIl0sWzAsMSwiR1giXSxbMiwxLCJHWSJdLFswLDEsIkZmIiwxXSxbMCwyLCJcXHBoaSIsMl0sWzIsMywiR2YiLDFdLFsxLDMsIlxccGhpIl0sWzUsNywiXFxtYXRoYmYgbl9mIiwxLHsic2hvcnRlbiI6eyJzb3VyY2UiOjIwLCJ0YXJnZXQiOjIwfX1dXQ==
        \begin{tikzcd}
          FX && FY \\
          GX && GY
          \arrow["Ff"{description}, from=1-1, to=1-3]
          \arrow[""{name=0, anchor=center, inner sep=0}, "\phi"', from=1-1, to=2-1]
          \arrow["Gf"{description}, from=2-1, to=2-3]
          \arrow[""{name=1, anchor=center, inner sep=0}, "\phi", from=1-3, to=2-3]
          \arrow["{\mathbf n_f}"{description}, shorten <=13pt, shorten >=13pt, Rightarrow, from=0, to=1]
        \end{tikzcd}
      \end{equation}
  \end{enumerate}
  such that, for all composable 1-cells $f,g\in\bicat C$,
  \begin{equation}
    % https://q.uiver.app/?q=WzAsNSxbMCwwLCJHZ0dmXFxwaGkiXSxbMiwwLCJHZ1xccGhpIEZmIl0sWzQsMCwiXFxwaGkgRmcgRmYiXSxbNCwxLCJcXHBoaSBGKGdmKSJdLFswLDEsIkcoZ2YpXFxwaGkiXSxbMCw0LCJcXG1hdGhiZiBkXnstMX1cXHBoaSIsMl0sWzAsMSwiR2dcXG1hdGhiZiBuIl0sWzEsMiwiXFxtYXRoYmYgbiBGZiJdLFsyLDMsIlxccGhpXFxtYXRoYmYgZF57LTF9Il0sWzQsMywiXFxtYXRoYmYgbiIsMl1d
    \begin{tikzcd}
      GgGf\phi && {Gg\phi Ff} && {\phi Fg Ff} \\
      {G(gf)\phi} &&&& {\phi F(gf)}
      \arrow["{\mathbf d^{-1}\phi}"', from=1-1, to=2-1]
      \arrow["{Gg\mathbf n}", from=1-1, to=1-3]
      \arrow["{\mathbf n Ff}", from=1-3, to=1-5]
      \arrow["{\phi\mathbf d^{-1}}", from=1-5, to=2-5]
      \arrow["{\mathbf n}"', from=2-1, to=2-5]
    \end{tikzcd}
  \end{equation}
  A 2-natural transformation is a pseudonatural transformation whose naturality 2-cell
  $\bicell n$ is the identity.
\end{definition}

\begin{example}\label{ex:cartesian_monoidal_structure}
  Let $\bicat{C}$ be a cartesian 2-category. We saw that the product 2-functor is merely
  an extension of the usual product functor. Because $\bicell n$ is the identity,
  each natural transformation between product functors yields a 2-natural transformation
  between product 2-functors. Thus the usual isomorphisms $\lambda,\alpha,\gamma$ are
  in fact 2-natural transformations.
\end{example}
