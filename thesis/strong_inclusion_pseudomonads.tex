\chapter{Strong inclusion pseudomonads}\label{sec:strong_relative_pseudomonads}

We have now developed all the theory required to present our approach to defining
a generalisation of strong monads that admits the presheaf construction as a model.
We do so in three major steps, each of which we will justify by applying it to the
presheaf construction and comparing it to related work

Firstly, we present a structure that is similar to that of a relative
pseudomonad. We justify the differences and show how the presheaf construction
gives rise to such a structure. Secondly, we state the axioms that this
structure is required to satisfy in order to induce a relative pseudomonad.
Finally, we extend the structure in a way that induces a strong pseudomonad
structure in the case where the inclusion is the identity. Unfortunately, we
are not able to postulate axioms that are sufficient to show that this induced
structure satsifies the axioms of a strong pseudomonad.

Fix an inclusion of cartesian 2-categories $J:\bicat{J}\to\bicat{C}$.

\section{Prestrong structure}

A strong monad is a monad equipped with a suitable natural transformation.
Hence the obvious way to define a strong relative pseudomonad is by equipping a relative
pseudomonad with a suitable pseudonatural transformation.
This approach would have two notable benefits: Firstly, it would allow us to leverage already
existing results about relative pseudomonads without any overhead. Secondly, it would
provide us with an intuitive connection to strong pseudomonads that would, presumably,
make some of the axioms easier to postulate.

We investigate another approach. Rather than adding the strength as additional structure,
we choose to build it in to the extension operator. That is, we only allow
1-cells $f:W\times JX\to TY$ to be extended to $f^\dagger:W\times TX\to TY$.
This relates the extension to the cartesian structure directly. Another reason why we
believe that this idea is worth entertaining is because it becomes very pleasant to
describe commutativity of such a structure.
One can demand the existence of the following natural isomorphism:

\begin{equation*}
  % https://q.uiver.app/?q=WzAsOCxbMCwwLCJcXHRleHR7SG9tfVtKWFxcdGltZXMgSlksVFpdIl0sWzEsMCwiXFx0ZXh0e0hvbX1bSlhcXHRpbWVzIFRZLFRaXSJdLFsyLDAsIlxcdGV4dHtIb219W1RZXFx0aW1lcyBKWCxUWl0iXSxbMiwxLCJcXHRleHR7SG9tfVtUWVxcdGltZXMgVFgsVFpdIl0sWzIsMiwiXFx0ZXh0e0hvbX1bVFhcXHRpbWVzIFRZLFRaXSJdLFsxLDIsIlxcdGV4dHtIb219W1RYXFx0aW1lcyBKWSxUWl0iXSxbMCwyLCJcXHRleHR7SG9tfVtKWVxcdGltZXMgVFgsVFpdIl0sWzAsMSwiXFx0ZXh0e0hvbX1bSllcXHRpbWVzIEpYLFRaXSJdLFswLDcsIi1cXGNpcmNcXGdhbW1hIiwyXSxbNyw2LCIoLSleXFxkYWdnZXIiLDJdLFs2LDUsIi1cXGNpcmNcXGdhbW1hIiwyLHsiY3VydmUiOjJ9XSxbNSw0LCIoLSleXFxkYWdnZXIiLDIseyJjdXJ2ZSI6Mn1dLFswLDEsIigtKV5cXGRhZ2dlciIsMCx7ImN1cnZlIjotMn1dLFsxLDIsIi1cXGNpcmNcXGdhbW1hIiwwLHsiY3VydmUiOi0yfV0sWzIsMywiKC0pXlxcZGFnZ2VyIl0sWzMsNCwiLVxcY2lyY1xcZ2FtbWEiXSxbNywzLCJcXGNvbmciLDEseyJzaG9ydGVuIjp7InNvdXJjZSI6MjAsInRhcmdldCI6MjB9LCJsZXZlbCI6Miwic3R5bGUiOnsiYm9keSI6eyJuYW1lIjoibm9uZSJ9LCJoZWFkIjp7Im5hbWUiOiJub25lIn19fV1d
  \begin{tikzcd}
    {\text{Hom}[JX\times JY,TZ]} & {\text{Hom}[JX\times TY,TZ]} & {\text{Hom}[TY\times JX,TZ]} \\
    {\text{Hom}[JY\times JX,TZ]} && {\text{Hom}[TY\times TX,TZ]} \\
    {\text{Hom}[JY\times TX,TZ]} & {\text{Hom}[TX\times JY,TZ]} & {\text{Hom}[TX\times TY,TZ]}
    \arrow["{-\circ\gamma}"', from=1-1, to=2-1]
    \arrow["{(-)^\dagger}"', from=2-1, to=3-1]
    \arrow["{-\circ\gamma}"', curve={height=12pt}, from=3-1, to=3-2]
    \arrow["{(-)^\dagger}"', curve={height=12pt}, from=3-2, to=3-3]
    \arrow["{(-)^\dagger}", curve={height=-12pt}, from=1-1, to=1-2]
    \arrow["{-\circ\gamma}", curve={height=-12pt}, from=1-2, to=1-3]
    \arrow["{(-)^\dagger}", from=1-3, to=2-3]
    \arrow["{-\circ\gamma}", from=2-3, to=3-3]
    \arrow["\cong"{description}, draw=none, from=2-1, to=2-3]
  \end{tikzcd}
\end{equation*}

Given that synthetic measure theory requires a commutative monad, it is possible that
this alternative structure is more convenient in the context of a generalised synthetic
measure theory.

\begin{definition}\label{def:prestrong_inclusion_pseudomonad_structure}
  A \emph{strong $J$-pseudomonad structure} consists of
  \begin{enumerate}
    \item for all $X\in\bicat{J}$, an object $TX\in\bicat{C}$;
    \item for all $X\in\bicat{J}$, a 1-cell $\eta_X:JX\to TX$ in $\bicat{C}$;
    \item for all $X,Y\in\bicat{J}$ and $W\in\bicat{C}$, a functor
      \begin{align}\label{eq:strong_pseudomonad_extension}
        \extend{\rr{-}}{\dagger}_{X,Y,W}:\Hom\bb{W\times JX,TY}\to\Hom\bb{W\times TX,TY};
      \end{align}
    \item for all $f:JW\times JX\to TY$ in $\bicat{C}$, an invertible 2-cell
      \begin{equation}\label{eq:strong_pseudomonad_r}
        % https://q.uiver.app/?q=WzAsMyxbMCwwLCJXXFx0aW1lcyBKWCJdLFsxLDEsIlRZIl0sWzIsMCwiV1xcdGltZXMgVFgiXSxbMiwxLCJmXlxcZGFnZ2VyIl0sWzAsMSwiZiIsMl0sWzAsMiwiV1xcdGltZXNcXGV0YSJdLFs0LDMsIlxcYmljZWxsIHJfZiIsMCx7InNob3J0ZW4iOnsic291cmNlIjoyMCwidGFyZ2V0IjoyMH19XV0=
        \begin{tikzcd}
          {W\times JX} && {W\times TX} \\
                       & TY
                       \arrow[""{name=0, anchor=center, inner sep=0}, "{f^\dagger}", from=1-3, to=2-2]
                       \arrow[""{name=1, anchor=center, inner sep=0}, "f"', from=1-1, to=2-2]
                       \arrow["W\times\eta", from=1-1, to=1-3]
                       \arrow["{\bicell r_f}", shorten <=7pt, shorten >=7pt, Rightarrow, from=1, to=0]
        \end{tikzcd}
      \end{equation}
    \item for all $X\in\bicat{J}$, an invertible 2-cell
      \begin{equation}\label{eq:strong_pseudomonad_l}
        % https://q.uiver.app/?q=WzAsMixbMCwwLCIxXFx0aW1lcyBUWCJdLFswLDIsIlRYIl0sWzAsMSwiXFxsYW1iZGEiLDIseyJjdXJ2ZSI6NX1dLFswLDEsIihcXGV0YVxcbGFtYmRhKV5cXGRhZ2dlciIsMCx7ImN1cnZlIjotNX1dLFsyLDMsIlxcbWF0aGJmIGxfWCIsMCx7InNob3J0ZW4iOnsic291cmNlIjoyMCwidGFyZ2V0IjoyMH19XV0=
        \begin{tikzcd}
          {1\times TX} \\
          \\
          TX
          \arrow[""{name=0, anchor=center, inner sep=0}, "\lambda"', curve={height=30pt}, from=1-1, to=3-1]
          \arrow[""{name=1, anchor=center, inner sep=0}, "{(\eta\lambda)^\dagger}", curve={height=-30pt}, from=1-1, to=3-1]
          \arrow["{\mathbf l_X}", shorten <=12pt, shorten >=12pt, Rightarrow, from=0, to=1]
        \end{tikzcd}
      \end{equation}
    \item for all $i:V\to W$, $f:JX\to TY$, and $g:W\times JY\to TZ$ in $\bicat{C}$,
      an invertible 2-cell
      \begin{equation}\label{eq:strong_pseudomonad_c}
        % https://q.uiver.app/?q=WzAsMyxbMCwwLCJWXFx0aW1lcyBUWCJdLFsyLDEsIlRaIl0sWzQsMCwiV1xcdGltZXMgVFkiXSxbMiwxLCJnXlQiXSxbMCwxLCIoZ15UKGlcXHRpbWVzIGYpKV5UIiwyXSxbMCwyLCJpXFx0aW1lcyAoZlxcbGFtYmRhKV5cXGRhZ2dlclxcbGFtYmRhXnstMX0iXSxbNCwzLCJcXG1hdGhiZiBjX3tmLGcsaX0iLDAseyJzaG9ydGVuIjp7InNvdXJjZSI6MjAsInRhcmdldCI6MjB9fV1d
        \begin{tikzcd}
          {V\times TX} &&&& {W\times TY} \\
                       && TZ
                       \arrow[""{name=0, anchor=center, inner sep=0}, "{g^T}", from=1-5, to=2-3]
                       \arrow[""{name=1, anchor=center, inner sep=0}, "{(g^T(i\times f))^T}"', from=1-1, to=2-3]
                       \arrow["{i\times (f\lambda)^\dagger\lambda^{-1}}", from=1-1, to=1-5]
                       \arrow["{\mathbf c_{f,g,i}}", shorten <=13pt, shorten >=13pt, Rightarrow, from=1, to=0]
        \end{tikzcd}
      \end{equation}
  \end{enumerate}
\end{definition}

We note that this extension operator is defined on a narrower selection of 1-cells
than the extension operator of a relative pseudomonad. This is only a temporary limitation.
Using the left unitor $\lambda$, we obtain the usual extension functor
\begin{align}\label{eq:relative_pseudomonad_extension}
  \extend{\rr{-}}{*}_{X,Y} = \extend{\rr{-\lambda_{JX}}}{\dagger}\inv\lambda_{JX};
\end{align}
as in the diagram
\begin{equation}
  % https://q.uiver.app/?q=WzAsNCxbMCwwLCJcXHRleHR7SG9tfVtKWCxUWV0iXSxbMiwwLCJcXHRleHR7SG9tfVtUWCxUWV0iXSxbMCwxLCJcXHRleHR7SG9tfVsxXFx0aW1lcyBKWCxUWV0iXSxbMiwxLCJcXHRleHR7SG9tfVsxXFx0aW1lcyBUWCxUWV0iXSxbMywxLCItXFxjaXJjXFxsYW1iZGFeey0xfSIsMl0sWzIsMywiKC0pXlxcZGFnZ2VyIiwyXSxbMCwyLCItXFxjaXJjXFxsYW1iZGEiLDJdLFswLDEsIigtKV4qIl1d
  \begin{tikzcd}
    {\text{Hom}[JX,TY]} && {\text{Hom}[TX,TY]} \\
    {\text{Hom}[1\times JX,TY]} && {\text{Hom}[1\times TX,TY]}
    \arrow["{-\circ\lambda^{-1}}"', from=2-3, to=1-3]
    \arrow["{(-)^\dagger}"', from=2-1, to=2-3]
    \arrow["{-\circ\lambda}"', from=1-1, to=2-1]
    \arrow["{(-)^*}", from=1-1, to=1-3]
  \end{tikzcd}
\end{equation}
This will help us state the axioms in the following section more cleanly.

The purpose of \ref{def:prestrong_inclusion_pseudomonad_structure} is to define a theory that admits
the presheaf construction as a model. While one might expect this result to be a straightforward
statement, some work is required to define the structure itself. One must pay special
attention when working with several layers of indirection.
For example, we have a natural transformation whose components are functors
$\lambda_{\scat X}:1\times\scat X\to\scat X$. This problem is amplified by the fact that
our main point of study are objects $[\scatop X,\Set]$ which are themselves categories
with functors as objects and natural transformations as morphisms.

Nonetheless, we define the entire structure in detail:

\begin{example}\label{ex:prestrong_presheaves}
  The \emph{prestrong presheaf construction} consists of:
  \begin{enumerate}
    \item for all $\scat X\in\biCat$, the object $\widehat{\scat X}=\bb{\scatop X,\Set}$;
    \item for all $\scat X\in\biCat$, the unit functor $\eta_{\scat X}$ given by
      $\eta_{\scat X}(X) = \Hom\rr{-,X}$;
    \item for all $\scat X,\scat Y\in\biCat$ and $\cat W\in\biCAT$, the extension functor
      $\extend{\rr{-}}{\dagger}_{\scat X,\scat Y,\cat W}$ given by
      \begin{enumerate}
        \item for all $F:\cat W\times\scat X\to \widehat{\scat Y}$, $W\in\cat W$,
          and $P\in\widehat{\scat X}$
          \begin{align}
            \extend{F}{\dagger}\rr{W,P} = \int^{X} PX\times F\rr{W,X}\rr{-}
          \end{align}
          and, for all morphisms $f$ in $\cat W$ and $\phi$ in $\widehat{\scat X}$,
          \begin{equation}
            \extend{F}{\dagger}\rr{f,\phi} = \int^{X} \phi_X\times F\rr{f,X}\rr{-};
          \end{equation}
        \item for all $F,G:\cat W\times\scat X\to\widehat{\scat Y}$,
          $\phi:F\Rightarrow G$, $W\in\cat W$, and $P\in\widehat{\scat X}$,
          \begin{align}
            \extend{\phi}{\dagger}_{W,P} = \int^{X} PX \times \rr{\phi_{W,P}}_{(-)};
          \end{align}
      \end{enumerate}
    \item for all $F:\cat W\times\scat X\to\widehat{\scat Y}$, $W\in\cat W$, and $X\in\scat X$,
      the natural transformation $(\bicell r_F)_{W,X}$ has the components
      \begin{equation}\label{eq:strong_presheaf_r}
        % https://q.uiver.app/?q=WzAsMyxbMCwwLCJGKFcsWClZIl0sWzIsMCwiXFx0ZXh0e0hvbX0oWCxYKVxcdGltZXMgRihXLFgpWSJdLFsyLDEsIlxcaW50XntYJ31cXHRleHR7SG9tfShYJyxYKVxcdGltZXMgRihXLFgnKVkiXSxbMCwxLCJcXGxhbmdsZSBcXERlbHRhXFx0ZXh0e2lkfSxGKFcsWClZXFxyYW5nbGUiXSxbMSwyLCJxIl0sWzAsMiwiKChcXG1hdGhiZiByX0YpX3tXLFh9KV9ZIiwyXV0=
        \begin{tikzcd}
          {F(W,X)Y} && {\text{Hom}(X,X)\times F(W,X)Y} \\
                    && {\int^{X'}\text{Hom}(X',X)\times F(W,X')Y}
                    \arrow["{\langle \Delta\text{id},F(W,X)Y\rangle}", from=1-1, to=1-3]
                    \arrow["q", from=1-3, to=2-3]
                    \arrow["{((\mathbf r_F)_{W,X})_Y}"', from=1-1, to=2-3]
        \end{tikzcd}
      \end{equation}
    \item for all $P\in\widehat{\scat X}$, the natural transformation $\rr{\bicell l_{\scat X}}_P$ has the components
      \begin{equation}\label{eq:strong_presheaf_l}
        % https://q.uiver.app/?q=WzAsMyxbMCwwLCJQWCJdLFsyLDEsIlxcaW50XntYJ30gUFgnXFx0aW1lc1xcdGV4dHtIb219KFgsWCcpIl0sWzIsMCwiUFhcXHRpbWVzXFx0ZXh0e0hvbX0oWCxYKSJdLFswLDIsIlxcbGFuZ2xlIFBYLFxcRGVsdGFcXHRleHR7aWR9XFxyYW5nbGUiXSxbMiwxLCJxIl0sWzAsMSwiKChcXG1hdGhiZiBsX3tcXG1hdGhiYiBYfSlfUClfWCIsMl1d
        \begin{tikzcd}
          PX && {PX\times\text{Hom}(X,X)} \\
             && {\int^{X'} PX'\times\text{Hom}(X,X')}
             \arrow["{\langle PX,\Delta\text{id}\rangle}", from=1-1, to=1-3]
             \arrow["q", from=1-3, to=2-3]
             \arrow["{((\mathbf l_{\mathbb X})_P)_X}"', from=1-1, to=2-3]
        \end{tikzcd}
      \end{equation}
    \item for all $F:\scat X\to\widehat{\scat Y}$, $G:\cat W\times\scat Y\to\widehat{\scat Z}$,
      $I:\cat V\to\scat W$, $P\in\widehat{\scat X}$, and $V\in\cat V$, the natural isomorphism
      $\rr{\bicell c_{F,G,I}}_{V,P}$ has as components the canonical cowedge
      isomorphisms given by
      \begin{equation*}
        % https://q.uiver.app/?q=WzAsNixbMCwyLCJcXGludF5YIFBYXFx0aW1lc1xcaW50XlkgKEZYKVlcXHRpbWVzIEcoSVYsWSlaIl0sWzEsMiwiXFxpbnReWVxcbGVmdChcXGludF5YIFBYXFx0aW1lcyAoRlgpWVxccmlnaHQpXFx0aW1lcyBHKElWLFkpWiJdLFswLDEsIlBYXFx0aW1lcyBcXGludF5ZIChGWClZXFx0aW1lcyBHKElWLFkpWiJdLFswLDAsIlBYXFx0aW1lc1xcbGVmdCgoRlgpWVxcdGltZXMgRyhJVixZKVpcXHJpZ2h0KSJdLFsxLDAsIihQWFxcdGltZXMgKEZYKVkpXFx0aW1lcyBHKElWLFkpWiJdLFsxLDEsIlxcbGVmdChcXGludF5YIFBYXFx0aW1lcyAoRlgpWVxccmlnaHQpXFx0aW1lcyBHKElWLFkpWiJdLFswLDEsIigoXFxtYXRoYmYgY197RixHLEl9KV97VixQfSlfWiIsMSx7ImN1cnZlIjozfV0sWzIsMCwicSIsMl0sWzUsMSwicSJdLFszLDIsIlBYXFx0aW1lcyBxIiwyXSxbNCw1LCJxXFx0aW1lcyBHKElWLFkpKC0pIl0sWzMsNCwiXFxjb25nIiwxLHsiY3VydmUiOi0zfV1d
        \begin{tikzcd}
          {PX\times\left((FX)Y\times G(IV,Y)Z\right)} & {(PX\times (FX)Y)\times G(IV,Y)Z} \\
          {PX\times \int^Y (FX)Y\times G(IV,Y)Z} & {\left(\int^X PX\times (FX)Y\right)\times G(IV,Y)Z} \\
          {\int^X PX\times\int^Y (FX)Y\times G(IV,Y)Z} & {\int^Y\left(\int^X PX\times (FX)Y\right)\times G(IV,Y)Z}
          \arrow["{((\mathbf c_{F,G,I})_{V,P})_Z}"{description}, curve={height=18pt}, from=3-1, to=3-2]
          \arrow["q"', from=2-1, to=3-1]
          \arrow["q", from=2-2, to=3-2]
          \arrow["{PX\times q}"', from=1-1, to=2-1]
          \arrow["{q\times G(IV,Y)(-)}", from=1-2, to=2-2]
          \arrow["\cong"{description}, curve={height=-18pt}, from=1-1, to=1-2]
        \end{tikzcd}
      \end{equation*}
  \end{enumerate}
\end{example}

Even from this detailed description it is not clear that we have provided the
necessary structure: we have to make sure that the structural 2-cells are in fact
invertible. For $\bicell c$, this is immediate and we already showed the result for
$\bicell l$ in \ref{ex:coend}. For $\bicell r$, the proof is similar:

\begin{lemma}\label{lemma:strong_presheaf_r_is_iso}
  The function (\ref{eq:strong_presheaf_r}) is an isomorphism.
  \begin{proof}
    For each $X'\in\scat X$, consider the function
    \begin{align*}
      w\rr{X'}:\Hom(X',X)\times F(W,X')Y&\to F(W,X)Y
    \end{align*}
    given by $\rr{f,u}\mapsto (F(W,f)Y)u$. We note that, for all 1-cells $f:X_1\to X_2$
    in $\scat X$,
    the following commutes:
    \begin{equation*}
      % https://q.uiver.app/?q=WzAsNCxbMSwxLCJGKFcsWClZIl0sWzAsMCwiXFx0ZXh0e0hvbX0oWF8yLFgpXFx0aW1lcyBGKFcsWF8xKVkiXSxbMSwwLCJcXHRleHR7SG9tfShYXzEsWClcXHRpbWVzIEYoVyxYXzEpWSJdLFswLDEsIlxcdGV4dHtIb219KFhfMixYKVxcdGltZXMgRihXLFhfMilZIl0sWzEsMywiXFx0ZXh0e0hvbX0oWF8yLFgpXFx0aW1lcyBGKFcsZilZIiwyXSxbMSwyLCJcXHRleHR7SG9tfShmLFgpXFx0aW1lcyBGKFcsWF8xKVkiLDAseyJjdXJ2ZSI6LTJ9XSxbMywwLCJ3KFhfMikiLDIseyJjdXJ2ZSI6Mn1dLFsyLDAsIncoWF8xKSJdXQ==
      \begin{tikzcd}
        {\text{Hom}(X_2,X)\times F(W,X_1)Y} & {\text{Hom}(X_1,X)\times F(W,X_1)Y} \\
        {\text{Hom}(X_2,X)\times F(W,X_2)Y} & {F(W,X)Y}
        \arrow["{\text{Hom}(X_2,X)\times F(W,f)Y}"', from=1-1, to=2-1]
        \arrow["{\text{Hom}(f,X)\times F(W,X_1)Y}", curve={height=-12pt}, from=1-1, to=1-2]
        \arrow["{w(X_2)}"', curve={height=12pt}, from=2-1, to=2-2]
        \arrow["{w(X_1)}", from=1-2, to=2-2]
      \end{tikzcd}
    \end{equation*}
    Thus $F(W,X)Y$ and $w$ define a cowedge. Consider the diagram
    \begin{equation}\label{eq:r_iso_proof}
      % https://q.uiver.app/?q=WzAsNCxbMSwwLCJGKFcsWClZIl0sWzAsMCwiXFx0ZXh0e0hvbX0oWCcsWClcXHRpbWVzIEYoVyxYJylZIl0sWzEsMiwiXFxpbnRee1gnfVxcdGV4dHtIb219KFgnLFgpXFx0aW1lcyBGKFcsWCcpWSJdLFsxLDEsIlxcdGV4dHtIb219KFgsWClcXHRpbWVzIEYoVyxYKVkiXSxbMSwwLCJ3KFgnKSJdLFsxLDIsInEiLDIseyJjdXJ2ZSI6NX1dLFszLDIsInEiLDJdLFswLDMsIlxcbGFuZ2xlIFxcRGVsdGFcXHRleHR7aWR9LEYoVyxYKVlcXHJhbmdsZSIsMl0sWzAsMiwiKChcXG1hdGhiZiByX0YpX3tXLFh9KV9ZIiwwLHsiY3VydmUiOi01fV1d
      \begin{tikzcd}
        {\text{Hom}(X',X)\times F(W,X')Y} & {F(W,X)Y} \\
                                          & {\text{Hom}(X,X)\times F(W,X)Y} \\
                                          & {\int^{X'}\text{Hom}(X',X)\times F(W,X')Y}
                                          \arrow["{w(X')}", from=1-1, to=1-2]
                                          \arrow["q"', curve={height=30pt}, from=1-1, to=3-2]
                                          \arrow["q"', from=2-2, to=3-2]
                                          \arrow["{\langle \Delta\text{id},F(W,X)Y\rangle}"', from=1-2, to=2-2]
                                          \arrow["{((\mathbf r_F)_{W,X})_Y}", curve={height=-100pt}, from=1-2, to=3-2]
      \end{tikzcd}
    \end{equation}
    Let $g:X'\to X$ and let $u\in F(W,X')Y$. Chasing elements we find
    \begin{align*}
      (q_X\circ\aa{\Delta\id, F(W,X)Y}\circ w(X'))(g,u)
      &= (q_X\circ\aa{\Delta\id, F(W,X)Y})((F(W,g)Y)u) \\
      &= q_X(\id, (F(W,g)Y)u)\\
      &= q_{X'}(g,u)
    \end{align*}
    where the last step follows from the cowedge property of the coend. Thus
    (\ref{eq:r_iso_proof}) commutes, making (\ref{eq:strong_presheaf_r}) into a
    cowedge morphism. By universality of the coend it follows that this must
    be an isomorphism.
  \end{proof}
\end{lemma}

\section{Induced relative pseudomonad structure}\label{sec:relative_pseudomonads}

For our generalisation to make sense, we require a strong $J$-pseudomonad to induce a
relative pseudomonad over $J$. The structure \ref{def:prestrong_inclusion_pseudomonad_structure}
already resembles that of a relative pseudomonad so it is straightforward to see
how the former gives rise to the latter.

\begin{definition}\label{def:induced_relative_pseudomonad_structure}
  Let $T$ be a strong $J$-pseudomonad. The \emph{relative pseudomonad structure induced
  by $T$} consists of
  \begin{itemize}
    \item for all $X,Y\in\bicat{J}$, the extension functor $\extend{\rr{-}}{*}$ as in
      \ref{eq:relative_pseudomonad_extension};
    \item for all 1-cells $f:JX\to TY$ and $g:JY\to TZ$, the 2-cell
      \begin{align*}
        \bicell m_{f,g}=\bicell c_{f,g\lambda}\inv\lambda_{TX}
      \end{align*}
      as in the diagram
      \begin{equation}
        % https://q.uiver.app/?q=WzAsNSxbNiwwLCJUWSJdLFswLDAsIlRYIl0sWzEsMCwiMVxcdGltZXMgVFgiXSxbMywxLCJUWiJdLFs1LDAsIjFcXHRpbWVzIFRZIl0sWzEsMiwiXFxsYW1iZGFeey0xfSIsMl0sWzIsMywiKChnXFxsYW1iZGEpXlxcZGFnZ2VyKDFcXHRpbWVzIGYpKV5cXGRhZ2dlciIsMl0sWzQsMywiKGdcXGxhbWJkYSleXFxkYWdnZXIiXSxbMCw0LCJcXGxhbWJkYV57LTF9Il0sWzIsNCwiMVxcdGltZXMgZl4qIl0sWzEsMCwiZl4qIiwwLHsiY3VydmUiOi01fV0sWzEsMywiKGdeKmYpXioiLDIseyJjdXJ2ZSI6NX1dLFswLDMsImdeKiIsMCx7ImN1cnZlIjotNX1dLFs2LDcsIlxcbWF0aGJmIGMiLDAseyJzaG9ydGVuIjp7InNvdXJjZSI6MjAsInRhcmdldCI6MjB9fV1d
        \begin{tikzcd}
          TX & {1\times TX} &&&& {1\times TY} & TY \\
             &&& TZ
             \arrow["{\lambda^{-1}}"', from=1-1, to=1-2]
             \arrow[""{name=0, anchor=center, inner sep=0}, "{((g\lambda)^\dagger(1\times f))^\dagger}"', from=1-2, to=2-4]
             \arrow[""{name=1, anchor=center, inner sep=0}, "{(g\lambda)^\dagger}", from=1-6, to=2-4]
             \arrow["{\lambda^{-1}}", from=1-7, to=1-6]
             \arrow["{1\times f^*}", from=1-2, to=1-6]
             \arrow["{f^*}", curve={height=-30pt}, from=1-1, to=1-7]
             \arrow["{(g^*f)^*}"', curve={height=30pt}, from=1-1, to=2-4]
             \arrow["{g^*}", curve={height=-30pt}, from=1-7, to=2-4]
             \arrow["{\mathbf c}", shorten <=13pt, shorten >=13pt, Rightarrow, from=0, to=1]
        \end{tikzcd}
      \end{equation}
    \item for all $f:JX\to TY$, the 2-cell
      \begin{align*}
        \bicell e_f = \mathbf r_{f\lambda}\inv\lambda_{JX}
      \end{align*}
      as in the diagram
      \begin{equation}
        % https://q.uiver.app/?q=WzAsNSxbMCwwLCJKWCJdLFsyLDEsIlRZIl0sWzQsMCwiVFgiXSxbMywwLCIxXFx0aW1lcyBUWCJdLFsxLDAsIjFcXHRpbWVzIEpYIl0sWzAsMiwiXFxldGEiLDAseyJjdXJ2ZSI6LTV9XSxbMiwzLCJcXGxhbWJkYV57LTF9Il0sWzMsMSwiKGZcXGxhbWJkYSleXFxkYWdnZXIiXSxbMCw0LCJcXGxhbWJkYV57LTF9IiwyXSxbNCwzLCIxXFx0aW1lcyBcXGV0YSJdLFs0LDEsImZcXGxhbWJkYSIsMl0sWzIsMSwiZl4qIiwwLHsiY3VydmUiOi0zfV0sWzAsMSwiZiIsMix7ImN1cnZlIjozfV0sWzEwLDcsIlxcbWF0aGJmIHIiLDAseyJzaG9ydGVuIjp7InNvdXJjZSI6MjAsInRhcmdldCI6MjB9fV1d
        \begin{tikzcd}
          JX & {1\times JX} && {1\times TX} & TX \\
             && TY
             \arrow["\eta", curve={height=-30pt}, from=1-1, to=1-5]
             \arrow["{\lambda^{-1}}", from=1-5, to=1-4]
             \arrow[""{name=0, anchor=center, inner sep=0}, "{(f\lambda)^\dagger}", from=1-4, to=2-3]
             \arrow["{\lambda^{-1}}"', from=1-1, to=1-2]
             \arrow["{1\times \eta}", from=1-2, to=1-4]
             \arrow[""{name=1, anchor=center, inner sep=0}, "f\lambda"', from=1-2, to=2-3]
             \arrow["{f^*}", curve={height=-18pt}, from=1-5, to=2-3]
             \arrow["f"', curve={height=18pt}, from=1-1, to=2-3]
             \arrow["{\mathbf r}", shorten <=7pt, shorten >=7pt, Rightarrow, from=1, to=0]
        \end{tikzcd}
      \end{equation}
    \item for all $X\in\bicat{J}$, the 2-cell
      \begin{align*}
        \bicell t_X=\mathbf l_{\eta_X}\inv\lambda_{TX}
      \end{align*}
      as in the diagram
      \begin{equation}
        % https://q.uiver.app/?q=WzAsMyxbMCwwLCJUWCJdLFsyLDIsIlRYIl0sWzIsMCwiMVxcdGltZXMgVFgiXSxbMCwxLCJcXGV0YV57VFxcbGFtYmRhfSIsMix7ImN1cnZlIjoyfV0sWzAsMiwiXFxsYW1iZGFeey0xfSJdLFsyLDEsIihcXGV0YVxcbGFtYmRhKV5UIiwyLHsiY3VydmUiOjN9XSxbMiwxLCJcXGxhbWJkYSIsMCx7ImN1cnZlIjotM31dLFs1LDYsIlxcbWF0aGJmIGwiLDAseyJzaG9ydGVuIjp7InNvdXJjZSI6MjAsInRhcmdldCI6MjB9fV1d
        \begin{tikzcd}
          TX && {1\times TX} \\
          \\
             && TX
             \arrow["{\eta^{T\lambda}}"', curve={height=12pt}, from=1-1, to=3-3]
             \arrow["{\lambda^{-1}}", from=1-1, to=1-3]
             \arrow[""{name=0, anchor=center, inner sep=0}, "{(\eta\lambda)^T}"', curve={height=18pt}, from=1-3, to=3-3]
             \arrow[""{name=1, anchor=center, inner sep=0}, "\lambda", curve={height=-18pt}, from=1-3, to=3-3]
             \arrow["{\mathbf l}", shorten <=7pt, shorten >=7pt, Rightarrow, from=0, to=1]
        \end{tikzcd}
      \end{equation}
  \end{itemize}
\end{definition}

When considering the presheaf construction, the induced relative pseudomonad structure
consists of several isomorphisms that may be familiar from coend calculus.
While we are not concerned with the details, it is worth investigating the structure
nonetheless:

\begin{example}
  For the presheaf construction, we obtain the following relative pseudomonad structure:
  \begin{enumerate}
    \item for all $F,G:\scat X\to\widehat{\scat Y}$, $\phi : F\Rightarrow G$,
      and $P\in\widehat{\scat X}$,
      \begin{align*}
        \extend{F}{*}(P) = \int^X PX\times FX(-), \hs
        \rr{\extend{\phi}{*}}_P = \int^X PX\times \phi_X;
      \end{align*}
    \item for all $X\in\scat X$, $\eta_{\scat X}(X) = \Hom\rr{-,X}$;
    \item for all $F:\scat X\to\widehat{\scat Y}$ and $X\in\scat X$, there is a
      natural isomorphism with components
      \begin{align*}
        \rr{\bicell e_F}_X:FX\to\int^{X'} \Hom(X',X)\times FX';
      \end{align*}
    \item for all $F:\scat X\to\widehat{\scat Y}$, $G:\scat Y\to\widehat{\scat
      Z}$, and $P\in\widehat{\scat X}$, $(\bicell m_{F,G})_P$ is a natural
      isomorphism with components
      \begin{align*}
        \int^X PX\times\int^Y (FX)Y\times (GY)Z
        \to\int^Y \rr{\int^X PX\times (FX)Y}\times (GY)Z;
      \end{align*}
    \item for all $P\in\widehat{\scat X}$, $\bicell t$ has as components isomorphisms
      \begin{align*}
        \rr{\rr{\bicell t_{\scat X}}_P}_X: \int^{X'} \Hom(X,X')\times PX'\to PX
      \end{align*}
  \end{enumerate}
\end{example}

\section{Prestrong axioms}

A prestrong $J$-pseudomonad should induce a relative pseudomonad over $J$. It is therefore
not surprising that the conditions which we impose are entirely analogous to those
stated in~\cite{fiore2017}.

\begin{definition}\label{def:prestrong_inclusion_pseudomonad_axioms}
  A \emph{prestrong $J$-pseudomonad} is a prestrong $J$-pseudomonad structure such that
  \begin{enumerate}
    \item $\bicell r_f$ is natural in $f$;
    \item $\bicell c_{f,g,i}$ is natural in $f$, $g$, and $i$;
    \item for all $f:W\times JX\to TY$,
      \begin{equation}\label{eq:strong_pseudomonad_square}
        % https://q.uiver.app/?q=WzAsNCxbMCwwLCJmXlxcZGFnZ2VyIl0sWzAsMSwiZl5cXGRhZ2dlciJdLFsyLDAsIihmXlxcZGFnZ2VyKDFcXHRpbWVzXFxldGEpKV5cXGRhZ2dlciJdLFsyLDEsImZeXFxkYWdnZXIoMVxcdGltZXNcXGV0YV4qKSJdLFswLDIsIlxcbWF0aGJmIHIiXSxbMiwzLCJcXG1hdGhiZiBjIl0sWzMsMSwiZl5cXGRhZ2dlcigxXFx0aW1lc1xcbWF0aGJmIGxcXGxhbWJkYV57LTF9KSJdLFswLDEsIiIsMix7ImxldmVsIjoyLCJzdHlsZSI6eyJoZWFkIjp7Im5hbWUiOiJub25lIn19fV1d
        \begin{tikzcd}
          {f^\dagger} && {(f^\dagger(1\times\eta))^\dagger} \\
          {f^\dagger} && {f^\dagger(1\times\eta^*)}
          \arrow["{\mathbf r}", from=1-1, to=1-3]
          \arrow["{\mathbf c}", from=1-3, to=2-3]
          \arrow["{f^\dagger(1\times\mathbf l\lambda^{-1})}", from=2-3, to=2-1]
          \arrow[Rightarrow, no head, from=1-1, to=2-1]
        \end{tikzcd}
      \end{equation}
    \item for all suitable 1-cells $f,g,h,i,j$,
      \begin{equation}\label{eq:strong_pseudomonad_house}
        % https://q.uiver.app/?q=WzAsNSxbMSwwLCIoKGheXFxkYWdnZXIoalxcdGltZXMgZykpXlxcZGFnZ2VyKGlcXHRpbWVzIGYpKV5cXGRhZ2dlciJdLFswLDEsIihoXlxcZGFnZ2VyKGppXFx0aW1lcyBnXipmKSleXFxkYWdnZXIiXSxbMCwzLCJoXlxcZGFnZ2VyKGppXFx0aW1lcyhnXipmKV5cXGRhZ2dlcikiXSxbMiwxLCIoaF5cXGRhZ2dlcihqXFx0aW1lcyBnKSleXFxkYWdnZXIoaVxcdGltZXMgZl4qKSJdLFsyLDMsImheXFxkYWdnZXIoamlcXHRpbWVzIGdeKmZeKikiXSxbMSwyLCJcXG1hdGhiZiBjIiwxXSxbMCwzLCJcXG1hdGhiZiBjIiwxXSxbMiw0LCJoXlxcZGFnZ2VyKGppXFx0aW1lc1xcbWF0aGJmIGNcXGxhbWJkYV57LTF9KSIsMV0sWzMsNCwiXFxtYXRoYmYgYyhpXFx0aW1lcyBmXiopIiwxXSxbMCwxLCIoXFxtYXRoYmYgYyhpXFx0aW1lcyBmKSleXFxkYWdnZXIiLDFdXQ==
        \begin{tikzcd}
  & {((h^\dagger(j\times g))^\dagger(i\times f))^\dagger} \\
          {(h^\dagger(ji\times g^*f))^\dagger} && {(h^\dagger(j\times g))^\dagger(i\times f^*)} \\
          \\
          {h^\dagger(ji\times(g^*f)^\dagger)} && {h^\dagger(ji\times g^*f^*)}
          \arrow["{\mathbf c}"{description}, from=2-1, to=4-1]
          \arrow["{\mathbf c}"{description}, from=1-2, to=2-3]
          \arrow["{h^\dagger(ji\times\mathbf c\lambda^{-1})}"{description}, from=4-1, to=4-3]
          \arrow["{\mathbf c(i\times f^*)}"{description}, from=2-3, to=4-3]
          \arrow["{(\mathbf c(i\times f))^\dagger}"{description}, from=1-2, to=2-1]
        \end{tikzcd}
      \end{equation}
  \end{enumerate}
\end{definition}

Now that we have stated some axioms, we need to make sure that they are sensible. The first
step is to verify that the presheaf construction remains a model.

\begin{proposition}
  The prestrong presheaf construction in \ref{ex:prestrong_presheaves} is a prestrong $J$-pseudomonad.
  \begin{proof}
    We verify the axioms:
    \begin{enumerate}
      \item We note that
        \begin{equation*}
          % https://q.uiver.app/?q=WzAsNixbMCwwLCJGKFcsWClZIl0sWzEsMCwiRyhXLFgpWSJdLFswLDIsIlxcdGV4dHtIb219KFgsWClcXHRpbWVzIEYoVyxYKVkiXSxbMSwyLCJcXHRleHR7SG9tfShYLFgpXFx0aW1lcyBHKFcsWClZIl0sWzAsNCwiXFxpbnRee1gnfVxcdGV4dHtIb219KFgnLFgpXFx0aW1lcyBGKFcsWCcpWSJdLFsxLDQsIlxcaW50XntYJ30gXFx0ZXh0e0hvbX0oWCcsWClcXHRpbWVzIEYoVyxYJylZIl0sWzAsMSwiXFxsZWZ0KFxccGhpX3tXLFh9XFxyaWdodClfWSIsMV0sWzIsMywiXFx0ZXh0e0hvbX0oWCxYKVxcdGltZXMgXFxsZWZ0KFxccGhpX3tXLFh9XFxyaWdodClfWSIsMix7ImN1cnZlIjoyfV0sWzAsMiwiXFxsYW5nbGUgXFxEZWx0YVxcdGV4dHtpZH0sRihXLFgpWVxccmFuZ2xlIiwxXSxbMSwzLCJcXGxhbmdsZSBcXERlbHRhXFx0ZXh0e2lkfSxHKFcsWClZXFxyYW5nbGUiLDFdLFsyLDQsInEiLDFdLFszLDUsInEiLDFdLFs0LDUsIlxcaW50XntYJ31cXHRleHR7SG9tfShYJyxYKVxcdGltZXMgXFxsZWZ0KFxccGhpX3tXLFh9XFxyaWdodClfWSIsMix7ImN1cnZlIjoyfV0sWzEsNSwiXFxtYXRoYmYgciIsMSx7ImN1cnZlIjotNX1dLFswLDQsIlxcbWF0aGJmIHIiLDEseyJjdXJ2ZSI6NX1dLFs2LDcsIlxcbGFuZ2xlXFxEZWx0YVxcdGV4dHtpZH0sLVxccmFuZ2xlXFx0ZXh0ey1uYXR9IiwxLHsic2hvcnRlbiI6eyJzb3VyY2UiOjIwLCJ0YXJnZXQiOjIwfSwic3R5bGUiOnsiYm9keSI6eyJuYW1lIjoibm9uZSJ9LCJoZWFkIjp7Im5hbWUiOiJub25lIn19fV0sWzcsMTIsIiIsMSx7InNob3J0ZW4iOnsic291cmNlIjoyMCwidGFyZ2V0IjoyMH0sInN0eWxlIjp7ImJvZHkiOnsibmFtZSI6Im5vbmUifSwiaGVhZCI6eyJuYW1lIjoibm9uZSJ9fX1dLFsxMCwxNCwiXFxtYXRoYmYgclxcdGV4dHstZGVmfSIsMSx7InNob3J0ZW4iOnsic291cmNlIjoyMH0sInN0eWxlIjp7ImJvZHkiOnsibmFtZSI6Im5vbmUifSwiaGVhZCI6eyJuYW1lIjoibm9uZSJ9fX1dLFsxMywxMSwiXFxtYXRoYmYgclxcdGV4dHstZGVmfSIsMSx7InNob3J0ZW4iOnsidGFyZ2V0IjoyMH0sInN0eWxlIjp7ImJvZHkiOnsibmFtZSI6Im5vbmUifSwiaGVhZCI6eyJuYW1lIjoibm9uZSJ9fX1dXQ==
\begin{tikzcd}
	{F(W,X)Y} & {G(W,X)Y} \\
	\\
	{\text{Hom}(X,X)\times F(W,X)Y} & {\text{Hom}(X,X)\times G(W,X)Y} \\
	\\
	{\int^{X'}\text{Hom}(X',X)\times F(W,X')Y} & {\int^{X'} \text{Hom}(X',X)\times F(W,X')Y}
	\arrow[""{name=0, anchor=center, inner sep=0}, "{\left(\phi_{W,X}\right)_Y}"{description}, from=1-1, to=1-2]
	\arrow[""{name=1, anchor=center, inner sep=0}, "{\text{Hom}(X,X)\times \left(\phi_{W,X}\right)_Y}"', curve={height=20pt}, from=3-1, to=3-2]
	\arrow["{\langle \Delta\text{id},F(W,X)Y\rangle}"{description}, from=1-1, to=3-1]
	\arrow["{\langle \Delta\text{id},G(W,X)Y\rangle}"{description}, from=1-2, to=3-2]
	\arrow[""{name=2, anchor=center, inner sep=0}, "q"{description}, from=3-1, to=5-1]
	\arrow[""{name=3, anchor=center, inner sep=0}, "q"{description}, from=3-2, to=5-2]
	\arrow[""{name=4, anchor=center, inner sep=0}, "{\int^{X'}\text{Hom}(X',X)\times \left(\phi_{W,X}\right)_Y}"', curve={height=20pt}, from=5-1, to=5-2]
	\arrow[""{name=5, anchor=center, inner sep=0}, "{\mathbf r}"{description}, curve={height=-100pt}, from=1-2, to=5-2]
	\arrow[""{name=6, anchor=center, inner sep=0}, "{\mathbf r}"{description}, curve={height=100pt}, from=1-1, to=5-1]
	\arrow["{\langle\Delta\text{id},-\rangle\text{-nat}}"{description}, draw=none, from=0, to=1]
	\arrow[draw=none, from=1, to=4]
	\arrow["{\mathbf r\text{-def}}"{description}, draw=none, from=2, to=6]
	\arrow["{\mathbf r\text{-def}}"{description}, draw=none, from=5, to=3]
\end{tikzcd}
        \end{equation*}
        commutes due to (\ref{eq:coend_natural_transformation}) on the right. This shows naturality
        of $\bicell r$.
      \item Naturality of $\bicell c$ follows from naturality of the
        underlying isomorphism.\footnote{
          Full diagram of naturality of $\bicell c_{F,G,I}$ in $F$ and $G$ in
        \href{https://q.uiver.app/?q=WzAsMTIsWzAsMCwiXFxpbnReWCBQWFxcdGltZXNcXGludF5ZKEZYKVlcXHRpbWVzIEcoSVYsWSlaIl0sWzUsMCwiXFxpbnReWVxcbGVmdChcXGludF5YIFBYXFx0aW1lcyAoRlgpWVxccmlnaHQpXFx0aW1lcyBHKElWLFkpWiJdLFswLDksIlxcaW50XlggUFhcXHRpbWVzXFxpbnReWShIWClZXFx0aW1lcyBLKElWLFkpWiJdLFs1LDksIlxcaW50XllcXGxlZnQoXFxpbnReWCBQWFxcdGltZXMoSFgpWVxccmlnaHQpXFx0aW1lcyBLKElWLFkpWiJdLFsxLDEsIlBYXFx0aW1lc1xcaW50XlkoRlgpWVxcdGltZXMgRyhJVixZKVoiXSxbNCwxLCJcXGxlZnQoXFxpbnReWCBQWFxcdGltZXMoRlgpWVxccmlnaHQpXFx0aW1lcyBHKElWLFkpWiJdLFsxLDgsIlBYXFx0aW1lc1xcaW50XlkoSFgpWVxcdGltZXMgSyhJVixZKVoiXSxbNCw4LCJcXGxlZnQoXFxpbnReWCBQWFxcdGltZXMoSFgpWVxccmlnaHQpXFx0aW1lcyBLKElWLFkpWiJdLFsyLDIsIlBYXFx0aW1lcygoRlgpWVxcdGltZXMgRyhJVixZKVopIl0sWzIsNywiUFhcXHRpbWVzKChIWClZXFx0aW1lcyBLKElWLFkpWikiXSxbMyw3LCIoUFhcXHRpbWVzIChIWClZKVxcdGltZXMgSyhJVixZKVoiXSxbMywyLCIoUFhcXHRpbWVzKEZYKVkpXFx0aW1lcyBHKElWLFkpWiJdLFsyLDMsIlxcbWF0aGJmIGMiXSxbMCwxLCJcXG1hdGhiZiBjIl0sWzAsMiwiXFxpbnReWCBQWFxcdGltZXNcXGludF5ZKFxccGhpX1gpX1lcXHRpbWVzKFxccHNpX3tJVixZfSlfWiIsMV0sWzcsMywicSIsMV0sWzYsMiwicSIsMV0sWzQsMCwicSIsMV0sWzUsMSwicSIsMl0sWzksMTAsIlxcY29uZyIsMSx7ImN1cnZlIjotMn1dLFs4LDExLCJcXGNvbmciLDEseyJjdXJ2ZSI6Mn1dLFs4LDQsIlBYXFx0aW1lcyBxIiwxXSxbMTEsNSwicVxcdGltZXMgRyhJVixZKVoiLDFdLFs5LDYsIlBYXFx0aW1lcyBxIiwxXSxbMTAsNywicVxcdGltZXMgSyhJVixZKVoiLDFdLFsxLDMsIlxcaW50XllcXGxlZnQoXFxpbnReWCBQWFxcdGltZXMgKFxccGhpX1gpX1lcXHJpZ2h0KVxcdGltZXMoXFxwc2lfe0lWLFl9KV9aIiwxXSxbOCw5LCJQWFxcdGltZXMoKFxccGhpX1gpX1lcXHRpbWVzKFxccHNpX3tJVixZfSlfWikiLDFdLFsxMSwxMCwiKFBYXFx0aW1lcyhcXHBoaV9YKV9ZKVxcdGltZXMoXFxwc2lfe0lWLFl9KV9aIiwxXV0=}{quiver}.}
      \item We note that the following commutes:
        \begin{equation*}
          % https://q.uiver.app/?q=WzAsNixbMCwwLCJQWFxcdGltZXMgRihXLFgpWSJdLFsxLDAsIlBYXFx0aW1lc1xcaW50XntYJ31cXHRleHR7SG9tfShYJyxYKVxcdGltZXMgRihXLFgnKVkiXSxbMSwyLCJQWFxcdGltZXMoXFx0ZXh0e0hvbX0oWCxYKVxcdGltZXMgRihXLFgpWSkiXSxbMSw1LCJcXGxlZnQoXFxpbnReWCBQWFxcdGltZXNcXHRleHR7SG9tfShYJyxYKVxccmlnaHQpXFx0aW1lcyBGKFcsWClZIl0sWzEsMywiKFBYXFx0aW1lc1xcdGV4dHtIb219KFgsWCkpXFx0aW1lcyBGKFcsWClZIl0sWzAsNSwiUFhcXHRpbWVzIEYoVyxYKVkiXSxbMiwxLCJQWFxcdGltZXMgcSIsMSx7ImN1cnZlIjozfV0sWzAsMSwiUFhcXHRpbWVzXFxtYXRoYmYgciIsMSx7ImN1cnZlIjotM31dLFswLDIsIlBYXFx0aW1lc1xcbGFuZ2xlXFxEZWx0YVxcdGV4dHtpZH0sRihXLFgpWVxccmFuZ2xlIiwxXSxbNCwzLCJxXFx0aW1lcyBGKFcsWClZIiwxLHsiY3VydmUiOi0zfV0sWzIsNCwiXFxhbHBoYSIsMV0sWzUsMywiXFxtYXRoYmYgbFxcdGltZXMgRihXLFgnKVkiLDEseyJjdXJ2ZSI6M31dLFswLDUsIiIsMSx7ImxldmVsIjoyLCJzdHlsZSI6eyJoZWFkIjp7Im5hbWUiOiJub25lIn19fV0sWzUsNCwiXFxsYW5nbGUgUFgsXFxEZWx0YVxcdGV4dHtpZH1cXHJhbmdsZVxcdGltZXMgRihXLFgpWSIsMV0sWzgsMSwiXFxtYXRoYmYgclxcdGV4dHstZGVmfSIsMSx7InNob3J0ZW4iOnsic291cmNlIjoyMH0sInN0eWxlIjp7ImJvZHkiOnsibmFtZSI6Im5vbmUifSwiaGVhZCI6eyJuYW1lIjoibm9uZSJ9fX1dLFsxMywzLCJcXG1hdGhiZiByXFx0ZXh0ey1kZWZ9IiwxLHsic2hvcnRlbiI6eyJzb3VyY2UiOjIwfSwic3R5bGUiOnsiYm9keSI6eyJuYW1lIjoibm9uZSJ9LCJoZWFkIjp7Im5hbWUiOiJub25lIn19fV1d
          \begin{tikzcd}
            {PX\times F(W,X)Y} & {PX\times\int^{X'}\text{Hom}(X',X)\times F(W,X')Y} \\
            \\
                               & {PX\times(\text{Hom}(X,X)\times F(W,X)Y)} \\
                               & {(PX\times\text{Hom}(X,X))\times F(W,X)Y} \\
                               \\
            {PX\times F(W,X)Y} & {\left(\int^X PX\times\text{Hom}(X',X)\right)\times F(W,X)Y}
            \arrow["{PX\times q}"{description}, curve={height=18pt}, from=3-2, to=1-2]
            \arrow["{PX\times\mathbf r}"{description}, curve={height=-18pt}, from=1-1, to=1-2]
            \arrow[""{name=0, anchor=center, inner sep=0}, "{PX\times\langle\Delta\text{id},F(W,X)Y\rangle}"{description}, from=1-1, to=3-2]
            \arrow["{q\times F(W,X)Y}"{description}, curve={height=-18pt}, from=4-2, to=6-2]
            \arrow["\alpha"{description}, from=3-2, to=4-2]
            \arrow["{\mathbf l\times F(W,X')Y}"{description}, curve={height=18pt}, from=6-1, to=6-2]
            \arrow[Rightarrow, no head, from=1-1, to=6-1]
            \arrow[""{name=1, anchor=center, inner sep=0}, "{\langle PX,\Delta\text{id}\rangle\times F(W,X)Y}"{description}, from=6-1, to=4-2]
            \arrow["{\mathbf r\text{-def}}"{description}, draw=none, from=0, to=1-2]
            \arrow["{\mathbf r\text{-def}}"{description}, draw=none, from=1, to=6-2]
          \end{tikzcd}
        \end{equation*}
        Postcomposition with the appropriate canonical morphisms yields
        (\ref{eq:strong_pseudomonad_square}).\footnote{
          Full diagram for proof of (\ref{eq:strong_pseudomonad_square}) in
          \href{https://q.uiver.app/?q=WzAsMTAsWzEsMSwiUFhcXHRpbWVzIEYoVyxYKVkiXSxbMiwxLCJQWFxcdGltZXNcXGludF57WCd9XFx0ZXh0e0hvbX0oWCcsWClcXHRpbWVzIEYoVyxYJylZIl0sWzIsMywiUFhcXHRpbWVzKFxcdGV4dHtIb219KFgsWClcXHRpbWVzIEYoVyxYKVkpIl0sWzIsNiwiXFxsZWZ0KFxcaW50XlggUFhcXHRpbWVzXFx0ZXh0e0hvbX0oWCcsWClcXHJpZ2h0KVxcdGltZXMgRihXLFgpWSJdLFsyLDQsIihQWFxcdGltZXNcXHRleHR7SG9tfShYLFgpKVxcdGltZXMgRihXLFgpWSJdLFsxLDYsIlBYXFx0aW1lcyBGKFcsWClZIl0sWzMsNywiXFxpbnRee1gnfSBcXGxlZnQoXFxpbnReWCBQWFxcdGltZXNcXHRleHR7SG9tfShYJyxYKVxccmlnaHQpXFx0aW1lcyBGKFcsWCcpWSJdLFszLDAsIlxcaW50XlhQWFxcdGltZXNcXGludF57WCd9XFx0ZXh0e0hvbX0oWCcsWClcXHRpbWVzIEYoVyxYJylZIl0sWzAsMCwiXFxpbnReWCBQWFxcdGltZXMgRihXLFgpWSJdLFswLDcsIlxcaW50XlggUFhcXHRpbWVzIEYoVyxYKVkiXSxbMiwxLCJQWFxcdGltZXMgcSIsMSx7ImN1cnZlIjo1fV0sWzAsMSwiUFhcXHRpbWVzXFxtYXRoYmYgciIsMSx7ImN1cnZlIjotM31dLFswLDIsIlBYXFx0aW1lc1xcbGFuZ2xlXFxEZWx0YSBYLEYoVyxYKVlcXHJhbmdsZSIsMV0sWzQsMywicVxcdGltZXMgRihXLFgpWSIsMSx7ImN1cnZlIjotNX1dLFsyLDQsIlxcYWxwaGEiLDFdLFs1LDMsIlxcbWF0aGJmIGxcXHRpbWVzIEYoVyxYJylZIiwxLHsiY3VydmUiOjN9XSxbMCw1LCIiLDEseyJsZXZlbCI6Miwic3R5bGUiOnsiaGVhZCI6eyJuYW1lIjoibm9uZSJ9fX1dLFs1LDQsIlxcbGFuZ2xlIFBYLFxcRGVsdGEgWFxccmFuZ2xlXFx0aW1lcyBGKFcsWClZIiwxXSxbMyw2LCJxIiwxXSxbNyw2LCJcXG1hdGhiZiBjIiwxXSxbMSw3LCJxIiwxXSxbOCw3LCJcXGludF5YUFhcXHRpbWVzXFxtYXRoYmYgciIsMV0sWzAsOCwicSIsMV0sWzYsOSwiXFxpbnRee1gnfVxcbWF0aGJmIGxeey0xfVxcdGltZXMgRihXLFgnKSIsMV0sWzgsOSwiIiwwLHsibGV2ZWwiOjIsInN0eWxlIjp7ImhlYWQiOnsibmFtZSI6Im5vbmUifX19XSxbNSw5LCJxIiwxXV0=}
        {quiver}.}
      \item The proof of (\ref{eq:strong_pseudomonad_house}) is similar to the above.\footnote{
          Full diagram for proof of (\ref{eq:strong_pseudomonad_house}) in \href{https://q.uiver.app/?q=WzAsMjAsWzAsMCwiXFxpbnReVyBQV1xcdGltZXMgXFxpbnReWChGVylYXFx0aW1lcyBcXGludF5ZIChHWClZXFx0aW1lcyBIKEpJVixZKVoiXSxbNywwLCJcXGludF5XIFBXXFx0aW1lcyBcXGludF5ZXFxsZWZ0KFxcaW50XlgoRlcpWFxcdGltZXMgKEdYKVlcXHJpZ2h0KVxcdGltZXMgSChKSVYsWSlaIl0sWzcsNSwiXFxpbnReWVxcbGVmdChcXGludF5XIFBXXFx0aW1lc1xcbGVmdChcXGludF5YKEZXKVhcXHRpbWVzIChHWClZXFxyaWdodClcXHJpZ2h0KVxcdGltZXMgSChKSVYsWSlaIl0sWzcsMTAsIlxcaW50XllcXGxlZnQoXFxpbnReWFxcbGVmdChcXGludF5XIFBXXFx0aW1lcyhGVylYXFxyaWdodClcXHRpbWVzIChHWClZXFxyaWdodClcXHRpbWVzIEgoSklWLFkpWiJdLFsxLDEsIlBXXFx0aW1lc1xcaW50XlgoRlcpWFxcdGltZXNcXGludF5ZKEdYKVlcXHRpbWVzIEgoSklWLFkpWiJdLFsyLDIsIlBXXFx0aW1lcyBcXGxlZnQoKEZXKVhcXHRpbWVzIFxcaW50XlkoR1gpWVxcdGltZXMgSChKSVYsWSlaXFxyaWdodCkiXSxbMCwxMCwiXFxpbnReWFxcbGVmdChcXGludF5XIFBXXFx0aW1lcyAoRlcpWFxccmlnaHQpXFx0aW1lcyBcXGludF5ZIChHWClZXFx0aW1lcyBIKEpJVixZKVoiXSxbMSw5LCJcXGxlZnQoXFxpbnReVyBQV1xcdGltZXMgKEZXKVhcXHJpZ2h0KVxcdGltZXMgXFxpbnReWSAoR1gpWVxcdGltZXMgSChKSVYsWSlaIl0sWzIsOCwiXFxsZWZ0KFBXXFx0aW1lcyAoRlcpWFxccmlnaHQpXFx0aW1lc1xcaW50XlkoR1gpWVxcdGltZXMgSChKSVYsWSlaIl0sWzQsMywiUFdcXHRpbWVzXFxsZWZ0KFxcbGVmdCgoRldYXFx0aW1lcyAoR1gpWVxccmlnaHQpXFx0aW1lcyBIKEpJVixZKVpcXHJpZ2h0KSJdLFs2LDEsIlBXXFx0aW1lcyBcXGludF5ZXFxsZWZ0KFxcaW50XlgoRlcpWFxcdGltZXMgKEdYKVlcXHJpZ2h0KVxcdGltZXMgSChKSVYsWSlaIl0sWzUsMiwiUFdcXHRpbWVzIFxcbGVmdChcXGxlZnQoXFxpbnReWChGVylYXFx0aW1lcyAoR1gpWVxccmlnaHQpXFx0aW1lcyBIKEpJVixZKVpcXHJpZ2h0KSJdLFszLDMsIlBXXFx0aW1lc1xcbGVmdCgoRlcpWFxcdGltZXNcXGxlZnQoKEdYKVlcXHRpbWVzIEgoSklWLFkpWlxccmlnaHQpXFxyaWdodCkiXSxbNiw5LCJcXGxlZnQoXFxpbnReWFxcbGVmdChcXGludF5XIFBXXFx0aW1lcyhGVylYXFxyaWdodClcXHRpbWVzIChHWClZXFxyaWdodClcXHRpbWVzIEgoSklWLFkpWiJdLFs1LDgsIlxcbGVmdChcXGxlZnQoXFxpbnReVyBQV1xcdGltZXMoRlcpWFxccmlnaHQpXFx0aW1lcyAoR1gpWVxccmlnaHQpXFx0aW1lcyBIKEpJVixZKVoiXSxbNCw3LCJcXGxlZnQoXFxsZWZ0KFBXXFx0aW1lcyhGVylYXFxyaWdodClcXHRpbWVzIChHWClZXFxyaWdodClcXHRpbWVzIEgoSklWLFkpWiJdLFs2LDUsIlxcbGVmdChcXGludF5XIFBXXFx0aW1lc1xcbGVmdChcXGludF5YKEZXKVhcXHRpbWVzIChHWClZXFxyaWdodClcXHJpZ2h0KVxcdGltZXMgSChKSVYsWSlaIl0sWzUsNSwiXFxsZWZ0KFBXXFx0aW1lc1xcbGVmdChcXGludF5YKEZXKVhcXHRpbWVzIChHWClZXFxyaWdodClcXHJpZ2h0KVxcdGltZXMgSChKSVYsWSlaIl0sWzQsNSwiXFxsZWZ0KFBXXFx0aW1lc1xcbGVmdCgoRlcpWFxcdGltZXMgKEdYKVlcXHJpZ2h0KVxccmlnaHQpXFx0aW1lcyBIKEpJVixZKVoiXSxbMyw3LCIoUFdcXHRpbWVzIChGVylYKVxcdGltZXMoKEdYKVlcXHRpbWVzIEgoSklWLFkpWikiXSxbMCwxLCJcXGludF5XIFBXXFx0aW1lc1xcbWF0aGJmIGMiLDFdLFsxLDIsIlxcbWF0aGJmIGMiLDFdLFsyLDMsIlxcaW50XllcXG1hdGhiZiBjXFx0aW1lcyBIKEpJVixZKVoiLDFdLFs0LDAsInEiXSxbNSw0LCJQV1xcdGltZXMgcSJdLFs2LDMsIlxcaW50XllcXG1hdGhiZiBjXFx0aW1lcyBIKEpJVixZKVoiLDFdLFswLDYsIlxcbWF0aGJmIGMiLDFdLFs3LDYsInEiXSxbOCw3LCJxIl0sWzUsOCwiXFxjb25nIiwxXSxbMTAsMSwicSIsMV0sWzExLDEwLCJQV1xcdGltZXMgcSIsMV0sWzksMTEsIlBXXFx0aW1lcyAocVxcdGltZXMgSChKSVYsWSlaKSIsMV0sWzQsMTAsIlBXXFx0aW1lc1xcbWF0aGJmIGMiLDFdLFsxMiw1LCJQV1xcdGltZXMoKEZXKVhcXHRpbWVzIHEpIiwxXSxbMTIsOSwiXFxjb25nIiwxLHsiY3VydmUiOi0yfV0sWzE1LDE0LCIocVxcdGltZXMoR1gpWSlcXHRpbWVzIEgoSklWLFkpWiIsMV0sWzE0LDEzLCJxXFx0aW1lcyBIKEpJVixZKVoiLDFdLFsxMywzLCJxIiwxXSxbMTgsMTcsIihQV1xcdGltZXMgcSlcXHRpbWVzIEgoSklWLFkpWiIsMSx7ImN1cnZlIjotM31dLFsxNywxNiwicVxcdGltZXMgSChKSVYsWSlaIiwxLHsiY3VydmUiOi0zfV0sWzE2LDIsInEiLDFdLFsxOCwxNSwiXFxjb25nIiwxXSxbMTYsMTMsIlxcbWF0aGJmIGMiLDFdLFsxMSwxNywiXFxjb25nIiwxXSxbMTksOCwiKFBXXFx0aW1lcyAoRlcpWClcXHRpbWVzIHEiLDFdLFsxOSwxNSwiXFxjb25nIiwxLHsiY3VydmUiOi0yfV0sWzEyLDE5LCJcXGNvbmciLDFdLFs5LDE4LCJcXGNvbmciLDFdLFs3LDEzLCJcXG1hdGhiZiBjIiwxXV0=}
        {quiver}.}
    \end{enumerate}
  \end{proof}
\end{proposition}

Only a little more work is required to make sure that a prestrong $J$-pseudomonad
does indeed correspond to a relative pseudomonad over $J$.

\begin{theorem}\label{thm:prestrong_inclusion_pseudomonads_are_relative_pseudomonads}
  Let $T$ be a strong $J$-pseudomonad. Then the relative pseudomonad structure induced
  by $T$ is a relative pseudomonad in the sense of~\cite[Definition 3.1]{fiore2017}.
  \begin{proof}
    We have the obvious correspondence between the structure in~\cite{fiore2017} and
    \ref{def:induced_relative_pseudomonad_structure}:
    $\extend{\rr{-}}{*}_{X,Y}$ is identical,
    $i_X$ is $\eta_X$,
    $\mu_{g,f}$ is $\bicell m_{f,g}$,
    $\eta_f$ is $\bicell e_f$,
    $\theta_X$ is $\bicell t_X$.
    We now verify the axioms:
    \begin{itemize}
      \item naturality of $\bicell e$ is just naturality of $\bicell r$;
      \item naturality of $\bicell m$ is just naturality of $\bicell c$;
      \item \cite[{(3.2)}]{fiore2017} holds by postcomposing (\ref{eq:strong_pseudomonad_square})
        with $\inv\lambda$ as in the diagram
        \begin{equation*}
          % https://q.uiver.app/?q=WzAsOCxbMCwwLCJmXioiXSxbNCwwLCIoZl4qXFxldGEpXioiXSxbNCwzLCJmXipcXGV0YV4qIl0sWzAsMywiZl4qIl0sWzEsMSwiKGZcXGxhbWJkYSleXFxkYWdnZXIiXSxbMywxLCIoKGZcXGxhbWJkYSleXFxkYWdnZXIoMVxcdGltZXNcXGV0YSkpXlxcZGFnZ2VyIl0sWzMsMiwiKGZcXGxhbWJkYSleXFxkYWdnZXIoMVxcdGltZXMoXFxldGFcXGxhbWJkYSleXFxkYWdnZXIpIl0sWzEsMiwiKGZcXGxhbWJkYSleXFxkYWdnZXIiXSxbMCwxLCIoXFxtYXRoYmYgZV9mKV4qIl0sWzEsMiwiXFxtYXRoYmYgbSJdLFswLDMsIiIsMix7ImxldmVsIjoyLCJzdHlsZSI6eyJoZWFkIjp7Im5hbWUiOiJub25lIn19fV0sWzIsMywiZl4qXFxtYXRoYmYgdCJdLFs3LDMsIi1cXGNpcmNcXGxhbWJkYV57LTF9IiwxXSxbMCw0LCItXFxjaXJjXFxsYW1iZGEiLDFdLFsxLDUsIi1cXGNpcmNcXGxhbWJkYSIsMV0sWzIsNiwiLVxcY2lyY1xcbGFtYmRhIiwxXSxbNSw2LCJcXG1hdGhiZiBjIl0sWzQsNSwiXFxtYXRoYmYgciJdLFs2LDcsIihmXFxsYW1iZGEpXlxcZGFnZ2VyKDFcXHRpbWVzXFxtYXRoYmYgbCkiXSxbNCw3LCIiLDEseyJsZXZlbCI6Miwic3R5bGUiOnsiaGVhZCI6eyJuYW1lIjoibm9uZSJ9fX1dXQ==
          \begin{tikzcd}
            {f^*} &&&& {(f^*\eta)^*} \\
                  & {(f\lambda)^\dagger} && {((f\lambda)^\dagger(1\times\eta))^\dagger} \\
                  & {(f\lambda)^\dagger} && {(f\lambda)^\dagger(1\times(\eta\lambda)^\dagger)} \\
            {f^*} &&&& {f^*\eta^*}
            \arrow["{(\mathbf e_f)^*}", from=1-1, to=1-5]
            \arrow["{\mathbf m}", from=1-5, to=4-5]
            \arrow[Rightarrow, no head, from=1-1, to=4-1]
            \arrow["{f^*\mathbf t}", from=4-5, to=4-1]
            \arrow["{-\circ\lambda^{-1}}"{description}, from=3-2, to=4-1]
            \arrow["{-\circ\lambda}"{description}, from=1-1, to=2-2]
            \arrow["{-\circ\lambda}"{description}, from=1-5, to=2-4]
            \arrow["{-\circ\lambda}"{description}, from=4-5, to=3-4]
            \arrow["{\mathbf c}", from=2-4, to=3-4]
            \arrow["{\mathbf r}", from=2-2, to=2-4]
            \arrow["{(f\lambda)^\dagger(1\times\mathbf l)}", from=3-4, to=3-2]
            \arrow[Rightarrow, no head, from=2-2, to=3-2]
          \end{tikzcd}
        \end{equation*}
      \item \cite[{(3.1)}]{fiore2017} holds by postcomposing (\ref{eq:strong_pseudomonad_house}) with
        $\lambda^{-1}$ just like above.
    \end{itemize}
  \end{proof}
\end{theorem}

We may thus conclude that the definitions that we have stated so far fit in with already
established work.

\section{Induced pseudofunctor}

Before we move on to the strong structure of a $J$-pseudomonad, we take some time to
appreciate what we have developed so far. Relative monads induce functors~\cite{altenkirch2015}.
It is therefore desirable that relative pseudomonads induce pseudofunctors. This was
not explicitly proven in~\cite{fiore2017}. We will show how to obtain the pseudofunctor
induced by a prestrong $J$-pseudomonad.

Fix a prestrong $J$-pseudomonad $T$. Analogous to \ref{def:pseudofunctor}, we define the following:

\begin{definition}
  The \emph{pseudofunctor structure induced by $T$} consists of
  \begin{enumerate}
    \item for all $X\in\bicat{J}$, the object $TX\in\bicat{C}$;
    \item for all $X,Y\in\bicat{J}$, the functor $T_{X,Y}$ is the composite
      \begin{equation}
        % https://q.uiver.app/?q=WzAsNCxbMCwwLCJcXHRleHR7SG9tfVtYLFldIl0sWzAsMSwiXFx0ZXh0e0hvbX1bSlgsSlldIl0sWzIsMSwiXFx0ZXh0e0hvbX1bSlgsVFldIl0sWzIsMCwiXFx0ZXh0e0hvbX1bVFgsVFldIl0sWzIsMywiKC0pXntUXFxsYW1iZGF9IiwyXSxbMSwyLCJcXGV0YVxcY2lyYy0iLDJdLFswLDEsIkpfe1gsWX0iLDJdLFswLDMsIlRfe1gsWX0iXV0=
        \begin{tikzcd}
          {\text{Hom}[X,Y]} && {\text{Hom}[TX,TY]} \\
          {\text{Hom}[JX,JY]} && {\text{Hom}[JX,TY]}
          \arrow["{(-)^{T\lambda}}"', from=2-3, to=1-3]
          \arrow["{\eta\circ-}"', from=2-1, to=2-3]
          \arrow["{J_{X,Y}}"', from=1-1, to=2-1]
          \arrow["{T_{X,Y}}", from=1-1, to=1-3]
        \end{tikzcd}
      \end{equation}
    \item for all $X\in\bicat{J}$, the 2-cell $\bicell i_X = \inv{\bicell l_X}\inv\lambda$;
    \item for all $f:X\to Y$ and $g:Y\to Z$ in $\bicat{J}$, the 2-cell
      \begin{align*}
        \bicell d_{f,g} = \bicell m_{\eta Jf,\eta Jg}\bullet\extend{\rr{\bicell e_{\eta Jg}Jf}}{T\lambda}
      \end{align*}
      as in the diagram
      \begin{equation}
        % https://q.uiver.app/?q=WzAsNCxbMCwwLCJUWCJdLFs3LDAsIlRZIl0sWzcsMiwiVFoiXSxbMCwyLCJUWiJdLFswLDEsIihcXGV0YSBKZileKiIsMV0sWzEsMiwiKFxcZXRhIEpnKV4qIiwxXSxbMCwyLCIoKFxcZXRhIEpnKV4qXFxldGEgSmYpXioiLDFdLFszLDIsIiIsMSx7ImxldmVsIjoyLCJzdHlsZSI6eyJoZWFkIjp7Im5hbWUiOiJub25lIn19fV0sWzAsMywiKFxcZXRhIEpnSmYpXioiLDFdLFswLDEsIlRmIiwwLHsiY3VydmUiOi01fV0sWzEsMiwiVGciLDAseyJjdXJ2ZSI6LTV9XSxbMCwzLCJUKGdmKSIsMix7ImN1cnZlIjo1fV0sWzYsMSwiXFxtYXRoYmYgbSIsMix7InNob3J0ZW4iOnsic291cmNlIjozMCwidGFyZ2V0IjozMH19XSxbMyw2LCIoXFxtYXRoYmYgZUpmKV4qIiwwLHsic2hvcnRlbiI6eyJzb3VyY2UiOjMwLCJ0YXJnZXQiOjMwfX1dXQ==
        \begin{tikzcd}
          TX &&&&&&& TY \\
          \\
          TZ &&&&&&& TZ
          \arrow["{(\eta Jf)^*}"{description}, from=1-1, to=1-8]
          \arrow["{(\eta Jg)^*}"{description}, from=1-8, to=3-8]
          \arrow[""{name=0, anchor=center, inner sep=0}, "{((\eta Jg)^*\eta Jf)^*}"{description}, from=1-1, to=3-8]
          \arrow[Rightarrow, no head, from=3-1, to=3-8]
          \arrow["{(\eta JgJf)^*}"{description}, from=1-1, to=3-1]
          \arrow["Tf", curve={height=-30pt}, from=1-1, to=1-8]
          \arrow["Tg", curve={height=-30pt}, from=1-8, to=3-8]
          \arrow["{T(gf)}"', curve={height=30pt}, from=1-1, to=3-1]
          \arrow["{\mathbf m}"', shorten <=31pt, shorten >=31pt, Rightarrow, from=0, to=1-8]
          \arrow["{(\mathbf eJf)^*}", shorten <=31pt, shorten >=31pt, Rightarrow, from=3-1, to=0]
        \end{tikzcd}
      \end{equation}
  \end{enumerate}
\end{definition}

\begin{example}
  The prestrong presheaf construction induces a pseudofunctor structure
  $\widehat{-} : \biCat\to\biCAT$ as described in \ref{ex:presheaf_pseudofunctor}.
\end{example}

Similar to the induced relative pseudomonad structure, we can prove a general
statement and no additional work is required to show that the presheaf
construction induces a pseudofunctor:

\begin{proposition}\label{prop:induced_pseudofunctor}
  The pseudofunctor structure induced by $T$ is a pseudofunctor.
  \begin{proof}
    We verify the axioms:
    \begin{enumerate}
      \item We have the commuting diagram
        \begin{equation}
          % https://q.uiver.app/?q=WzAsMTAsWzAsMCwiVChoZ2YpIl0sWzAsMiwiVChoZylUZiJdLFswLDUsIlRoIFRnIFRmIl0sWzIsMCwiVGggVChnZikiXSxbMCw0LCIoVGhcXGV0YSBKZyleKlRmIl0sWzAsMSwiKFQoaGcpXFxldGEgSmYpXntUXFxsYW1iZGF9Il0sWzEsMCwiKFRoXFxldGEgSihnZikpXioiXSxbMiw1LCJUaCgoXFxldGEgSmcpXipcXGV0YSBKZileKiJdLFsxLDMsIigoXFxldGEgSmgpXiooXFxldGEgSmcpXipcXGV0YSBKZileKiJdLFsxLDEsIigoVGhcXGV0YSBKZyleKlxcZXRhIEpmKV4qIl0sWzEsNCwiKFxcbWF0aGJmIGUgSmcpXipUZiIsMl0sWzQsMiwiXFxtYXRoYmYgbVRmIiwyXSxbMCw1LCIoXFxtYXRoYmYgZUpmKV4qIiwyXSxbNSwxLCJcXG1hdGhiZiBtIiwyXSxbMCw2LCIoXFxtYXRoYmYgZUooZ2YpKV4qIl0sWzYsMywiXFxtYXRoYmYgbSJdLFszLDcsIlRoKFxcbWF0aGJmIGVKZileKiIsMV0sWzcsMiwiKFxcZXRhIEpoKV4qXFxtYXRoYmYgbSJdLFs2LDgsIigoXFxldGEgSmgpXipcXG1hdGhiZiBlIEpmKV4qIiwxLHsiY3VydmUiOi01fV0sWzgsNywiXFxtYXRoYmYgbSJdLFs5LDQsIlxcbWF0aGJmIG0iLDJdLFs1LDksIigoXFxtYXRoYmYgZUpnKV4qIFxcZXRhIEpmKV4qIiwxLHsiY3VydmUiOjN9XSxbOSw4LCIoXFxtYXRoYmYgbVxcZXRhIEpmKV4qIiwxXSxbNiw5LCIoXFxtYXRoYmYgZSBKZileKiIsMl0sWzIsOCwiKFxccmVme2VxOnN0cm9uZ19wc2V1ZG9tb25hZF9ob3VzZX0pIiwxLHsic3R5bGUiOnsiYm9keSI6eyJuYW1lIjoibm9uZSJ9LCJoZWFkIjp7Im5hbWUiOiJub25lIn19fV0sWzEyLDIzLCJcXG1hdGhiZiBlXFx0ZXh0ey1uYXR9IiwyLHsic2hvcnRlbiI6eyJzb3VyY2UiOjIwLCJ0YXJnZXQiOjIwfSwic3R5bGUiOnsiYm9keSI6eyJuYW1lIjoibm9uZSJ9LCJoZWFkIjp7Im5hbWUiOiJub25lIn19fV0sWzQsMjEsIlxcbWF0aGJmIG5cXHRleHR7LW5hdH0iLDEseyJzaG9ydGVuIjp7InRhcmdldCI6MjB9LCJzdHlsZSI6eyJib2R5Ijp7Im5hbWUiOiJub25lIn0sImhlYWQiOnsibmFtZSI6Im5vbmUifX19XSxbMTUsMTYsIlxcbWF0aGJmIG1cXHRleHR7LW5hdH0iLDEseyJzaG9ydGVuIjp7InNvdXJjZSI6MjAsInRhcmdldCI6MjB9LCJzdHlsZSI6eyJib2R5Ijp7Im5hbWUiOiJub25lIn0sImhlYWQiOnsibmFtZSI6Im5vbmUifX19XV0=
          \begin{tikzcd}
            {T(hgf)} & {(Th\eta J(gf))^*} & {Th T(gf)} \\
            {(T(hg)\eta Jf)^{T\lambda}} & {((Th\eta Jg)^*\eta Jf)^*} \\
            {T(hg)Tf} \\
                                        & {((\eta Jh)^*(\eta Jg)^*\eta Jf)^*} \\
                                        {(Th\eta Jg)^*Tf} \\
            {Th Tg Tf} && {Th((\eta Jg)^*\eta Jf)^*}
            \arrow["{(\mathbf e Jg)^*Tf}"', from=3-1, to=5-1]
            \arrow["{\mathbf mTf}"', from=5-1, to=6-1]
            \arrow[""{name=0, anchor=center, inner sep=0}, "{(\mathbf eJf)^*}"', from=1-1, to=2-1]
            \arrow["{\mathbf m}"', from=2-1, to=3-1]
            \arrow["{(\mathbf eJ(gf))^*}", from=1-1, to=1-2]
            \arrow[""{name=1, anchor=center, inner sep=0}, "{\mathbf m}", from=1-2, to=1-3]
            \arrow[""{name=2, anchor=center, inner sep=0}, "{Th(\mathbf eJf)^*}"{description}, from=1-3, to=6-3]
            \arrow["{(\eta Jh)^*\mathbf m}", from=6-3, to=6-1]
            \arrow["{((\eta Jh)^*\mathbf e Jf)^*}"{description}, curve={height=-70pt}, from=1-2, to=4-2]
            \arrow["{\mathbf m}", from=4-2, to=6-3]
            \arrow["{\mathbf m}"', from=2-2, to=5-1]
            \arrow[""{name=3, anchor=center, inner sep=0}, "{((\mathbf eJg)^* \eta Jf)^*}"{description}, curve={height=18pt}, from=2-1, to=2-2]
            \arrow["{(\mathbf m\eta Jf)^*}"{description}, from=2-2, to=4-2]
            \arrow[""{name=4, anchor=center, inner sep=0}, "{(\mathbf e Jf)^*}"', from=1-2, to=2-2]
            \arrow["{(\ref{eq:strong_pseudomonad_house})}"{description}, draw=none, from=6-1, to=4-2]
            \arrow["{\mathbf e\text{-nat}}"', draw=none, from=0, to=4]
            \arrow["{\mathbf n\text{-nat}}"{description}, draw=none, from=5-1, to=3]
            \arrow["{\mathbf m\text{-nat}}"{description}, draw=none, from=1, to=2]
          \end{tikzcd}
        \end{equation}
        where the remaining face follows from~\cite[Lemma 3.2 (i)]{fiore2017}.
        We have shown \ref{eq:pseudofunctor_coherence_associativity}.
      \item The conditions \ref{eq:pseudofunctor_coherence_identity} are just~\cite[(3.1)]{fiore2017}.
    \end{enumerate}
  \end{proof}
\end{proposition}

\section{Strong structure}

We are now ready for our final step towards the definition of a strong relative pseudomonad.
As we are not be able to prove all the results that we would have liked, we provide
some additional insight into how one might come up with the definition of a strong $J$-pseudomonad
structure.

Firstly, we notice that the prestrong structure already gives rise to a
1-cell $X\times TY\to T(X\times Y)$ by extending the unit $\eta_{X\times Y}$. We will
therefore think of $\eta^\dagger$ as the strength. Secondly, we observe that of the four
structural 2-cells in~\cite[Definitions 8 and 9]{saville2023} only one involves repeated
applications of the object map. The others are thus easily translated. To avoid the
repeated applications in the problematic case, we take inspiration from the definition
of a strong relative monad in~\cite{tarmo}. The result is the rather unintuitive
family of invertible 2-cells $\bicell q$ which will allow us to construct the usual
pentagon in the case where $J$ is the identity.

\begin{definition}\label{def:strong_inclusion_pseudomonad_structure}
  A \emph{strong $J$-pseudomonad structure} consists of
  \begin{enumerate}
    \item a prestrong $J$-pseudomonad $T$;
    \item for all $X,Y,Z\in\bicat{J}$, an invertible 2-cell
      \begin{equation}\label{eq:strong_pseudomonad_p}
        % https://q.uiver.app/?q=WzAsNSxbMCwwLCIoSlhcXHRpbWVzIEpZKVxcdGltZXMgVFoiXSxbMCwxLCJKWFxcdGltZXMoSllcXHRpbWVzIFRaKSJdLFsyLDAsIlQoKFhcXHRpbWVzIFkpXFx0aW1lcyBaKSJdLFsyLDEsIlQoWFxcdGltZXMoWVxcdGltZXMgWikpIl0sWzEsMSwiSlhcXHRpbWVzIFQoWVxcdGltZXMgWikiXSxbMSw0LCJcXGV0YV5cXGRhZ2dlciIsMl0sWzQsMywiXFxldGFeXFxkYWdnZXIiLDJdLFswLDEsIlxcYWxwaGEiLDJdLFswLDIsIlxcZXRhXlxcZGFnZ2VyIl0sWzIsMywiVFxcYWxwaGEiXSxbNyw5LCJcXG1hdGhiZiBwX3tYLFksWn0iLDEseyJzaG9ydGVuIjp7InNvdXJjZSI6MjAsInRhcmdldCI6MjB9fV1d
        \begin{tikzcd}
          {(JX\times JY)\times TZ} && {T((X\times Y)\times Z)} \\
          {JX\times(JY\times TZ)} & {JX\times T(Y\times Z)} & {T(X\times(Y\times Z))}
          \arrow["{\eta^\dagger}"', from=2-1, to=2-2]
          \arrow["{\eta^\dagger}"', from=2-2, to=2-3]
          \arrow[""{name=0, anchor=center, inner sep=0}, "\alpha"', from=1-1, to=2-1]
          \arrow["{\eta^\dagger}", from=1-1, to=1-3]
          \arrow[""{name=1, anchor=center, inner sep=0}, "T\alpha", from=1-3, to=2-3]
          \arrow["{\mathbf p_{X,Y,Z}}"{description}, shorten <=29pt, shorten >=29pt, Rightarrow, from=0, to=1]
        \end{tikzcd}
      \end{equation}
    \item for all invertible 2-cells
      \begin{equation*}
        % https://q.uiver.app/?q=WzAsMyxbMCwwLCJKV1xcdGltZXMgSlgiXSxbMiwwLCJKV1xcdGltZXMgVFkiXSxbMSwxLCJUKFdcXHRpbWVzIFkpIl0sWzEsMiwiXFxldGFeXFxkYWdnZXIiXSxbMCwxLCJKV1xcdGltZXMgZiJdLFswLDIsImciLDJdLFs1LDMsIlxcbWF0aGJmIHUiLDAseyJzaG9ydGVuIjp7InNvdXJjZSI6MjAsInRhcmdldCI6MjB9fV1d
        \begin{tikzcd}
          {JW\times JX} && {JW\times TY} \\
                        & {T(W\times Y)}
                        \arrow[""{name=0, anchor=center, inner sep=0}, "{\eta^\dagger}", from=1-3, to=2-2]
                        \arrow["{JW\times f}", from=1-1, to=1-3]
                        \arrow[""{name=1, anchor=center, inner sep=0}, "g"', from=1-1, to=2-2]
                        \arrow["{\mathbf u}", shorten <=10pt, shorten >=10pt, Rightarrow, from=1, to=0]
        \end{tikzcd}
      \end{equation*}
      an invertible 2-cell
      \begin{equation}\label{eq:strong_pseudomonad_q}
        % https://q.uiver.app/?q=WzAsNCxbMCwwLCJKV1xcdGltZXMgVFgiXSxbMywwLCJUKFdcXHRpbWVzIFgpIl0sWzAsMSwiSldcXHRpbWVzIFRZIl0sWzMsMSwiVChXXFx0aW1lcyBZKSJdLFswLDEsIlxcZXRhXlxcZGFnZ2VyIl0sWzIsMywiXFxldGFeXFxkYWdnZXIiLDJdLFsxLDMsImdeKiJdLFswLDIsIkpXXFx0aW1lcyBmXioiLDJdLFs3LDYsIlxcbWF0aGJmIHFfe1xcbWF0aGJmIHV9IiwxLHsic2hvcnRlbiI6eyJzb3VyY2UiOjIwLCJ0YXJnZXQiOjIwfX1dXQ==
        \begin{tikzcd}
          {JW\times TX} &&& {T(W\times X)} \\
          {JW\times TY} &&& {T(W\times Y)}
          \arrow["{\eta^\dagger}", from=1-1, to=1-4]
          \arrow["{\eta^\dagger}"', from=2-1, to=2-4]
          \arrow[""{name=0, anchor=center, inner sep=0}, "{g^*}", from=1-4, to=2-4]
          \arrow[""{name=1, anchor=center, inner sep=0}, "{JW\times f^*}"', from=1-1, to=2-1]
          \arrow["{\mathbf q_{\mathbf u}}"{description}, shorten <=23pt, shorten >=23pt, Rightarrow, from=1, to=0]
        \end{tikzcd}
      \end{equation}
    \item for all $f:JW\times JX\to TY$ in $\bicat{C}$, an invertible 2-cell
      \begin{equation}\label{eq:strong_pseudomonad_s}
        % https://q.uiver.app/?q=WzAsMyxbMCwwLCJKV1xcdGltZXMgVFgiXSxbMiwxLCJUWSJdLFs0LDAsIlQoV1xcdGltZXMgWCkiXSxbMCwyLCJcXGV0YV5cXGRhZ2dlciJdLFsyLDEsImZeKiJdLFswLDEsImZeXFxkYWdnZXIiLDJdLFs1LDQsIlxcbWF0aGJmIHNfZiIsMCx7InNob3J0ZW4iOnsic291cmNlIjoyMCwidGFyZ2V0IjoyMH19XV0=
        \begin{tikzcd}
          {JW\times TX} &&&& {T(W\times X)} \\
                        && TY
                        \arrow["{\eta^\dagger}", from=1-1, to=1-5]
                        \arrow[""{name=0, anchor=center, inner sep=0}, "{f^*}", from=1-5, to=2-3]
                        \arrow[""{name=1, anchor=center, inner sep=0}, "{f^\dagger}"', from=1-1, to=2-3]
                        \arrow["{\mathbf s_f}", shorten <=13pt, shorten >=13pt, Rightarrow, from=1, to=0]
        \end{tikzcd}
      \end{equation}
  \end{enumerate}
\end{definition}

Let us now verify that it is indeed possible to obtain such a structure for the
presheaf construction. Of particular interest to us is the structural 2-cells
$\bicell q_{\bicell u}$ because they are different to all the others that we
have seen so far.

\begin{example}
  We extend the prestrong presheaf construction as follows:
  \begin{enumerate}
    \item for all $X\in\scat X$, $Y\in\scat Y$, and $P\in\widehat{\scat Z}$,
      $\rr{\bicell p_{\scat X,\scat Y,\scat Z}}_{X,Y,P}$ is the isomorphism of coends
      \begin{align*}
        &\int^{X',Y',Z'}\rr{\int^Z PZ\times\Hom(((X',Y'),Z'),((X,Y),Z))} \times \Hom(-,(X',(Y',Z'))) \\
        &\cong \int^{Y',Z'}\rr{\int^Z PZ\times\Hom((Y',Z'),(Y,Z))} \times \Hom(-,(X,(Y',Z')))
      \end{align*}
      which may be obtained by composition of $\bicell l$ and $\bicell c$;\footnote{
        Construction of $\bicell p$ in \href{https://q.uiver.app/?q=WzAsMTEsWzQsMCwiXFxpbnRee1gnLFknLFonfVxcbGVmdChcXGludF5aIFBaXFx0aW1lc1xcdGV4dHtIb219KCgoWCcsWScpLFonKSwoKFgsWSksWikpXFxyaWdodClcXHRpbWVzXFx0ZXh0e0hvbX0oLSxYJywoWScsWicpKSJdLFs0LDYsIlxcaW50XntZJyxaJ31cXGxlZnQoXFxpbnReWiBQWlxcdGltZXNcXHRleHR7SG9tfSgoWScsWicpLChZLFopKVxccmlnaHQpXFx0aW1lc1xcdGV4dHtIb219KC0sKFgsKFknLFonKSkpIl0sWzIsMCwiXFxpbnRee1gnfVxcbGVmdChcXGludF5aIFBaXFx0aW1lc1xcdGV4dHtIb219KCgoWCcsWScpLFonKSwoKFgsWSksWikpXFxyaWdodClcXHRpbWVzXFx0ZXh0e0hvbX0oLSwoWCcsKFknLFonKSkpIl0sWzIsNiwiXFxsZWZ0KFxcaW50XlpQWlxcdGltZXNcXHRleHR7SG9tfSgoWScsWicpLChZLFopKVxccmlnaHQpXFx0aW1lc1xcdGV4dHtIb219KC0sKFgsKFknLFonKSkpIl0sWzIsMiwiXFxpbnReWiBQWlxcdGltZXNcXGludF57WCd9XFx0ZXh0e0hvbX0oKChYJyxZJyksWicpLCgoWCxZKSxaKSlcXHRpbWVzXFx0ZXh0e0hvbX0oLSwoWCcsKFknLFonKSkpIl0sWzAsNiwiXFxsZWZ0KFBaXFx0aW1lc1xcdGV4dHtIb219KChZJyxaJyksKFksWikpXFxyaWdodClcXHRpbWVzXFx0ZXh0e0hvbX0oLSwoWCwoWScsWicpKSkiXSxbMCwyLCJQWlxcdGltZXNcXGludF57WCd9XFx0ZXh0e0hvbX0oKChYJyxZJyksWicpLCgoWCxZKSxaKSlcXHRpbWVzXFx0ZXh0e0hvbX0oLSwoWCcsKFknLFonKSkpIl0sWzIsMywiXFxpbnReWiBQWlxcdGltZXNcXGludF57WCd9KFxcdGV4dHtIb219KChZJyxaJyksKFksWikpXFx0aW1lc1xcdGV4dHtIb219KC0sKFgnLChZJyxaJykpKSlcXHRpbWVzXFx0ZXh0e0hvbX0oWCcsWCkiXSxbMiw1LCJcXGludF5aUFpcXHRpbWVzKFxcdGV4dHtIb219KChZJyxaJyksKFksWikpXFx0aW1lc1xcdGV4dHtIb219KC0sKFgsKFknLFonKSkpKSJdLFswLDMsIlBaXFx0aW1lc1xcaW50XntYJ31cXHRleHR7SG9tfSgoWScsWicpLChZLFopKVxcdGltZXNcXHRleHR7SG9tfSgtLChYJywoWScsWicpKSlcXHRpbWVzXFx0ZXh0e0hvbX0oWCcsWCkiXSxbMCw1LCJQWlxcdGltZXMoXFx0ZXh0e0hvbX0oKFknLFonKSwoWSxaKSlcXHRpbWVzXFx0ZXh0e0hvbX0oLSwoWCwoWScsWicpKSkpIl0sWzAsMSwiXFxtYXRoYmYgcCJdLFszLDEsInEiLDJdLFsyLDAsInEiXSxbMiw0LCJcXG1hdGhiZiBjXnstMX0iLDJdLFs2LDQsInEiXSxbNSwzLCJxXFx0aW1lc1xcdGV4dHtIb219KC0sKFgsKFknLFonKSkpIiwyXSxbNCw3LCJcXGNvbmciLDFdLFs3LDgsIlxcaW50XlpQWlxcdGltZXNcXG1hdGhiZiBsXnstMX0iLDFdLFs4LDMsIlxcY29uZyIsMV0sWzEwLDgsInEiLDFdLFs5LDcsInEiLDFdLFsxMCw1LCJcXGNvbmciLDFdLFs5LDEwLCJQWlxcdGltZXNcXG1hdGhiZiBsXnstMX0iLDFdLFs2LDksIlxcY29uZyIsMV1d}
      {quiver}}
    \item for all natural isomorphisms with components
      \begin{align*}
        \bicell u_{W,X} : G(W,X)\cong \int^Y (FX)Y\times\Hom(-,(W,Y))
      \end{align*}
      we have the natural isomorphism $\bicell q_{\bicell u}$ whose components are themselves
      natural isomorphisms
      \begin{align*}
        &\int^Y \rr{\int^X PX\times(FX)Y}\times\Hom\rr{-,(W,Y)} \\
        &\cong \int^{W',X'}\rr{\int^X PX\times\Hom((W',X'),(W,X))}\times G(W',X')(-)
      \end{align*}
      which are obtained by composing $\mathbf c$, $\mathbf l$, and  $\mathbf u$;\footnote{
        Construction of $\bicell q$ in \href{
        https://q.uiver.app/?q=WzAsMTAsWzMsMCwiXFxpbnReWVxcbGVmdChcXGludF5YIFBYXFx0aW1lcyhGWClZXFxyaWdodClcXHRpbWVzXFx0ZXh0e0hvbX0oLSwoVyxZKSkiXSxbMyw0LCJcXGludF57VycsWCd9XFxsZWZ0KFxcaW50XlggUFhcXHRpbWVzXFx0ZXh0e0hvbX0oKFcnLFgnKSwoVyxYKSlcXHJpZ2h0KVxcdGltZXMgRyhXJyxYJykoLSkiXSxbMiwwLCJcXGludF5YIFBYIFxcdGltZXNcXGludF5ZKEZYKVlcXHRpbWVzXFx0ZXh0e0hvbX0oLSwoVyxZKSkiXSxbMiw0LCJcXGludF5YIFBYXFx0aW1lc1xcaW50XntXJyxYJ31cXHRleHR7SG9tfSgoVycsWCcpLChXLFgpKVxcdGltZXMgRyhXJyxYJykoLSkiXSxbMCwwLCJQWCBcXHRpbWVzXFxpbnReWShGWClZXFx0aW1lc1xcdGV4dHtIb219KC0sKFcsWSkpIl0sWzAsNCwiUFhcXHRpbWVzXFxpbnRee1cnLFgnfVxcdGV4dHtIb219KChXJyxYJyksKFcsWCkpXFx0aW1lcyBHKFcnLFgnKSgtKSJdLFswLDEsIlBYXFx0aW1lcyBHKFcsWCkoLSkiXSxbMCwzLCJQWFxcdGltZXNcXGludF57VycsWCd9RyhXJyxYJykoLSlcXHRpbWVzXFx0ZXh0e0hvbX0oKFcnLFgnKSwoVyxYKSkiXSxbMiwxLCJcXGludF5YUFhcXHRpbWVzIEcoVyxYKSgtKSJdLFsyLDMsIlxcaW50XlhQWFxcdGltZXNcXGludF57VycsWCd9RyhXJyxYJykoLSlcXHRpbWVzXFx0ZXh0e0hvbX0oKFcnLFgnKSwoVyxYKSkiXSxbMCwxLCIoXFxtYXRoYmYgcV97XFxtYXRoYmYgdX0pX3tXLFB9Il0sWzMsMSwiXFxtYXRoYmYgYyIsMV0sWzAsMiwiXFxtYXRoYmYgY157LTF9IiwxXSxbNSwzLCJxIiwxXSxbNCwyLCJxIiwxXSxbNCw2LCJQWFxcdGltZXNcXG1hdGhiZiB1IiwxXSxbNiw3LCJQWFxcdGltZXNcXG1hdGhiZiBsIiwxXSxbNyw1LCJQWFxcdGltZXNcXGludFxcZ2FtbWEiLDFdLFs2LDgsInEiLDFdLFs3LDksInEiLDFdLFs5LDMsIlxcaW50XlggUFhcXHRpbWVzIFxcaW50XFxnYW1tYSIsMV0sWzIsOCwiXFxpbnReWCBQWFxcdGltZXNcXG1hdGhiZiB1IiwxXSxbOCw5LCJcXGludF5YIFBYXFx0aW1lc1xcbWF0aGJmIGwiLDFdXQ==}
      {quiver}.}
    \item for all $F:\scat W\times\scat X\to \widehat{\scat Y}$, the natural isomorphism
      $\bicell s_F$ has as components further natural isomorphisms
      \begin{align*}
        &\int^X PX\times F(W,X)(-)\\
        &\cong \int^X PX\times \int^{W',X'} F(W',X')(-)\times\Hom((W',X'),(W,X)) \\
        &\cong \int^X PX\times \int^{W',X'} \Hom((W',X'),(W,X))\times F(W',X')(-) \\
        &\cong \int^{W',X'}\rr{\int^X PX\times\Hom((W',X'),(W,X))}\times F(W',X')(-).
      \end{align*}
  \end{enumerate}
\end{example}

\section{Towards an induced strong pseudomonad}\label{sec:induced_strong_pseudomonad}

The structure \ref{def:strong_inclusion_pseudomonad_structure} requires some further
axioms to be useful. Ideally, we would like for the structural 2-cells
to be coherent. However, proving such a result may be very difficult. A more attainable
goal is to show that, in the case where the inclusion is the identity, we obtain a strong
pseudomonad.

We are not able to state any further axioms or prove any of the results above. Instead
we are going to outline how one can obtain the structure of a strong pseudomonad.
We begin by constructing the strength and proceed by adding the structural 2-cells that
promote it to a strength of the induced pseudofunctor and subsequently the induced
pseudomonad.

Fix a strong $J$-pseudomonad structure $T$.

We have already discussed how to obtain the 1-cells $JW\times TX\to T(W\times X)$ that
resemble the components of a pseudonatural transformation. We are now able to extend this
structure to include the naturality 2-cell. Combining these, we obtain the structure
of a pseudonatural transformation.

\begin{definition}\label{def:induced_strength}
  The \emph{strength induced by $T$} consists of
  \begin{enumerate}
    \item for all $W,X\in\bicat{J}$, the 1-cell
      \begin{equation}\label{eq:strong_inclusion_pseudomonad_induced_strength}
        \sigma_{W,X} = \extend{\rr{\eta_{W\times X}}}{\dagger} : JW\times TX\to T(W\times X)
      \end{equation}
    \item for all $f:W\to W'$ and $g:X\to X'$ in $\bicat{J}$, the naturality 2-cell
      \begin{align*}
        \bicell n_{f,g} = \mathbf c\bullet(\mathbf rJ(f\times g))^\dagger\bullet\inv{\mathbf s}
      \end{align*}
      as in the diagram
      \begin{equation}
        % https://q.uiver.app/?q=WzAsNixbMCwyLCJKV1xcdGltZXMgVFgiXSxbMCw1LCJKVydcXHRpbWVzIFRYJyJdLFszLDAsIlQoV1xcdGltZXMgWCkiXSxbMywzLCJUKFcnXFx0aW1lcyBYJykiXSxbMCwwLCJKV1xcdGltZXMgVFgiXSxbMyw1LCJUKFcnXFx0aW1lcyBYJykiXSxbMiwzLCJUKGZcXHRpbWVzIGcpIiwxXSxbMCwxLCJKZlxcdGltZXMgVGciLDFdLFsxLDUsIlxcZXRhXlxcZGFnZ2VyIiwxXSxbMyw1LCIiLDEseyJsZXZlbCI6Miwic3R5bGUiOnsiaGVhZCI6eyJuYW1lIjoibm9uZSJ9fX1dLFs0LDIsIlxcZXRhXlxcZGFnZ2VyIiwxXSxbNCwzLCIoXFxldGEgKEpmXFx0aW1lcyBKZykpXlxcZGFnZ2VyIiwxXSxbMCw1LCIoXFxldGFeXFxkYWdnZXIgKEpmXFx0aW1lc1xcZXRhIEpnKSleXFxkYWdnZXIiLDFdLFs0LDAsIiIsMSx7ImxldmVsIjoyLCJzdHlsZSI6eyJoZWFkIjp7Im5hbWUiOiJub25lIn19fV0sWzExLDEyLCIoXFxtYXRoYmYgciAoSmZcXHRpbWVzIEpnKSleXFxkYWdnZXIiLDEseyJzaG9ydGVuIjp7InNvdXJjZSI6MjAsInRhcmdldCI6MjB9fV0sWzEyLDgsIlxcbWF0aGJmIGMiLDEseyJzaG9ydGVuIjp7InNvdXJjZSI6MjAsInRhcmdldCI6MjB9fV0sWzEwLDExLCJcXG1hdGhiZiBzXnstMX0iLDEseyJzaG9ydGVuIjp7InNvdXJjZSI6MzAsInRhcmdldCI6MzB9fV1d
        \begin{tikzcd}
          {JW\times TX} &&& {T(W\times X)} \\
          \\
          {JW\times TX} \\
                        &&& {T(W'\times X')} \\
                        \\
          {JW'\times TX'} &&& {T(W'\times X')}
          \arrow["{T(f\times g)}"{description}, from=1-4, to=4-4]
          \arrow["{Jf\times Tg}"{description}, from=3-1, to=6-1]
          \arrow[""{name=0, anchor=center, inner sep=0}, "{\eta^\dagger}"{description}, from=6-1, to=6-4]
          \arrow[Rightarrow, no head, from=4-4, to=6-4]
          \arrow[""{name=1, anchor=center, inner sep=0}, "{\eta^\dagger}"{description}, from=1-1, to=1-4]
          \arrow[""{name=2, anchor=center, inner sep=0}, "{(\eta (Jf\times Jg))^\dagger}"{description}, from=1-1, to=4-4]
          \arrow[""{name=3, anchor=center, inner sep=0}, "{(\eta^\dagger (Jf\times\eta Jg))^\dagger}"{description}, from=3-1, to=6-4]
          \arrow[Rightarrow, no head, from=1-1, to=3-1]
          \arrow["{(\mathbf r (Jf\times Jg))^\dagger}"{description}, shorten <=9pt, shorten >=9pt, Rightarrow, from=2, to=3]
          \arrow["{\mathbf c}"{description}, shorten <=6pt, shorten >=6pt, Rightarrow, from=3, to=0]
          \arrow["{\mathbf s^{-1}}"{description}, shorten <=10pt, shorten >=10pt, Rightarrow, from=1, to=2]
        \end{tikzcd}
      \end{equation}
  \end{enumerate}
\end{definition}

With the axioms we have included so far, we are not able to prove that this is indeed
a pseudonatural transformation. Further, we have not been able to identify any intuitive or obvious axioms that
would achieve this goal.\footnote{For future reference, \href{https://q.uiver.app/?q=WzAsMjEsWzAsMCwiVChnXFx0aW1lcyBnJylUKGZcXHRpbWVzIGYnKVxcc2lnbWEiXSxbMCw3LCJUKGdmXFx0aW1lcyBnJ2YnKVxcc2lnbWEiXSxbMTQsNywiXFxzaWdtYShKKGdmKVxcdGltZXMgVChnJ2YnKSkiXSxbMTQsMCwiXFxzaWdtYShKKGdmKVxcdGltZXMgVGcnVGYnKSJdLFs3LDAsIlQoZ1xcdGltZXMgZycpXFxzaWdtYShKZlxcdGltZXMgVGYnKSJdLFsxLDYsIihcXGV0YSBKKGdmXFx0aW1lcyBnJ2YnKSlee1RcXGxhbWJkYX1cXGV0YV5UIl0sWzEsMSwiKFxcZXRhIEooZ1xcdGltZXMgZycpKV57VFxcbGFtYmRhfShcXGV0YSBKKGZcXHRpbWVzIGYnKSlee1RcXGxhbWJkYX1cXGV0YV5UIl0sWzEzLDEsIlxcZXRhXlQoSihnZilcXHRpbWVzIChcXGV0YSBKZylee1RcXGxhbWJkYX0oXFxldGEgSmYpXntUXFxsYW1iZGF9KSJdLFsxMyw2LCJcXGV0YV5UKEooZ2YpXFx0aW1lcyhcXGV0YSBKKGcnZicpKV57VFxcbGFtYmRhfSkiXSxbNywxLCIoXFxldGEgSihnXFx0aW1lcyBnJykpXntUXFxsYW1iZGF9XFxldGFeVChKZlxcdGltZXMgKFxcZXRhIEpmJylee1RcXGxhbWJkYX0pIl0sWzUsNiwiKFxcZXRhIEooZ2ZcXHRpbWVzIGcnZicpKV57VH0iXSxbOSw2LCIoXFxldGFeVChKKGdmKVxcdGltZXMoXFxldGEgSihnJ2YnKSkpKV5UIl0sWzMsMSwiKFxcZXRhIEooZ1xcdGltZXMgZycpKV57VFxcbGFtYmRhfShcXGV0YSBKKGZcXHRpbWVzIGYnKSleVCJdLFs1LDEsIihcXGV0YSBKKGdcXHRpbWVzIGcnKSlee1RcXGxhbWJkYX0oXFxldGFeVChKZlxcdGltZXMgXFxldGEgSmYnKSleVCJdLFs5LDEsIihcXGV0YSBKKGdcXHRpbWVzIGcnKSleVChKZlxcdGltZXMgKFxcZXRhIEpmJylee1RcXGxhbWJkYX0pIl0sWzExLDEsIihcXGV0YV5UKEpnXFx0aW1lcyBcXGV0YSBKZycpKV5UKEpmXFx0aW1lcyBcXGV0YSBKZicpXntUXFxsYW1iZGF9Il0sWzEzLDMsIlxcZXRhXlQoSihnZilcXHRpbWVzKChcXGV0YSBKZylee1RcXGxhbWJkYX1cXGV0YSBKZilee1RcXGxhbWJkYX0pIl0sWzEsNCwiKChcXGV0YSBKKGdcXHRpbWVzIGcnKSlee1RcXGxhbWJkYX1cXGV0YSBKKGZcXHRpbWVzIGYnKSlee1RcXGxhbWJkYX1cXGV0YV5UIl0sWzMsNCwiKChcXGV0YSBKKGdcXHRpbWVzIGcnKSlee1RcXGxhbWJkYX1cXGV0YSBKKGZcXHRpbWVzIGYnKSlee1R9Il0sWzYsNCwiKChcXGV0YSBKKGdcXHRpbWVzIGcnKSlee1RcXGxhbWJkYX1cXGV0YV5UIChKZlxcdGltZXMgXFxldGEgSmYnKSlee1R9Il0sWzksNCwiKChcXGV0YSBKKGdcXHRpbWVzIGcnKSleVChKZlxcdGltZXMgXFxldGEgSmYnKSleVCJdLFs0LDMsIlxcbWF0aGJmIG4oSmZcXHRpbWVzIFRmJykiLDFdLFswLDQsIlQoZ1xcdGltZXMgZycpXFxtYXRoYmYgbiIsMV0sWzAsMSwiXFxtYXRoYmYgZF57LTF9XFxzaWdtYSIsMV0sWzMsMiwiXFxzaWdtYShKKGdmKVxcdGltZXMgXFxtYXRoYmYgZF57LTF9KSIsMV0sWzEsMiwiXFxtYXRoYmYgbiIsMV0sWzAsNl0sWzEsNV0sWzMsN10sWzgsMl0sWzUsMTAsIlxcbWF0aGJmIHNeey0xfSIsMV0sWzEwLDExLCIoXFxtYXRoYmYgcihKKGdmKVxcdGltZXMgSihnJ2YnKSkpXlQiLDFdLFsxMSw4LCJcXG1hdGhiZiBjIiwxXSxbNiwxMiwiKFxcZXRhIEooZ1xcdGltZXMgZycpKV5UXFxtYXRoYmYgc157LTF9IiwxLHsiY3VydmUiOi0zfV0sWzEyLDEzLCIoXFxldGEgSihnXFx0aW1lcyBnJykpXntUXFxsYW1iZGF9KFxcbWF0aGJmIHIoSmZcXHRpbWVzIEpmJykpXlQiLDEseyJjdXJ2ZSI6LTN9XSxbMTMsOSwiKFxcZXRhIEooZ1xcdGltZXMgZycpKV57VFxcbGFtYmRhfVxcbWF0aGJmIGMiLDFdLFs5LDE0LCJcXG1hdGhiZiBzXnstMX0oSmZcXHRpbWVzIChcXGV0YSBKZicpXntUXFxsYW1iZGF9KSIsMSx7ImN1cnZlIjotM31dLFsxNCwxNSwiKFxcbWF0aGJmIHIoSmdcXHRpbWVzIEpnJykpXlQoSmZcXHRpbWVzIFxcZXRhIEpmJylee1RcXGxhbWJkYX0iLDEseyJjdXJ2ZSI6LTN9XSxbMTUsNywiXFxtYXRoYmYgYyhKZlxcdGltZXMgXFxldGEgSmYnKV57VFxcbGFtYmRhfSIsMSx7ImN1cnZlIjotM31dLFs2LDE3LCJcXG1hdGhiZiBtXnstMX1cXGV0YV5UIiwxXSxbMTcsNSwiKFxcbWF0aGJmIGVeey0xfSBKKGZcXHRpbWVzIGYnKSlee1RcXGxhbWJkYX1cXGV0YV5UIiwxXSxbNywxNiwiXFxldGFeVChKKGdmKVxcdGltZXMgXFxtYXRoYmYgbV57LTF9KSIsMV0sWzE2LDgsIlxcZXRhXlQoSihnZilcXHRpbWVzIChcXG1hdGhiZiBlXnstMX1KZilee1RcXGxhbWJkYX0pIiwxXSxbMTcsMTgsIlxcbWF0aGJmIHNeey0xfSIsMV0sWzE4LDEwLCIoXFxtYXRoYmYgZV57LTF9IEooZlxcdGltZXMgZicpKV5UIiwxXSxbNCw5XSxbMTgsMTksIigoXFxldGEgSihnXFx0aW1lcyBnJykpXntUXFxsYW1iZGF9XFxtYXRoYmYgckooZlxcdGltZXMgZicpKV5UIiwxXSxbMTksMjAsIihcXG1hdGhiZiBzXnstMX0oSmZcXHRpbWVzIFxcZXRhIEpmJykpXlQiLDFdLFsxMCwyMCwiKFxcbWF0aGJmIHJeey0xfSBKKGZcXHRpbWVzIGYnKSleVCIsMV0sWzE0LDIwLCJcXG1hdGhiZiBjXnstMX0iLDFdLFs0MCw0NCwiXFxtYXRoYmYgc1xcdGV4dHstbmF0fSIsMSx7InNob3J0ZW4iOnsic291cmNlIjoyMCwidGFyZ2V0IjoyMH0sInN0eWxlIjp7ImJvZHkiOnsibmFtZSI6Im5vbmUifSwiaGVhZCI6eyJuYW1lIjoibm9uZSJ9fX1dXQ==}
{the diagram} that we require to commute}.

\begin{example}
  For the strong presheaf construction, the strength is given by
  \begin{enumerate}
    \item the functor $\sigma:\scat W\times\widehat{\scat X}\to\widehat{\scat W\times\scat X}$
      acting on objects by
      \begin{align*}
        \sigma(W,P) = \int^X PX \times \Hom(-,(W,X));
      \end{align*}
    \item the natural transformation $\bicell n_{F,G}:JF\times \widehat{G}\to \widehat{F\times G}$
      which has as components natural isomorphisms
      \begin{align*}
        &\int^{X',Y'}\rr{\int^Y PY\times\Hom((X',Y'),(X,Y))}\times\Hom(-,(FX',GY'))\\
        &\cong \int^{Y'}\rr{\int^Y PY\times\Hom(Y',GY)}\times\Hom(-,(FX,Y')).
      \end{align*}
  \end{enumerate}
\end{example}

In order for a strong $J$-pseudomonad to induce a strong pseudomonad in the
sense of~\cite{saville2023}, we require $J$ to be the identity. We describe how
one obtains the appropriate structure:

\begin{definition}\label{def:induced_strong_pseudomonad}
  Let $J:\bicat C\to\bicat C$ be the identity and $T$ a strong $J$-pseudomonad structure.
  The \emph{strong pseudomonad structure induced by $T$} consists of
  \begin{enumerate}
    \item the strength \ref{def:induced_strength};
    \item for all $X\in\bicat{C}$, the invertible 2-cell $\bicell x_X=\bicell s_{\eta\lambda}\bullet \bicell l_X$;
    \item for all $X,Y,Z\in\bicat{C}$, the invertible 2-cell $\bicell y_{X,Y,Z}=\bicell p_{X,Y,Z}$;
    \item for all $X,Y\in\bicat{C}$, the invertible 2-cell $\bicell w_{X,Y}$ given by
      \begin{equation}
        % https://q.uiver.app/?q=WzAsNixbMCwwLCJXXFx0aW1lcyBUXjJYIl0sWzAsMSwiVChXXFx0aW1lcyBUWCkiXSxbMCwzLCJUXjIoV1xcdGltZXMgWCkiXSxbNywwLCJXXFx0aW1lcyBUWCJdLFs3LDEsIlQoV1xcdGltZXMgWCkiXSxbNywzLCJUKFdcXHRpbWVzIFgpIl0sWzAsMywiWFxcdGltZXMgXFx0ZXh0e2lkfV4qIiwxXSxbMyw0LCJcXGV0YV5cXGRhZ2dlciIsMV0sWzAsMSwiXFxldGFeXFxkYWdnZXIiLDFdLFsyLDUsIlxcdGV4dHtpZH1eKiIsMV0sWzQsNSwiIiwxLHsibGV2ZWwiOjIsInN0eWxlIjp7ImhlYWQiOnsibmFtZSI6Im5vbmUifX19XSxbMSw1LCIoXFxsYW1iZGFeXFxkYWdnZXIoMVxcdGltZXNcXGV0YVxcZXRhXlxcZGFnZ2VyKSleXFxkYWdnZXJcXGxhbWJkYV57LTF9IiwxXSxbMSwyLCIoXFxldGFcXGV0YV5cXGRhZ2dlcileKiIsMV0sWzEsNCwiKFxcZXRhXlxcZGFnZ2VyKV4qIiwxXSxbMTEsMTAsIihcXG1hdGhiZiByXnstMX0oMVxcdGltZXNcXGV0YV5cXGRhZ2dlcikpXlxcZGFnZ2VyXFxsYW1iZGFeey0xfSIsMCx7InNob3J0ZW4iOnsic291cmNlIjozMCwidGFyZ2V0IjozMH19XSxbMTIsMTEsIlxcbWF0aGJmIGNeey0xfSIsMix7InNob3J0ZW4iOnsic291cmNlIjozMCwidGFyZ2V0IjozMH19XSxbOCw3LCJcXG1hdGhiZiBxXnstMX0iLDEseyJzaG9ydGVuIjp7InNvdXJjZSI6MjAsInRhcmdldCI6MjB9fV1d
        \begin{tikzcd}
          {W\times T^2X} &&&&&&& {W\times TX} \\
          {T(W\times TX)} &&&&&&& {T(W\times X)} \\
          \\
          {T^2(W\times X)} &&&&&&& {T(W\times X)}
          \arrow["{X\times \text{id}^*}"{description}, from=1-1, to=1-8]
          \arrow[""{name=0, anchor=center, inner sep=0}, "{\eta^\dagger}"{description}, from=1-8, to=2-8]
          \arrow[""{name=1, anchor=center, inner sep=0}, "{\eta^\dagger}"{description}, from=1-1, to=2-1]
          \arrow["{\text{id}^*}"{description}, from=4-1, to=4-8]
          \arrow[""{name=2, anchor=center, inner sep=0}, Rightarrow, no head, from=2-8, to=4-8]
          \arrow[""{name=3, anchor=center, inner sep=0}, "{(\lambda^\dagger(1\times\eta\eta^\dagger))^\dagger\lambda^{-1}}"{description}, from=2-1, to=4-8]
          \arrow[""{name=4, anchor=center, inner sep=0}, "{(\eta\eta^\dagger)^*}"{description}, from=2-1, to=4-1]
          \arrow["{(\eta^\dagger)^*}"{description}, from=2-1, to=2-8]
          \arrow["{(\mathbf r^{-1}(1\times\eta^\dagger))^\dagger\lambda^{-1}}", shorten <=37pt, shorten >=37pt, Rightarrow, from=3, to=2]
          \arrow["{\mathbf c^{-1}}"', shorten <=37pt, shorten >=37pt, Rightarrow, from=4, to=3]
          \arrow["{\mathbf q^{-1}}"{description}, shorten <=49pt, shorten >=49pt, Rightarrow, from=1, to=0]
        \end{tikzcd}
      \end{equation}
    \item for all $X,Y\in\bicat{C}$, the invertible 2-cell $\bicell z_{X,Y}=\inv{\bicell r}_\eta$.
  \end{enumerate}
\end{definition}

Beyond the previously mentioned pseudonaturality, it is straightforward to postulate
axioms that ensure the coherence axioms for strong pseudofunctors are satisfied.
This is because none of these coherence axioms contain repeated applications of the
pseudofunctor. Thus the notion generalises from endofunctors to arbitrary pseudofunctors.

