\chapter{Symmetric monoidal categories}

\section{Definition}\label{sec:symmetric_monoidal_categories}

Monoidal categories formalise a general way of combining objects and morphisms between them
in a way that captures the behaviour of many mathematical operations such as categorical
products (cf.~\ref{sec:cartesian_closed_categories_are_symmetric_monoidal}, categorical
coproducts, and the tensor product.

\begin{definition}\label{def:monoidal_category}
  A monoidal category consists of
  \begin{enumerate}
    \item a category $\cat{C}$;
    \item an object $I\in\cat{C}$;
    \item a bifunctor $\otimes:\cat{C}\times\cat{C}\to\cat{C}$;
    \item a natural isomorphism given by arrows
      \begin{align*}
        \alpha:\rr{X\otimes Y}\otimes Z\to X\otimes\rr{Y\otimes Z}
      \end{align*}
      for $X,Y,Z\in\cat{C}$;
    \item a natural isomorphism given by arrows
      \begin{align*}
        \lambda:I\otimes X\to X
      \end{align*}
      for $X\in\cat{C}$;
    \item a natural isomorphism given by arrows
      \begin{align*}
        \rho:X\otimes I\to X
      \end{align*}
      for $X\in\cat{C}$
  \end{enumerate}
  such that, for all $W,X,Y,Z\in\cat{C}$ the following commute:
  \begin{equation}
    \label{eq:monoidal_pentagon}
    % https://q.uiver.app/?q=WzAsNSxbMSwwLCIoV1xcb3RpbWVzIFgpXFxvdGltZXMoWVxcb3RpbWVzIFopIl0sWzAsMSwiKChXXFxvdGltZXMgWClcXG90aW1lcyBZKVxcb3RpbWVzIFoiXSxbMCwzLCIoV1xcb3RpbWVzIChYXFxvdGltZXMgWSkpXFxvdGltZXMgWiJdLFsyLDMsIldcXG90aW1lcygoWFxcb3RpbWVzIFkpXFxvdGltZXMgWikiXSxbMiwxLCJXXFxvdGltZXMgKFhcXG90aW1lcyAoWVxcb3RpbWVzIFopIl0sWzEsMCwiXFxhbHBoYV97V1xcb3RpbWVzIFgsWSxafSJdLFsxLDIsIlxcYWxwaGFfe1csWCxZfVxcb3RpbWVzIDFfWCIsMl0sWzIsMywiXFxhbHBoYV97VyxYXFxvdGltZXMgWSxafSIsMl0sWzMsNCwiMV9XXFxvdGltZXMgXFxhbHBoYV97WCxZLFp9IiwyXSxbMCw0LCJcXGFscGhhX3tXLFgsWVxcb3RpbWVzIFp9Il1d
    \begin{tikzcd}
                & {(W\otimes X)\otimes(Y\otimes Z)} \\
      {((W\otimes X)\otimes Y)\otimes Z} && {W\otimes (X\otimes (Y\otimes Z)} \\
      \\
      {(W\otimes (X\otimes Y))\otimes Z} && {W\otimes((X\otimes Y)\otimes Z)}
      \arrow["{\alpha_{W\otimes X,Y,Z}}", from=2-1, to=1-2]
      \arrow["{\alpha_{W,X,Y}\otimes X}"', from=2-1, to=4-1]
      \arrow["{\alpha_{W,X\otimes Y,Z}}"', from=4-1, to=4-3]
      \arrow["{W\otimes \alpha_{X,Y,Z}}"', from=4-3, to=2-3]
      \arrow["{\alpha_{W,X,Y\otimes Z}}", from=1-2, to=2-3]
    \end{tikzcd}
  \end{equation}
  \begin{equation}
    \label{eq:monoidal_triangle}
    % https://q.uiver.app/?q=WzAsMyxbMCwwLCIoWFxcb3RpbWVzIDEpXFxvdGltZXMgWSJdLFsyLDAsIlhcXG90aW1lcygxXFxvdGltZXMgWSkiXSxbMSwxLCJYXFxvdGltZXMgWSJdLFswLDEsIlxcYWxwaGFfe1gsMSxZfSJdLFsxLDIsIjFfWFxcb3RpbWVzXFxsYW1iZGFfWSJdLFswLDIsIlxccmhvX1hcXG90aW1lcyAxX1kiLDJdXQ==
    \begin{tikzcd}
      {(X\otimes I)\otimes Y} && {X\otimes(I\otimes Y)} \\
                              & {X\otimes Y}
                              \arrow["{\alpha_{X,I,Y}}", from=1-1, to=1-3]
                              \arrow["{X\otimes\lambda_Y}", from=1-3, to=2-2]
                              \arrow["{\rho_X\otimes Y}"', from=1-1, to=2-2]
    \end{tikzcd}
  \end{equation}
\end{definition}

In many cases the monoidal product $\otimes$ is commutative up to a canonical isomorphism.
This is captured by the notion of a symmetric monoidal category.

\begin{definition}
  A symmetric monoidal category consists of
  \begin{enumerate}
    \item a monoidal category $(\cat{C},I,\otimes,\alpha,\lambda,\rho)$;
    \item a natural isomorphism given by the arrows
      \begin{align*}
        \gamma_{X,Y}:X\otimes Y\to Y\otimes X
      \end{align*}
      for $X,Y\in\cat{C}$
  \end{enumerate}
  such that, for all $X,Y,Z\in\cat{C}$, the following commutes:
  \begin{equation}\label{eq:symmetric_monoidal_hexagon}
    % https://q.uiver.app/?q=WzAsNixbMCwwLCIoWFxcb3RpbWVzIFkpXFxvdGltZXMgWiJdLFs2LDAsIihZXFxvdGltZXMgWClcXG90aW1lcyBaIl0sWzAsMiwiWFxcb3RpbWVzKFlcXG90aW1lcyBaKSJdLFswLDQsIihZXFxvdGltZXMgWilcXG90aW1lcyBYIl0sWzYsNCwiWVxcb3RpbWVzKFpcXG90aW1lcyBYKSJdLFs2LDIsIllcXG90aW1lcyhYXFxvdGltZXMgWikiXSxbMCwyLCJcXGFscGhhX3tYXFxvdGltZXMgWSxafSIsMl0sWzIsMywiXFxnYW1tYV97WCxZXFxvdGltZXMgWn0iLDJdLFswLDEsIlxcZ2FtbWFfe1gsWX1cXG90aW1lcyAxX1oiXSxbMSw1LCJcXGFscGhhX3tZXFxvdGltZXMgWCxafSJdLFs1LDQsIjFfWVxcb3RpbWVzIFxcZ2FtbWFfe1gsWn0iXSxbMyw0LCJcXGFscGhhX3tZXFxvdGltZXMgWixYfSIsMl1d
    \begin{tikzcd}
      {(X\otimes Y)\otimes Z} &&&&&& {(Y\otimes X)\otimes Z} \\
      \\
      {X\otimes(Y\otimes Z)} &&&&&& {Y\otimes(X\otimes Z)} \\
      \\
      {(Y\otimes Z)\otimes X} &&&&&& {Y\otimes(Z\otimes X)}
      \arrow["{\alpha_{X\otimes Y,Z}}"', from=1-1, to=3-1]
      \arrow["{\gamma_{X,Y\otimes Z}}"', from=3-1, to=5-1]
      \arrow["{\gamma_{X,Y}\otimes Z}", from=1-1, to=1-7]
      \arrow["{\alpha_{Y\otimes X,Z}}", from=1-7, to=3-7]
      \arrow["{Y\otimes \gamma_{X,Z}}", from=3-7, to=5-7]
      \arrow["{\alpha_{Y\otimes Z,X}}"', from=5-1, to=5-7]
    \end{tikzcd}
  \end{equation}
\end{definition}

All the previously mentioned monoidal categories are in fact symmetric monoidal.
In particular, in any category with products $X\times Y\cong Y\times X$, in any
category with coproducts $X+Y\cong Y+ X$, and for all vector spaces $U,V$,
$U\otimes V\cong V\otimes U$. Moroever, these isomorphisms are natural and coherent
in the sense made precise above.

\section{Cartesian categories are symmetric monoidal}\label{sec:cartesian_closed_categories_are_symmetric_monoidal}

Let $\cat C$ be a category with a choice of terminal object $1\in\cat C$
and, for all $X,Y\in{\cat C}$, a binary product $X\times Y\in\cat C$
with the projection morphisms
\begin{align}
  \label{eq:monoidal_projection1}
  \pi^1_{X,Y} : X \times Y \to X \\
  \label{eq:monoidal_projection2}
  \pi^2_{X,Y} : X \times Y \to Y
\end{align}

\begin{definition}\label{def:monoidal_tensor_product}
  Define a bifunctor structure
  \begin{align*}
    \otimes : \cat C \times \cat C \to \cat C
  \end{align*}
  as follows: for all $X,X',Y,Y'\in{\cat{C}}$, let $X\otimes X'$ be the
  product $X\times X'$ in $\cat C$, and, for each pair of arrows
  $f:X\to Y$, $g:X'\to Y'$, let
  \begin{align*}
    f \otimes g : X \otimes X' \to Y \otimes Y'
  \end{align*}
  be the unique arrow such that both squares in the following commute:
  \begin{equation}\label{eq:monoidal_tensor_product}
    \begin{tikzcd}[row sep=huge, column sep=huge]
      X \arrow{d}{f} &
      X \otimes X'
      \arrow{l}[swap]{\pi^1_{X,X'}}
      \arrow{r}{\pi^2_{X,X'}}
      \arrow{d}{f\otimes g} &
      X' \arrow{d}{g}\\
      Y &
      Y \otimes Y'
      \arrow{l}[swap]{\pi^1_{Y,Y'}}
      \arrow{r}{\pi^2_{Y, Y'}} &
      Y'
    \end{tikzcd}
  \end{equation}
\end{definition}

\begin{lemma}
  $\otimes : \cat C \times \cat C \to \cat C$ is a bifunctor.
  Furthermore, the transformations $\pi^1,\pi^2$ given by the maps
  (\ref{eq:monoidal_projection1}) and (\ref{eq:monoidal_projection2}), respectively,
  are natural.
  \begin{proof}
    Consider arrows
    \begin{align*}
      f : X & \to Y, & f': X' & \to Y', \\
      g : Y & \to Z, & g': Y' & \to Z'.
    \end{align*}
    in $\cat{C}$. We then note that the following commutes:
    \begin{center}
      \begin{tikzcd}[row sep=huge, column sep=huge]
        X \arrow{d}{f} &
        X \otimes X'
        \arrow{l}[swap]{\pi^1_{X, X'}}
        \arrow{r}{\pi^2_{X, X'}}
        \arrow{d}{f\otimes f'} &
        X' \arrow{d}{f'}\\
        Y \arrow{d}{g}&
        Y \otimes Y'
        \arrow{l}[swap]{\pi^1_{Y, Y'}}
        \arrow{r}{\pi^2_{Y, Y'}}
        \arrow{d}{g\otimes g'} &
        Y' \arrow{d}{g'} \\
        Z &
        Z \otimes Z'
        \arrow{l}[swap]{\pi^1_{Z, Z'}}
        \arrow{r}{\pi^2_{Z, Z}} &
        Z'
      \end{tikzcd}
    \end{center}
    In particular, $(g\otimes g')\circ(f\otimes f')$ makes the outer diagram
    commute. By~\ref{eq:monoidal_tensor_product}, we then have
    \begin{align*}
      (g\otimes g')\circ(f\otimes f') = (g\circ f)\otimes(g'\circ f').
    \end{align*}
    Finally, the following commutes trivially:
    \begin{equation*}
      \begin{tikzcd}[row sep=huge, column sep=huge]
        X
        \arrow{d}{X} &
        X \otimes X'
        \arrow{l}[swap]{\pi^1_{X,X'}}
        \arrow{d}{{X\otimes X'}}
        \arrow{r}{\pi^2_{X,X'}} &
        X'
        \arrow{d}{{X'}} \\
        X &
        X \otimes X'
        \arrow{l}[swap]{\pi^1_{X,X'}}
        \arrow{r}{\pi^2_{X,X'}} &
        X
      \end{tikzcd}
    \end{equation*}
    I.e. $1_X \otimes 1_{X'} = 1_{X\otimes X'}$.
    Thus $\otimes:\cat C\times\cat C\to\cat C$ is a bifunctor.
  \end{proof}
\end{lemma}

\begin{definition}[Associator]\label{def:monoidal_associator}
  For all $X,Y,Z\in\cat C$, define
  \begin{align*}
    \alpha_{X,Y,Z} : (X \otimes Y) \otimes Z \to X \otimes (Y \otimes Z)
  \end{align*}
  to be the unique morphism that makes the following commute:
  \begin{equation}\label{eq:monoidal_a}
    \begin{tikzcd}[row sep=huge, column sep=huge]
      X\otimes Y \arrow{d}{\pi^1_{X,Y}} &
      \rr{X\otimes Y}\otimes Z
      \arrow{l}[swap]{\pi^1_{X\otimes Y,Z}}
      \arrow{d}{\alpha_{X,Y,Z}}
      \arrow{dr}{\pi^2_{X,Y}\otimes Z} \\
      X &
      X\otimes\rr{Y\otimes Z}
      \arrow{l}[swap]{\pi^1_{X,Y\otimes Z}}
      \arrow{r}{\pi^2_{X,Y\otimes Z}} &
      Y\otimes Z
    \end{tikzcd}
  \end{equation}
\end{definition}

\begin{lemma}
  The transformation given by the maps
  $\alpha_{X,Y,Z}:(X\otimes Y)\otimes Z \to X\otimes(Y\otimes Z)$
  forms a natural isomorphism in all three arguments.
  \begin{proof}
    For all $X,Y,Z\in\cat C$, define
    \begin{align*}
      \inv \alpha_{X,Y,Z}:X\otimes\rr{Y\otimes Z}\to\rr{X\otimes Y}\otimes Z
    \end{align*}
    to be the unique morphism that makes the following commute:
    \begin{equation}\label{eq:monoidal_inv_a}
      \begin{tikzcd}[row sep=huge, column sep=huge]
                &
                X \otimes\rr{Y\otimes Z}
                \arrow{dl}[swap]{X\otimes\pi^1_{Y,Z}}
                \arrow{d}{\inv \alpha_{X,Y,Z}}
        \arrow{r}{\pi^2_{X,Y\otimes Z}} &
        Y\otimes Z
        \arrow{d}{\pi^2_{Y,Z}}\\
        X\otimes Y &
        \rr{X\otimes Y}\otimes Z
        \arrow{l}[swap]{\pi^1_{X\otimes Y,Z}}
        \arrow{r}{\pi^2_{X\otimes Y,Z}} &
        Z
      \end{tikzcd}
    \end{equation}
    We intend to show
    \begin{equation}\label{eq:monoidal_a_iso}
      \begin{tikzcd}
                &&&& {(X\otimes Y)\otimes Z} \\
                \\
                {X \otimes (Y \otimes Z)} \\
                &&&&&& {} \\
                &&&& {(X\otimes Y)\otimes Z}
                \arrow["{\alpha_{X,Y,Z}}"', bend right=30, from=1-5, to=3-1]
                \arrow["{\inv\alpha_{X,Y,Z}}"', bend right=30, from=3-1, to=5-5]
                \arrow["{{(X\otimes Y)\otimes Z}}", bend left=30, from=1-5, to=5-5]
      \end{tikzcd}
    \end{equation}
    Firstly, this diagram commutes:
    \begin{equation}\label{eq:monoidal_a_iso_pi1_pi1}
      % https://q.uiver.app/?q=WzAsNyxbMCwxXSxbMCwyLCJYXFxvdGltZXMoWVxcb3RpbWVzIFopIl0sWzQsMCwiKFhcXG90aW1lcyBZKVxcb3RpbWVzIFoiXSxbNCw0LCIoWFxcb3RpbWVzIFkpXFxvdGltZXMgWiJdLFsyLDIsIlgiXSxbNSwyXSxbNCwyLCJYXFxvdGltZXMgWSJdLFsyLDMsIjFfeyhYXFxvdGltZXMgWSlcXG90aW1lcyBafSIsMCx7ImN1cnZlIjotNX1dLFsyLDEsIlxcYWxwaGFfe1gsWSxafSIsMix7ImN1cnZlIjo1fV0sWzEsMywiXFxhbHBoYV57LTF9X3tYLFksWn0iLDIseyJjdXJ2ZSI6NX1dLFsyLDYsIlxccGleMV97WFxcb3RpbWVzIFksWn0iLDJdLFsxLDQsIlxccGleMV97WCxZXFxvdGltZXMgWn0iLDJdLFs2LDQsIlxccGleMV97WCxZfSJdLFsxLDYsIjFfWFxcb3RpbWVzXFxwaV4xX3tZLFp9IiwwLHsiY3VydmUiOi0zfV0sWzMsNiwiXFxwaV4xX3tYXFxvdGltZXMgWSxafSJdLFs0LDEzLCJcXGV4cGxhaW57ZXE6bW9ub2lkYWwtdGVuc29yX3Byb2R1Y3R9IiwxLHsic2hvcnRlbiI6eyJ0YXJnZXQiOjIwfSwic3R5bGUiOnsiYm9keSI6eyJuYW1lIjoibm9uZSJ9LCJoZWFkIjp7Im5hbWUiOiJub25lIn19fV0sWzYsOSwiXFxleHBsYWlue2VxOm1vbm9pZGFsLWludl9hfSIsMSx7InNob3J0ZW4iOnsidGFyZ2V0IjoyMH0sInN0eWxlIjp7ImJvZHkiOnsibmFtZSI6Im5vbmUifSwiaGVhZCI6eyJuYW1lIjoibm9uZSJ9fX1dLFsyLDEzLCJcXGV4cGxhaW57ZXE6bW9ub2lkYWwtYX0iLDEseyJzaG9ydGVuIjp7InRhcmdldCI6MjB9LCJzdHlsZSI6eyJib2R5Ijp7Im5hbWUiOiJub25lIn0sImhlYWQiOnsibmFtZSI6Im5vbmUifX19XV0=
      \begin{tikzcd}
                &&&& {(X\otimes Y)\otimes Z} \\
                {} \\
        {X\otimes(Y\otimes Z)} && X && {X\otimes Y} & {} \\
        \\
                               &&&& {(X\otimes Y)\otimes Z}
                               \arrow["{{(X\otimes Y)\otimes Z}}", bend left=30, from=1-5, to=5-5]
                               \arrow["{\alpha_{X,Y,Z}}"', bend right=30, from=1-5, to=3-1]
                               \arrow[""{name=0, anchor=center, inner sep=0}, "{\alpha^{-1}_{X,Y,Z}}"', bend right=30, from=3-1, to=5-5]
                               \arrow["{\pi^1_{X\otimes Y,Z}}"', from=1-5, to=3-5]
                               \arrow["{\pi^1_{X,Y\otimes Z}}"', from=3-1, to=3-3]
                               \arrow["{\pi^1_{X,Y}}", from=3-5, to=3-3]
                               \arrow[""{name=1, anchor=center, inner sep=0}, "{X\otimes\pi^1_{Y,Z}}", bend left=18, from=3-1, to=3-5]
                               \arrow["{\pi^1_{X\otimes Y,Z}}", from=5-5, to=3-5]
                               \arrow["{\explain{eq:monoidal_tensor_product}}"{description}, Rightarrow, draw=none, from=3-3, to=1]
                               \arrow["{\explain{eq:monoidal_inv_a}}"{description}, Rightarrow, draw=none, from=3-5, to=0]
                               \arrow["{\explain{eq:monoidal_a}}"{description}, Rightarrow, draw=none, from=1-5, to=1]
      \end{tikzcd}
    \end{equation}
    Secondly, we have this commutative diagram:
    \begin{equation}\label{eq:monoidal_a_iso_pi1_pi2}
      % https://q.uiver.app/?q=WzAsOSxbMCwxXSxbMCwyLCJYXFxvdGltZXMoWVxcb3RpbWVzIFopIl0sWzQsMCwiKFhcXG90aW1lcyBZKVxcb3RpbWVzIFoiXSxbNCw0LCIoWFxcb3RpbWVzIFkpXFxvdGltZXMgWiJdLFszLDMsIlhcXG90aW1lcyBZIl0sWzMsMiwiWSJdLFsyLDEsIllcXG90aW1lcyBaIl0sWzIsMl0sWzUsMl0sWzIsMywiMV97KFhcXG90aW1lcyBZKVxcb3RpbWVzIFp9IiwwLHsiY3VydmUiOi01fV0sWzIsMSwiXFxhbHBoYV97WCxZLFp9IiwyLHsiY3VydmUiOjV9XSxbMSwzLCJcXGFscGhhXnstMX1fe1gsWSxafSIsMix7ImN1cnZlIjo1fV0sWzIsNCwiXFxwaV4xX3tYXFxvdGltZXMgWSxafSJdLFszLDQsIlxccGleMV97WFxcb3RpbWVzIFksWn0iXSxbMSw2LCJcXHBpXjJfe1gsWVxcb3RpbWVzIFp9Il0sWzYsNSwiXFxwaV4xX3tZLFp9Il0sWzQsNSwiXFxwaV4yX3tYLFl9Il0sWzEsNCwiMV9YXFxvdGltZXMgXFxwaV4xX3tZLFp9Il0sWzIsNiwiXFxwaV4yX3tYLFl9XFxvdGltZXMgMV9aIl0sWzYsMTcsIlxcZXhwbGFpbntlcTptb25vaWRhbC10ZW5zb3JfcHJvZHVjdH0iLDEseyJzaG9ydGVuIjp7InRhcmdldCI6MjB9LCJzdHlsZSI6eyJib2R5Ijp7Im5hbWUiOiJub25lIn0sImhlYWQiOnsibmFtZSI6Im5vbmUifX19XSxbMTIsNiwiXFxleHBsYWlue2VxOm1vbm9pZGFsLXRlbnNvcl9wcm9kdWN0fSIsMSx7InNob3J0ZW4iOnsic291cmNlIjoyMH0sInN0eWxlIjp7ImJvZHkiOnsibmFtZSI6Im5vbmUifSwiaGVhZCI6eyJuYW1lIjoibm9uZSJ9fX1dLFs0LDksIiIsMSx7InNob3J0ZW4iOnsic291cmNlIjo0MCwidGFyZ2V0Ijo0MH0sInN0eWxlIjp7ImJvZHkiOnsibmFtZSI6Im5vbmUifSwiaGVhZCI6eyJuYW1lIjoibm9uZSJ9fX1dLFs0LDExLCJcXGV4cGxhaW57ZXE6bW9ub2lkYWwtaW52X2F9IiwxLHsic2hvcnRlbiI6eyJ0YXJnZXQiOjIwfSwic3R5bGUiOnsiYm9keSI6eyJuYW1lIjoibm9uZSJ9LCJoZWFkIjp7Im5hbWUiOiJub25lIn19fV0sWzYsMTAsIlxcZXhwbGFpbntlcTptb25vaWRhbC1hfSIsMSx7InNob3J0ZW4iOnsidGFyZ2V0IjoyMH0sInN0eWxlIjp7ImJvZHkiOnsibmFtZSI6Im5vbmUifSwiaGVhZCI6eyJuYW1lIjoibm9uZSJ9fX1dXQ==
      \begin{tikzcd}
                &&&& {(X\otimes Y)\otimes Z} \\
        {} && {Y\otimes Z} \\
        {X\otimes(Y\otimes Z)} && {} & Y && {} \\
                               &&& {X\otimes Y} \\
                               &&&& {(X\otimes Y)\otimes Z}
                               \arrow[""{name=0, anchor=center, inner sep=0}, "{{(X\otimes Y)\otimes Z}}", bend left=30, from=1-5, to=5-5]
                               \arrow[""{name=1, anchor=center, inner sep=0}, "{\alpha_{X,Y,Z}}"', bend right=30, from=1-5, to=3-1]
                               \arrow[""{name=2, anchor=center, inner sep=0}, "{\alpha^{-1}_{X,Y,Z}}"', bend right=30, from=3-1, to=5-5]
                               \arrow[""{name=3, anchor=center, inner sep=0}, "{\pi^1_{X\otimes Y,Z}}", from=1-5, to=4-4]
                               \arrow["{\pi^1_{X\otimes Y,Z}}", from=5-5, to=4-4]
                               \arrow["{\pi^2_{X,Y\otimes Z}}", from=3-1, to=2-3]
                               \arrow["{\pi^1_{Y,Z}}", from=2-3, to=3-4]
                               \arrow["{\pi^2_{X,Y}}", from=4-4, to=3-4]
                               \arrow[""{name=4, anchor=center, inner sep=0}, "{X\otimes \pi^1_{Y,Z}}", from=3-1, to=4-4]
                               \arrow["{\pi^2_{X,Y}\otimes Z}", from=1-5, to=2-3]
                               \arrow["{\explain{eq:monoidal_tensor_product}}"{description}, Rightarrow, draw=none, from=2-3, to=4]
                               \arrow["{\explain{eq:monoidal_tensor_product}}"{description}, Rightarrow, draw=none, from=3, to=2-3]
                               \arrow[Rightarrow, draw=none, from=4-4, to=0]
                               \arrow["{\explain{eq:monoidal_inv_a}}"{description}, Rightarrow, draw=none, from=4-4, to=2]
                               \arrow["{\explain{eq:monoidal_a}}"{description}, Rightarrow, draw=none, from=2-3, to=1]
      \end{tikzcd}
    \end{equation}
    Thus we conclude that the following commutes:
    \begin{equation}
      \label{eq:monoidal_a_iso_pi1}
      % https://q.uiver.app/?q=WzAsNSxbMCwxXSxbMCwyLCJYXFxvdGltZXMoWVxcb3RpbWVzIFopIl0sWzQsMCwiKFhcXG90aW1lcyBZKVxcb3RpbWVzIFoiXSxbNCw0LCIoWFxcb3RpbWVzIFkpXFxvdGltZXMgWiJdLFszLDIsIlhcXG90aW1lcyBZIl0sWzIsMywiMV97KFhcXG90aW1lcyBZKVxcb3RpbWVzIFp9IiwwLHsiY3VydmUiOi01fV0sWzIsMSwiXFxhbHBoYV97WCxZLFp9IiwyLHsiY3VydmUiOjV9XSxbMSwzLCJcXGFscGhhXnstMX1fe1gsWSxafSIsMix7ImN1cnZlIjo1fV0sWzEsNCwiMV9YXFxvdGltZXMgXFxwaV4xX3tZLFp9Il0sWzIsNCwiXFxwaV4xX3tYXFxvdGltZXMgWSxafSJdLFszLDQsIlxccGleMV97WFxcb3RpbWVzIFksWn0iXV0=
      \begin{tikzcd}
                &&&& {(X\otimes Y)\otimes Z} \\
                {} \\
        {X\otimes(Y\otimes Z)} &&& {X\otimes Y} \\
        \\
                               &&&& {(X\otimes Y)\otimes Z}
                               \arrow["{{(X\otimes Y)\otimes Z}}", bend left=30, from=1-5, to=5-5]
                               \arrow["{\alpha_{X,Y,Z}}"', bend right=30, from=1-5, to=3-1]
                               \arrow["{\alpha^{-1}_{X,Y,Z}}"', bend right=30, from=3-1, to=5-5]
                               \arrow["{\pi^1_{X\otimes Y,Z}}", from=1-5, to=3-4]
                               \arrow["{\pi^1_{X\otimes Y,Z}}", from=5-5, to=3-4]
      \end{tikzcd}
    \end{equation}
    Similarly, we have the diagram:
    \begin{equation}
      \label{eq:monoidal_a_iso_pi2}
      \begin{tikzcd}
                &&&& {(X\otimes Y)\otimes Z} \\
                \\
        {X \otimes (Y \otimes Z)} && {Y \otimes Z} & Z \\
                                  &&&&&& {} \\
                                  &&&& {(X\otimes Y)\otimes Z}
                                  \arrow["{\alpha_{X,Y,Z}}"', bend right=30, from=1-5, to=3-1]
                                  \arrow["{\inv\alpha_{X,Y,Z}}"', bend right=30, from=3-1, to=5-5]
                                  \arrow["{{(X\otimes Y)\otimes Z}}", bend left=30, from=1-5, to=5-5]
                                  \arrow["{\pi^2_{X,Y\otimes Z}}"', from=3-1, to=3-3]
                                  \arrow["{\pi^2_{Y,Z}}"', from=3-3, to=3-4]
                                  \arrow["{\pi^2_{X\otimes Y,Z}}", from=1-5, to=3-4]
                                  \arrow["{\pi^2_{X,Y}\otimes Z}"', bend right=12, from=1-5, to=3-3]
                                  \arrow["{\pi^2_{X\otimes Y,Z}}"', from=5-5, to=3-4]
      \end{tikzcd}
    \end{equation}
    Combining (\ref{eq:monoidal_a_iso_pi1}) and (\ref{eq:monoidal_a_iso_pi2}),
    (\ref{eq:monoidal_a_iso}) commutes. We omit the proof that
    $\alpha_{X,Y,Z}\circ\inv\alpha_{X,Y,Z}=1_{(X\otimes Y)\otimes Z}$
    and conclude that $\alpha_{X,Y,Z}$ is an isomorphism.

    To show naturality, we require that the following commutes:
    \begin{equation}
      \label{eq:monoidal_a-naturality}
      \begin{tikzcd}[row sep=huge, column sep=huge]
        \rr{X\otimes Y}\otimes Z
        \arrow{r}{\alpha_{X,Y,Z}}
        \arrow{d}{\rr{f\otimes g}\otimes h}&
        X \otimes \rr{Y\otimes Z}
        \arrow{d}{f \otimes \rr{g \otimes h}} \\
        \rr{X'\otimes Y'}\otimes Z'
        \arrow{r}{\alpha_{X',Y',Z'}} &
        X' \otimes \rr{Y' \otimes Z'}
      \end{tikzcd}
    \end{equation}

    Firstly, we consider the following diagram:
    \begin{equation*}
      \begin{tikzcd}
        {(X\otimes Y)\otimes Z} &&&& {(X'\otimes Y')\otimes Z'} \\
                                & {X\otimes Y} && {X'\otimes Y'} \\
                                \\
                                & X && {X'} \\
        {X\otimes(Y\otimes Z)} &&&& {X'\otimes(Y'\otimes Z')}
        \arrow["{\alpha_{X,Y,Z}}"', from=1-1, to=5-1]
        \arrow["{\alpha_{X',Y',Z'}}", from=1-5, to=5-5]
        \arrow["{f\otimes(g\otimes h)}"', from=5-1, to=5-5]
        \arrow["{(f\otimes g)\otimes h}", from=1-1, to=1-5]
        \arrow["{\pi^1_{X,Y}}"', from=2-2, to=4-2]
        \arrow["f"', from=4-2, to=4-4]
        \arrow["{\pi^1_{X',Y'}}", from=2-4, to=4-4]
        \arrow["{\pi^1_{X'\otimes Y',Z'}}", from=1-5, to=2-4]
        \arrow["{\pi^1_{X\otimes Y,Z}}"', from=1-1, to=2-2]
        \arrow["{\pi^1_{X,Y\otimes Z}}", from=5-1, to=4-2]
        \arrow["{\pi^1_{X',Y'\otimes Z'}}"', from=5-5, to=4-4]
        \arrow["{f\otimes g}", from=2-2, to=2-4]
      \end{tikzcd}
    \end{equation*}
    Here the left and right rectangles commute by~\ref{def:monoidal_associator}.
    Then remaining three rectangles then commute due to naturality of
    $\pi^1$.
    Secondly, we consider
    \begin{equation*}
      \begin{tikzcd}
        {(X\otimes Y)\otimes Z} &&&& {(X'\otimes Y')\otimes Z'} \\
        \\
                                & {Y\otimes Z} && {Y'\otimes Z'} \\
                                \\
        {X\otimes(Y\otimes Z)} &&&& {X'\otimes(Y'\otimes Z')}
        \arrow["{\alpha_{X,Y,Z}}"', from=1-1, to=5-1]
        \arrow["{\alpha_{X',Y',Z'}}", from=1-5, to=5-5]
        \arrow["{f\otimes(g\otimes h)}"', from=5-1, to=5-5]
        \arrow["{(f\otimes g)\otimes h}", from=1-1, to=1-5]
        \arrow["{\pi^2_{X,Y\otimes Z}}"', from=5-1, to=3-2]
        \arrow["{g\otimes h}"', from=3-2, to=3-4]
        \arrow["{\pi^2_{X',Y'\otimes Z'}}", from=5-5, to=3-4]
        \arrow["{\pi^2_{X,Y}\otimes Z}", from=1-1, to=3-2]
        \arrow["{\pi^2_{X',Y'}\otimes {Z'}}"', from=1-5, to=3-4]
      \end{tikzcd}
    \end{equation*}
    Here the two triangles are just the triangles in (\ref{eq:monoidal_a}).
    The rectangles commute due to functoriality of $\otimes$ and naturality of
    $\pi^2$.

    It follows that (\ref{eq:monoidal_a-naturality}) commutes, i.e. $\alpha_{X,Y,Z}$ is natural
    in all three arguments.
  \end{proof}
\end{lemma}

\begin{definition}[Unitors]
  For each $X\in\cat C$, define maps
  \begin{align*}
    \lambda_X = \pi^2_{1,X} : 1 \otimes X & \to X \\
    \rho_X = \pi^1_{X,1}: X \otimes 1     & \to X
  \end{align*}
\end{definition}

\begin{lemma}\label{eq:monoidal_lambda}
  The transformation $\lambda$  given by the maps $\lambda_X:1\otimes X\to X$ is a natural isomorphism.
  \begin{proof}
    For each $X\in{\cat C}$, define $\inv\lambda_X:X\to 1\otimes X$
    to be the unique arrow that makes the following commute:
    \begin{center}
      \begin{tikzcd}[row sep=huge, column sep=huge]
                &
                X
                \arrow{dl}{}
                \arrow{d}{\inv\lambda_X}
                \arrow{dr}{X}
                &
                \\
        1 &
        1 \otimes X
        \arrow{l}{\pi^1_{1,X}}
        \arrow{r}[swap]{\pi^2_{1,X}}&
        X
      \end{tikzcd}
    \end{center}
    We note that $1\in{\cat C}$ is terminal so $\pi^1_{1,X}$
    is the unique arrow $1\otimes X\to 1$. Further, $\pi^2_{1,X}=\lambda_X$
    so $\lambda_X\circ\inv\lambda_X = 1_X$. Similarly, it can be shown that
    $\inv\lambda_X\circ\lambda_X = 1_{1\otimes X}$.
    Thus $\lambda_X$ is an isomorphism.

    Noting $\lambda_X=\pi^2_{1,X}$ and $\lambda_Y=\pi^2_{1,X}$ it follows
    from \ref{eq:monoidal_tensor_product} that the following commutes:
    \begin{center}
      \begin{tikzcd}[row sep=huge, column sep=huge]
        1\otimes X
        \arrow{d}{1\otimes f}
        \arrow{r}{\lambda_X} &
        X
        \arrow{d}{f}\\
        1\otimes Y
        \arrow{r}{\lambda_Y} &
        Y
      \end{tikzcd}
    \end{center}
    Thus $\lambda$ is a natural isomorphism.
  \end{proof}
\end{lemma}

\begin{lemma}\label{eq:monoidal_rho}
  The transformation $\rho$ given by the maps $\rho:X\otimes 1\to X$ is a natural transformation.
  \begin{proof}
    Analogous to~\ref{eq:monoidal_lambda}.
  \end{proof}
\end{lemma}

\begin{lemma}\label{eq:monoidal}
  $(\cat C, \otimes, 1, \alpha, \lambda, \rho)$ forms a monoidal category.
  \begin{proof}
    Let $W,X,Y,Z\in\cat C$. We intend to show \ref{eq:monoidal_triangle}.
    Firstly, we consider the following diagram:
    \begin{equation}
      \label{eq:monoidal_triangle_pi1}
      % https://q.uiver.app/?q=WzAsNSxbMCwwLCIoWFxcb3RpbWVzIDEpXFxvdGltZXMgWSJdLFs2LDAsIlhcXG90aW1lcygxXFxvdGltZXMgWSkiXSxbMywzLCJYXFxvdGltZXMgWSJdLFs0LDEsIlgiXSxbMiwxLCJYXFxvdGltZXMgMSJdLFswLDEsIlxcYWxwaGFfe1gsMSxZfSJdLFsxLDIsIjFfWFxcb3RpbWVzXFxsYW1iZGFfWSIsMCx7ImN1cnZlIjotNX1dLFswLDIsIlxccmhvX1hcXG90aW1lcyAxX1kiLDIseyJjdXJ2ZSI6NX1dLFsyLDMsIlxccGleMV97WCxZfSJdLFsxLDMsIlxccGleMV97WCwxXFxvdGltZXMgWX0iXSxbMCw0LCJcXHBpXjFfe1hcXG90aW1lcyAxLCBZfSJdLFs0LDMsIlxccGleMV97WCwxfSJdXQ==
      \begin{tikzcd}
        {(X\otimes 1)\otimes Y} &&&&&& {X\otimes(1\otimes Y)} \\
                                && {X\otimes 1} && X \\
                                \\
                                &&& {X\otimes Y}
                                \arrow["{\alpha_{X,1,Y}}", from=1-1, to=1-7]
                                \arrow["{X\otimes\lambda_Y}", bend left=30, from=1-7, to=4-4]
                                \arrow["{\rho_X\otimes Y}"', bend right=30, from=1-1, to=4-4]
                                \arrow["{\pi^1_{X,Y}}", from=4-4, to=2-5]
                                \arrow["{\pi^1_{X,1\otimes Y}}", from=1-7, to=2-5]
                                \arrow["{\pi^1_{X\otimes 1, Y}}", from=1-1, to=2-3]
                                \arrow["{\pi^1_{X,1}}", from=2-3, to=2-5]
      \end{tikzcd}
    \end{equation}
    Here the top rectangle is just the rectangle in (\ref{eq:monoidal_a}) and the
    remaining commute due to naturality of $\pi^1$, noting the definition of $\rho_X$.
    Secondly, we have:
    \begin{equation}
      \label{eq:monoidal_triangle_pi2}
      % https://q.uiver.app/?q=WzAsNSxbMCwwLCIoWFxcb3RpbWVzIDEpXFxvdGltZXMgWSJdLFs2LDAsIlhcXG90aW1lcygxXFxvdGltZXMgWSkiXSxbMywzLCJYXFxvdGltZXMgWSJdLFsyLDIsIlkiXSxbNSwxLCIxXFxvdGltZXMgWSJdLFswLDEsIlxcYWxwaGFfe1gsMSxZfSJdLFsxLDIsIjFfWFxcb3RpbWVzXFxsYW1iZGFfWSIsMCx7ImN1cnZlIjotNX1dLFswLDIsIlxccmhvX1hcXG90aW1lcyAxX1kiLDIseyJjdXJ2ZSI6NX1dLFsxLDQsIlxccGleMl97WCwxXFxvdGltZXMgWX0iXSxbMiwzLCJcXHBpXjJfe1gsWX0iXSxbMCwzLCJcXHBpXjJfe1hcXG90aW1lcyAxLFl9Il0sWzQsMywiXFxwaV4yX3sxLFl9Il0sWzAsNCwiXFxwaV4yX3tYLDF9XFxvdGltZXMgMV9ZIl1d
      \begin{tikzcd}
        {(X\otimes 1)\otimes Y} &&&&&& {X\otimes(1\otimes Y)} \\
                                &&&&& {1\otimes Y} \\
                                && Y \\
                                &&& {X\otimes Y}
                                \arrow["{\alpha_{X,1,Y}}", from=1-1, to=1-7]
                                \arrow["{X\otimes\lambda_Y}", bend left=30, from=1-7, to=4-4]
                                \arrow["{\rho_X\otimes Y}"', bend right=30, from=1-1, to=4-4]
                                \arrow["{\pi^2_{X,1\otimes Y}}", from=1-7, to=2-6]
                                \arrow["{\pi^2_{X,Y}}", from=4-4, to=3-3]
                                \arrow["{\pi^2_{X\otimes 1,Y}}", from=1-1, to=3-3]
                                \arrow["{\pi^2_{1,Y}}", from=2-6, to=3-3]
                                \arrow["{\pi^2_{X,1}\otimes Y}", from=1-1, to=2-6]
      \end{tikzcd}
    \end{equation}
    Her the top top triangle is just the triangle in (\ref{eq:monoidal_a}) and the remaining
    commute due to naturality of $\pi^2$, once again noting the definition of $\lambda_Y$.
    Combining (\ref{eq:monoidal_triangle_pi1}) and (\ref{eq:monoidal_triangle_pi2}) we find
    that (\ref{eq:monoidal_triangle}) holds.

    We intend to show \ref{eq:monoidal_pentagon}.
    Firstly, we consider
    \begin{equation}
      \label{eq:monoidal_pentagon_pi1}
      % https://q.uiver.app/?q=WzAsOSxbMSwwLCIoV1xcb3RpbWVzIFgpXFxvdGltZXMoWVxcb3RpbWVzIFopIl0sWzAsMSwiKChXXFxvdGltZXMgWClcXG90aW1lcyBZKVxcb3RpbWVzIFoiXSxbMCw0LCIoV1xcb3RpbWVzIChYXFxvdGltZXMgWSkpXFxvdGltZXMgWiJdLFszLDQsIldcXG90aW1lcygoWFxcb3RpbWVzIFkpXFxvdGltZXMgWikiXSxbMywxLCJXXFxvdGltZXMgKFhcXG90aW1lcyAoWVxcb3RpbWVzIFopIl0sWzEsMiwiKFdcXG90aW1lcyBYKVxcb3RpbWVzIFkiXSxbMSwzLCJXXFxvdGltZXMoWFxcb3RpbWVzIFkpIl0sWzIsMywiVyJdLFsyLDIsIldcXG90aW1lcyBYIl0sWzEsMCwiXFxhbHBoYV97V1xcb3RpbWVzIFgsWSxafSJdLFsxLDIsIlxcYWxwaGFfe1csWCxZfVxcb3RpbWVzIFgiLDJdLFsyLDMsIlxcYWxwaGFfe1csWFxcb3RpbWVzIFksWn0iLDJdLFszLDQsIldcXG90aW1lcyBcXGFscGhhX3tYLFksWn0iLDJdLFswLDQsIlxcYWxwaGFfe1csWCxZXFxvdGltZXMgWn0iXSxbNSw4LCJcXHBpXjFfe1dcXG90aW1lcyBYLFl9Il0sWzYsNywiXFxwaV4xX3tXLFhcXG90aW1lcyBZfSJdLFs4LDcsIlxccGleMV97VyxYfSIsMl0sWzUsNiwiXFxhbHBoYV97VyxYLFl9IiwyXSxbMSw1LCJcXHBpXjFfeyhXXFxvdGltZXMgWClcXG90aW1lcyBZLFp9IiwyXSxbMiw2LCJcXHBpXjFfe1dcXG90aW1lcyhYXFxvdGltZXMgWSksWn0iXSxbMCw4LCJcXHBpXjFfe1dcXG90aW1lcyBYLFlcXG90aW1lcyBafSIsMl0sWzQsNywiXFxwaV4xX3tXLFhcXG90aW1lcyhZXFxvdGltZXMgWil9IiwyXSxbMyw3LCJcXHBpXjFfe1csKFhcXG90aW1lcyBZKVxcb3RpbWVzIFp9IiwyXV0=
      \begin{tikzcd}
  & {(W\otimes X)\otimes(Y\otimes Z)} \\
        {((W\otimes X)\otimes Y)\otimes Z} &&& {W\otimes (X\otimes (Y\otimes Z)} \\
                                           & {(W\otimes X)\otimes Y} & {W\otimes X} \\
                                           & {W\otimes(X\otimes Y)} & W \\
        {(W\otimes (X\otimes Y))\otimes Z} &&& {W\otimes((X\otimes Y)\otimes Z)}
        \arrow["{\alpha_{W\otimes X,Y,Z}}", from=2-1, to=1-2]
        \arrow["{\alpha_{W,X,Y}\otimes X}"', from=2-1, to=5-1]
        \arrow["{\alpha_{W,X\otimes Y,Z}}"', from=5-1, to=5-4]
        \arrow["{W\otimes \alpha_{X,Y,Z}}"', from=5-4, to=2-4]
        \arrow["{\alpha_{W,X,Y\otimes Z}}", from=1-2, to=2-4]
        \arrow["{\pi^1_{W\otimes X,Y}}", from=3-2, to=3-3]
        \arrow["{\pi^1_{W,X\otimes Y}}", from=4-2, to=4-3]
        \arrow["{\pi^1_{W,X}}"', from=3-3, to=4-3]
        \arrow["{\alpha_{W,X,Y}}"', from=3-2, to=4-2]
        \arrow["{\pi^1_{(W\otimes X)\otimes Y,Z}}"', from=2-1, to=3-2]
        \arrow["{\pi^1_{W\otimes(X\otimes Y),Z}}", from=5-1, to=4-2]
        \arrow["{\pi^1_{W\otimes X,Y\otimes Z}}"', from=1-2, to=3-3]
        \arrow["{\pi^1_{W,X\otimes(Y\otimes Z)}}"', from=2-4, to=4-3]
        \arrow["{\pi^1_{W,(X\otimes Y)\otimes Z}}"', from=5-4, to=4-3]
      \end{tikzcd}
    \end{equation}
    Here the rectangle on the left and the triangle on the right commute due to
    naturality of $\pi^1$. All the remaining rectangles are just the rectangle in
    (\ref{eq:monoidal_a}). Thus (\ref{eq:monoidal_pentagon_pi1}) commutes, i.e.
    \begin{align*}
      \pi^1_{W,X\otimes\rr{Y\otimes Z}} \circ W\otimes \alpha_{X,Y,Z} \circ \alpha_{W,X\otimes Y,Z} \circ \alpha_{W,X,Y}\otimes Z
      = \pi^1_{W,X\otimes\rr{Y\otimes Z}} \circ \alpha_{W,X,Y\otimes Z} \circ \alpha_{W\otimes X,Y,Z}
    \end{align*}
    Similarly, we consider
    \begin{equation}
      \label{eq:monoidal_pentagon_pi2}
      % https://q.uiver.app/?q=WzAsNyxbMiwwLCIoV1xcb3RpbWVzIFgpXFxvdGltZXMoWVxcb3RpbWVzIFopIl0sWzAsMSwiKChXXFxvdGltZXMgWClcXG90aW1lcyBZKVxcb3RpbWVzIFoiXSxbMCw0LCIoV1xcb3RpbWVzIChYXFxvdGltZXMgWSkpXFxvdGltZXMgWiJdLFs0LDQsIldcXG90aW1lcygoWFxcb3RpbWVzIFkpXFxvdGltZXMgWikiXSxbNCwxLCJXXFxvdGltZXMgKFhcXG90aW1lcyAoWVxcb3RpbWVzIFopIl0sWzMsMiwiWFxcb3RpbWVzKFlcXG90aW1lcyBaKSJdLFsyLDMsIihYXFxvdGltZXMgWSlcXG90aW1lcyBaIl0sWzEsMCwiXFxhbHBoYV97V1xcb3RpbWVzIFgsWSxafSJdLFsxLDIsIlxcYWxwaGFfe1csWCxZfVxcb3RpbWVzIDFfWCIsMl0sWzIsMywiXFxhbHBoYV97VyxYXFxvdGltZXMgWSxafSIsMl0sWzMsNCwiMV9XXFxvdGltZXMgXFxhbHBoYV97WCxZLFp9IiwyXSxbMCw0LCJcXGFscGhhX3tXLFgsWVxcb3RpbWVzIFp9Il0sWzIsNiwiXFxwaV4yX3tXLFhcXG90aW1lcyBZfVxcb3RpbWVzIDFfWiJdLFszLDYsIlxccGleMl97WFxcb3RpbWVzIFksIFp9IiwyXSxbNiw1LCJcXGFscGhhX3tYLFksWn0iLDJdLFsxLDYsIihcXHBpXjJfe1csWH1cXG90aW1lcyAxX1kpXFxvdGltZXMgMV9aIl0sWzAsNSwiXFxwaV4yX3tXLFh9XFxvdGltZXMgKDFfWVxcb3RpbWVzIDFfWikiLDJdLFs0LDUsIlxccGleMl97VyxYXFxvdGltZXMoWVxcb3RpbWVzIFopfSIsMl1d
      \begin{tikzcd}
                && {(W\otimes X)\otimes(Y\otimes Z)} \\
        {((W\otimes X)\otimes Y)\otimes Z} &&&& {W\otimes (X\otimes (Y\otimes Z)} \\
                                           &&& {X\otimes(Y\otimes Z)} \\
                                           && {(X\otimes Y)\otimes Z} \\
        {(W\otimes (X\otimes Y))\otimes Z} &&&& {W\otimes((X\otimes Y)\otimes Z)}
        \arrow["{\alpha_{W\otimes X,Y,Z}}", from=2-1, to=1-3]
        \arrow["{\alpha_{W,X,Y}\otimes X}"', from=2-1, to=5-1]
        \arrow["{\alpha_{W,X\otimes Y,Z}}"', from=5-1, to=5-5]
        \arrow["{W\otimes \alpha_{X,Y,Z}}"', from=5-5, to=2-5]
        \arrow["{\alpha_{W,X,Y\otimes Z}}", from=1-3, to=2-5]
        \arrow["{\pi^2_{W,X\otimes Y}\otimes Z}", from=5-1, to=4-3]
        \arrow["{\pi^2_{X\otimes Y, Z}}"', from=5-5, to=4-3]
        \arrow["{\alpha_{X,Y,Z}}"', from=4-3, to=3-4]
        \arrow["{(\pi^2_{W,X}\otimes Y)\otimes Z}", from=2-1, to=4-3]
        \arrow["{\pi^2_{W,X}\otimes ({Y\otimes Z})}"', from=1-3, to=3-4]
        \arrow["{\pi^2_{W,X\otimes(Y\otimes Z)}}"', from=2-5, to=3-4]
      \end{tikzcd}
    \end{equation}
    Here the leftmost triangle is just the triangle in (\ref{eq:monoidal_a}).
    The rectangle at the top follows due to naturality of $\alpha$ and the rest
    commutes due to naturality of $\pi^2$. Thus (\ref{eq:monoidal_pentagon_pi2})
    commutes.

    Combining (\ref{eq:monoidal_pentagon_pi1}) and (\ref{eq:monoidal_pentagon_pi2}) we
    find that (\ref{eq:monoidal_pentagon}) holds.
  \end{proof}
\end{lemma}

\begin{definition}[Braiding]\label{def:cartesian_braiding}
  For all $X,Y\in\cat C$, define
  \begin{align*}
    \gamma : X \otimes Y \to Y \otimes X
  \end{align*}
  to be the unique arrow that makes the following commute:
  \begin{equation}
    \label{eq:monoidal_y}
    \begin{tikzcd}[row sep=huge, column sep=huge]
            &
            X \otimes Y
            \arrow{dl}[swap]{\pi^2_{X,Y}}
            \arrow{d}{\gamma_{X,Y}}
            \arrow{dr}{\pi^1_{X,Y}}\\
      Y &
      Y \otimes X
      \arrow{l}{\pi^1_{Y,X}}
      \arrow{r}[swap]{\pi^2_{Y,X}} &
      X
    \end{tikzcd}
  \end{equation}
\end{definition}

\begin{proposition}
  \label{eq:monoidal-braiding}
  The transformation given by the maps $\gamma_{X,Y}: X\otimes Y\to Y\otimes X$ is a natural isomorphism in both
  arguments.
  \begin{proof}
    Let $X,X',Y,Y'\in\cat C$.
    The fact that $\gamma$ is an isomorphism is immediate as there exists
    a unique morphism
    \begin{align}
      \label{eq:monoidal_inv_y_y}
      \inv\gamma_{X,Y}=\gamma_{Y,X}:Y\otimes X\to X\otimes Y
    \end{align}
    that makes the following commute:
    \begin{equation*}
      \label{eq:monoidal_inv_y}
      \begin{tikzcd}[row sep=huge, column sep=huge]
                &
                Y \otimes X
                \arrow{dl}[swap]{\pi^2_{Y,X}}
                \arrow{d}{\inv\gamma_{X,Y}}
                \arrow{dr}{\pi^1_{Y,X}}\\
        X &
        X \otimes Y
        \arrow{l}{\pi^1_{X,Y}}
        \arrow{r}[swap]{\pi^2_{X,Y}} &
        Y
      \end{tikzcd}
    \end{equation*}
    Using~\ref{eq:monoidal_y} it is then straightforward to show
    $\inv\gamma_{X,Y}\circ\gamma_{X,Y} = {X\otimes Y}$ and
    $\gamma_{X,Y}\circ\inv\gamma_{X,Y} = {Y\otimes X}$.

    Further, let $f:X\to X'$ and $g:Y\to Y'$. Then the following two
    diagrams commute:
    \begin{equation*}
      % https://q.uiver.app/?q=WzAsNixbMCwwLCJYXFxvdGltZXMgWSJdLFszLDAsIlgnXFxvdGltZXMgWSciXSxbMywyLCJZJ1xcb3RpbWVzIFgnIl0sWzAsMiwiWVxcb3RpbWVzIFgiXSxbMSwxLCJYIl0sWzIsMSwiWCciXSxbMCwxLCJmXFxvdGltZXMgZyJdLFszLDIsImdcXG90aW1lcyBmIiwyXSxbMSwyLCJcXHVwc2lsb25fe1gnLFknfSJdLFswLDMsIlxcdXBzaWxvbl97WCxZfSIsMl0sWzAsNCwiXFxwaV4xX3tYLFl9Il0sWzQsNSwiZiJdLFsyLDUsIlxccGleMl97WScsWCd9Il0sWzMsNCwiXFxwaV4yX3tZLFh9IiwyXSxbMSw1LCJcXHBpXjFfe1gnLFknfSIsMl0sWzQsOSwiXFxleHBsYWlue2VxOm1vbm9pZGFsLXl9IiwxLHsic2hvcnRlbiI6eyJzb3VyY2UiOjIwLCJ0YXJnZXQiOjIwfSwic3R5bGUiOnsiYm9keSI6eyJuYW1lIjoibm9uZSJ9LCJoZWFkIjp7Im5hbWUiOiJub25lIn19fV0sWzgsNSwiXFxleHBsYWlue2VxOm1vbm9pZGFsLXl9IiwxLHsic2hvcnRlbiI6eyJzb3VyY2UiOjIwfSwic3R5bGUiOnsiYm9keSI6eyJuYW1lIjoibm9uZSJ9LCJoZWFkIjp7Im5hbWUiOiJub25lIn19fV0sWzcsMTEsIlxcZXhwbGFpbntlcTptb25vaWRhbC10ZW5zb3JfcHJvZHVjdH0iLDEseyJzaG9ydGVuIjp7InNvdXJjZSI6MjAsInRhcmdldCI6MjB9LCJzdHlsZSI6eyJib2R5Ijp7Im5hbWUiOiJub25lIn0sImhlYWQiOnsibmFtZSI6Im5vbmUifX19XSxbNiwxMSwiXFxleHBsYWlue2VxOm1vbm9pZGFsLXRlbnNvcl9wcm9kdWN0fSIsMSx7InNob3J0ZW4iOnsic291cmNlIjoyMCwidGFyZ2V0IjoyMH0sInN0eWxlIjp7ImJvZHkiOnsibmFtZSI6Im5vbmUifSwiaGVhZCI6eyJuYW1lIjoibm9uZSJ9fX1dXQ==
      \begin{tikzcd}
        {X\otimes Y} &&& {X'\otimes Y'} \\
                     & X & {X'} \\
        {Y\otimes X} &&& {Y'\otimes X'}
        \arrow[""{name=0, anchor=center, inner sep=0}, "{f\otimes g}", from=1-1, to=1-4]
        \arrow[""{name=1, anchor=center, inner sep=0}, "{g\otimes f}"', from=3-1, to=3-4]
        \arrow[""{name=2, anchor=center, inner sep=0}, "{\gamma_{X',Y'}}", from=1-4, to=3-4]
        \arrow[""{name=3, anchor=center, inner sep=0}, "{\gamma_{X,Y}}"', from=1-1, to=3-1]
        \arrow["{\pi^1_{X,Y}}", from=1-1, to=2-2]
        \arrow[""{name=4, anchor=center, inner sep=0}, "f", from=2-2, to=2-3]
        \arrow["{\pi^2_{Y',X'}}", from=3-4, to=2-3]
        \arrow["{\pi^2_{Y,X}}"', from=3-1, to=2-2]
        \arrow["{\pi^1_{X',Y'}}"', from=1-4, to=2-3]
        \arrow["{\explain{eq:monoidal_y}}"{description}, Rightarrow, draw=none, from=2-2, to=3]
        \arrow["{\explain{eq:monoidal_y}}"{description}, Rightarrow, draw=none, from=2, to=2-3]
        \arrow["{\explain{eq:monoidal_tensor_product}}"{description}, Rightarrow, draw=none, from=1, to=4]
        \arrow["{\explain{eq:monoidal_tensor_product}}"{description}, Rightarrow, draw=none, from=0, to=4]
      \end{tikzcd}
    \end{equation*}
    \begin{equation*}
      % https://q.uiver.app/?q=WzAsNixbMCwwLCJYXFxvdGltZXMgWSJdLFszLDAsIlgnXFxvdGltZXMgWSciXSxbMywyLCJZJ1xcb3RpbWVzIFgnIl0sWzAsMiwiWVxcb3RpbWVzIFgiXSxbMSwxLCJZIl0sWzIsMSwiWSciXSxbMCwxLCJmXFxvdGltZXMgZyJdLFszLDIsImdcXG90aW1lcyBmIiwyXSxbMSwyLCJcXHVwc2lsb25fe1gnLFknfSJdLFswLDMsIlxcdXBzaWxvbl97WCxZfSIsMl0sWzMsNCwiXFxwaV4xX3tZLFh9IiwyXSxbMCw0LCJcXHBpXjFfe1gsWX0iXSxbNCw1LCJnIl0sWzIsNSwiXFxwaV4xX3tZJyxYJ30iXSxbMSw1LCJcXHBpXjJfe1gnLFknfSIsMl0sWzYsMTIsIlxcZXhwbGFpbntlcTptb25vaWRhbC10ZW5zb3JfcHJvZHVjdH0iLDEseyJzaG9ydGVuIjp7InNvdXJjZSI6MjAsInRhcmdldCI6MjB9LCJzdHlsZSI6eyJib2R5Ijp7Im5hbWUiOiJub25lIn0sImhlYWQiOnsibmFtZSI6Im5vbmUifX19XSxbNywxMiwiXFxleHBsYWlue2VxOm1vbm9pZGFsLXRlbnNvcl9wcm9kdWN0fSIsMSx7InNob3J0ZW4iOnsic291cmNlIjoyMCwidGFyZ2V0IjoyMH0sInN0eWxlIjp7ImJvZHkiOnsibmFtZSI6Im5vbmUifSwiaGVhZCI6eyJuYW1lIjoibm9uZSJ9fX1dLFs4LDUsIlxcZXhwbGFpbntlcTptb25vaWRhbC15fSIsMSx7InNob3J0ZW4iOnsic291cmNlIjoyMH0sInN0eWxlIjp7ImJvZHkiOnsibmFtZSI6Im5vbmUifSwiaGVhZCI6eyJuYW1lIjoibm9uZSJ9fX1dLFs5LDQsIlxcZXhwbGFpbntlcTptb25vaWRhbC15fSIsMSx7InNob3J0ZW4iOnsic291cmNlIjoyMH0sInN0eWxlIjp7ImJvZHkiOnsibmFtZSI6Im5vbmUifSwiaGVhZCI6eyJuYW1lIjoibm9uZSJ9fX1dXQ==
      \begin{tikzcd}
        {X\otimes Y} &&& {X'\otimes Y'} \\
                     & Y & {Y'} \\
        {Y\otimes X} &&& {Y'\otimes X'}
        \arrow[""{name=0, anchor=center, inner sep=0}, "{f\otimes g}", from=1-1, to=1-4]
        \arrow[""{name=1, anchor=center, inner sep=0}, "{g\otimes f}"', from=3-1, to=3-4]
        \arrow[""{name=2, anchor=center, inner sep=0}, "{\gamma_{X',Y'}}", from=1-4, to=3-4]
        \arrow[""{name=3, anchor=center, inner sep=0}, "{\gamma_{X,Y}}"', from=1-1, to=3-1]
        \arrow["{\pi^1_{Y,X}}"', from=3-1, to=2-2]
        \arrow["{\pi^2_{X,Y}}", from=1-1, to=2-2]
        \arrow[""{name=4, anchor=center, inner sep=0}, "g", from=2-2, to=2-3]
        \arrow["{\pi^1_{Y',X'}}", from=3-4, to=2-3]
        \arrow["{\pi^2_{X',Y'}}"', from=1-4, to=2-3]
        \arrow["{\explain{eq:monoidal_tensor_product}}"{description}, Rightarrow, draw=none, from=0, to=4]
        \arrow["{\explain{eq:monoidal_tensor_product}}"{description}, Rightarrow, draw=none, from=1, to=4]
        \arrow["{\explain{eq:monoidal_y}}"{description}, Rightarrow, draw=none, from=2, to=2-3]
        \arrow["{\explain{eq:monoidal_y}}"{description}, Rightarrow, draw=none, from=3, to=2-2]
      \end{tikzcd}
    \end{equation*}
    Thus the outer diagram
    \begin{equation*}
      % https://q.uiver.app/?q=WzAsNCxbMCwwLCJYXFxvdGltZXMgWSJdLFszLDAsIlgnXFxvdGltZXMgWSciXSxbMywyLCJZJ1xcb3RpbWVzIFgnIl0sWzAsMiwiWVxcb3RpbWVzIFgiXSxbMCwxLCJmXFxvdGltZXMgZyJdLFszLDIsImdcXG90aW1lcyBmIiwyXSxbMSwyLCJcXHVwc2lsb25fe1gnLFknfSJdLFswLDMsIlxcdXBzaWxvbl97WCxZfSIsMl1d
      \begin{tikzcd}
        {X\otimes Y} &&& {X'\otimes Y'} \\
        \\
        {Y\otimes X} &&& {Y'\otimes X'}
        \arrow["{f\otimes g}", from=1-1, to=1-4]
        \arrow["{g\otimes f}"', from=3-1, to=3-4]
        \arrow["{\gamma_{X',Y'}}", from=1-4, to=3-4]
        \arrow["{\gamma_{X,Y}}"', from=1-1, to=3-1]
      \end{tikzcd}
    \end{equation*}
    commutes, showing naturality.
  \end{proof}
\end{proposition}


\begin{proposition}
  \label{eq:monoidal-symmetric}
  $(\cat C, \otimes, 1, a, \lambda, \rho, \gamma)$ forms a symmetric monoidal category.
  \begin{proof}
    We have the unit axiom
    \begin{equation*}
      % https://q.uiver.app/?q=WzAsMyxbMCwwLCJYXFxvdGltZXMgMSJdLFsyLDAsIjFcXG90aW1lcyBYIl0sWzEsMSwiWCJdLFswLDIsIlxccmhvX1giLDJdLFsxLDIsIlxcbGFtYmRhX1giXSxbMCwxLCJcXHVwc2lsb25fe1gsMX0iXSxbMiw1LCJcXGV4cGxhaW57ZXE6bW9ub2lkYWwteX0iLDEseyJzaG9ydGVuIjp7InRhcmdldCI6MjB9LCJzdHlsZSI6eyJib2R5Ijp7Im5hbWUiOiJub25lIn0sImhlYWQiOnsibmFtZSI6Im5vbmUifX19XV0=
      \begin{tikzcd}
        {X\otimes 1} && {1\otimes X} \\
                     & X
                     \arrow["{\rho_X}"', from=1-1, to=2-2]
                     \arrow["{\lambda_X}", from=1-3, to=2-2]
                     \arrow[""{name=0, anchor=center, inner sep=0}, "{\gamma_{X,1}}", from=1-1, to=1-3]
                     \arrow["{\explain{eq:monoidal_y}}"{description}, Rightarrow, draw=none, from=2-2, to=0]
      \end{tikzcd}
    \end{equation*}
    and the inverse axiom
    \begin{equation*}
      % https://q.uiver.app/?q=WzAsMyxbMCwwLCJYXFxvdGltZXMgWSJdLFsyLDAsIlhcXG90aW1lcyBZIl0sWzEsMSwiWVxcb3RpbWVzIFgiXSxbMCwxLCIxX3tYXFxvdGltZXMgWX0iXSxbMCwyLCJcXHVwc2lsb25fe1gsWX0iLDJdLFsyLDEsIlxcdXBzaWxvbl97WSxYfSIsMl0sWzIsMywiXFxleHBsYWlue2VxOm1vbm9pZGFsLXl9IiwxLHsic2hvcnRlbiI6eyJ0YXJnZXQiOjIwfSwic3R5bGUiOnsiYm9keSI6eyJuYW1lIjoibm9uZSJ9LCJoZWFkIjp7Im5hbWUiOiJub25lIn19fV1d
      \begin{tikzcd}
        {X\otimes Y} && {X\otimes Y} \\
                     & {Y\otimes X}
                     \arrow[""{name=0, anchor=center, inner sep=0}, "{{X\otimes Y}}", from=1-1, to=1-3]
                     \arrow["{\gamma_{X,Y}}"', from=1-1, to=2-2]
                     \arrow["{\gamma_{Y,X}}"', from=2-2, to=1-3]
                     \arrow["{\explain{eq:monoidal_inv_y_y}}"{description}, Rightarrow, draw=none, from=2-2, to=0]
      \end{tikzcd}
    \end{equation*}
    Now we find that the following commutes:
    \begin{equation*}
      % https://q.uiver.app/?q=WzAsMTAsWzAsMCwiKFhcXG90aW1lcyBZKVxcb3RpbWVzIFoiXSxbNiwwLCIoWVxcb3RpbWVzIFgpXFxvdGltZXMgWiJdLFswLDIsIlhcXG90aW1lcyhZXFxvdGltZXMgWikiXSxbMCw0LCIoWVxcb3RpbWVzIFopXFxvdGltZXMgWCJdLFs2LDQsIllcXG90aW1lcyhaXFxvdGltZXMgWCkiXSxbNiwyLCJZXFxvdGltZXMoWFxcb3RpbWVzIFopIl0sWzIsMywiWVxcb3RpbWVzIFoiXSxbNCwzLCJZIl0sWzQsMSwiWVxcb3RpbWVzIFgiXSxbMiwxLCJYXFxvdGltZXMgWSJdLFswLDIsIlxcYWxwaGFfe1hcXG90aW1lcyBZLFp9IiwyXSxbMiwzLCJcXHVwc2lsb25fe1gsWVxcb3RpbWVzIFp9IiwyXSxbMCwxLCJcXHVwc2lsb25fe1gsWX1cXG90aW1lcyAxX1oiXSxbMSw1LCJcXGFscGhhX3tZXFxvdGltZXMgWCxafSJdLFs1LDQsIjFfWVxcb3RpbWVzIFxcdXBzaWxvbl97WCxafSJdLFszLDQsIlxcYWxwaGFfe1lcXG90aW1lcyBaLFh9IiwyXSxbMCw2LCJcXHBpXjJfe1gsWX1cXG90aW1lcyAxX1oiXSxbMCw5LCJcXHBpXjFfe1hcXG90aW1lcyBZLFp9Il0sWzEsOCwiXFxwaV4xX3tZXFxvdGltZXMgWCxafSJdLFs5LDgsIlxcdXBzaWxvbl97WCxZfSJdLFs4LDcsIlxccGleMV97WSxYfSJdLFs5LDcsIlxccGleMl97WCxZfSJdLFs2LDcsIlxccGleMV97WSxafSJdLFs1LDcsIlxccGleMV97WSxYXFxvdGltZXMgWn0iXSxbNCw3LCJcXHBpXjFfe1ksWlxcb3RpbWVzIFh9Il0sWzMsNiwiXFxwaV4xX3tZXFxvdGltZXMgWixYfSIsMl0sWzIsNiwiXFxwaV4yX3tYLFlcXG90aW1lcyBafSJdLFsyMywxOCwiXFxleHBsYWlue2VxOm1vbm9pZGFsLWF9IiwxLHsic2hvcnRlbiI6eyJzb3VyY2UiOjIwLCJ0YXJnZXQiOjIwfSwic3R5bGUiOnsiYm9keSI6eyJuYW1lIjoibm9uZSJ9LCJoZWFkIjp7Im5hbWUiOiJub25lIn19fV0sWzE5LDEyLCJcXGV4cGxhaW5uYXR7XFxwaV4xfSIsMSx7InNob3J0ZW4iOnsic291cmNlIjoyMCwidGFyZ2V0IjoyMH0sInN0eWxlIjp7ImJvZHkiOnsibmFtZSI6Im5vbmUifSwiaGVhZCI6eyJuYW1lIjoibm9uZSJ9fX1dLFs3LDE0LCJcXGV4cGxhaW5uYXR7XFxwaV4xfSIsMSx7InNob3J0ZW4iOnsidGFyZ2V0IjoyMH0sInN0eWxlIjp7ImJvZHkiOnsibmFtZSI6Im5vbmUifSwiaGVhZCI6eyJuYW1lIjoibm9uZSJ9fX1dLFsxNiwyMSwiXFxleHBsYWlubmF0e1xccGleMX0iLDEseyJzaG9ydGVuIjp7InNvdXJjZSI6MjAsInRhcmdldCI6MjB9LCJzdHlsZSI6eyJib2R5Ijp7Im5hbWUiOiJub25lIn0sImhlYWQiOnsibmFtZSI6Im5vbmUifX19XSxbMiwxNiwiXFxleHBsYWlue2VxOm1vbm9pZGFsLWF9IiwxLHsic2hvcnRlbiI6eyJ0YXJnZXQiOjIwfSwic3R5bGUiOnsiYm9keSI6eyJuYW1lIjoibm9uZSJ9LCJoZWFkIjp7Im5hbWUiOiJub25lIn19fV0sWzExLDYsIlxcZXhwbGFpbmRlZntcXHVwc2lsb259IiwxLHsic2hvcnRlbiI6eyJzb3VyY2UiOjIwfSwic3R5bGUiOnsiYm9keSI6eyJuYW1lIjoibm9uZSJ9LCJoZWFkIjp7Im5hbWUiOiJub25lIn19fV0sWzE1LDIyLCJcXGV4cGxhaW57ZXE6bW9ub2lkYWwtYX0iLDEseyJzaG9ydGVuIjp7InNvdXJjZSI6MjAsInRhcmdldCI6MjB9LCJzdHlsZSI6eyJib2R5Ijp7Im5hbWUiOiJub25lIn0sImhlYWQiOnsibmFtZSI6Im5vbmUifX19XSxbMjEsOCwiXFxleHBsYWluZGVme1xcdXBzaWxvbn0iLDEseyJzaG9ydGVuIjp7InNvdXJjZSI6MjB9LCJzdHlsZSI6eyJib2R5Ijp7Im5hbWUiOiJub25lIn0sImhlYWQiOnsibmFtZSI6Im5vbmUifX19XV0=
      \begin{tikzcd}
        {(X\otimes Y)\otimes Z} &&&&&& {(Y\otimes X)\otimes Z} \\
                                && {X\otimes Y} && {Y\otimes X} \\
        {X\otimes(Y\otimes Z)} &&&&&& {Y\otimes(X\otimes Z)} \\
                               && {Y\otimes Z} && Y \\
        {(Y\otimes Z)\otimes X} &&&&&& {Y\otimes(Z\otimes X)}
        \arrow["{\alpha_{X\otimes Y,Z}}"', from=1-1, to=3-1]
        \arrow[""{name=0, anchor=center, inner sep=0}, "{\gamma_{X,Y\otimes Z}}"', from=3-1, to=5-1]
        \arrow[""{name=1, anchor=center, inner sep=0}, "{\gamma_{X,Y}\otimes Z}", from=1-1, to=1-7]
        \arrow["{\alpha_{Y\otimes X,Z}}", from=1-7, to=3-7]
        \arrow[""{name=2, anchor=center, inner sep=0}, "{Y\otimes \gamma_{X,Z}}", from=3-7, to=5-7]
        \arrow[""{name=3, anchor=center, inner sep=0}, "{\alpha_{Y\otimes Z,X}}"', from=5-1, to=5-7]
        \arrow[""{name=4, anchor=center, inner sep=0}, "{\pi^2_{X,Y}\otimes Z}", from=1-1, to=4-3]
        \arrow["{\pi^1_{X\otimes Y,Z}}", from=1-1, to=2-3]
        \arrow[""{name=5, anchor=center, inner sep=0}, "{\pi^1_{Y\otimes X,Z}}", from=1-7, to=2-5]
        \arrow[""{name=6, anchor=center, inner sep=0}, "{\gamma_{X,Y}}", from=2-3, to=2-5]
        \arrow["{\pi^1_{Y,X}}", from=2-5, to=4-5]
        \arrow[""{name=7, anchor=center, inner sep=0}, "{\pi^2_{X,Y}}", from=2-3, to=4-5]
        \arrow[""{name=8, anchor=center, inner sep=0}, "{\pi^1_{Y,Z}}", from=4-3, to=4-5]
        \arrow[""{name=9, anchor=center, inner sep=0}, "{\pi^1_{Y,X\otimes Z}}", from=3-7, to=4-5]
        \arrow["{\pi^1_{Y,Z\otimes X}}", from=5-7, to=4-5]
        \arrow["{\pi^1_{Y\otimes Z,X}}"', from=5-1, to=4-3]
        \arrow["{\pi^2_{X,Y\otimes Z}}", from=3-1, to=4-3]
        \arrow["{\explain{eq:monoidal_a}}"{description}, draw=none, from=9, to=5]
        \arrow["{\explainnat{\pi^1}}"{description}, draw=none, from=6, to=1]
        \arrow["{\explainnat{\pi^1}}"{description}, draw=none, from=4-5, to=2]
        \arrow["{\explainnat{\pi^1}}"{description}, draw=none, from=4, to=7]
        \arrow["{\explain{eq:monoidal_a}}"{description}, draw=none, from=3-1, to=4]
        \arrow["{\explaindef{\gamma}}"{description}, draw=none, from=0, to=4-3]
        \arrow["{\explain{eq:monoidal_a}}"{description}, draw=none, from=3, to=8]
        \arrow["{\explaindef{\gamma}}"{description}, draw=none, from=7, to=2-5]
      \end{tikzcd}
    \end{equation*}
    Now we note that
    \begin{equation*}
      % https://q.uiver.app/?q=WzAsNSxbMCwwLCIoWFxcb3RpbWVzIFkpXFxvdGltZXMgWiJdLFs2LDAsIlhcXG90aW1lcyhZXFxvdGltZXMgWikiXSxbMywzLCJYXFxvdGltZXMgWiJdLFszLDEsIlgiXSxbMiwxLCJYXFxvdGltZXMgWSJdLFswLDEsIlxcYWxwaGFfe1hcXG90aW1lcyBZLFp9Il0sWzEsMiwiMV9YXFxvdGltZXMgXFxwaV4yX3tZLFp9IiwwLHsiY3VydmUiOi01fV0sWzAsMiwiXFxwaV4xX3tYLFl9XFxvdGltZXMgMV9aIiwyLHsiY3VydmUiOjV9XSxbMiwzLCJcXHBpXjFfe1gsWn0iXSxbMSwzLCJcXHBpXjFfe1gsWVxcb3RpbWVzIFp9Il0sWzAsNCwiXFxwaV4xX3tYXFxvdGltZXMgWSxafSJdLFs0LDMsIlxccGleMV97WCxZfSJdLFszLDYsIlxcZXhwbGFpbm5hdHtcXHBpXjF9IiwxLHsic2hvcnRlbiI6eyJ0YXJnZXQiOjIwfSwic3R5bGUiOnsiYm9keSI6eyJuYW1lIjoibm9uZSJ9LCJoZWFkIjp7Im5hbWUiOiJub25lIn19fV0sWzUsMTEsIlxcZXhwbGFpbmRlZntcXGFscGhhfSIsMSx7InNob3J0ZW4iOnsic291cmNlIjoyMCwidGFyZ2V0IjoyMH0sInN0eWxlIjp7ImJvZHkiOnsibmFtZSI6Im5vbmUifSwiaGVhZCI6eyJuYW1lIjoibm9uZSJ9fX1dLFs3LDgsIlxcZXhwbGFpbm5hdHtcXHBpXjF9IiwxLHsic2hvcnRlbiI6eyJzb3VyY2UiOjIwLCJ0YXJnZXQiOjIwfSwic3R5bGUiOnsiYm9keSI6eyJuYW1lIjoibm9uZSJ9LCJoZWFkIjp7Im5hbWUiOiJub25lIn19fV1d
      \begin{tikzcd}
        {(X\otimes Y)\otimes Z} &&&&&& {X\otimes(Y\otimes Z)} \\
                                && {X\otimes Y} & X \\
                                \\
                                &&& {X\otimes Z}
                                \arrow[""{name=0, anchor=center, inner sep=0}, "{\alpha_{X\otimes Y,Z}}", from=1-1, to=1-7]
                                \arrow[""{name=1, anchor=center, inner sep=0}, "{X\otimes \pi^2_{Y,Z}}", bend left=30, from=1-7, to=4-4]
                                \arrow[""{name=2, anchor=center, inner sep=0}, "{\pi^1_{X,Y}\otimes Z}"', bend right=30, from=1-1, to=4-4]
                                \arrow[""{name=3, anchor=center, inner sep=0}, "{\pi^1_{X,Z}}", from=4-4, to=2-4]
                                \arrow["{\pi^1_{X,Y\otimes Z}}", from=1-7, to=2-4]
                                \arrow["{\pi^1_{X\otimes Y,Z}}", from=1-1, to=2-3]
                                \arrow[""{name=4, anchor=center, inner sep=0}, "{\pi^1_{X,Y}}", from=2-3, to=2-4]
                                \arrow["{\explainnat{\pi^1}}"{description}, draw=none, from=2-4, to=1]
                                \arrow["{\explaindef{\alpha}}"{description}, draw=none, from=0, to=4]
                                \arrow["{\explainnat{\pi^1}}"{description}, draw=none, from=2, to=3]
      \end{tikzcd}
    \end{equation*}
    and
    \begin{equation*}
      % https://q.uiver.app/?q=WzAsNSxbMCwwLCIoWFxcb3RpbWVzIFkpXFxvdGltZXMgWiJdLFs2LDAsIlhcXG90aW1lcyhZXFxvdGltZXMgWikiXSxbMywzLCJYXFxvdGltZXMgWiJdLFszLDEsIloiXSxbNCwxLCJZXFxvdGltZXMgWiJdLFswLDEsIlxcYWxwaGFfe1hcXG90aW1lcyBZLFp9Il0sWzEsMiwiMV9YXFxvdGltZXMgXFxwaV4yX3tZLFp9IiwwLHsiY3VydmUiOi01fV0sWzAsMiwiXFxwaV4xX3tYLFl9XFxvdGltZXMgMV9aIiwyLHsiY3VydmUiOjV9XSxbMCwzLCJcXHBpXjJfe1hcXG90aW1lcyBZLFp9Il0sWzEsNCwiXFxwaV4yX3tYLFlcXG90aW1lcyBafSIsMl0sWzIsMywiXFxwaV4yX3tYLFp9IiwyXSxbNCwzLCJcXHBpXjJfe1ksWn0iXSxbNywzLCJcXGV4cGxhaW5uYXR7XFxwaV4yfSIsMSx7InNob3J0ZW4iOnsic291cmNlIjoyMH0sInN0eWxlIjp7ImJvZHkiOnsibmFtZSI6Im5vbmUifSwiaGVhZCI6eyJuYW1lIjoibm9uZSJ9fX1dLFs1LDExLCJcXGV4cGxhaW5kZWZ7XFxpbnZcXGFscGhhfSIsMSx7InNob3J0ZW4iOnsic291cmNlIjoyMCwidGFyZ2V0IjoyMH0sInN0eWxlIjp7ImJvZHkiOnsibmFtZSI6Im5vbmUifSwiaGVhZCI6eyJuYW1lIjoibm9uZSJ9fX1dLFs2LDEwLCJcXGV4cGxhaW5uYXR7XFxwaV4yfSIsMSx7InNob3J0ZW4iOnsic291cmNlIjoyMH0sInN0eWxlIjp7ImJvZHkiOnsibmFtZSI6Im5vbmUifSwiaGVhZCI6eyJuYW1lIjoibm9uZSJ9fX1dXQ==
      \begin{tikzcd}
        {(X\otimes Y)\otimes Z} &&&&&& {X\otimes(Y\otimes Z)} \\
                                &&& Z & {Y\otimes Z} \\
                                \\
                                &&& {X\otimes Z}
                                \arrow[""{name=0, anchor=center, inner sep=0}, "{\alpha_{X\otimes Y,Z}}", from=1-1, to=1-7]
                                \arrow[""{name=1, anchor=center, inner sep=0}, "{X\otimes \pi^2_{Y,Z}}", bend left=30, from=1-7, to=4-4]
                                \arrow[""{name=2, anchor=center, inner sep=0}, "{\pi^1_{X,Y}\otimes Z}"', bend right=30, from=1-1, to=4-4]
                                \arrow["{\pi^2_{X\otimes Y,Z}}", from=1-1, to=2-4]
                                \arrow["{\pi^2_{X,Y\otimes Z}}"', from=1-7, to=2-5]
                                \arrow[""{name=3, anchor=center, inner sep=0}, "{\pi^2_{X,Z}}"', from=4-4, to=2-4]
                                \arrow[""{name=4, anchor=center, inner sep=0}, "{\pi^2_{Y,Z}}", from=2-5, to=2-4]
                                \arrow["{\explainnat{\pi^2}}"{description}, draw=none, from=2, to=2-4]
                                \arrow["{\explaindef{\inv\alpha}}"{description}, draw=none, from=0, to=4]
                                \arrow["{\explainnat{\pi^2}}"{description}, draw=none, from=1, to=3]
      \end{tikzcd}
    \end{equation*}
    commute. Thus we have
    \begin{equation}
      \label{eq:monoidal_symmetric_triangle}
      % https://q.uiver.app/?q=WzAsMyxbMCwwLCIoWFxcb3RpbWVzIFkpXFxvdGltZXMgWiJdLFs2LDAsIlhcXG90aW1lcyhZXFxvdGltZXMgWikiXSxbMywzLCJYXFxvdGltZXMgWiJdLFswLDEsIlxcYWxwaGFfe1hcXG90aW1lcyBZLFp9Il0sWzEsMiwiMV9YXFxvdGltZXMgXFxwaV4yX3tZLFp9Il0sWzAsMiwiXFxwaV4xX3tYLFl9XFxvdGltZXMgMV9aIiwyXV0=
      \begin{tikzcd}
        {(X\otimes Y)\otimes Z} &&&&&& {X\otimes(Y\otimes Z)} \\
        \\
        \\
                                &&& {X\otimes Z}
                                \arrow["{\alpha_{X\otimes Y,Z}}", from=1-1, to=1-7]
                                \arrow["{X\otimes \pi^2_{Y,Z}}", from=1-7, to=4-4]
                                \arrow["{\pi^1_{X,Y}\otimes Z}"', from=1-1, to=4-4]
      \end{tikzcd}
    \end{equation}
    Using this, we show the following:
    \begin{equation*}
      % https://q.uiver.app/?q=WzAsOCxbMCwwLCIoWFxcb3RpbWVzIFkpXFxvdGltZXMgWiJdLFs2LDAsIihZXFxvdGltZXMgWClcXG90aW1lcyBaIl0sWzAsMiwiWFxcb3RpbWVzKFlcXG90aW1lcyBaKSJdLFswLDQsIihZXFxvdGltZXMgWilcXG90aW1lcyBYIl0sWzYsNCwiWVxcb3RpbWVzKFpcXG90aW1lcyBYKSJdLFs2LDIsIllcXG90aW1lcyhYXFxvdGltZXMgWikiXSxbNCwzLCJaXFxvdGltZXMgWCJdLFs0LDIsIlhcXG90aW1lcyBaIl0sWzAsMiwiXFxhbHBoYV97WFxcb3RpbWVzIFksWn0iLDJdLFsyLDMsIlxcZ2FtbWFfe1gsWVxcb3RpbWVzIFp9IiwyXSxbMCwxLCJcXGdhbW1hX3tYLFl9XFxvdGltZXMgMV9aIl0sWzEsNSwiXFxhbHBoYV97WVxcb3RpbWVzIFgsWn0iXSxbNSw0LCIxX1lcXG90aW1lcyBcXGdhbW1hX3tYLFp9Il0sWzMsNCwiXFxhbHBoYV97WVxcb3RpbWVzIFosWH0iLDJdLFszLDYsIlxccGleMl97WSxafVxcb3RpbWVzIDFfWCJdLFs0LDYsIlxccGleMl97WSxaXFxvdGltZXMgWH0iLDJdLFs1LDcsIlxccGleMl97WSxYXFxvdGltZXMgWn0iXSxbNyw2LCJcXGdhbW1hX3tYLFp9Il0sWzEsNywiXFxwaV4yX3tZLFp9XFxvdGltZXMgMV9aIiwyXSxbMiw3LCIxX1hcXG90aW1lcyBcXHBpXjJfe1ksWn0iXSxbMCw3LCJcXHBpXjFfe1gsWX1cXG90aW1lcyAxX1oiXSxbMTMsNiwiXFxleHBsYWluZGVme1xcYWxwaGF9IiwxLHsic2hvcnRlbiI6eyJzb3VyY2UiOjIwfSwic3R5bGUiOnsiYm9keSI6eyJuYW1lIjoibm9uZSJ9LCJoZWFkIjp7Im5hbWUiOiJub25lIn19fV0sWzIsMTQsIlxcZXhwbGFpbmRlZntcXGdhbW1hfSIsMSx7InNob3J0ZW4iOnsidGFyZ2V0IjoyMH0sInN0eWxlIjp7ImJvZHkiOnsibmFtZSI6Im5vbmUifSwiaGVhZCI6eyJuYW1lIjoibm9uZSJ9fX1dLFsxNywxMiwiXFxleHBsYWlubmF0e1xccGleMn0iLDEseyJzaG9ydGVuIjp7InNvdXJjZSI6MjAsInRhcmdldCI6MjB9LCJzdHlsZSI6eyJib2R5Ijp7Im5hbWUiOiJub25lIn0sImhlYWQiOnsibmFtZSI6Im5vbmUifX19XSxbMTgsNSwiXFxleHBsYWluZGVme1xcYWxwaGF9IiwxLHsic2hvcnRlbiI6eyJzb3VyY2UiOjIwfSwic3R5bGUiOnsiYm9keSI6eyJuYW1lIjoibm9uZSJ9LCJoZWFkIjp7Im5hbWUiOiJub25lIn19fV0sWzcsMTAsIlxcZXhwbGFpbmRlZntcXGdhbW1hfSIsMSx7InNob3J0ZW4iOnsidGFyZ2V0IjoyMH0sInN0eWxlIjp7ImJvZHkiOnsibmFtZSI6Im5vbmUifSwiaGVhZCI6eyJuYW1lIjoibm9uZSJ9fX1dLFsyMCwyLCJcXGV4cGxhaW57ZXE6bW9ub2lkYWwtc3ltbWV0cmljLXRyaWFuZ2xlfSIsMSx7InNob3J0ZW4iOnsic291cmNlIjoyMH0sInN0eWxlIjp7ImJvZHkiOnsibmFtZSI6Im5vbmUifSwiaGVhZCI6eyJuYW1lIjoibm9uZSJ9fX1dXQ==
      \begin{tikzcd}
        {(X\otimes Y)\otimes Z} &&&&&& {(Y\otimes X)\otimes Z} \\
        \\
        {X\otimes(Y\otimes Z)} &&&& {X\otimes Z} && {Y\otimes(X\otimes Z)} \\
                               &&&& {Z\otimes X} \\
        {(Y\otimes Z)\otimes X} &&&&&& {Y\otimes(Z\otimes X)}
        \arrow["{\alpha_{X\otimes Y,Z}}"', from=1-1, to=3-1]
        \arrow["{\gamma_{X,Y\otimes Z}}"', from=3-1, to=5-1]
        \arrow[""{name=0, anchor=center, inner sep=0}, "{\gamma_{X,Y}\otimes Z}", from=1-1, to=1-7]
        \arrow["{\alpha_{Y\otimes X,Z}}", from=1-7, to=3-7]
        \arrow[""{name=1, anchor=center, inner sep=0}, "{Y\otimes \gamma_{X,Z}}", from=3-7, to=5-7]
        \arrow[""{name=2, anchor=center, inner sep=0}, "{\alpha_{Y\otimes Z,X}}"', from=5-1, to=5-7]
        \arrow[""{name=3, anchor=center, inner sep=0}, "{\pi^2_{Y,Z}\otimes X}", from=5-1, to=4-5]
        \arrow["{\pi^2_{Y,Z\otimes X}}"', from=5-7, to=4-5]
        \arrow["{\pi^2_{Y,X\otimes Z}}", from=3-7, to=3-5]
        \arrow[""{name=4, anchor=center, inner sep=0}, "{\gamma_{X,Z}}", from=3-5, to=4-5]
        \arrow[""{name=5, anchor=center, inner sep=0}, "{\pi^2_{Y,Z}\otimes Z}"', from=1-7, to=3-5]
        \arrow["{X\otimes \pi^2_{Y,Z}}", from=3-1, to=3-5]
        \arrow[""{name=6, anchor=center, inner sep=0}, "{\pi^1_{X,Y}\otimes Z}", from=1-1, to=3-5]
        \arrow["{\explaindef{\alpha}}"{description}, draw=none, from=2, to=4-5]
        \arrow["{\explaindef{\gamma}}"{description}, draw=none, from=3-1, to=3]
        \arrow["{\explainnat{\pi^2}}"{description}, draw=none, from=4, to=1]
        \arrow["{\explaindef{\alpha}}"{description}, draw=none, from=5, to=3-7]
        \arrow["{\explaindef{\gamma}}"{description}, draw=none, from=3-5, to=0]
        \arrow["{\explain{eq:monoidal_symmetric_triangle}}"{description}, draw=none, from=6, to=3-1]
      \end{tikzcd}
    \end{equation*}
    Thus \ref{eq:symmetric_monoidal_hexagon} holds.
  \end{proof}
\end{proposition}

\section{Symmetric monoidal 2-categories}\label{sec:symmetric_monoidal_2categories}

\begin{definition}
  A symmetric monoidal 2-category consists of
  \begin{enumerate}
    \item a 2-category $\bicat C$;
    \item a 2-functor $\otimes : \bicat C\times\bicat C\to\bicat C$;
    \item an object $I\in\bicat C$;
    \item a 2-natural transformations $\alpha,\lambda,\rho,\gamma$ with components
      \begin{align*}
        \alpha_{X,Y,Z}:(X\otimes Y)\otimes Z &\to X\otimes(Y\otimes Z),\\
        \lambda_X:I\otimes X &\to X,\\
        \rho_X:X\otimes I &\to X,\\
        \gamma_{X,Y}:X\otimes Y &\to Y\otimes X
      \end{align*}
  \end{enumerate}
  such that
  \begin{enumerate}
    \item for all $W,X,Y,Z\in\bicat C$,
      \begin{equation}
        % https://q.uiver.app/?q=WzAsNSxbMCwwLCIoKFdcXG90aW1lcyBYKVxcb3RpbWVzIFkpXFxvdGltZXMgWiJdLFsxLDAsIihXXFxvdGltZXMgKFhcXG90aW1lcyBZKSlcXG90aW1lcyBaIl0sWzIsMCwiV1xcb3RpbWVzICgoWFxcb3RpbWVzIFkpXFxvdGltZXMgWikiXSxbMiwxLCJXXFxvdGltZXMoWFxcb3RpbWVzKFlcXG90aW1lcyBaKSkiXSxbMCwxLCIoV1xcb3RpbWVzIFgpXFxvdGltZXMoWVxcb3RpbWVzIFopIl0sWzAsNCwiXFxhbHBoYSIsMl0sWzAsMSwiXFxhbHBoYVxcb3RpbWVzIFoiXSxbMSwyLCJcXGFscGhhIl0sWzQsMywiXFxhbHBoYSIsMl0sWzIsMywiV1xcb3RpbWVzIFxcYWxwaGEiXV0=
        \begin{tikzcd}
          {((W\otimes X)\otimes Y)\otimes Z} & {(W\otimes (X\otimes Y))\otimes Z} & {W\otimes ((X\otimes Y)\otimes Z)} \\
          {(W\otimes X)\otimes(Y\otimes Z)} && {W\otimes(X\otimes(Y\otimes Z))}
          \arrow["\alpha"', from=1-1, to=2-1]
          \arrow["{\alpha\otimes Z}", from=1-1, to=1-2]
          \arrow["\alpha", from=1-2, to=1-3]
          \arrow["\alpha"', from=2-1, to=2-3]
          \arrow["{W\otimes \alpha}", from=1-3, to=2-3]
        \end{tikzcd}
      \end{equation}
    \item for all $X,Y\in\bicat C$,
      \begin{equation}
        % https://q.uiver.app/?q=WzAsMyxbMCwwLCIoWFxcb3RpbWVzIEkpXFxvdGltZXMgWSJdLFsyLDAsIlhcXG90aW1lcyhJXFxvdGltZXMgWSkiXSxbMSwxLCJYXFxvdGltZXMgWSJdLFswLDEsIlxcYWxwaGEiXSxbMCwyLCJcXHJob1xcb3RpbWVzIFkiLDJdLFsxLDIsIlhcXG90aW1lc1xcbGFtYmRhIl1d
        \begin{tikzcd}
          {(X\otimes I)\otimes Y} && {X\otimes(I\otimes Y)} \\
                                  & {X\otimes Y}
                                  \arrow["\alpha", from=1-1, to=1-3]
                                  \arrow["{\rho\otimes Y}"', from=1-1, to=2-2]
                                  \arrow["X\otimes\lambda", from=1-3, to=2-2]
        \end{tikzcd}
      \end{equation}
    \item for all $X,Y\in\bicat C$,
      \begin{equation}
        % https://q.uiver.app/?q=WzAsMyxbMCwwLCIoSVxcb3RpbWVzIFgpXFxvdGltZXMgWSJdLFsyLDAsIklcXG90aW1lcyAoWFxcb3RpbWVzIFkpIl0sWzEsMSwiWFxcb3RpbWVzIFkiXSxbMCwyLCJcXGxhbWJkYVxcb3RpbWVzIFkiLDJdLFsxLDIsIlxcbGFtYmRhIl0sWzAsMSwiXFxhbHBoYSJdXQ==
        \begin{tikzcd}
          {(I\otimes X)\otimes Y} && {I\otimes (X\otimes Y)} \\
                                  & {X\otimes Y}
                                  \arrow["{\lambda\otimes Y}"', from=1-1, to=2-2]
                                  \arrow["\lambda", from=1-3, to=2-2]
                                  \arrow["\alpha", from=1-1, to=1-3]
        \end{tikzcd}
      \end{equation}
    \item for all $X,Y\in\bicat C$,
      \begin{equation}
        % https://q.uiver.app/?q=WzAsMyxbMCwwLCIoWFxcb3RpbWVzIFkpXFxvdGltZXMgSSJdLFsyLDAsIlhcXG90aW1lcyhZXFxvdGltZXMgSSkiXSxbMSwxLCJYXFxvdGltZXMgWSJdLFswLDIsIlxccmhvIiwyXSxbMSwyLCJYXFxvdGltZXMgXFxyaG8iXSxbMCwxLCJcXGFscGhhIl1d
        \begin{tikzcd}
          {(X\otimes Y)\otimes I} && {X\otimes(Y\otimes I)} \\
                                  & {X\otimes Y}
                                  \arrow["\rho"', from=1-1, to=2-2]
                                  \arrow["{X\otimes \rho}", from=1-3, to=2-2]
                                  \arrow["\alpha", from=1-1, to=1-3]
        \end{tikzcd}
      \end{equation}
  \end{enumerate}
\end{definition}

