\chapter{Introduction}\label{sec:introduction}

The problem addressed in this report seems to have been solved entirely by
Slattery in \cite{slattery2023}. Our work is fully independent and was submitted before \cite{slattery2023} was published.

\section{Synthetic measure theory}

Kock observed that it is possible to characterise measure theory abstractly by a commutative monad
over a locally small cartesian closed category satisfying two further axioms~\cite{kock2011}.

For practical purposes, it makes sense to consider models that account for traditional Lebesgue integration.
However, function spaces are never measurable~\cite{aumann1961} so the category of measurable
spaces is not cartesian closed. This means that conventional measure theory is not a model of synthetic measure theory.
It is therefore interesting to find models that are close to conventional measure theory.

Measurable function spaces are desirable for other reasons, e.g.\ to formalise higher-order probability
theory, which is a problem that arises naturally when considering probabilisitic programming languages.
It is possible to extend measurable spaces to quasi-Borel spaces in order to achieve cartesian closure~\cite{heunen2017}. While some care is required when comparing quasi-Borel spaces to measure theory
and rephrasing probability theory in these new terms, they bring with them several convenient properties.
In particular, quasi-Borel spaces form a model of synthetic measure theory and the associated integral
is precisely the measure space integral~\cite{scibor2018}.

Since synthetic measure theory turns out to be a practically useful tool, it is now also
valuable to consider models that are far removed from conventional measure theory. One
may think of this as testing the limits of the theory.

A classical example of a monad from functional programming is the list monad.
This fails to be commutative. If we forget about the ordering of elements, we
obtain the powerset monad taking sets to their powersets. This monad may be
endowed with a commutative strong structure that, moreover, forms a model of synthetic
measure theory. In this model, the measures over a set $X$ are subsets
$\mu\subseteq X$ and the integral of a function $f:X\to\cc{0,1}$ with respect
to $\mu$ is
\begin{align*}
  \int_X \mu\rr{dx}f\rr{x} = \begin{cases}
    0 & \text{if }\forall  x\in \mu. f\rr{x}=0 \\
    1 & \text{if }\exists  x\in \mu. f\rr{x}=1
  \end{cases}
\end{align*}
In other words, if we consider $f$ to be the characteristic function of a subset $B\subseteq X$
then the integral is zero if and only if $\mu\cap B=\emptyset$.

It is worth noting that Kock's synthetic measure theory is not the only attempt at
formalising probability theory categorically. An alternative approach was taken by
Fritz in~\cite{fritz2020} who developed Markov categories by generalising probability
theory directly. This allows for more abstract reasoning,
thereby improving clarity. In particular, Fritz' theory unifies different types of probability
theory and does not rely on measure theoretic probability theory to the same degree.

\section{Presheaves}

Category theory has its own notion of integration: coends. A coend of a
functor $\catop{C}\times\cat{C}\to\cat{D}$ is a particular colimit in
$\cat{D}$. We tend to think of coends as integrals because they behave
like such.

However, as of right now there have been little to no formal developments
relating coends to integration. The motivation for this project is to
establish such a connection by considering a suitable model of synthetic
measure theory. To construct this model, we require the following:

\begin{enumerate}
  \item a cartesian closed category $\cat{C}$ and
  \item a commutative monad $T$ on $\cat{C}$ such that
  \item the extension operation of $T$ involves coends.
\end{enumerate}

These conditions are contradictory. Coends are not unique, so defining an
extension operation involves choosing particular coends. It is then not
reasonable to demand that this choice is consistent in the sense that all the
monad laws are satisfied up to equality. Thus we will not obtain a monad in the
strict sense. Fortunately, monads have been generalised to pseudomonads to
account for this problem~\cite{marmolejo2013}.

Now the obvious candidate model is the presheaf construction, taking each small
category $\scat{C}$ to its presheaf category $\widehat{\scat C} =
\bb{\scatop{C}, \Set}$. This is a great choice because a functor
$\scat{C}\to\widehat{\scat C}$ may be thought of as a functor
$\scatop{C}\times\scat{C}\to\Set$. Thus coends are closely related to
presheaves. Secondly, it is known that the presheaf construction exhibits
monad-like behaviour~\cite{fiore2017}. Unfortunately, the presheaf category of
a small category is itself not necessarily small. Thus the presheaf
construction fails to be a pseudomonad.

One way to turn the presheaf construction into a pseudomonad is by considering
small presheaves~\cite{day2007}. This is still insufficient for our pruposes,
as the functor category $\bb{\scat{C},\scat{D}}$ of small categories $\scat{C}$
and $\scat{D}$ is itself not small. This violates our first requirement of
cartesian closure. We have tried and failed to find a cartesian closed
2-category on which the presheaf construction remains a pseudomonad.

We thus opt for relative pseudomonads~\cite{fiore2017}. This generalisation
allows for the possibility that the pseudofunctor underlying a pseudomonad is
itself not an endomorphism. We know that the presheaf construction is a
relative pseudomonad. Before we can extend synthetic measure theory to include
relative pseudomonads, we need to define the corresponding notion of strength.
This turns out to be a difficult problem that we are not able to solve
entirely.

\section{Related work}

There are four pieces of work that this project builds on:

\begin{itemize}
  \item Kock's synthetic measure theory from~\cite{kock2011} is our main motivation. It
    is therefore especially helpful in making sure what requirements our theory needs
    to satisfy and what simplifications we are able to make.
  \item Fiore et al.~developed the theory theory of relative pseudomonads in~\cite{fiore2017}.
    This is our main reference point for the first part of our project.
  \item  Paquet and Saville defined strong pseudomonads in~\cite{saville2023}. This
    serves as a guide for the second part of our theory and will most likely be a
    valuable reference for future work.
  \item Uustalu outlined in~\cite{tarmo} how to construct a strong relative monad. While
    this extended abstract does not justify the correctness of the definition, it has allowed
    us to come up with a sensible structure of a strong relative pseudomonad nonetheless.
\end{itemize}

\section{Contributions}

The following developments in this report are worth noting:

\begin{itemize}
  \item We partially define strong relative pseudomonads with a particular 
    focus on generalising synthetic measure theory.
  \item We show that our definition coincides with relative pseudomonads
    in the sense of~\cite{fiore2017}.
  \item We show how our definition yields the structure of a strong
    pseudomonad in the sense of~\cite{saville2023}.
  \item We demonstrate how to endow the presheaf construction with the
    necessary structure and verify that it satisfies the corresponding
    axioms.
\end{itemize}

\section{Overview}

We structure this report as follows:
\begin{itemize}
  \item Chapter 2 revisits introductory category theory. We define commutative
    monads and relative monads in order to develop an intuition for later
    developments. We then focus our attention towards coends, getting used to
    the notation and proving results that will be useful later on.
  \item Chapter 3 rigorously introduces 2-categories, pseudofunctors, and
    pseudonatural transformations. We pay particular attention to the strictness
    requirements, making sure that our definitions are not more general than required.
  \item Chapter 4 consists entirely of novel developments: step by step we work towards
    the definition of a strong relative pseudomonad. Along the way we show how the
    structures and axioms relate to previous work and to the presheaf construction.
  \item Chapter 5 contains a critical evaluation.
  \item Chapter 6 summarises the remaining work and lists possible extensions.
\end{itemize}


